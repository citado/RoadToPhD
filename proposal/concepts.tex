\فصل{مفاهیم}

\قسمت{مقدمه}

اینترنت اشیا اولین بار توسط \متن‌لاتین{Kevin Ashton} در سال ۱۹۹۹ پیشنهاد شد، او اینترنت اشیا را به عنوان
شبکه‌ای از اشیا هم‌کنشپذیر، قابل شناسایی به وسیله‌ی فرکانس رادیویی (\متن‌لاتین{RFID}) تعریف می‌کند.
کلمه‌های ``اینترنت'' و ``اشیا'' به معنی شبکه‌ای متصل و جهانی بر پایه حسگرها، ارتباطات، شبکه‌سازی و تکنولوژی‌های پردازش داده است
که نسخه‌ی جدیدی از فناوری اطلاعات یا \متن‌لاتین{ICT} را عرضه می‌کند.
پیشرفت تکنولوژی‌های حسگر بی‌سیم کمک شایانی به قابلیت‌های دستگاه‌های حسگر و بنابراین مفهوم پایه اینترنت اشیا کرده است.
اهمیت اینترنت اشیا برای کشورهای در حال توسعه و توسعه یافته بسیار واضح است و این کشورها شروع به سرمایه‌گذاری در این حوزه نموده‌اند.
\مرجع{Li2014}

یکی از بحث‌های در اینترنت اشیا نبود استانداردها و تاثیر آن بر گسترش آن است. در این سال‌ها فعالیت‌های زیادی در حوزه طراحی استانداردها
صورت گرفته است. مهمترین این استانداردها برای میان‌افزارها و رابط‌ها بوده‌اند. این استانداردها که هر یک توسط گروه‌ها و ارگان‌های خاصی تهیه می‌شوند
خود می‌بایست هماهنگی و ارتباط داشته باشند.
\مرجع{Li2014}

تحقیقات و پژوهش‌های متفاوت اینترنت اشیا را به شکل‌های مختلفی تعریف کرده‌اند که از جمله‌ی آن‌ها می‌توان به
\شروع{نقل}
یک زیرساخت شبکه‌ای پویا و جهانی با قابلیت‌های خودپیکربندی بر پایه پروتکل‌های ارتباطی استاندارد و هم‌کنشپذیر که در آن اشیا فیزیکی و مجازی دارای شناسه، صفات فیزیکی
و شخصیت مجازی بوده، از رابط‌های هوشمند استفاده می‌کنند و به صورت یکپارچه در شکبه اطلاعاتی مجمتع می‌شوند.
\پایان{نقل}

\شروع{نقل}
یک زیرساخت جهانی برای اجماع اطلاعاتی که سرویس‌های پیشرفته‌ای را با برقراری ارتباط میان اشیا فیزیکی و مجازی بر پایه تکنولوژی‌های حاضر، هم‌کنشپذیر و رو به گسترش اطلاعاتی
و ارتباطی فراهم می‌آورد.
\پایان{نقل}
اشاره کرد. با توجه به آغاز اینترنت اشیا از \متن‌لاتین{RFID}ها بسیاری از پژوهش‌ها رشد این تکنولوژی در بستر \متن‌لاتین{RFID} را مورد بحث قرار داده‌اند.
\مرجع{Ray2018}

در سال‌های اخیر با کاهش قسمت حسگرهای و عملگرها، تعداد دستگاه‌های اینترنت اشیا به سرعت در حال گسترش است و به سرعت در حال تبدیل کردن خود به یکی از اجزا زندگی ما می‌باشند.
در نتیجه رد پای فاحش اینترنت اشیا امروزه در همه جا قابل مشاهده است.
\مرجع{Mishra2021}

\قسمت{اینترنت اشیا}

اینترنت اشیا از اجزای مختلفی تشکیل شده است که کنار یکدیگر سیستم اینترنت اشیا را تشکیل می‌دهند. در این قسمت قصد داریم مروری بر این اجزا داشته باشیم.

در شکل \رجوع{شکل: اجزای کارکردی یک دستگاه اینترنت اشیا} اجزای کارکردی یک دستگاه اینترنت اشیا آورده شده است.
اینترنت اشیا بر پایه همین دستگاه‌ها که دارای حسگر و عملگر بوده و می‌توانند عملیات مانیتورینگ و کنترل را انجام دهند، بنا شده است.
این دستگاه‌ها می‌توانند با سایر دستگاه‌ها و برنامه‌های کاربردی متصل تبادل داده داشته باشند یا می‌توانند داده‌ها را از سایر دستگاه‌ها جمع‌آوری کرده
و آن‌ها را پردازش کنند یا در جهت پردازش آن‌ها را به سرورهای ابری ارسال کنند. این تقسیم پردازشی بین پردازش محلی و ابری می‌تواند بر پایه محدودیت‌های پردازشی و حافظه‌ای
صورت بپذیرد.
\مرجع{Ray2018}

یک دستگاه اینترنت اشیا می‌توانند چند رابط جهت ارتباط با سایر اشیا به صورت بی‌سیم یا دارای سیم داشته باشد.
رابط‌های یک دستگاه اینترنت اشیا را می‌توان به اقسام زیر تقسیم کرد:

\شروع{فقرات}
\فقره رابط‌های ورودی و خروجی برای ارتباط با حسگرها و عملگرها
\فقره رابط‌های شبکه
\فقره رابط‌های ذخیره‌سازی و حافظه
\فقره رابط‌های صوتی و تصویری
\پایان{فقرات}

دستگاه‌های اینترنت اشیا می‌توانند در اشکال متفاوتی ظاهر شوند که از جمله‌ی آن می‌توان به شکل‌های، حسگرهای پوشیدنی، ساعت‌های هوشمند،
روشنایی‌های \متن‌لاتین{LED}، خودرو و ماشین‌های صعنعی. اشیا می‌توانند داده‌هایی در فرمت‌های گوناگون تولید کنند که بعد از پردازش توسط سیستم‌های
پردازش داده باعث اطلاعات کاربردی می‌شوند که می‌توان از آن‌ها برای واکنش‌های مستقیم یا غیرمستقیم استفاده نمود.
\مرجع{Ray2018}

\شروع{شکل}
\درج‌تصویر[width=\textwidth]{./img/iot-device-components.png}
\تنظیم‌ازوسط
\شرح{اجزای کارکردی یک دستگاه اینترنت اشیا \مرجع{Ray2018}}
\برچسب{شکل: اجزای کارکردی یک دستگاه اینترنت اشیا}
\پایان{شکل}

قسمت بعدی در اینترنت اشیا ارتباطات است، این قسمت وظیفه برقراری ارتباط میان دستگاه‌ها و سرورها را برعهده دارد.
سرویس‌ها قسمت سوم در اینترنت اشیا هستند. سرویس‌ها می‌توانند کارکردهای مختلفی برای مدل کردن دستگاه‌ها، کنترل دستگاه،
انتشار داده‌ها، آنالیز داده‌ها و اکتشاف دستگاه‌ها را فراهم آورند.
\مرجع{Ray2018}

بلوک مدیریت در اینترنت اشیا زیرساخت قانون‌گذاری در اینترنت اشیا را با استفاده از کارکردهای مدیریتی فراهم می‌آورد.
بلوک امنیت کارکردهایی چون احراز هویت، سطوح دسترسی، حریم خصوصی، یکپارچی پیام‌ها، یکپارچی محتوا و امنیت داده‌ها را فراهم می‌آورد.
در نهایت بلوک برنامه‌های کاربردی قرار دارد که رابط کاربری مشتریان بوده و ماژول‌های لازم در جهت کنترل و نظارت بر جنبه‌های مختلف سیستم اینترنت اشیا
را فراهم می‌آورد. برنامه‌های کاربردی به کاربران اجازه می‌دهند داده‌های خود را نمایش دهند، سیستم را ارزیابی کنند و گاها جنبه‌های مختلف آن را پیش‌بینی کنند.
\مرجع{Ray2018}

سیستم‌های اینترنت اشیا می‌بایست پویا بوده و خود تطبیق‌پذیر باشند. مثلا دوربین‌های نظارتی باید بتوانند به صورت خودکار دید در شب را فعال کنند یا در صورت تشخیص حرکت
کیفیت ضبط خود و سایر دوربین‌ها را افزایش دهند. سیستم‌های اینترنت اشیا می‌توانند قابلیت خودپیکربندی داشته باشند. آن‌ها ممکن است بتوانند بروزرسانی‌های سیستم خود یا
تنظیمات رابط شبکه‌ای را با کمترین دخالت کاربران انجام بدهند. دستگاه‌های اینترنت اشیا از پروتکل‌های هم‌کنشپذیر برای ارتباط با زیرساخت و سایر اشیا استفاده می‌کنند.
این دستگاه‌ها در سیستم اینترنت اشیا می‌بایست یک شناسه یکتا مانند آدرس \متن‌لاتین{IP} یا \متن‌لاتین{URI} داشته باشند. دستگاه‌های اینترنت اشیا عموما با شبکه‌های اطلاعاتی
یکپارچه‌سازی می‌گردند. در این یکپارچه‌سازی اشیا می‌توانند به صورت پویا در شبکه کشف شوند و بتوانند ویژگی‌ها خود برای سایر اشیا و برنامه‌های کاربردی ارائه دهند. این یکپارچی
سیستم اینترنت اشیا را هوشمندتر می‌کند چرا که اشیا و زیرساخت در کنار یکدیگر می‌توانند هوشمندی بالاتری داشته باشند.
در نهایت سیستم‌های اینترنت اشیا می‌توانند با استفاده از اشیای متفاوت تصمیمات هوشمندانه‌تری بگیرند.
\مرجع{Ray2018}

در نهایت می‌توان ویژگی‌های اینترنت اشیا به شرح زیر خلاصه کرد:
\شروع{فقرات}
\فقره \متن‌سیاه{اتصال متقابل}: اینترنت اشیا می‌تواند به زیرساخت ارتباطات جهانی متصل شود.
\فقره \متن‌سیاه{سرویس‌های مرتبط با اشیا}: اینتنرت اشیا در ارائه اشیا فیزیکی و مجازی، حریم خصوصی و سرویس‌های سازگار معنایی در محدوده اشیا مهارت دارد.
\فقره \متن‌سیاه{تنوع}: دستگاه‌های اینترنت اشیا وابسته به زیرساخت‌های سخت‌افزاری و شبکه‌های متنوع هستند.
\فقره \متن‌سیاه{منابع محدود}: دستگاه‌های اینترنت اشیا محدودیت‌های پردازشی و توان مصرفی دارند.
\فقره \متن‌سیاه{تغییرات پویا}: وضعیت دستگاه‌ها و محیط در اینترنت اشیا ممکن است به صورت پویا تغییر کند.
\فقره \متن‌سیاه{محیط‌های کنترل نشده}: دستگاه‌های اینترنت اشیا در محیط‌هایی با تنظیمات کنترل‌نشده مستقر می‌شوند.
\فقره \متن‌سیاه{ابعاد بسیار بزرگ}: دستگاه‌هایی که در اینترنت اشیا می‌بایست نظارت شوند یا آن‌هایی که با یکدیگر ارتباط دارند بسیار زیاد هستند و به صورت نمهایی به رشد خود در آینده ادامه می‌دهند.
\پایان{فقرات}
\مرجع{Angel2021}

\زیرقسمت{سیستم‌های سایبر فیزیکال}

اگر بخواهیم خیلی کلی صحبت کنیم، سیستم‌های سایبر فیزیکال یا انحصرا \متن‌لاتین{CPS}، سیستم‌هایی هستند که به صورت بهینه قسمت‌های فیزیکی و سایبری را به وسیله ادغام تکنولوژی‌های ارتباطی و محاسباتی مدرن،
ادغام می‌کنند و هدفشان تغییر روش تعامل انسان، سایبر و محیط فیزیکی است.
\متن‌لاتین{CPS} ترکیب اجزای فیزیکی، سنسورها، عملگرها، شبکه‌های ارتباطی و مراکز کنترل است. سنسورها برای نظارت و اندازه‌گیری وضعیت اجزای فیزیکی مستقر می‌شوند.
عملگرها برای اطمینان از عملیات‌های مورد نظر بر اجزای فیزیکی مستقر می‌شوند. شبکه‌های ارتباطی برای رساندن داده‌های اندازه‌گیری شده و دستورات بازخوردی در میان حسگرها، عملگرها و مراکز کنترلی استفاده می‌شود.
مراکز کنترلی برای آنالیز داده‌ّهای اندازه‌گیری شده و ارسال دستورات بازخورد به عملگرها و اطمینان از عملکرد کلی سیستم در وضعیت موردنظر استفاده می‌شوند.
\مرجع{Lin2017}

با توجه به تعریفی که از \متن‌لاتین{CPS} ارائه شد، می‌دانیم که سیستم‌های سایبر فیزیکال و اینترنت اشیا هر دو قصد ایجاد تعامل میان دنیای سایبری و فیزیکی را دارند.
با توجه به شباهت‌های میان سیستم‌های سایبر فیزیکال و اینترنت اشیا، نیاز فوری به تفیکیک میان این دو موضوع وجود دارد.
سیستم‌های سایبر فیزیکال به ماهیت سیستم تاکید دارد، در \متن‌لاتین{CPS} لایه حسگر و عملگر در جهت جمع‌اوری داده همزمان و اجرای دستورات،
لایه ارتباط برای رساندن داده‌ها به لایه بالاتر (لایه کاربرد) و رساندن دستورات به لایه پایین‌تر (لایه حسگر و عملکگر) و لایه کاربر یا کنترل در جهت پردازش داده‌ها
و تصمیم‌گیری وجود دارند. بنابراین می‌توان گفت معماری \متن‌لاتین{CPS} یک معماری عمودی است.
اما اینترنت اشیا یک معماری مبتنی بر شبکه‌سازی است که قصد دارد تعداد زیادی از دستگاه‌ها را برای نظارت و کنترل به وسیله‌ی تکنولوژی‌های مدرن در دنیای سایبری، به یکدیگر متصل کند.
بنابراین هدف اینترنت اشیا اتصال شبکه‌های مختلف است که به وسیله‌ی آن جمع‌اوری داده، اشتراک منابع، پردازش و مدیریت بتواند در میان شبکه‌های مختلف رخ دهد.
در نهایت می‌توان نتیجه گرفت که اینترنت اشیا یک معماری افقی است که قصد دارد لایه‌های ارتباط از سیستم‌های سایبر فیزیکال گوناگون را برای بدست آوردن ارتباط به یکدیگر متصل کند.
\مرجع{Lin2017}

اگر بخواهیم مثالی از ارتباط میان سیستم‌های سایبر فیزیکال و اینترنت اشیا بزنیم، شهرهای هوشمند مثال خوبی خواهد بود. شهرهای هوشمند از سیستم‌های سایبر فیزیکال متنوعی چون
شبکه برق هوشمند، حمل و نقل هوشمند، سلامت هوشمند و \نقاط‌خ تشکیل شده‌اند. لایه ارتباطی همه این سیستم‌ها به یکدیگر متصل شده‌اند و یک سرویس یکپارچه برای شهر هوشمند را
تشکیل داده‌اند.
\مرجع{Lin2017}

با توجه به آنچه بیان شد، لایه کنترل می‌بایست به گونه‌ای طراحی شود که بتواند از شبکه‌های گوناگون و اشتراکی برای داده‌ها استفاده کند.
در واقع، لایه کنترل در اینترنت اشیا بسیار پیچیده‌تر از اینترنت است و تا به حال به خوبی به آن پرداخته نشده است.
\مرجع{Lin2017}

\زیرقسمت{معماری‌های اینترنت اشیا}

در این سال‌های معماری‌های زیادی برای اینترنت اشیا پیشنهاد شده است که بسیاری از آن‌ها در حوزه‌های مشخصی بوده‌اند.
یکی از نگاه‌های در معماری اینترنت اشیا نگاه \متن‌لاتین{Service Oriented Architecture} یا مختصرا \متن‌لاتین{SOA} است.
در این نگاه از سرویس‌ها استفاده می‌شود و آنچه از \متن‌لاتین{SOA} امروزه در اینترنت اشیا استفاده می‌شود شامل میان‌افزار است، میان‌افزار نرم‌افزاری است که
میان برنامه‌های کاربردی و اشیا قرار گرفته و پیچیدگی‌ها را از دید کاربر پنهان می‌کند. این لایه زمان توسعه را کاهش داده و کمک می‌کند محصول برای بازار
زودتر حاضر شود. معماری‌هایی بر همین اساس برای شبکه‌های سنسور بی‌سیم، کشاورزی هوشمند، ارزیابی کیفیت آب هوشمند، \نقاط‌خ مطرح شده است که در پژوهش \مرجع{Ray2018} به طور خلاصه آمده‌اند.
\مرجع{Ray2018}

در این سال‌های سیستم‌های ابری زیادی برای اینترنت اشیا شکل گرفته‌اند که کارکردهایی مانند مدیریت اشیا، مدیریت سیستم، مدیریت تنوع، مدیریت داده‌ها، پردازش و
نظارت را فراهم می‌آورند. این سرویس‌های ابری عموما دارای \متن‌لاتین{Gateway} برای ارسال داده‌ها به ابر هستند. دو نمونه از چالش‌های مهم در این حوزه ارتباطات یکسان و
شناسه‌ی یکسان برای اشیا در این سیستم‌های ابری است. سازمان \متن‌لاتین{ETSI} در حال تلاش برای استانداردسازی این موارد است.
\مرجع{Ray2018}

یکی از مسائل مطرح در اینرتنت اشیا حجم داده‌ها است و معماری‌هایی برای پردازش و نگهداری این ابرداده‌ها پیشنهاد شده‌اند. این معماری‌ها از سیستم‌هایی چون \متن‌لاتین{Spark} یا \متن‌لاتین{Flink} استفاده می‌کنند.
از مسائل دیگر در معماری اینترنت اشیا می‌توان به \متن‌لاتین{Fog Computing} یا پردازش مِه اشاره کرد.
\مرجع{Ray2018}

اگر بخواهیم به معماری مبتنی بر سرویس مطابق با آنچه در شکل \رجوع{شکل: معماری مبتنی بر سرویس اینترنت اشیا} دقیق‌تر نگاه کنیم، آن را می‌توان در چهار لایه با کارکردهای متفاوت دسته‌بندی کرد:

\شروع{فقرات}
\فقره \متن‌سیاه{لایه حسگر} با اشیا حاضر در هم آمیخته می‌شود تا بتواند وضعیت اشیا را حس کند. در واقع در این لایه سنسورها می‌توانند به صورت خودکار وضعیت محیط را ارزیابی کرده و با سایر اشیا تبادل اطلاعات داشته باشند.
در این لایه هزینه، اندازه، منابع و مصرف انرژی باید بهینه باشد. در این لایه شبکه‌بندی اشیا، ارتباطات و انتخاب پروتکل ارتباطی مناسب، تنوع و نحوه استقرار از سایر دغدغه‌ها است.
\فقره \متن‌سیاه{لایه شبکه} زیرساختی است که ارتباط بی‌سیم یا دارای سیم میان اشیا را پشتیبانی می‌کند. در این لایه اشیا به یکدیگر متصل شده و می‌توانند از اطراف باخبر شوند. این لایه می‌تواند داده‌ها را با زیرساخت فناوری اطلاعات فعلی
ترکیب کرده یا آن‌ها را برای واحد تصمیم‌گیری شامل سرویس‌های سطح بالا و پیچیده ارسال کند. در لایه شبکه دغدغه‌هایی مانند مدیریت شبکه، مصرف انرژی شبکه، نیازمندی‌های کیفیت سرویس، پیدا کردن اشیا و امنیت مطرح است.
\فقره \متن‌سیاه{لایه سرویس}، سرویس‌های لازم برای کاربران یا برنامه‌های کاربردی را مدیریت کرده یا می‌سازد. این لایه از تکنولوژی میان‌افزار استفاده می‌کند، که یک تکنولوژی کلیدی در سرویس‌ها و برنامه‌های کاربردی اینترنت اشیا است.
سرویس‌ها را می‌توان به عنوان یک رفتار، شامل جمع‌آوری، جابجایی و ذخیره‌سازی داده یا ترکیب این‌ها برای رسیدن به یک هدف مشخص، در نظر گرفت.
این لایه خود از اجزا زیر تشکیل شده است.
\شروع{فقرات}
\فقره \متن‌سیاه{سرویس اکتشاف} این سرویس اشیایی که اطلاعات یا سرویس مشخصی را فراهم می‌آورند، پیدا می‌کند.
\فقره \متن‌سیاه{ترکیب سرویس‌ها} ارتباط میان اشیا متصل را ممکن می‌سازد. سرویس اکتشاف اشیا را برای یافتن سرویس موردنظر جستجو می‌کند و ترکیب سرویس‌ها با زمان‌بندی و باز ساخت سرویس‌های بیشتر مطمئن‌ترین سرویس را ارائه میدهد.
در واقع در اینترنت اشیا برخی از نیازمندی‌ها می‌توانند با یک سرویس برآورده شوند و برخی برای برآورده شدن به ترکیب سرویس‌ها نیاز دارند.
\فقره \متن‌سیاه{مدیریت ارزشمندی} درک می‌کند که داده‌های هر سرویس می‌بایست چگونه پردازش شوند.
\فقره \متن‌سیاه{رابط‌های کاربری سرویس‌ها} چگونگی ارتباط سرویس‌ها با کاربران و اشیا را مشخص می‌کنند.
\پایان{فقرات}
\فقره \متن‌سیاه{لایه رابط‌ها} شامل روش‌هایی برای ارتباط با کاربران و برنامه‌ها است. همانطور که بیان شد اینترنت اشیا از تعداد زیادی دستگاه متنوع تشکیل شده است و یک رابط کاربری کارا برای ساده‌سازی مدیریت و ارتباط اشیا لازم است.
\پایان{فقرات}

معماری مبتنی بر سرویس یک سیستم پیچیده را در قالب یک مجموعه خوش تعریف از اشیا یا زیر سیستم‌های ساده می‌بیند.
این اجزای به صورت مستقل قابلیت استفاده و تغییر دارند بنابراین اجزای نرم‌افزاری و سخت‌افزاری در اینترنت اشیا می‌توانند به صورت کارایی
باز استفاده یا به روزرسانی شوند.
\مرجع{Li2014}

میان‌افزار در لایه سرویس، در واقع یک نرم‌افزار است که انتزاعی را میان تکنولوژی‌های اینترنت اشیا و برنامه‌های کاربردی دخیل می‌کند.
در میان‌افزار جزئیات تکنولوژی پنهان می‌شود و رابط‌های استانداری تعریف می‌شود که اجازه می‌دهد برنامه‌نویسان بدون دغدغه هماهنگی میان برنامه‌ها و زیرساخت، بر توسعه برنامه‌ها تمرکز کنند.
بنابراین با استفاده از میان‌افزارها برنامه‌ها و دستگاه‌هایی با رابط‌های متفاوت می‌توانند با یکدیگر تبادل اطلاعات داشته و منابع را به اشتراک بگذارند.
\مرجع{Lin2017}

برای میان‌افزار می‌توان سودهای زیر را در نظر گرفت:
\شروع{فقرات}
\فقره میان‌افزارها می‌توانند از برنامه‌های کاربردی متنوع پشتیبانی کنند.
\فقره میان افزار می‌تواند روی زیرساخت‌ها و سیستم‌عامل‌های مختلفی اجرا شود.
\فقره میان‌افزار می‌تواند از محاسبات توزیع شده و تعامل میان سرویس‌ها در بین شبکه‌ها، اشیا و برنامه‌های کاربردی گوناگون پشتیبانی کند.
\فقره میان‌افزار می‌تواند از پروتکل‌های استاندارد پشتیبانی کند.
\فقره میان‌افزار می‌تواند رابط‌های استاندارد فراهم آورد، قابلیت انتقال فراهم آورد و می‌تواند پروتکل‌های استاندارد در جهت همکنش‌پذیری فراهم آورد که باعث می‌شود میان‌افزار نقش مهمی در استانداردسازی داشته باشد.
\فقره میان‌افزار می‌تواند یک رابط سطح بالا برای برنامه‌های کاربردی فراهم آورد.
\پایان{فقرات}
\مرجع{Lin2017}

پژوهش‌های میان‌افزارها را به ۵ دسته کلی تقسیم کرده‌اند:
\شروع{فقرات}
\فقره \متن‌سیاه{میان‌افزارهای پیام محور}: این میان‌افزارها می‌توانند جایجایی اطلاعات مطمئن میان پلتفرم‌ها و پروتکل‌های ارتباطی را فراهم آورند.
\فقره \متن‌سیاه{میان‌افزارهای مبتنی بر وب معنایی}: این میان‌افزارها می‌توانند تعامل و همکنش‌پذیری را میان شبکه‌های حسگر مختلف فراهم کنند.
\فقره \متن‌سیاه{میان‌افزارهای سرویس موقعیت و نظارتی}: این میان‌افزارها موقعیت اشیا را با سایر اطلاعات ترکیب کرده و از آن برای فراهم آوردن سرویس با ارزش یکپارچه استفاده می‌کنند.
\فقره \متن‌سیاه{میان‌افزارهای ارتباطی}: این میان‌افزارها ارتباطات قابل اطمینان میان اشیا و برنامه‌های کاربردی متنوع فراهم می‌آورند.
\فقره \متن‌سیاه{میان‌افرارهای فراگیر}: این میان‌افزارها برای محیط‌های محاسباتی فراگیر طراحی شده‌اند و می‌توانند روی چندین زیرساخت گوناگون اجرا شوند.
\پایان{فقرات}
\مرجع{Lin2017}

\شروع{شکل}
\درج‌تصویر[width=\textwidth]{./img/iot-soa-architecture.png}
\تنظیم‌ازوسط
\شرح{معماری مبتنی بر سرویس اینترنت اشا \مرجع{Li2014}}
\برچسب{شکل: معماری مبتنی بر سرویس اینترنت اشیا}
\پایان{شکل}

یک سرویس می‌تواند به صورت مجموعه‌ای از داده‌ها و رفتارهای وابسته یا قسمتی از یک دستگاه یا ویژگی از یک دستگاه دیده شود یک کارکرد خاص را صورت می‌دهد.
برای معرفی سرویس‌ها نیاز به یک استاندارد است که در این استاندادرهای متنوعی پیشنهاد شده است مثل استفاده از \متن‌لاتین{XML} در استاندارد پیشنهادی از \متن‌لاتین{Special Interest Group} مربوط به \متن‌لاتین{Bluetooth}.
\مرجع{Li2014}

یکی دیگر از معماری‌های مرسوم در اینترنت اشیا، معماری‌های لایه‌ای هستند. ساده‌ترین این معماری‌ها سه لایه \متن‌لاتین{Perception}، \متن‌لاتین{Network} و \متن‌لاتین{Application}
است. از سایر معماری‌های لایه‌ای می‌توان به معماری ۴ لایه‌ای \متن‌لاتین{Things}، \متن‌لاتین{Edge}، \متن‌لاتین{Network} و \متن‌لاتین{Application} و معماری ۵ لایه‌ای
\متن‌لاتین{Business}، \متن‌لاتین{Application}، \متن‌لاتین{Service}، \متن‌لاتین{Object Abstraction} و \متن‌لاتین{Objects} اشاره کرد. در ادامه به توضیح بیشتر این معماری‌ها می‌پردازیم.
\مرجع{FerrndezPastor2018}

همانطور که بیان شد، معماری سه‌لایه‌ای یک معماری ساده و پایه‌ای مشتمل بر لایه‌های \متن‌لاتین{Perception} یا ادراک، \متن‌لاتین{Network} یا شبکه و \متن‌لاتین{Application} یا کاربرد است.
این لایه‌های در ادامه بیشتر بحث خواهند شد.
\شروع{فقرات}
\فقره \متن‌سیاه{لایه ادراک} که با نام لایه‌ی حسگر نیز شناخته می‌شود در پایین این معماری اینترنت اشیا پیاده‌سازی می‌شود. این لایه با اجزا و دستگاه‌های فیزیکی به وابسته دستگاه‌های هوشمند ارتباط برقرار می‌کند.
هدف اصلی این لایه اتصال اشیا به شبکه اینترنت اشیا است تا بتوان اطلاعات وضعیتی این اشیا را به واسطه دستگاه‌های هوشمند مستقر شده، اندازه‌گیری، جمع‌اوری و پردازش کرد و در نهایت داده‌های پردازش شده به
وسیله‌ی رابط لایه، به لایه‌ی بالاتر ارسال شود.
\فقره \متن‌سیاه{لایه شبکه} که با نام لایه انتقال نیز شناخته می‌شود به عنوان یک لایه‌ی میانی در این معماری اینترنت اشیا پیاده‌سازی می‌شود. این لایه داده‌های پردازش شده توسط لایه ادراک را دریافت کرده و
مسیرهایی را برای انتقال این داده‌ها و اطلاعات به دستگاه‌ّها، هاب و برنامه‌های کاربردی توسط شبکه مجتمع مشخص می‌کند. لایه شبکه در اینترنت اشیا لایه‌ی مهمی است چرا که دستگاه‌های زیادی (مانند
هاب، سوئیچ‌ها، \متن‌لاتین{Gateway}ها، زیرساخت محاسبات ابری و \نقاط‌خ) و تکنولوژی‌های ارتباطی زیادی (مانند \متن‌لاتین{Bluetooth}، \متن‌لاتین{WiFi} و \نقاط‌خ) در این لایه جمع شده‌اند.
\فقره \متن‌سیاه{لایه کاربرد} که با نام لایه کسب و کار هم شناخته می‌شود در بالای این معماری اینترنت اشیا پیاده‌سازی شده است. این لایه با دریافت داده‌های ارسالی از لایه‌ی شبکه از آن‌ها برای فراهم آوردن
سرویس‌ها و عملیات‌های لازم استفاده می‌کند.
\پایان{فقرات}

با وجود سادگی معماری سه لایه، کارکردهای لایه‌ی شبکه و کاربرد بسیار متنوع و گسترده است. به صورت مثال لایه شبکه در کنار مسیریابی می‌تواند سرویس‌هایی برای تجمبع‌داده‌ها ارائه دهد یا لایه کاربرد
می‌تواند سرویس‌هایی برای آنالیز داده‌ها ارائه کند. بنابراین لایه سرویس میان لایه‌ی کاربرد و شبکه ایجاد می‌شود که باعث معماری مبتنی بر سرویس خواهد که پیشتر به آن پرداختیم.
\مرجع{Lin2017}

\قسمت{شبکه‌های \متن‌لاتین{LPWAN}}

نیاز کاربردهای اینترنت اشیا روز به روز به تکنولوژی‌هایی که می‌توانند عملکرد توان پایین داشته باشند
و دستگاه‌های انتهایی که بتوانند ارتباط بی‌سیم در مسافت‌های طولانی را با هزینه و پیچیدگی پایین برقرار کنند، بیشتر می‌شود.
در بیشتر کاربردها، دستگاه‌های انتهایی اینترنت اشیا حسگرهایی با باتری می‌باشند، که پروفایل مصرف توان آن‌ها در جهت افزایش طول عمر
باتریشان می‌بایست با دقت طراحی شده باشد.
برد ارتباطی نیاز دارد از چند صد متر تا چندین کیلومتر را شامل شود چرا که دستگاه‌های ارتباطی در محیط عملیاتی بزرگی گسترده‌اند.
با نظر گرفتن همه ویژگی‌های نامبرده، این امر تنها با استفاده از تکنولوژی‌های حوزه شبکه‌های توان پایین با برد بالا\پانویس{LPWAN} ممکن است.
\مرجع{sensors-18-03995}

\قسمت{نیازمندی‌های شبکه‌های \متن‌لاتین{LPWAN}}

یکی از بحث‌های اصلی این رساله شبکه‌های با گستره بالا و توان پایین است. این شبکه نیازمندی‌های خاصی دارند و تکنولوژی‌های پیشنهاد شده در این حوزه
هر یک روش‌هاص خاص خود را برای پاسخ به این نیازمندی‌های دنبال کرده‌اند.

\زیرقسمت{نیازمندی‌های ترافیک}

در شبکه‌های اینترنت اشیا، حسگرها و سنسورها همگی از رفتار ترافیکی یکسانی پیروی نمی‌کنند. برخی نیاز دارند که پیام‌هایشان بلافاصله به مقصد برسد و این در حالی است که برخی می‌توانند تاخیر را تحمل کنند.
از سوی دیگر با افزایش تعداد اشیا برای فرآهم آوردن کیفیت سرویس نیاز به یک زمان‌بند با الویت است. پیام‌های ارسالی خود می‌توانند تعداد یا نرخ و اندازه‌های متفاوتی داشته باشند.
در نتیجه شبکه‌های اینترنت اشیا می‌بایست ظرفیت این تنوع ترافیکی را داشته باشند.
\مرجع{Chaudhari2020}

به صورت پایه‌ای شبکه‌های می‌بایست نوعی از مدیریت ترافیک کاربران و کنترل پذیرش را داشته باشند.
این امر البته به معماری شبکه دسترسی هم وابسته است، برخی از پروتکل‌های ارتباطی بدون هیچ نوعی از کنترل پذیرش
داده‌ها را ارسال یا دریافت می‌کنند و در برخی نیاز به نوعی پذیرش وجود دارد.
در شبکه‌هایی که از اشیا متنوع پشتیبانی می‌کنند ممکن است این نیازمندی گسترش پیدا کند، به طور مثال
شبکه ممکن است پذیرش درخواست‌ها را بر پایه اولویت انجام دهد یا شرایط اذحام را مدیریت کند.
\مرجع{Chaudhari2020}

\زیرقسمت{ظرفیت و چگالی}

یکی از نیازمندی‌ها پشتیبانی از تعداد بالا اشیا است. شبکه‌ها می‌بایست بتوانند بدون اختلال در فعالیت نودهای کنونی نودهای جدیدی را به شبکه اضافه کنند.
با توجه به سادگی نودها بار اصلی این کار بر عهده \متن‌لاتین{Gateway}ها و نقاط دسترسی است.
\مرجع{Chaudhari2020}

در شبکه‌های بزرگ نیاز به یک شناسه‌ی یکتا جهانی وجود دارد.
در این شبکه‌ها با توجه به تعداد زیاد اشیا لینک‌های ارتباطی می‌بایست قابل اطمینان و کارا باشند.
از سوی دیگر تداخل در این شبکه‌ها می‌تواند بسیار باشد و نیاز است با شیوه‌هایی همچون استفاده از کانال‌های مختلف،
ارسال‌های تکراری و استفاده از روش‌های تطبیقی با این امر مقابله کرد.
برای روش‌های تطبیقی نیاز به ذخیره‌سازی وضعیت ارتباطی دستگاه‌ها در ایستگاه‌های پایه است و این امر با توجه به تعداد زیاد
اشیا نیاز به بهینه‌سازی دارد.
\مرجع{Chaudhari2020}

\زیرقسمت{توان مصرفی}

این شبکه‌ها نیاز دارند توان مصرفی پایینی داشته باشند چرا که بیشتر نودها با باتری فعالیت می‌کنند
و در برخی موارد جایگزینی باتری هم عمل سختی است.
\مرجع{Chaudhari2020}

دستگاه‌ها در این شبکه‌ها نرخ ارسال پایین دارند و برای ارسال نیز داده‌ها بسیار کوچک هستند
برای همین استفاده از مدهای خواب مناسب و غیر فعال کردن ماژول‌های پر مصرف مثل ماژول
ارتباطی برای زمان‌هایی که از آن‌ها استفاده نمی‌شود می‌تواند به میزان زیادی توان مصرفی را کاهش دهد.
\مرجع{Chaudhari2020}

اشیا در این شبکه‌ها می‌توانند از منابعی مانند باد یا انرژی خورشیدی برای شارژ کردن باتری خود استفاده کنند.
این بهبودها سربار دارند و می‌بایست بین این سربار و افزایش طول عمر باتری مصالحه نبود.
\مرجع{Chaudhari2020}

به صورت کلی مصرف توان برای پردازش‌های دستگاه بسیار کمتر از مصرف توان در ارسال و دریافت داده‌ها می‌باشد
بنابراین ساختاربندی داده‌ها پیش از ارسال می‌تواند به کاهش توان مصرفی کمک کند. از سوی دیگر می‌توان پیچیدگی‌های
ارسال را با در نظر گرفتن لایه‌ی مدیریت دسترسی همزمان ساده‌تر کاهش داد.
\مرجع{Chaudhari2020}

\زیرقسمت{پوشش‌دهی}

پوشش این شبکه‌ها در مناطق غیر شهری بین ۱۰ تا ۴۰ کیلومتر و
در مناطق شهری بین ۱ تا ۵ کیلومتر است.
این شبکه گاهاً نیاز دارند در مناطقی با دسترسی سخت یا داخل ساختمان‌ها
عملیاتی شوند. این شبکه‌ها با استفاده از باندهای زیرگیگاهرتز تلاش می‌کنند تا پوشش بیشتری را با توان کمتری بدست بیاورند.
\مرجع{Chaudhari2020}

مفهوم پوشش‌دهی خود می‌تواند به افزایش پوشش جفرافیایی، دسترسی به محیط‌های اطراف موانع و پوشش محیط‌های داخلی اشاره کند.
تکنیک‌های غلبه بر تضعیف سیگنال مانند افزایش توان ارسال، افزایش حساسیت آنتن‌ها، ارسال چندباره یک بسته و کاهش نرخ ماژولیشن اینجا کمک کننده هستند.
\مرجع{Chaudhari2020}

\زیرقسمت{موقعیت‌یابی}

یکی از نیازمندی‌ها دنبال کردن اشیا یا تشخیص رویدادهایی همچون تغییر موقعیت مکانی آن‌ها است.
برای این امر می‌توان از سیستم \متن‌لاتین{GPS} یا زیر ساخت شبکه استفاده کرد و دقت‌های مختلفی
از سانتی‌متر تا متر بدست آورد.
\مرجع{Chaudhari2020}

برای موقعیت یابی می‌توان از اطلاعات زمانی مربوط به زمان رسیدن سیگنال‌ها در جهت مکان‌یابی استفاده کرد. این روش
یکی از کم هزینه‌ترین روش‌ها است. در زمانی که از شبکه‌های سلولی استفاده می‌کنیم می‌توان از اطلاعات سلول نیز
در مکان‌یابی استفاده کرد. راهکارهای مبتنی بر ماهواره در کاربردهایی که حساس به توان مصرفی نباشد می‌تواند استفاده شود.
\مرجع{Chaudhari2020}

\زیرقسمت{امنیت و حریم خصوصی}

امنیت یکی از مسائل مهم در این شبکه‌ها است و می‌بایست مسائل پایه‌ای
مثل احراز هویت، سطوح دسترسی، اطمینان، محرمانگی، امنیت داده‌ها و عدم همسان‌سازی را در نظر بگیرد.
از سوی دیگری مسائلی چون حملات توزیع شده جلوگیری از دسترسی یا تزریق کد مخرب به شبکه و \نقاط‌خ
نیز وجود دارند که باید برای آن‌ها چاره اندیشی شود.
داده‌های می‌بایست در ارسال و دریافت رمزگذاری شوند تا امنیت آن‌ها و حریم خصوصی کاربران تضمین شود.
\مرجع{Chaudhari2020}

\زیرقسمت{هزینه مناسب}

از آنجایی که تعداد زیادی از اشیا در این شبکه دخیل هستند، هزینه نگهداری از آن‌ها و شبکه می‌بایست کم باشد.
از سوی دیگر به روزرسانی‌های نرم‌افزاری از ویژگی‌های مهم اشیا است که باعث کاهش هزینه‌ها می‌شود.
\مرجع{Chaudhari2020}

استفاده از سخت‌افزار ساده‌تر یکی از مهمترین گام‌ها در کاهش هزینه‌ها می‌باشد. از سوی دیگر به روزرسانی‌های
نرم‌افزاری که باعث کاهش تعداد به روزرسانی‌های سخت‌افزاری شوند می‌توانند به کاهش هزینه‌ها کمک کنند.
کارهای پردازشی و ارتباطی نیاز دارند که ساده باشند. سربار لایه‌های مختلف می‌بایست به گونه‌ای طراحی شوند
که کمینه باشند.
\مرجع{Chaudhari2020}

\زیرقسمت{اشیا با سخت‌افزارهایی با پیچیدگی کم}

از آنجایی که قصد داریم تعداد زیادی از اشیا را در برد بلندی و با هزینه پایین پوشش دهیم طراحی دستگاه‌ها با پیچیدگی کم
و ابعاد کوچک از نیازمندی‌های اساسی به شمار می‌آید. این اشیا نیازی به توان پردازشی بالا ندارند، معماری شبکه‌ای و پروتکل‌های
ساده‌ای را می‌بایست پشتیبانی کنند. دستگاه‌های ارتباطی آن‌ها ساده بوده و باید بتوانند به صورت نرم‌افزاری تنظیم شوند.
\مرجع{Chaudhari2020}

استفاده از تکنیک‌های رادیوهای نرم‌افزار بنیان می‌تواند اینجا بسیار کمک کننده باشد
البته باید در نظر داشت که استفاده از سخت‌افزارهای ساده و تکنیک‌های نرم‌افزاری
باعث ایجاد فرکانس‌های رادیویی غیر ایده‌آل می‌شود که برای رفع آن‌ها نیاز به استفاده
از پردازش‌های نرم‌افزاری سنگین است. بنابراین در اینجا مصالحه‌ای برای پیاده‌سازی
این تکنیک‌های ساده وجود دارد.
\مرجع{Chaudhari2020}

\زیرقسمت{گستردگی راه‌حل‌ها}

اشیا می‌بایست از شبکه‌های داری لایسنس و بدون لایسنس پشتیبانی کنند.
این اشیا می‌بایست از همبندی‌های متفاوت شبکه مانند \متن‌لاتین{Mesh}، \متن‌لاتین{Tree} و \متن‌لاتین{Star} پشتیبانی کنند.
\مرجع{Chaudhari2020}

راه‌های کارهای زیادی برای پیاده‌سازی شبکه وجود دارد بنابراین نودها می‌بایست چندین حالت و فرکانس را پشتیبانی کنند.
حالت‌های مختلف به نودها اجازه می‌دهد که در شبکه‌های مختلف که هر یک ویژگی‌های منحصر به فرد خود را دارند
فعالیت کند و از سوی دیگر فرکانس‌های مختلف به نود اجازه می‌دهد در یک تکنولوژی از چندین فرکانس مختلف استفاده کند.
\مرجع{Chaudhari2020}

\زیرقسمت{عملکرد، ارتباطات و روابط بین شبکه‌ای}

امروز شبکه‌های مختلفی با ویژگی‌های متفاوت وجود دارند اما با گسترگی \متن‌لاتین{IP} انتخاب آن به عنوان
یک استاندارد ارتباطی مطرح است. شبکه‌ها تلاش می‌کنند تا به شبکه‌های \متن‌لاتین{IP} و پروتکل‌هایی مانند
\متن‌لاتین{CoAP} متصل شوند و آن‌ها را پشتیبانی کنند.
در شبکه‌های \متن‌لاتین{LPWAN} اندازه داده‌ها کوچک است و از این رو برای ارسال بسته‌های \متن‌لاتین{IP} و
به خصوص \متن‌لاتین{IPv6} نیاز به فشرده‌سازی وجود دارد.
\مرجع{Chaudhari2020}

\قسمت{همبندی‌های شبکه‌های \متن‌لاتین{LPWAN}}

به صورت کلی دو همبندی در این شبکه‌ها مطرح است. همبندی \متن‌لاتین{Mesh} یا توری
و همبندی \متن‌لاتین{Start} یا ستاره.
در همبندی \متن‌لاتین{Mesh} همه نودها به یکدیگر متصل هستند از این رو زمانی که هدف
افزایش پوشش‌دهی و کاهش توان مصرفی است این همبندی ترجیح داده نمی‌شود. همبندی
\متن‌لاتین{Star} همبندی انتخابی در شبکه‌های \متن‌لاتین{LPWAN} می‌باشد.
در این همبندی با استفاده از یک نود می‌توان به تعداد زیادی نود دسترسی داشت که
خود باعث کاهش هزینه است.
\مرجع{Chaudhari2020}

در شبکه‌های \متن‌لاتین{Star} نود به صورت مستقیم با یک یا چند \متن‌لاتین{Gateway} در ارتباط هستند
و با یکدیگر ارتباطی ندارند. در این شبکه‌ها اطلاعات توسط \متن‌لاتین{Gateway} برای لایه‌های بالاتر ارسال می‌شود.
خود \متن‌لاتین{Gateway} ارتباطی با یکدیگر ندارند و از این رو این شبکه‌ها بسیار از شبکه‌های \متن‌لاتین{Mesh} ساده‌تر هستند.
در این شبکه‌ها نودها دارای ایراد به سادگی شناسایی می‌شوند ولی در صورت ایراد \متن‌لاتین{Gateway} تمام نودهای
متصل به آن دست می‌روند.
\مرجع{Chaudhari2020}

در شبکه‌های \متن‌لاتین{Mesh} کامل همه نودها به یکدیگر متصل هستند ولی در شبکه‌های \متن‌لاتین{Mesh} جزئی یا \متن‌لاتین{Partial Mesh} نودها
همه با یکدیگر ارتباط ندارند و با نودهایی که بیشترین پیام را رد و بدل کرده‌اند در ارتباط هستند.
مزیت این شبکه‌ها در وجود چندین راه ارتباطی بین نودها، امکان ارسال و دریافت همزمان داده‌ها از مسیرهای متفاوت،
گسترش‌پذیری آسان و قابلیت بهبود خودکار است. از سوی دیگر معایب آن شامل افزایش تاخیر به خاطر مسیرهایی با چند گام و
افزایش هزینه و پیچیدگی است.
\مرجع{Chaudhari2020}

\قسمت{معماری شبکه‌های \متن‌لاتین{LPWAN}}

معماری معمول شبکه‌های \متن‌لاتین{LPWAN} در شکل \رجوع{شکل: معماری معمول شبکه‌های LPWAN} آمده است.
کارکرد پایه یک دستگاه \متن‌لاتین{LPWAN} جمع‌اوری داده و پاسخ به ورودی‌های دریافتی از شبکه است.
داده‌های جمع‌اوری شده در یک لینک رادیویی مشخص برای ایستگاه دسترسی بی‌سیم ارسال می‌گردد.
ایستگاه بی‌سیم لینک رادیویی، برای مدیریت دستگاه و تبادل اطلاعات فراهم می‌آورد.
این ایستگاه در ارتباط با \متن‌لاتین{Gateway} یا \متن‌لاتین{Concentrator} قرار دارد که در برخی از موارد هسته نامیده می‌شود.
هسته وظیفه پشتبانی از لایه‌های کاربر و کنترل را دارد و همچنین وظیفه تبدیل پروتکل‌های قابل فهم برای شبکه و برنامه‌های کاربردی نیز از وظایف هسته است.
\مرجع{Chaudhari2020}

به خاطر نزدیکی \متن‌لاتین{Gateway} به اشیا در برخی از موارد از آن برای پردازش در لبه در کاربردهای همزمان استفاده می‌گردد.
از سوی دیگر پردازش و ذخیره‌سازی در لبه می‌توان بار پردازش ابری را کاهش دهد. در برخی از تکنولوژی‌های
از \متن‌لاتین{Gateway} برای کنترل درخواست و اولویت‌دهی نیز استفاده می‌گردد.
\مرجع{Chaudhari2020}

\شروع{شکل}
\درج‌تصویر[width=\textwidth]{./img/lpwan-arch.png}
\تنظیم‌ازوسط
\شرح{معماری معمول شبکه‌های \متن‌لاتین{LPWAN} \مرجع{Chaudhari2020}}
\برچسب{شکل: معماری معمول شبکه‌های LPWAN}
\پایان{شکل}

در معماری مرسوم، دستگاه به طور مستقیم با شبکه‌ی \متن‌لاتین{LPWAN} در ارتباط است.
تنظیمات دسترسی دیگری را نیز می‌توان در نظر گرفت. دو نمونه از معروف‌ترین آن‌ها
در ادامه آمده است.
معماری اول حالتی است که ارتباط اشیا از طریق تکنولوژی‌هایی مانند \متن‌لاتین{Zigbee}،
\متن‌لاتین{WiFi} و \نقاط‌خ فراهم می‌آید. \متن‌لاتین{Gateway} متناظر این اشیا از طریق
شبکه \متن‌لاتین{LPWAN} متصل می‌شود.
معماری دوم حالتی است که اشیا از چند شبکه \متن‌لاتین{LPWAN} به صورت همزمان استفاده می‌کنند.
\مرجع{Chaudhari2020}

\قسمت{زیرساخت‌های اینترنت اشیا}

حجم داده و پردازش مورد نیاز در اینترنت اشیا بسیار زیاد است و برای همین نیاز به زیرساختی قابل گسترش وجود دارد.
برای پاسخ به این نیاز پردازش ابری با گسترش‌پذیری و ظرفیت بالا، بهترین گزینه است.
در پردازش ابری عموما برنامه‌ها به صورت مجازی‌سازی شده اجرا می‌شوند که این امر کارایی، امنیت، گسترش‌پذیری و کاهش هزینه
را نسبت به اجرای روی \متن‌لاتین{bare-metal} به ارمغان می‌آورد.
\مرجع{Botez2021}

استفاده از کانتینرها برای اجرای برنامه‌ها بر بسترهای ابری به جای استفاده از مجازی‌سازی می‌تواند گسترش‌پذیری
بیشتری را به همراه داشته باشد.
\مرجع{Botez2021}

در شبکه‌های \متن‌لاتین{5G} بحث استفاده از کارکردهای مجازی شبکه مطرح شده است و هدف حذف کارکردهای فیزیکی
و استفاده از پیاده‌سازی‌های نرم‌افزاری آن‌ها است. اما هنوز مشکلاتی وجود دارد،
این پیاده‌سازی‌ها نرم‌افزاری در قالب ماشین‌های مجازی سربار زیادی دارند
و به سادگی نمی‌توانند آن‌ها را گسترش داد، مدیریت کرد یا هماهنگ نمود.
از این روی \متن‌لاتین{Cloud-Native Network Functions}ها مطرح می‌شوند که پیاده‌سازی کارکردها به صورت ابرزی بوده
و می‌توان برای مدیریت آن‌ها به زیرساخت‌هایی چون \متن‌لاتین{Kubernetes} استفاده کرد که خود بسترهایی برای
گسترش خودکار، خطاپذیری و \نقاط‌خ را فراهم می‌آورد.
\مرجع{Botez2021}

\قسمت{ارتباطات و شبکه‌ها}

ارتباطات و شبکه بخش مهمی از بلاک‌های کارکردی اینترنت اشیا را تشکیل می‌دهند.
این پروتکل‌ها با توجه به گستردگی اشیا و تنوع قابلیت‌های آن‌ها، بسیار متنوع شده‌اند و البته نباید نیازمندی‌های کیفیت سرویس متنوع برای اشیا را نیز از یاد برد.
قطعا پروتکل \متن‌لاتین{IPv6} نقش بسیار مهمی در اینتنرت اشیا خواهد شد و بسیاری از پروتکل‌ها تلاش برای استفاده از آن در محیط‌هایی با محدودیت‌های گوناگون دارند.
این تکنولوژی‌های ارتباطی می‌توانند در سه گروه کلی طبقه‌بندی شوند که در ادامه به همراه مثال‌هایی آورده شده‌اند:

\شروع{فقرات}
\فقره \متن‌سیاه{\متن‌لاتین{Session/Application}}: \متن‌لاتین{MQTT}، \متن‌لاتین{CoAP}، \متن‌لاتین{AMQP}، \متن‌لاتین{HTTP}، \متن‌لاتین{SOAP} و \نقاط‌خ
\فقره \متن‌سیاه{\متن‌لاتین{Network}}: \متن‌لاتین{6LowPAN}، \متن‌لاتین{RPL}، \متن‌لاتین{IPsec}، \متن‌لاتین{TCP/UDP}، \متن‌لاتین{DTLS}، \متن‌لاتین{CORPL} و \نقاط‌خ
\فقره \متن‌سیاه{\متن‌لاتین{Perception/Things}}: \متن‌لاتین{WiFi}، \متن‌لاتین{Bluetooth Low Energy}، \متن‌لاتین{Z-Wave}، \متن‌لاتین{ZigBee}، \متن‌لاتین{LoRaWAN}، \متن‌لاتین{LTE} و \نقاط‌خ
\پایان{فقرات}

به صورت کلی می‌توان این تکنولوژی‌های ارتباطی لایه فیزیکی را در دسته‌های برد کوتاه، متوسط و بلند دسته‌بندی کرد.
از سوی دیگر یکی از پارامترهای مهم برای این دسته از پروتکل‌ها نرخ داده‌ای است. یکی دیگر از پارامترهای مهم در شبکه‌های اینترنت اشیا توان مصرفی شبکه است، به صورت کلی اگر شبکه‌های توان پایین را
در نظر بگیریم، می‌توانیم آن را به دو دسته تقسیم کنیم \مرجع{Augustin2016}\مرجع{Li2014}:

\شروع{فقرات}
\فقره شبکه‌های محلی توان پایین که عموما بردشان زیر یک کیلومتر است. این دسته از شبکه‌ها شامل \متن‌لاتین{IEEE 802.15.4}، \متن‌لاتین{IEEE 802.11ah}، \متن‌لاتین{Bluetooth}، \متن‌لاتین{BLE} و \نقاط‌خ است.
این شبکه‌ها عموما از همبندی \متن‌لاتین{Mesh} استفاده می‌کنند و با این شیوه می‌توان پوشش آن‌ها را گسترش داد.

\فقره شبکه‌های گسترده توان پایین که عموما بردشان بالای یک کیلومتر است و تحت قالب \متن‌لاتین{LPWAN} به آن‌ها پرداختیم.
\پایان{فقرات}

پروتکل‌هایی مانند \متن‌لاتین{ModBus} برای اتوماسیون صنعتی، \متن‌لاتین{KNX} برای هوشمند‌سازی ساختمان‌ها و \متن‌لاتین{Wireless M-Bus} برای اندازه‌گیری آب و گاز مصرفی از سال‌های پیش وجود داشته‌اند
و برای یک کاربردهای خاص تعریف شده‌اند.
در شبکه‌های محلی توان پایین، شبکه‌هایی مانند \متن‌لاتین{Zigbee} وجود دارند که بر پایه \متن‌لاتین{IEEE 802.15.4} بوده اما لایه شبکه را نیز افزون بر لایه‌های پیوند داده و فیزیکی دارا می‌باشند. شبکه‌های \متن‌لاتین{Zigbee}
از همبندی‌های متنوعی پشتیبانی می‌کنند و حتی الگوریتم‌های مسیریابی نیز برای آن‌ها وجود دارد.
پیشتر به معرفی \متن‌لاتین{LPWAN} پرداختیم، تکنولوژی‌های بسیاری در حوزه \متن‌لاتین{LPWAN} به بازار عرضه شده‌اند که از جمله‌ی آن‌ها می‌توان به \متن‌لاتین{SigFox}، \متن‌لاتین{NB-IoT}، \متن‌لاتین{LTE Cat-M} و \متن‌لاتین{LoRaWAN}
اشاره کرد.

\متن‌لاتین{SigFox} قصد دارد یک پوشش جهانی را در قالب یک اپراتور شبکه که در کشورهای مختلف با استفاده از شرکت‌های تابعه اجرا می‌شود، فراهم آورد.
این شبکه به صورت کامل از سخت‌افزار تا لایه شبکه در انحصار همین شرکت است و همکاری با آن تنها راه برای عملیاتی کردن این شبکه است.
\متن‌لاتین{NB-IoT} توسط شرکت‌های مخابراطی به عنوان یک جایگزین در ارتباطات اینترنت اشیا، نسبت به تکنولوژی‌های زیرگیگاهرتز \متن‌لاتین{LPWAN} ارائه می‌شود.
از آنجایی \متن‌لاتین{NB-IoT} در طیف فرکانسی دارای لایسنس فعالیت می‌کند، می‌تواند قابلیت اطمینان بیشتری در ترافیک نسبت به سایر تکنولوژی‌های زیرگیگاهرتز ارائه دهد.
برخلاف \متن‌لاتین{SigFox} و \متن‌لاتین{NB-IoT}، \متن‌لاتین{LoRaWAN} قابلیت ارائه به صورت شبکه‌های خصوصی و ادغام آسان با پلتفرم‌های شبکه‌ای جهانی مانند \متن‌لاتین{The Things Network} را فراهم می‌آورد.
به همین دلیل و از سوی دیگر باز بودن استاندارد، \متن‌لاتین{LoRaWAN} توجه جامعه محققان را از اولین نمود خود در بازار جلب کرده است.
\مرجع{sensors-18-03995}
\مرجع{Mekki2019}

در جدول \رجوع{جدول: مقایسه تکنولوژی‌های LPWAN} تکنولوژی‌های مطرح \متن‌لاتین{LPWAN} در معیارهای مختلف مقایسه شده‌اند.

\begin{table}
\caption{مقایسه تکنولوژی‌های \متن‌لاتین{LPWAN} \مرجع{SanchezIborra2016} \مرجع{Mekki2019} \مرجع{Naik2018}}
\label{جدول: مقایسه تکنولوژی‌های LPWAN}
\begin{latin}\begin{tabularx}
  {\textwidth}
  {|*{6}{X|}}
  \toprule
  &
  LoRaWAN &
  Sigfox &
  NB-IoT &
  Ingenu &
  Telensa \\
  \midrule
  Band &
  433/868/ 780/915 MHz &
  868/915 MHz &
  Cellular &
  2.4 GHz &
  868/915 MHz \\
  \midrule
  Data Rate &
  50 kbps &
  100 bps &
  200 kbps &
  19 kbps &
  346 Mbps \\
  \midrule
  Range &
  5 km &
  10 km &
  35 km &
  15 km &
  1 km \\
  \midrule
  Number of Channels &
  6 &
  333 &
  --- &
  --- &
  --- \\
  \midrule
  MAC &
  ALOHA &
  none &
  Non-Access Stratum &
  --- &
  --- \\
  \midrule
  Topology &
  Star-of-Stars &
  Star &
  Star &
  Star / Tree &
  Star / Tree \\
  \midrule
  Adaptive Data Rate &
  Yes &
  No &
  Yes &
  --- &
  --- \\
  \midrule
  Payload Length &
  256 B &
  12 B &
  1600 B &
  10 kB &
  65 kB \\
  \midrule
  Handover &
  No &
  No &
  Yes &
  --- &
  --- \\
  \midrule
  Authentication / Encryption &
  AES 128 &
  No &
  LTE Encryption &
  --- &
  --- \\
  \midrule
  Over the air update &
  --- &
  --- &
  --- &
  --- &
  --- \\
  \midrule
  Battery life &
  10Y+ &
  10Y+ &
  --- &
  --- &
  10Y+ \\
  Bi-Directional &
  Yes &
  Yes &
  Yes &
  Yes &
  Yes \\
  \bottomrule
\end{tabularx}\end{latin}
\end{table}

شبکه‌ها \متن‌لاتین{Sigfox} پهنای باند بسیار کمی داشته و محدودیت‌های زیادی برای اندازه بسته و تعداد بسته‌ها در نظر گرفته است. با توجه به مدل تجاری خاص آن که پیشتر
به آن پرداخته شد، توجه‌ها بیشتر به سوی \متن‌لاتین{LoRaWAN} معطوف شده است.
\مرجع{Adelantado2017}

مستقل از ارتباط رادیویی که برای شکل دادن شبکه‌ی \متن‌لاتین{M2M} از آن استفاده شده است، دستگاه انتهایی یا ماشین می‌بایست داده خود را از طریق اینرتنت قابل دسترسی کنند.
دستگاه اینترنت اشیا عموما منابع محدودی دارند و این به آن معناست که باید با حافظه، توان پردازشی، توان شبکه‌ای و باتری محدودی فعالیت کنند.
بنابراین کارایی ارتباط ماشین به ماشین وابستگی زیادی به پروتکل زیرین مورد استفاده در اپلیکشن اینرتنت اشیا دارد.
\مرجع{Mishra2021}

پروتکل‌های ارتباطی زیادی در لایه شبکه و کاربرد اینترنت اشیا مطرح است که می‌توان از بین آن‌ها به \متن‌لاتین{MQTT}، \متن‌لاتین{CoAP}، \متن‌لاتین{AMQP} و \متن‌لاتین{HTTP} اشاره کرد.
\مرجع{Mishra2021}

\زیرقسمت{\متن‌لاتین{NB-IoT}}

همانطور که اشاره شد شبکه \متن‌لاتین{NB-IoT} از باند دارای لایسنس استفاده می‌کند. سه مد عملیاتی برای شبکه‌های \متن‌لاتین{NB-IoT} وجود دارد.
در مد اول یا \متن‌لاتین{Stand-Alone} از باندهای فرکانسی \متن‌لاتین{GSM} استفاده می‌شود.
در مد دوم یا \متن‌لاتین{Guard-Band} از منابع استفاده نشده در باند محافظ \متن‌لاتین{LTE} استفاده می‌شود.
در مد سوم یا \متن‌لاتین{In-Band} از منابع داخلی \متن‌لاتین{LTE} استفاده می‌شود.
\مرجع{Mekki2019}

نودها در شبکه‌ی \متن‌لاتین{NB-IoT} می‌توانند دو حالت \متن‌لاتین{eDRX} و \متن‌لاتین{PSM} را برای صرفه‌جویی انتخاب کنند.
در مد \متن‌لاتین{eDRX} دستگاه برای مدت تا ۱۷۵ دقیقه مودم خود را خاموش می‌کند.
در \متن‌لاتین{DRX} که پیشتر هم در شبکه‌های سلولی وجود داشته است همین رویه برای بازه‌ی کوتاه $2.56$ ثانیه خاموش می‌شده است
و تفاوت \متن‌لاتین{eDRX} در همین مدت زمان است.
در نظر داشته باشید که مساله زمان از این جهت مطرح است که در صورت خاموش بودن مودم پاسخ \متن‌لاتین{downlink} با تاخیر مواجه می‌شود.
در روش \متن‌لاتین{PSM} مودم برای مدتهای طولانی مانند چندین ماه خاموش می‌شود.
این حالت برای سنسورهایی که صرفا در شرایط مشخصی \متن‌لاتین{uplink} دارند، کاربرد دارد.
\مرجع{Lee2017}

\زیرقسمت{\متن‌لاتین{LoRa}}

لایه‌ی فیزیکی \متن‌لاتین{LoRa} که در \متن‌لاتین{LoRaWAN} استفاده می‌شود، در سال ۲۰۱۴ توسط \متن‌لاتین{Semtech} به ثبت رسید
و بنابراین برای بررسی‌ها کاملا باز نیست. مطالبی که در ادامه می‌آید بخشی بر اساس قسمت‌های باز استاندارد و بخشی بر اساس آزمایش‌های
تجربی بدست آمده‌اند.
از ویژگی‌های \متن‌لاتین{LoRa} می‌توان به توان عملیاتی پایین، نرخ پایین داده و برد ارتباطی بالا اشاره کرد.
\مرجع{sensors-18-03995}
\مرجع{Adelantado2017}

از سال ۲۰۱۵ جامعه تحقیقاتی شروع به مطالعه در رابطه با کارآیی و ویژگی‌های مختلف تکنولوژی‌های \متن‌لاتین{LoRa} و \متن‌لاتین{LoRaWAN} کرد.
از آن تاریخ مقلات متعددی در ژورنال‌ها و کنفرانس‌های عملی در سراسر دنیا چاپ و ارائه شده‌اند.
\مرجع{sensors-18-03995}

ماژولیشن آن بر پایه \متن‌لاتین{Chirp Spread Spectrum} بوده و به صورت دوره‌ای سیگنال‌های \متن‌لاتین{chirp}ای تولید می‌کنند که همه آن‌ها بازه زمانی یکسانی دارند.
\متن‌لاتین{chirp} یک سیگنال سینوسی است که فرکانس آن با زمان به صورت خطی افزایش یا کاهش پیدا می‌کند.
یک \متن‌لاتین{chirp} به وسیله‌ی پروفایل زمانی فرکانس لحظه‌ی آن که در بازه‌ی زمانی \متن‌لاتین{T} از فرکانس $f_0$ به فرکانس $f_1$
تغییر می‌کند، شناخته می‌شود.
در \متن‌لاتین{LoRa} دو نوع \متن‌لاتین{chirp} تعریف شده است. \متن‌لاتین{chirp} پایه که فرکانس پروفایل زمانی آن با فرکانس مینیمال
\(f_{\min} = -\frac{BW}{2}\)
شروع شده و با فرکانس ماکسیمال
\(f_{\max} = \frac{BW}{2}\)
خاتمه می‌یابد.
برای ورودی‌های دیجیتال مختلف، یک ماژولاتور \متن‌لاتین{chirp}های مختلفی تولید می‌کند که نسبت به \متن‌لاتین{chirp} پایه شیف زمانی خورده‌اند.
\مرجع{sensors-18-03995}

\شروع{شکل}
\درج‌تصویر[width=\textwidth]{./img/lora-mod.png}
\تنظیم‌ازوسط
\شرح{ماژولیشن \متن‌لاتین{LoRa}}
\پایان{شکل}

\شروع{شکل}
\درج‌تصویر[width=\textwidth]{./img/lora-chirp-sf.png}
\تنظیم‌ازوسط
\شرح{\متن‌لاتین{chirp}های پایه}
\پایان{شکل}

\متن‌لاتین{LoRa} از باند فرکانسی بدون مجوز استفاده می‌کند بنابراین برای راه‌اندازی شبکه‌ی آن نیاز به تهیه هیچ مجوزی نیست. البته باید در نظر داشته که نرخ پیام در این باندهای بدون مجوز توسط قانون‌گذاران محدود شده است.
یکی از محدودیت‌های مهم در شبکه‌های \متن‌لاتین{LoRa} محدودیت \متن‌لاتین{Duty Cycle} است که استفاده از کانال را محدود می‌کند. این محدودیت بیان می‌کند برای استفاده از کانال به مدت $T_{a}$ شی می‌بایست
حداقل به اندازه $T_{s}$ که از رابطه \رجوع{معادله: چرخه وظیفه} بدست می‌آید، ارسالی نداشته باشد.
\مرجع{Cruz2021}
\مرجع{Adelantado2017}

\begin{align}
  \label{معادله: چرخه وظیفه}
  T_{s} = T_{a}\left( \frac{1}{d} - 1 \right)
\end{align}

لایه فیزیکی \متن‌لاتین{LoRa} با توجه به ویژگی‌های گسترده‌ای که دارد در راهکارهای دیگری به جز \متن‌لاتین{LoRaWAN} نیز استفاده شده است که از جمله‌ی آن می‌توان به \متن‌لاتین{Meshed LoRa} اشاره کرد.
\مرجع{Beltramelli2021}

پارامترهای فاکتور گسترش یا به اختصار \متن‌لاتین{SF}، پهنای باند و نرخ‌کدگذاری قابل تنظیم می‌باشند و می‌توانند روی زمان ارسال بسته، نرخ ارسال، مصرف انرژی و برد ارتباطی تاثیر داشته باشند.
در ادامه به مرور این پارامترها و تاثیرشان می‌پردازیم.

به صورت غیر رسمی فاکتور گسترش لگاریتم مبنای ۲ تعداد \متن‌لاتین{chirp}ها در هر علامت است. مقدار فاکتور گسترش بین ۷ تا ۱۲ است.
با افزایش فاکتور گسترش پوشش‌دهی بیشتر می‌شود اما بهای آن کاهش نرخ بیت و افزایش زمان ارسال\پانویس{Time on Air} است (معادله \رجوع{معادله: زمان علامت در LoRa}).
\مرجع{Augustin2016}

در بسته‌های \متن‌لاتین{LoRa} از تصحیح خطا جلورونده یا مختصرا \متن‌لاتین{FEC} استفاده می‌شود.
در این فرآیند بیت‌های تصحیح خطا به داده‌های ارسال اضافه می‌شوند.
این بیت‌های اضافه شده کمک می‌کنند تا داده‌های از دست رفته به خاطر تداخل بازگردانی شوند.
بیت‌های بیشتر این پروسه بازگردانی را ساده‌تر می‌کنند اما باعث هدر رفت پهنای باند و عمر باتری می‌شوند.
در \متن‌لاتین{LoRa} ما نرخ‌های کدگذاری $4/5$، $4/6$، $4/7$ و $4/8$ را داریم.

\begin{table}
\caption{توانایی \متن‌لاتین{LoRa} در تشخیص و تصحیح خطا \مرجع{Pham2020}}
\begin{latin}\begin{tabularx}
  {\textwidth}
  {|*{3}{X|}}
  \toprule
  Coding rates &
  Error detection (bits) &
  Error correction (bits) \\
  \midrule
  $4/5$ &
  0 &
  0 \\
  \midrule
  $4/6$ &
  1 &
  0 \\
  \midrule
  $4/7$ &
  2 &
  1 \\
  \midrule
  $4/8$ &
  3 &
  1 \\
  \bottomrule
\end{tabularx}\end{latin}
\end{table}

پهنانی باند در \متن‌لاتین{LoRa} می‌توان بین ۱۲۵ تا ۵۰۰ کیلوهرتز باشد و با توجه به استفاده از باند بدون لایسنس این پهنای باند وابسته به پارامتر‌های منطقه‌ای و فاکتور گسترش می‌باشد.
به طور مثال در باند فرکانسی ۸۶۸ مگاهرتز ۸ کانال متفاوت وجود دارد که ۷ کانال ابتدایی تنها با پهنای باند ۱۲۵ کیلوهرتز کار می‌کنند و کانال آخر می‌تواند با پهنای باند‌های
۱۲۵، ۲۵۰ و ۵۰۰ کیلوهرتز کار کند. از این بین ۳ کانال ۱۲۵ کیلوهرتزی اجباری بوده و می‌توان در صورت لزوم کانال‌های بیشتری را نیز فعال کرد.

ارسال‌هایی که روی یک کانال صورت می‌پذیرند اما فاکتورهای گسترش متفاوتی دارند می‌توانند تقریبا بدون تداخل ارسال شوند پس زوج کانال و فاکتور گسترش در \متن‌لاتین{LoRa} می‌تواند به عنوان
یک کانال مجازی شناخته شود. از آنجایی که این کانال‌های مجازی فاکتورهای گسترش متفاوتی دارند نرخ بیت ارسالی روی همه آن‌ها یکسان نخواهد بود.
\مرجع{Augustin2016}

\شروع{شکل}
\درج‌تصویر[width=\textwidth]{./img/lora-868-channels.jpg}
\تنظیم‌ازوسط
\شرح{کانال‌های \متن‌لاتین{LoRa} در باند فرکانسی ۸۶۸ مگاهرتز}
\پایان{شکل}

در متن‌لاتین{LoRa} نرخ باد یا نرخ علائم از رابطه‌ی زیر محاسبه می‌گردد:

\begin{align}
  \label{معادله: نرخ باد یا علائم در LoRa}
  R_{s} = \frac{BW}{2^{SF}}
\end{align}

\begin{align}
  \label{معادله: زمان علامت در LoRa}
  T_{s} = \frac{2^{SF}}{BW}
\end{align}

که در آن \متن‌لاتین{BW} پهنای باند و \متن‌لاتین{SF} فاکتور گسترش می‌باشد.
\مرجع{Augustin2016}

در ادامه نرخ داده‌ی ارسالی را می‌توان با استفاده از رابطه زیر محاسبه کرد:

\begin{align}
  \label{معادله: نرخ داده در LoRa}
  R_{b} = SF \times \frac{BW}{2^{SF}} \times CR
\end{align}

در این رابطه \متن‌لاتین{CR} نرخ کدگذاری، \متن‌لاتین{SF} فاکتور گسترش و \متن‌لاتین{BW} پهنای باند می‌باشد.
\مرجع{Augustin2016}

\شروع{شکل}
\درج‌تصویر[width=.5\textwidth]{./img/lora-packet.png}
\تنظیم‌ازوسط
\شرح{ساختار بسته \متن‌لاتین{LoRa} \مرجع{Augustin2016}}
\برچسب{شکل: بسته LoRa}
\پایان{شکل}

رابطه زیر مشخص می‌کند برای ارسال یک داده به چه تعداد علامت نیاز داریم. این پارامتر با $n_{s}$ نمایش داده می‌شود.

\begin{align}
  \label{معادله: تعداد علائم مورد نیاز در LoRa}
  n_{s} = 8 + \max\left( \left\lceil \frac{8PL - 4SF + 8 + CRC + H}{4 \times (SF - DE)} \right\rceil \times \frac{4}{CR}, 0 \right)
\end{align}

در این رابطه در صورت فعال بودن \متن‌لاتین{CRC} مقدار آن برابر ۱۶ و در غیر این صورت برابر صفر است.
\متن‌لاتین{CR} نرخ کدگذاری،
\متن‌لاتین{PL} اندازه داده،
\متن‌لاتین{SF} فاکتور گسترش است.
در این رابطه \متن‌لاتین{H} اندازه سرآیند بوده که در صورت فعال بودن برابر ۲۰ و در غیر این صورت صفر است.
در این رابطه \متن‌لاتین{DE} در صورت فعال بودن حالت نرخ داده پایین یا \متن‌لاتین{low data rate} برابر ۲ و در غیر این صورت برابر صفر است.
\مرجع{Augustin2016}
\مرجع{Pham2020}

همانطور که محاسبات دیده می‌شود استفاده از مقدارهای بالاتر برای فاکتور گسترش زمان ارسال را بیشتر کرده و تاثیر \متن‌لاتین{Duty Cycle} را بیشتر می‌کند.
و پژوهش \مرجع{Adelantado2017} بیان می‌کند احتمال استفاده از فاکتورهای گسترش بالاتر بیشتر است.

ساختار فریم \متن‌لاتین{LoRa} در شکل \رجوع{شکل: بسته LoRa} قابل مشاهده است. استفاده از سرآیند اختیاری بوده است و در صورتی که مواردی مانند اندازه بسته،
نرخ کدگذاری و وجود \متن‌لاتین{CRC} از پیش هماهنگ شده باشند نیازی به استفاده از آن نیست. سرآیند از نرخ کدگذاری $4/8$ استفاده کرده و دارای یک \متن‌لاتین{CRC}
برای خود است.

\زیرقسمت{\متن‌لاتین{LR-FHSS}}

یکی از تکنیک‌ها در شبکه‌های بی‌سیم استفاده از \متن‌لاتین{Frequency Hopping} است. در این تکنیک با هماهنگی در میان ارسال کننده و گیرنده فرکانس‌های ارسال در زمان
تغییر می‌کند. پیاده‌سازی این شیوه در شبکه‌های \متن‌لاتین{LoRaWAN} در قالب \متن‌لاتین{LR-FHSS} یا
\متن‌لاتین{Frequency Hopping Spread Spectrum}
صورت می‌پذیرد. در این روش هر کانال به تعدادی زیرکانال شکسته شده و سرآیند بسته روی همه این زیرکانال‌ها ارسال می‌شود.
خود داده اما قطعه قطعه شده و هر قطعه به وسیله‌ی یک زیرکانال ارسال می‌گردد.
از آنجایی که \متن‌لاتین{Gateway} روی همه‌ی این کانال‌ها گوش می‌دهد می‌تواند بسته را دوباره بازسازی کند بنابراین
در \متن‌لاتین{Gateway} نیازی به از پیش دانستن ترتیب این پرش‌ها یا فرکانس و پهنای باند دقیق این کانال‌ها ندارد
و این اطلاعات از طریق سرآیندهای بسته که به صورت تکراری ارسال می‌گردند قابل بازیابی است.
ذکر این نکته نیز خالی از لطف نیست که این ماژولیشن سربار بیشتری برای تشخیص سیگنال در گیرنده نسبت به \متن‌لاتین{LoRa} دارد.
\مرجع{Boquet2021}

\متن‌لاتین{LR-FHSS} در سال ۲۰۲۰ توسط \متن‌لاتین{Semtech} برای پشتیبانی از شبکه‌های وسیع پیشنهاد شده است.
ای ماژولیشن با \متن‌لاتین{LoRa} سازگار بوده و تنها برای \متن‌لاتین{uplink} پیشنهاد شده است و برای \متن‌لاتین{downlink} از همان
\متن‌لاتین{LoRa} استفاده خواهد شد.
\مرجع{Boquet2021}

پژوهش \مرجع{Boquet2021} با استفاده از شبیه‌سازی دست به ارزیابی این لایه فیزیکی زده است و یادآور می‌شود که هدف از ارائه این لایه فیزیکی
افزایش تعداد بسته‌های ارسالی توسط یک شی نیست بلکه افزایش کلی ظرفیت شبکه است. در شکل \رجوع{شکل: مقایسه LoRa و LR-FHSS} این افزایش ظرفیت شبکه بر پایه همین
شبیه‌سازی کاملا مشهود است. در شکل \رجوع{شکل: مقایسه LoRa و LR-FHSS} همچنین مشخص است که با افزایش تعداد اشیا در شبکه کارایی \متن‌لاتین{LR-FHSS} بیشتر از \متن‌لاتین{LoRa}
می‌گردد. از سوی دیگر این پژوهش بیان می‌کند در صورت افزایش حجم داده از ۱۰ بایت به ۵۰ بایت این بهبود بسیار زودتر به وقوع می‌پیوندد.
\مرجع{Boquet2021}

\شروع{شکل}
\درج‌تصویر[width=\textwidth]{./img/lr-fhss-limits.png}
\تنظیم‌ازوسط
\شرح{مقایسه دریافت صحیح اطلاعات با افزایش تعداد اشیا برای بسته‌های ۱۰ بیتی در \متن‌لاتین{LoRa} و \متن‌لاتین{LR-FHSS} \مرجع{Boquet2021}}
\برچسب{شکل: مقایسه LoRa و LR-FHSS}
\پایان{شکل}

از گپ‌های تحقیقاتی این حوزه می‌توان به مشخص کردن دنباله‌ی پرش‌های فرکانسی در \متن‌لاتین{LR-FHSS} و همزیستی \متن‌لاتین{LoRa} و \متن‌لاتین{LR-FHSS} اشاره کرد.

\زیرقسمت{\متن‌لاتین{LoRaWAN}}

\متن‌لاتین{LoRaWAN} پروتکل لایه لینک و شبکه می‌باشد که شامل پروتکل کنترل دسترسی چندگانه\پانویس{MAC} نیز می‌باشد.
این پروتکل اجازه می‌دهد تا دستگاه‌هایی با لایه فیزیکی \متن‌لاتین{LoRa} با برنامه‌های کاربردی ارتباط برقرار کنند.
این پروتکل توسط \متن‌لاتین{LoRa Alliance} توسعه پیدا کرده و برای همگان قابل استفاده است.
این پروتکل برای ارتباط دستگاه به دستگاه ایجاد نشده است و تنها هدف آن ارتباط اشیا با \متن‌لاتین{Gateway} و \متن‌لاتین{Network Server} است.
در صورت نیاز به ارتباط بین دستگاه‌ها می‌بایست از \متن‌لاتین{Gateway} و \متن‌لاتین{Network Server} استفاده کرد یا اینکه
تنها لایه‌ی فیزیکی \متن‌لاتین{LoRa} را مورد استفاده قرار داد.
از سوی دیگر \متن‌لاتین{LoRaWAN} از لایه‌های فیزیکی \متن‌لاتین{LoRa}، \متن‌لاتین{FSK} و \متن‌لاتین{LR-FHSS} پشتیبانی می‌کند.
\مرجع{Cruz2021}
\مرجع{Augustin2016}

یک شبکه‌ی \متن‌لاتین{LoRaWAN} در ساده‌ترین شکل از اجزای زیر تشکیل شده است:

\شروع{شمارش}
\فقره یک دستگاه حسگر یا عملگر که توان مصرفی و محاسبات محدودی دارد.
\فقره یک \متن‌لاتین{Gateway} که عنصر شبکه‌ای برای دریافت و ارسال اطلاعات از و به دستگاه‌ها است. این عنصر شبکه برای ارتباط سرور شبکه
از زیرساخت \متن‌لاتین{IP} و لایه فیزیکی با گذردهی بالا مانند \متن‌لاتین{Ethernet} استفاده می‌کند.
\فقره سرور شبکه که پیام‌های دریافت شده از یک مجموعه \متن‌لاتین{Gateway}ها را به برنامه‌های کاربردی می‌رساند و برعکس.
سرور شبکه وظیفه حذف بسته‌های تکراری
و رمزگشایی آن‌ها برعهده دارد. از سوی دیگر بسته‌های ارسالی به دستگاه‌ها در سرور شبکه ساخته می‌شوند.
\فقره برنامه کاربردی که می‌تواند در بستر اینترنت قرار داشته باشد و داده‌ها را از طریق سرور شبکه برای اشیا ارسال و دریافت کند.
\پایان{شمارش}

\شروع{شکل}
\درج‌تصویر[width=\textwidth]{./img/nrm-home.png}
\تنظیم‌ازوسط
\شرح{مدل مرجع شبکه \متن‌لاتین{LoRaWAN} - شبکه‌ی خانگی}
\پایان{شکل}

\شروع{شکل}
\درج‌تصویر[width=\textwidth]{./img/nrm-roaming.png}
\تنظیم‌ازوسط
\شرح{مدل مرجع شبکه \متن‌لاتین{LoRaWAN} - شبکه‌ی فراگرد}
\پایان{شکل}

\شروع{شکل}
\درج‌تصویر[width=\textwidth]{./img/lora-architecture-osi.png}
\تنظیم‌ازوسط
\شرح{معماری شبکه \متن‌لاتین{LoRaWAN} از نگاه مدل لایه‌ای \متن‌لاتین{OSI} \مرجع{Ertrk2019}}
\پایان{شکل}

برخلاف شبکه‌های سلولی سنتی، در \متن‌لاتین{LoRaWAN} ارتباطی میان \متن‌لاتین{Gateway} و دستگاه انتهایی شکل نمی‌گیرد.
\متن‌لاتین{Gateway}ها در واقع نقش رله‌ای در لایه لینک را ایفا می‌کنند و بعد از افزودن اطلاعات مبنی بر کیفیت پیام دریافتی آن
را به سرور شبکه ارسال می‌کنند. بنابراین دستگاه‌های انتهایی با سرور شبکه ارتباط دارند که وظیفه آن رمزگشایی بسته‌ها، حذف بسته‌های تکراری و
انتخاب \متن‌لاتین{Gateway} مناسب جهت ارسال بسته به دستگاه انتهایی است.
بنابراین می‌توان گفت که در شبکه‌ی \متن‌لاتین{LoRaWAN} عملا \متن‌لاتین{Gateway} از دید دستگاه‌های انتهایی پنهان است.
\مرجع{Augustin2016}

در حوزه امنیت \متن‌لاتین{LoRaWAN} دولایه از امنیت را تعریف می‌کند. لایه اول امنیت میان شی و شبکه است در حالی که لایه دوم میان شی و برنامه کاربردی می‌باشد.
به این صورت می‌توان مطمئن شد که تنها برنامه کاربردی است که می‌تواند داده‌های ارسالی توسط دستگاه را رمزگشایی کند.
\مرجع{Cruz2021}

در ضمن \متن‌لاتین{LoRaWAN} ویژگی‌های دیگری مانند نرخ داده تطبیقی\پانویس{ADR} را اضافه می‌کند. در نرخ داده تطبیقی شبکه با دستگاه در رابطه با پارامترهای لایه‌ی فیزیکی \متن‌لاتین{LoRa} مذاکره می‌کند
که در نتجیه آن کارآیی مصرف بهینه می‌شود. شکل \رجوع{شکل: لایه‌های لورا} مدل لایه‌ای \متن‌لاتین{LoRa} و \متن‌لاتین{LoRaWAN} را نمایش می‌دهد.
\مرجع{Cruz2021}

\شروع{شکل}
\درج‌تصویر[width=\textwidth]{./img/lora-layers.png}
\تنظیم‌ازوسط
\شرح{مدل لایه‌ای \متن‌لاتین{LoRa} و \متن‌لاتین{LoRaWAN} \مرجع{Cruz2021}}
\برچسب{شکل: لایه‌های لورا}
\پایان{شکل}

در شبکه‌های \متن‌لاتین{LoRaWAN} سه کلاس کاری می‌توان برای اشیا در نظر گرفت.

\شروع{فقرات}
\فقره در کلاس ($A$ یا \متن‌لاتین{All}) شی هر زمان که به خواهد شروع به ارسال داده کرده و دو پریود متوالی آینده را برای دریافت \متن‌لاتین{Downlink} خواهد داشت. این کلاس پایین‌ترین مصرف انرژی را دارد چرا که شی تنها در زمان‌هایی که لازم است
روشن می‌شود و می‌تواند دوباره خاموش شود. با توجه به ساختار دریافت \متن‌لاتین{Downlink} در این کلاس می‌توان به سادگی برای پیام‌ها \متن‌لاتین{Ack} دریافت کرد اما برای سایر پیام‌های \متن‌لاتین{Downlink} می‌بایست تا تصمیم شی برای ارسال داده صبر کرد.
\فقره در کلاس ($B$ یا \متن‌لاتین{Beacon}) نود به صورت همگام می‌تواند \متن‌لاتین{Downlink} دریافت کند برای اینکار نود به جز دو بازه دریافت که در کلاس $A$ تعریف شده بود یک بازه دریافت قابل پیش‌بینی نیز دارد.
این کلاس مصرف بالاتری دارد چرا که نیاز است یک بازه دوره‌ای برای دریافت \متن‌لاتین{Downlink} روشن شود. مزیت این کلاس قابلیت دریافت \متن‌لاتین{Downlink} حتی در زمان‌هایی که ارسالی ندارد می‌باشد.
\فقره در کلاس ($C$ یا \متن‌لاتین{Continues}) بیشترین مصرف توان را داشته و شی در هر زمان می‌تواند داده دریافت کند.
\پایان{فقرات}

پیشتر به ساختار بسته‌ها در لایه‌ی فیزیکی \متن‌لاتین{LoRa} پرداختیم. در \متن‌لاتین{LoRaWAN} برای پیام‌های \متن‌لاتین{uplink} استفاده از سرآیند و \متن‌لاتین{CRC} اجباری است
و بنابراین نمی‌توان در \متن‌لاتین{LoRaWAN} از فاکتور گسترش ۶ استفاده کرد. از سوی دیگر تنها استفاده از سرآیند در بسته‌های \متن‌لاتین{downlink} اجباری است.
نرخ کدگذاری در \متن‌لاتین{LoRaWAN} مشخص نشده و قابل تغییر است.
\مرجع{Augustin2016}

\شروع{شکل}
\درج‌تصویر[width=\textwidth]{./img/lorawan-packet.png}
\تنظیم‌ازوسط
\شرح{بسته‌های پروتکل \متن‌لاتین{LoRaWAN} (سایز فیلدها به بیت آمده است) \مرجع{Augustin2016}}
\برچسب{شکل: بسته‌های LoRaWAN}
\پایان{شکل}

ساختار پیام‌های \متن‌لاتین{LoRaWAN} در شکل \رجوع{شکل: بسته‌های LoRaWAN} آورده شده است. \متن‌لاتین{DevAddr} آدرس دستگاه است.
\متن‌لاتین{FPort} پورت برای مالتی‌پلکس است و در صورتی که مقدار آن صفر باشد نشان‌دهنده‌ی این موضوع است که بسته تنها شامل
دستورات لایه‌ی \متن‌لاتین{MAC} است. بسته‌های \متن‌لاتین{uplink} ادرس مقصد و بسته‌های \متن‌لاتین{downlink} آدرس مبدا ندارند
چرا که هر شی تنها با یک سرور شبکه در ارتباط است.
\مرجع{Augustin2016}

در \متن‌لاتین{LoRaWAN} از الگوریتم کنترل دسترسی همزمان \متن‌لاتین{ALOHA} استفاده میشود. در این الگوریتم بسته‌ها می‌توانند اندازه‌های متغیر داشته باشند، نودها هر زمان که قصد داشته باشند داده ارسال کنند و نیازی به همگام‌سازی زمانی ندارد.
مشکل اصلی این روش در تعداد برخوردهای بالا در زمانی است که شبکه نودهای زیادی دارد، از این رو روش‌های ``گوش‌دادن پیش از حرف‌زدن'' مانند \متن‌لاتین{CSMA/CA} کارایی بهتری دارند.
چنین الگوریتم‌هایی در مقابل نیاز به همگام‌سازی دارند به این معنی که می‌بایست اشیا یک \متن‌لاتین{clock} محلی مشترک داشته باشند.
\مرجع{Beltramelli2021}

به صورت کلی می‌توان الگوریتم‌های دسترسی همزمان را در سه گروه اصلی قرار داد:

\شروع{فقرات}
\فقره الگوریتم‌های قطعه‌بندی کانال
\فقره الگوریتم‌های دسترسی تصادفی
\فقره الگوریتم‌های نوبت‌دهی
\پایان{فقرات}

که الگوریتم \متن‌لاتین{ALOHA} یا \متن‌لاتین{CSMA} در دسته الگوریتم‌های دسترسی تصادفی است. روش \متن‌لاتین{ALOHA} خود به دو گونه \متن‌لاتین{Pure ALOHA} و \متن‌لاتین{Slotted ALOHA}
قابل پیاده‌سازی است. در روش \متن‌لاتین{Pure ALOHA} که در \متن‌لاتین{LoRaWAN} نیز استفاده می‌شود، نودها هر زمان که داده‌ای داشته باشند می‌توانند آن را ارسال کنند و در صورت رخ دادن تصادم
با احتمال $p$ داده را باز ارسال می‌کنند. در روش \متن‌لاتین{Slotted ALOHA} نیاز است که نودها با یکدیگری همگام بوده و در ابتدای بازه‌های مشخصی شروع به ارسال کنند و در صورت
وقوع تصادم بعد از صبر کردن در ابتدای بازه‌ی بعدی با احتمال $p$ باز ارسال را شروع می‌کنند. کارایی روش \متن‌لاتین{Slotted ALOHA} از روش \متن‌لاتین{ALOHA} بیشتر بوده است ولی نیاز به همگام‌سازی نودها دارد
که خود موضوعی هزینه‌بر است. پژوهش‌های متعددی روش‌های همگام‌سازی برای \متن‌لاتین{LoRaWAN} در جهت استفاده از \متن‌لاتین{Slotted ALOHA} پیشنهاد داده‌اند.
می‌توان نشان داد کارایی پروتکل \متن‌لاتین{ALOHA} برابر $\frac{1}{2e}$ است و از سوی دیگر کارایی پروتکل \متن‌لاتین{Slotted ALOHA} تقریبا دو برابر بوده و برابر $\frac{1}{e}$ است.

ذکر این نکته خالی از لطف نیست که تفاوت آنچه در \متن‌لاتین{LoRaWAN} به عنوان \متن‌لاتین{ALOHA} پیاده‌سازی شده است و آنچه به عنوان \متن‌لاتین{ALOHA} شناخته می‌شود در اندازه بسته‌ها است،
در \متن‌لاتین{LoRaWAN} اندازه بسته‌ها متغیر است ولی در \متن‌لاتین{ALOHA} اندازه‌ی بسته‌ها ثابت در نظر گرفته می‌شود.
\مرجع{Augustin2016}

ارسال اطلاعات همگام‌سازی در صورتی که از طریق همان بستر بی‌سیم رخ بدهد به آن همگام‌سازی داخل باند گفته می‌شود. در این روش نیاز است که به محدودیت‌های بستر رادیویی احترام گذاشت.
در این شیوه ممکن است کیفیت ارتباط رادیویی به علت ارسال همین اطلاعات به میزان غیرقابل قبولی افت کند. برای حل این مشکل می‌توان از راه‌حل‌های خارج از باند استفاده کرد.
\مرجع{Beltramelli2021}

پروتکل \متن‌لاتین{LoRaWAN} یک پروتکل تک گام است و از مسیریابی یا ارسال چند گامه پیام پشتیبانی نمی‌کند. یکی از مهم‌ترین ویژگی‌های این پروتکل نرخ داده تطبیق‌پذیر و تشخیص
موقعیت مکانی بدون نیاز به \متن‌لاتین{GPS} و تنها با ۳ \متن‌لاتین{Gateway} است.
\مرجع{Ertrk2019}

اولین نسخه از استاندارد \متن‌لاتین{LoRaWAN} در سال ۲۰۱۵ منتشر شد. در طی این سال‌ها بهبودهایی در آن حاصل شد که مهمترین نسخه‌های آن ۱.۰.۳ و ۱.۱ است.
\مرجع{Ertrk2019}

\شروع{شکل}
\درج‌تصویر[width=\textwidth]{./img/lorawan-gps.png}
\تنظیم‌ازوسط
\شرح{تشخیص موقعیت مکانی در \متن‌لاتین{LoRaWAN} \مرجع{Ertrk2019}}
\پایان{شکل}

برای پیوستن یک شی به شبکه \متن‌لاتین{LoRaWAN} نیاز است که آن شی فعال‌سازی شود. دو راه برای فعال‌سازی اشیا وجود دارد: \متن‌لاتین{Over-The-Air Activation} یا مختصرا \متن‌لاتین{OTAA}
و \متن‌لاتین{Activation By Personalization} یا مختصرا \متن‌لاتین{ABP}.
\مرجع{Augustin2016}

پروسه فعال‌سازی می‌بایست اطلاعات زیر را در اختیار اشیا قرار بدهد:
\شروع{فقرات}
\فقره آدرس دستگاه انتهایی (\متن‌لاتین{DevAddr}): یک شناسه‌ی ۳۲ بیتی که ۷ بیت آن نماینده شبکه و ۲۵ بیت آن آدرس دستگاه انتهایی در شبکه است.
\فقره شناسه برنامه کاربردی (\متن‌لاتین{AppEUI}): شناسه عمومی برنامه کاربردی که از فضای آدرس \متن‌لاتین{IEEE EUI64} انتخاب شده و به صورت یکتا صاحب این دستگاه انتهایی را مشخص می‌کند.
\فقره کلید نشست شبکه (\متن‌لاتین{NetSKey}): کلیدی که میان دستگاه و سرور شبکه برای محاسبه و اطمینان از یکپارچگی اطلاعات استفاده می‌شود.
\فقره کلید نشست برنامه کاربردی (\متن‌لاتین{AppSKey}): کلیدی که میان دستگاه و برنامه کاربردی برای رمزگذاری و رمزگشایی اطلاعات استفاده می‌شود.
\پایان{فقرات}

در \متن‌لاتین{OTAA} فرآیند عضویت شامل یک درخواست عضویت و پاسخی مبتنی بر پذیرفته شدن عضویت است که برای هر نشست جدید استفاده می‌شود.
بنابر پاسخی که مبتنی بر پذیرفته شدن عضویت می‌آید، دستگاه انتهایی کلیدهای جدید نشست شبکه و برنامه کاربردی را دریافت می‌کند.
در فرایند \متن‌لاتین{ABP} این کلیدها به صورت مستقیم روی دستگاه‌های انتهایی ذخیره شده‌اند.
\مرجع{Augustin2016}

در \متن‌لاتین{LoRaWAN} دستورات \متن‌لاتین{MAC}ای تعریف شده است که اجازه می‌دهد پارامترهای دستگاه‌های انتهایی سفارشی شود.
از بین این دستورات تنها \متن‌لاتین{LinkCheckReq} از سمت شی ارسال شده و هدف آن بررسی ارتباط است.
سایر این دستورات از سوی سرور شبکه ارسال می‌شوند و می‌توانند پارامترهایی مانند نرخ داده تنظیم پذیر را تغییر دهند.
\مرجع{Augustin2016}

آنجه در \متن‌لاتین{LoRaWAN} بیان می‌شود در رابطه با دستگاه‌ّها است و در رابطه با سرور شبکه حرفی به میان نیامده است اما سرور شبکه نقشه حیاتی در کارکرد شبکه ایفا می‌کند.
اگر بخواهیم در شبکه میلیون دستگاه را هندل کنیم در این صورت بهینه‌سازی این دستگاه‌ّها بر عهده سرور شبکه خواهد بود. در صورتی که سرور شبکه دستورات \متن‌لاتین{MAC}
درستی را ارسال نکند یا پارامترها را به شکل اشتباهی تغییر دهد می‌تواند کاملا کارکرد شبکه را مختل کند.
\مرجع{Augustin2016}

از سوی دیگر در رابطه با نقش \متن‌لاتین{Gateway}ها در \متن‌لاتین{LoRaWAN} تنها به عنوان یک رله اکتفا می‌شود و مسئولیت انتخاب بهترین \متن‌لاتین{Gateway} برای ارسال
برعهده سرور شبکه خواهد بود. تنها مسئولیت \متن‌لاتین{Gateway} زمان‌بندی صحیح در جهت ارسال است و این زمان‌بندی می‌بایست به گونه‌ای باشد که از تصادم در هنگام ارسال
\متن‌لاتین{Downlink} جلوگیری کند. البته استاندارد \متن‌لاتین{LoRaWAN} در رابطه با این زمان‌بندی صحیتی به میان نمیاورد.
\مرجع{Augustin2016}

\زیرقسمت{چالش‌های \متن‌لاتین{LoRa} و \متن‌لاتین{LoRaWAN}}

با وجود گسترش و استفاده روزافزون شبکه‌های \متن‌لاتین{LoRaWAN} هنوز چالش‌هایی در این شبکه وجود دارد. اولین چالش مربوط به تاثیر
\متن‌لاتین{Duty Cycle} بر اندازه شبکه است. با افزایش تعداد نودها در شبکه نرخ پیام‌هایی که به درستی دریافت می‌شوند توسط تصادم و
\متن‌لاتین{Duty Cycle} محدود می‌شوند. همانطور که پیشتر بیان شد \متن‌لاتین{LoRaWAN} از \متن‌لاتین{ALOHA} استفاده می‌کند
که خود یکی از دلایل اصلی در ایجاد تصادم با افزایش تعداد نودها در شبکه است. آنچه پژوهش \مرجع{Adelantado2017} در این حوزه
با شبیه‌سازی بر پایه سه کانال با بسته‌های ۱۰ بایتی محاسبه کرده است در شکل \رجوع{شکل: محدودیت‌های LoRaWAN} قابل مشاهده است.
\مرجع{Adelantado2017}

\شروع{شکل}
\درج‌تصویر[width=.5\textwidth]{./img/lora-limits-1}
\تنظیم‌ازوسط
\شرح{تاثیر افزایش اشیا و نرخ ارسال با در نظر گرفتن سه کانال بر بسته‌های دریافتی در پروتکل \متن‌لاتین{LoRaWAN} \مرجع{Adelantado2017}}
\برچسب{شکل: محدودیت‌های LoRaWAN}
\پایان{شکل}

دومین چالش شبکه‌های \متن‌لاتین{LoRaWAN} بحث قابلیت اطمینان است. قابلیت اطمینان در این شبکه‌ها توسط \متن‌لاتین{Acknowledgement} تامین می‌شود.
اما باز به دلیل وجود \متن‌لاتین{Duty Cycle} و تاثیر آن بر \متن‌لاتین{Gateway} طراحی شبکه و برنامه‌های می‌بایست به گونه‌ای باشد که تعداد \متن‌لاتین{Acknowledgement}ها کمینه شوند.
همین مشکل استفاده از \متن‌لاتین{LoRaWAN} برای کاربردهایی که قابلیت اطمینان بسیار بالا می‌خواهند را زیر سوال می‌برد.
\مرجع{Adelantado2017}

اگر کاربردهایی مانند کشاورزی هوشمند یا کنترل محیط را در نظر بگیریم که داده‌های تولیدی به صورت دوره‌ای یا غیردوره‌ای بوده و محدودیت‌های سخت‌گیرانه‌ای از نگاه تاخیر ندارند و تنها نیاز به پوشش بالا وجود دارد،
استفاده از \متن‌لاتین{LoRaWAN} گزینه‌ی مناسبی است. در رابطه با کاربردهایی مانند اتوماسیون صنعتی که نیاز به تاخیر مشخص و محدود شده وجود دارد، استفاده از \متن‌لاتین{LoRaWAN} نیاز به در نظر گرفتن تمهیدات
خاصی است از جمله، کوچک نگاه داشتن فاکتور گسترش یا به عبارت دیگر نزدیکی اشیا به \متن‌لاتین{Gateway} که باعث کاهش زمان ارسال و تاثیر \متن‌لاتین{Duty Cycle} می‌گردد و از سوی دیگر تعداد کانال‌ها می‌بایست
با دقت انتخاب شود تا با شکست ارسال در یک کانال بتوان بدون مشکل \متن‌لاتین{Duty Cycle} از کانال دیگری برای ارسال بهره برد.
\مرجع{Adelantado2017}

با توجه به تعریف فراگَردی در پروتکل \متن‌لاتین{LoRaWAN} این پروتکل به یک گزینه خوب در کنترل و مدیریت ناوگان بدل شده است.

\زیرقسمت{\متن‌لاتین{WiFi 7}}

اندکی پس از انتشار \متن‌لاتین{WiFi 6} کارگروه \متن‌لاتین{IEEE 802.11} به همراه \متن‌لاتین{WiFi Alliance} شروع به طراحی نسل بعدی آن در شبکه‌های بی‌سیم محلی با نام \متن‌لاتین{WiFi 7} کردند.
یکی از اجزای \متن‌لاتین{WiFi 7}، \متن‌لاتین{IEEE 802.11be} می‌باشد و قرار است در این نسل از \متن‌لاتین{Time-Sensitive Networking} یا \متن‌لاتین{TSN} برای ارتباط‌هایی با تاخیر کم و قابلیت
اطمینان بالا پشتیبانی شود.
\مرجع{Adame2021}

\متن‌لاتین{TSN} در ابتدا برای شبکه‌ها اترنت (\متن‌لاتین{IEEE 802.3}) طراحی شده بود اما به آرامی راه خود را به شبکه‌های بی‌سیم باز می‌کند. در \متن‌لاتین{TSN} سعی می‌شود
هیچ بسته‌ای به خاطر ازدحام بافرها از دست نرود، بسته‌های کمی در خرابی تجهیزات از دست بروند و تاخیر انتها به انتها گارانتی شده باشد.
کارگروه \متن‌لاتین{IEEE P802.11be} برای طراحی لایه \متن‌لاتین{MAC} و \متن‌لاتین{PHY} در می ۲۰۱۹ شکل گرفت. یکی از اهداف \متن‌لاتین{WiFi 7} کاهش بدترین حالت تاخیر و \متن‌لاتین{Jitter} می‌باشد
که برای آن، کارگروه در حال بررسی استانداردهای \متن‌لاتین{TSN} می‌باشد.
\مرجع{Adame2021}

با وجود اینکه هرگز \متن‌لاتین{WiFi} نخواهد توانست تاخیر محدودی را با توجه به ماهیت خود در استفاده از باندهای فرکانسی بدون مجوز، ارائه دهد اما استفاده از مفاهیم \متن‌لاتین{TSN}
می‌تواند آن را در زمره فناوری‌های پیشرو در \متن‌لاتین{6G} قرار دهد.
\مرجع{Adame2021}

به صورت سنتی \متن‌لاتین{WiFi} برای مدیریت دسترسی همزمان از \متن‌لاتین{Distributed Coordination Function} یا مختصرا \متن‌لاتین{DCF} استفاده می‌کند.
این شیوه بر پایه حس حامل و عقب‌نشینی نمایی عمل می‌کند. از مشکلات اصلی آن می‌توان به عدم قابلیت برای اولویت‌دهی ترافیک و از سوی دیگر غیرقابل پیش‌بینی بودن
آن اشاره کرد. در واقع در \متن‌لاتین{DCF} چند ایستگاه می‌توانند باعث اشباع شدن کانل شده و بنابراین نمی‌توان گارانتی از نظر زمانی برای داده‌ها ارائه داد.
\مرجع{Adame2021}

برای حل این مشکل روش \متن‌لاتین{EDCF} یا \متن‌لاتین{Enhanced DCF} در \متن‌لاتین{IEEE 802.11e} پیشنهاد شد. در این روش امکان اولویت‌دهی بر پایه
کاتالوگ‌های دسترسی اضافه شد. در ادامه این شیوده در \متن‌لاتین{IEEE 802.11aa} برای ارتباطات صدا و تصویر بهبود بیشتری یافت.
با این حال هیچ یک از این استانداردها کیفیت سرویس را در شرایطی که \متن‌لاتین{WiFi} دارای بار اضافه است، گارانتی نمی‌کنند.
\مرجع{Adame2021}

در لایه انتقال وجود بافر در پروتکل \متن‌لاتین{TCP} باعث تاخیرهای زیادی می‌شود و این امر کار برای انتقال جریان‌های ترافیکی \متن‌لاتین{TCP}
با استانداردهای \متن‌لاتین{TSN} سخت می‌کند. از سوی دیگر تکنیک‌های شبکه‌های سیمی مانند روش‌های نوین مدیریت صف و \نقاط‌خ در اینجا
کارایی زیادی ندارد.
\مرجع{Adame2021}

در استاندارد \متن‌لاتین{IEEE 802.11be} حالت عملیاتی چند کاناله وجود دارد. با استفاده از این حالت امکان افزایش بهره‌وری با ارسال همزمان
روی چند کانال به وجود می‌آید و از سوی دیگر می‌توان یک بسته یکسان را در چند کانال ارسال کرده تا از رسیدن آن مطمئن شد. در نهایت ارسال‌کننده
می‌تواند کانال با تاخیر کمتر را انتخاب کرده و تاخیر را کاهش دهد. این حالت عملیاتی خود می‌تواند در دو حالت همزمان و غیرهمزمان استفاده شود.
در حالت همزمان بعد از ارسال از کانال اصلی یک مدتی صبر شده و بعد می‌توان از کانال ثانویه استفاده کرد این در حالتی است که در حالت غیرهمزمان
هر دو کانال می‌توانند همزمان استفاده شوند ولی امکان تداخل میان آن‌ها وجود دارد.
\مرجع{Adame2021}

\قسمت{پروتکل‌های ارتباطی سطح بالا}

\زیرقسمت{مقدمه}

به جز پروتکل‌های لایه پیوند داده و فیزیکی، پروتکل‌های سطح بالای زیادی برای اینترنت اشیا پیشنهاد شده‌اند.
مثلا پروتکل \متن‌لاتین{mDNS} می‌تواند به عملیات \متن‌لاتین{Name Resolution} در اینرتنت اشیا کمک کند یا پروتکل
\متن‌لاتین{DNS-DS} میتواند به وسیله‌ی کلاینت‌ها برای پیدا کردن سرویس‌های موردنظرشان به وابسته‌ی \متن‌لاتین{mDNS} استفاده شود.
\مرجع{Lin2017}

\زیرقسمت{\متن‌لاتین{AMQP}}

پروتکل \متن‌لاتین{Advanced Message Queuing Protocol} یا اختصارا \متن‌لاتین{AMQP} یک پروتکل صف پیام استاندارد و متن باز است
که از آن برای سرویس پیام شامل مسیریابی، صف کردن، قابلیت اطمینان و \نقاط‌خ در لایه کاربرد استفاده می‌شود.
\مرجع{Lin2017}

\زیرقسمت{\متن‌لاتین{MQTT}}

پروتکل‌های اینترنت اشیا امروزه قلب اصلی ارتباط‌های ماشین به ماشین (\متن‌لاتین{M2M}) را تشکیل می‌دهند. فارغ از تکنولوژی رادیویی که برای پیاده‌سازی شبکه‌ی اینتنرت اشیا و ماشین به ماشین استفاده می‌شود، همه داده‌هایی که توسط سنسورها و عملگرهای اینترنت اشیا
تولید می‌شوند وابستگی زیادی به پروتکل ارتباطی که برای ارتباط ماشین به ماشین در اپلیکیشن اینترنت اشیا استفاده شده است، دارند.
با افزایش تقاضا برای سرویس‌های مبتنی بر اینترنت اشیا، نیاز برای کاهش توان دستگاه‌ها و سرویس‌های اینترنت اشیا نیز در جهت محیط زیست پایدار برای نسل‌های آینده، افزایش یافته است.

پروتکل \متن‌لاتین{Messaging Queue Telemetry Transport} که مختصرا \متن‌لاتین{MQTT} نامیده می‌شود یکی از پروتکل‌ها پر استفاده در اینترنت اشیا می‌باشد.
این پروتکل یک پروتکل با معماری انتشار و اشتراک است که توان مصرفی پایینی دارد.
\متن‌لاتین{MQTT} یک پروتکل لایه کاربرد است که برای لایه انتقال از \متن‌لاتین{TCP/IP} و پورت‌های ۱۸۸۳ و ۸۸۸۳ (به ترتیب برای ارتباط رمز شده و ارتباط رمز نشده) استفاده می‌کند. البته پژوهش‌هایی چون \مرجع{Fernndez2021} سعی در تغییر لایه انتقال به \متن‌لاتین{UDP/Quic} داشته‌اند.
\مرجع{Mishra2021}

سه نقش در معماری \متن‌لاتین{MQTT} تعریف شده است. نقش اول کلاینت تولید کننده داده می‌باشد که به آن \متن‌لاتین{Producer} گفته می‌شود. نقش دوم سرور دلال پیام می‌باشد و نقش سوم کلاینت دریافت کننده داده است که به آن \متن‌لاتین{Subscriber} گفته می‌شود.
از \متن‌لاتین{Topic} برای مشخص کردن جریان‌های داده‌ای استفاده می‌شود و در تولید کننده داده می‌بایست برای داده‌ی خود \متن‌لاتین{Topic} داشته باشد و هر دریافت کننده داده روی \متن‌لاتین{Topic} خاصی مشترک می‌شود.
این \متن‌لاتین{Topic}ها می‌توانند به صورت سلسله مراتبی نیز می‌تواند تشکیل شود.
\مرجع{Mishra2021}

در پروتکل \متن‌لاتین{‌MQTT} سه سطح مختلف از کیفیت سرویس تعریف می‌شود. در کیفیت سرویس \متن‌لاتین{QoS0} پیام‌ها به صورت ارسال و فراموش کردن ارسال می‌شوند و هیچ تضمینی برای موفیت این ارسال وجود ندارد.
در کیفیت سرویس \متن‌لاتین{QoS1} پیام‌ها حداقل یکبار تحویل داده خواهند شد، در این حالت کلاینت ارسال‌کننده پس از ارسال منتظر پیام \متن‌لاتین{PUBACK} می‌ماند و در صورت عدم دریافت آن فرآیند ارسال را دوباره تکرار می‌کند.
در کیفیت سرویس \متن‌لاتین{QoS2} پیام‌ها دقیقا یکبار تحویل داده می‌شوند، این بالاترین کیفیت سرویس بوده و منابع زیادی را مصرف می‌کند.
\مرجع{Mishra2021}

در نظر داشته باشید که کیفیت سرویس پروتکل \متن‌لاتین{MQTT} به صورت انتها به انتها نیست و به ارتباط میان دلال پیام و کلاینت‌ها وابسته است.
کیفیت سرویس پیام دریافت شده از سوی \متن‌لاتین{Subscriber} وابسته کیفیت سرویس عملیات انتشار و اشتراک است. اگر کلاینت الف عملیات انتشار را با کیفیت سرویس بالاتری
نسبت به عملیات اشتراک در کلاینت ب انجام دهد کیفیت سرویسی که سرور پیام را به دست کلاینت ب می‌رساند کیفیت سرویسی است که کلاینت ب در عملیات اشتراک استفاده کرده است.
اگر کلاینت الف عملیات انتشار را با کیفیت سرویس پایین‌تری نسبت به عملیات اشتراک در کلاینت ب انجام دهد کیفیت سرویسی که سرور پیام را به دست کلاینت ب می‌رساند کیفیت سرویسی است که
کلاینت الف در عملیات انتشار استفاده کرده است.
\مرجع{MQTTQoS}

\شروع{لوح}
\شرح{چگونگی محاسبه کیفیت سرویس پیام دریافتی در پروتکل \متن‌لاتین{MQTT}\مرجع{MQTTQoS}}
\فضای‌و{5mm}
\begin{tabularx}{\textwidth}{|X|X|X|}
\خط‌پر
کیفیت سرویس عملیات انتشار & کیفیت سرویس عملیات اشتراک & کیفیت سرویس پیام دریافتی \\
\خط‌پر
۰ & ۰ & ۰ \\
\خط‌پر
۰ & ۱ & ۰ \\
\خط‌پر
۰ & ۲ & ۰ \\
\خط‌پر
۱ & ۰ & ۰ \\
\خط‌پر
۱ & ۱ & ۱ \\
\خط‌پر
۱ & ۲ & ۱ \\
\خط‌پر
۲ & ۰ & ۰ \\
\خط‌پر
۲ & ۱ & ۱ \\
\خط‌پر
۲ & ۲ & ۲ \\
\خط‌پر
\end{tabularx}
\پایان{لوح}

\شروع{شکل}
\تنظیم‌ازوسط
\درج‌تصویر[width=\textwidth]{./img/mqtt-qos.png}
\فضای‌و{5mm}
\شرح{سطوح مختلف کیفیت سرویس در پروتکل \متن‌لاتین{MQTT}\مرجع{Mishra2021}}
\پایان{شکل}

به صورت کلی می‌توان دلال‌های پیام در پروتکل \متن‌لاتین{MQTT} را به دو دسته کلی تقسم کرد. دلال‌هایی که از تعداد مشخصی نخ استفاده می‌کنند و نمی‌توانند در صورت لزوم از همه منابع سیستم استفاده کنند و دلال‌هایی که به صورت چند پروسه‌ای یا چند نخی طراحی شده‌اند
و می‌توانند در صورت لزوم تمام منابع سیستم را مصرف کنند. دسته اول برای زمانی که سیستم منابع زیادی ندارد یا در لبه می‌توانند کاربرد داشته باشند و دسته دوم عموما در زیرساخت‌های بزرگ و ابری می‌توانند مورد استفاده قرار بگیرند.
\مرجع{Mishra2021}

همانطور که اشاره شد پروتکل \متن‌لاتین{MQTT} در کاربردهای اینترنت اشیا می‌تواند برای ارتباط مستقیم میان اشیا و اپلیکیشن مورد استفاده قرار بگیرد. از سوی دیگر این پروتکل با توجه به ماهیت غیرهمزمانی که دارد یکی از راه‌های شناخته شده برای ارتباط میان سرور شبکه \متن‌لاتین{LoRaWAN}
و برنامه‌های کاربردی و \متن‌لاتین{Gateway} می‌باشد. این امر در معماری سامانه \متن‌لاتین{Chirpstack} (شکل \رجوع{شکل:معماری سامانه Chripstack}) که یکی از بسترهای شناخته شده و متن باز برای \متن‌لاتین{LoRa} می‌باشد مشهود است.

\شروع{شکل}
\درج‌تصویر[width=\textwidth]{./img/chirpstack-architecture.png}
\تنظیم‌ازوسط
\شرح{معماری \متن‌لاتین{LoRaWAN} سرور متن باز \متن‌لاتین{Chirpstack}}
\برچسب{شکل:معماری سامانه Chripstack}
\پایان{شکل}

\زیرقسمت{\متن‌لاتین{QUIC}}

امروز با گسترش سرویس‌های حساس به تاخیر در دنیای وب و استفاده از بستر وب برای برنامه‌های کاربردی باعث فشار برای کاهش تاخیر وب شده است. تاخیر وب تاثر مستقیمی بر تجربه کاربران دارد
و دم تاخیر باعث می‌شود که گسترش بستر وب با مشکل مواجه شود.
از سوی دیگر وب امروز از سمت ترافیک غیر امن به ترافیک امن حرکت کرده است و این خود باعث افزایش تاخیر شده است.
کاهش تاخیر عموما در لایه انتقال با محدودیت‌های \متن‌لاتین{TCP/TLS} مواجه شده‌اند. در ادامه به مرور این محدودیت‌ها می‌پردازیم.
\مرجع{10.1145/3098822.3098842}

\متن‌سیاه{بهبود پروتکل}: با وجود ارائه پروتکل‌های انتقال جدید برای برآورده کردن نیازهای برنامه‌های کاربردی در ورای سرویس ساده \متن‌لاتین{TCP}، این پروتکل‌ها به صورت گسترده پیاده‌سازی نشده‌اند.
چرا که جعبه‌های میانی، نقطه کنترل در معماری اینترنت هستند، دیوارهای آتش ترافیک‌های غیرمعمول را به دلایل امنیتی بلوکه می‌کنند و ترجمه‌کنندگان آدرس شبکه (\متن‌لاتین{NAT})
سرآیند انتقال بسته‌ها را بازنویسی می‌کنند بنابراین نیاز است که به صورت مشخص از پروتکل انتقال جدید پشتیبانی کنند.
هر بسته‌ای که بدون امنیت انتها به انتها در بستر اینترنت در حرکت است مانند سرآیند بسته‌های \متن‌لاتین{TCP} ممکن است توسط جعبه‌های میانی مورد تغییر یا بررسی قرار بگیرد.
گسترش \متن‌لاتین{TCP} به حدی است که کوچکترین تغییرات پروتکل نیاز به سال‌ها زمان در جهت پیاده‌سازی گسترده دارد.
\مرجع{10.1145/3098822.3098842}

\متن‌سیاه{بهبود پیاده‌سازی}: پروتکل \متن‌لاتین{TCP} عموما در سطح سیستم عامل پیاده‌سازی شده است. با گسترش اینترنت و تهدیدهای امنیتی نیاز است که تغییرات کلاینت‌ها به سرعت اتفاق بیافتد.
این درهم تنیدگی پیاده‌سازی پروتکل \متن‌لاتین{TCP} و سیستم عامل باعث می‌شود که به روزرسانی این پیاده‌سازی منجر به یک به روزرسانی در سطح سیستم شود که عموما به روزرسانی‌ها در این سطح
با احتیاط انجام می‌گیرند. البته با پیشرفت‌های این سال‌ها پروسه به روزرسانی سیستم عامل بسیار سریعتر شده است اما هنوز ماه‌ها زمان برای یک نسخه قابل اتکا از سیستم عامل لازم است و این باعث
محدود شدن سرعت برای حتی ساده‌ترین تغییرات شبکه‌ای می‌شود.
\مرجع{10.1145/3098822.3098842}

\متن‌سیاه{تاخیر دست‌داد}: عمومیت پروتکل‌های \متن‌لاتین{TCP} و \متن‌لاتین{TLS} در حال خدمت به گسترش اینترنت است اما سربار لایه‌بندی با نیازمندی تاخیر در پشته \متن‌لاتین{HTTPS}،
بسیار مشهود شده است. یک ارتباط \متن‌لاتین{TCP} پیش از آنکه هیچ داده‌ی کاربردی بتواند ارسال شود نیاز به یک تاخیر رفت و برگشت برای ساخت ارتباط دارد و \متن‌لاتین{TLS} دو رفت و برگشت
به این تاخیر اضافه می‌کند. اهمیت این مساله زمانی مشخص می‌شود که دقت کنیم در این سال‌های پهنای باند گسترش پیدا کرده است اما سرعت نور ثابت مانده است. از سوی دیگر بیشتر تراکنش‌های دنیای وب
کوتاه بوده و بسیار تحت تاثیر این دست‌دادهای غیرضروری هستند.
\مرجع{10.1145/3098822.3098842}

\متن‌سیاه{تاخیر بلاک شدن ابتدای خط}: در نسخه \متن‌لاتین{HTTP/1.1} پیشنهاد می‌شد برای کاهش تاخیر از چند ارتباط همزمان استفاده شود و سرورها نیز تعداد ارتباط‌های همزمان را محدود کنند.
در نسخه \متن‌لاتین{HTTP/2} از مالتی پلکس کردن روی یک ارتباط استفاده می‌شد و بنابراین پیشنهاد می‌شد با هر سرور از یک ارتباط استفاده شود.
اما انتزاع \متن‌لاتین{TCP} از جریان بایت‌ها باعث می‌شود تا برنامه نتواند کنترلی برای فرآیند فریم‌بندی ارتباط داشته باشد و فریم‌های اطلاعاتی می‌بایست برای باز ارسال فریم‌های قبلتر منتظر بمانند.
\مرجع{10.1145/3098822.3098842}

پروتکل \متن‌لاتین{QUIC} توسط گوگل در سال ۲۰۱۳ پیشنهاد شد و ۳ سال بعد کاگروهی در \متن‌لاتین{IETF} برای استانداردسازی آن شکل گرفت. این پروتکل در لایه کاربر بوده و بر پایه \متن‌لاتین{UDP} کار می‌کند.
هدف این پروتکل جایگزین کردن پشته سابق \متن‌لاتین{HTTP2}، \متن‌لاتین{TLS} و \متن‌لاتین{TCP} است.
پیاده‌سازی این پروتکل در لایه کاربر فرآیند توسعه و سازگاری آن را ساده می‌کند و از سوی دیگر استفاده از \متن‌لاتین{UDP} اجازه می‌دهد به سادگی بر بستر شبکه‌های حاضر فعالیت کند.
این پروتکل از اپراتورهای شبکه و سازندگان تجهیزات مستقل بوده و اجازه می‌دهد برنامه‌های کاربردی بتوانند به صورت مستقیم از منافع آن بهره ببرند.
\مرجع{10.1145/3098822.3098842}

\شروع{فقرات}
\فقره این پروتکل با ترکیب دست‌داد لایه‌های انتقال و رمزنگاری زمان رفت و برگشت‌های لازم برای شکل‌گیری ارتباط را کمینه می‌کند.
\فقره این پروتکل با مالتی‌پلکس کردن چند درخواست‌ها و پاسخ‌هایشان روی یک ارتباط به وسیله فراهم آوردن یک جریان برای هر کدام باعث می‌شود تا هیچ پاسخی توسط دیگری بلاک نشود.
\فقره این پروتکل باعث می‌شود تا ارتباط‌ها با تغییر آدرس \متن‌لاتین{IP} ادامه پیدا کنند. برای این کار \متن‌لاتین{QUIC} از شناسه ارتباط به جای پنج‌تایی \متن‌لاتین{IP/Port/Protocol} استفاده می‌کند.
\فقره این پروتکل بازیابی بسته‌ها را به وسیله استفاده از شناسه یکتا برای بسته‌ها، جلوگیری از ابهام در باز ارسال و سیگنال \متن‌لاتین{ACK} صریح برای محاسبه دقیق زمان رفت و برگشت بهبود می‌بخشد. اگر بخواهیم بهتر بیان کنیم برخلاف \متن‌لاتین{TCP} در این پروتکل
بسته‌های باز ارسال شده شناسه متفاوتی از بسته‌ی اصلی دارند و همین امر محاسبه زمان رفت و برگشت برای \متن‌لاتین{ACK} آن‌ها را ساده‌تر می‌کند. در نظر داشته باشید که ترتیب بسته‌ها با توجه به شناسه‌ی آن‌ها در جریانشان مشخص می‌شود.
\فقره این پروتکل بسته‌ها را رمزنگاری و اهراز هویت می‌کند و از این رو جلو خوانده یا تغییر آن‌ها توسط جعبه‌های میانی گرفته می‌شود.
\فقره این پروتکل کنترل جریان فراهم می‌آورد و مطمئن می‌شود که همه بافر در گیرنده توسط یک جریان مصرف نشود و از سوی دیگر میزان داده‌ی بافر شده در گیرنده آهسته را محدود می‌کند. در واقع این پروتکل در سطح ارتباط و در سطح جریان کنترل صورت می‌دهد.
\فقره این پروتکل دارای یک رابط ماژولار برای کنترل ازدحام است.
\پایان{فقرات}
\مرجع{10.1145/3098822.3098842}

در نهایت گوگل برای ارزیابی این پروتکل از زیرساخت تست \متن‌لاتین{A/B} در نرم‌افزار کروم استفاده کرده است که به آن‌ها اجازه می‌داده متریک‌های متنوعی را ارزیابی و تست‌های مختلفی را انجام دهند
و از سوی دیگر این پروتکل به وابسته‌ی این تست‌ها در کنار استفاده از متریک‌های لایه انتقال توانسته از متریک‌های لایه‌ی کاربرد نیز برای بهبود خود استفاده کند که این امر نقطه برتری در طراحی این پروتکل است.
همین ارزیابی بر پایه برنامه‌های موبایل \متن‌لاتین{Google} و \متن‌لاتین{Youtube} نیز صورت پذیرفته است.
اولین پیاده‌سازی در نرم‌افزار کروم در سال ۲۰۱۳ صورت پذیرفت و پس از آن در سال ۲۰۱۴ برای درصد پایینی از کاربران فعال شد. در سال ۲۰۱۷ تقریبا همه کاربران نرم‌افزار کروم از این پروتکل استفاده می‌کردند.
\مرجع{10.1145/3098822.3098842}

یکی از ارزیابی‌های صورت پذیرفته در گوگل با ساختاری که پیشتر به آن پرداخته شد مربوط به تاخیر دست‌داد است. برای این ارزیابی دو گروه از کاربران وجود دارند، گروهی که از پروتکل \متن‌لاتین{QUIC} استفاده می‌کنند
و در صورت عدم موفقیت آن از پروتکل \متن‌لاتین{TCP/TLS} استفاده می‌کنند که با نماد $QUIC_{g}$ نمایش داده می‌شوند و گروهی که از پروتکل \متن‌لاتین{TCP/TLS} استفاده می‌کنند و با نماد $TCP_{g}$
نمایش داده می‌شوند. مشخص شده است که درصدی از کاربران که نتوانسته‌اند ارتباط درستی با \متن‌لاتین{QUIC} داشته باشند و مجبور به استفاده از \متن‌لاتین{TCP/TLS} شده‌اند بسیار اندک هستند.
در شکل \رجوع{شکل: مقایسه تاخیر دست‌داد در پروتکل‌های QUIC و TCP/TLS} مقایسه تاخیر دست‌داد در این دو گروه آمده است. همانطور که می‌بینید در صورت استفاده پروتکل \متن‌لاتین{QUIC} در حالت \متن‌لاتین{0-RTT}
عملا وابستگی به زمان رفت و برگشت وجود ندارد و حتی در صورتی که نیاز باشد پروتکل \متن‌لاتین{QUIC} از حالت \متن‌لاتین{1-RTT} نیز استفاده کند باز تاخیر نسبت به \متن‌لاتین{TCP/TLS} کمتر است.
\مرجع{10.1145/3098822.3098842}

\شروع{شکل}
\درج‌تصویر[width=\textwidth]{./img/quic-handshake.png}
\تنظیم‌ازوسط
\شرح{مقایسه تاخیر دست‌داد در پروتکل‌های \متن‌لاتین{QUIC} و \متن‌لاتین{TCP/TLS} \مرجع{10.1145/3098822.3098842}}
\برچسب{شکل: مقایسه تاخیر دست‌داد در پروتکل‌های QUIC و TCP/TLS}
\پایان{شکل}

یکی از چالش‌های پیش رو در این پروتکل مصرف بیشتر منابع پردازشی در سمت سرورها است. در ابتدای کار این مصرف نزدیک به سه برابر \متن‌لاتین{TCP/TLS} بود اما با بهبودهایی که گوگل در پیاده‌سازی خود ایجاد کرد
این مقدار به دو برابر رسید اما هنوز جای کار و بهبود دارد. گوگل فکر می‌کند با حذف قسمت‌هایی از هسته سیستم عامل در روند پردازش داده‌ها می‌تواند بهبود بیشتری ایجاد کند.
در نهایت این پژوهش بیان می‌کند در زمانی که سیستم از کانکشن‌های از پیش ساخته شده استفاده کند، پهنای باند زیادی در اختیار داشته باشد، تاخیر اندکی داشته باشد و یا محدودیت پردازشی داشته باشد نسخه پیشنهادی
\متن‌لاتین{QUIC} کارایی زیادی نخواهد داشت. البته تلاش برای بهبود \متن‌لاتین{QUIC} بر بسترهای موبایل هنوز ادامه دارد.
\مرجع{10.1145/3098822.3098842}

\قسمت{پردازش در لبه}

با گسترش اینترنت اشیا و مطرح شدن داده‌های حجیم در چند سال اخیر مشخص شده است که استفاده تنها از سرورهای ابری نمی‌تواند برای این پردازش مفید باشد و باعث هدر رفت منابعی چون پهنای باند و \نقاط‌خ می‌گردد.
برای پاسخ به این مشکل، بحث پردازش مِه یا \متن‌لاتین{Fog Computing} پیشنهاد شد که در آن پیشنهاد می‌شود تا پردازش‌ها به جای انجام شدن در سرورهای ابری در لبه شبکه انجام شوند.
اگر بخواهیم بهتر بیان کنیم داده‌ّهای جمع‌آوری شده به جای ارسال و پردازش در ابر در همان نودهای لبه پردازش شوند.
\مرجع{Perera2017}

اگر بخواهیم چالش‌هایی که در اینترنت اشیا باعث روی آوردن به پردازش در لبه و مه شده است را جمع‌بندی کنیم:
\شروع{فقرات}
\فقره \متن‌سیاه{تاخیر کم}: برنامه‌های اینترنت اشیا و سیستم‌های کنترل صنعتی خواهان تاخیر در بازه چند میلی‌ثانیه هستند که به سختی در مدل ابری حاضر قابل برآورده شدن است.
\فقره \متن‌سیاه{پهنای باند بالای شبکه}: افزایش دستگاه‌های اینترنت اشیا باعث تولید حجم بالایی از داده شده است که ممکن است با توجه به پهنای باند زیادی که برای
انتقال این داده‌ها به ابر مصرف می‌شود یا حذف آن‌ها برای جلوگیری از نقص حریم خصوصی در نهایت
بدون استفاده باشند، بنابراین بهتر است که با آن‌ها به صورت محلی برخورد شود.
\فقره \متن‌سیاه{منابع محدود}: بسیار از دستگاه‌های متصل در اینترنت اشیا توانایی ارتباط مستقیم با ابر به خاطر منابع محدود را ندارند. عموما چنین ارتباط‌های پردازش زیاد و پروتکل‌های پیچیده‌ای را انتظار دارد.
\فقره \متن‌سیاه{همپوشانی فناوری عملیات و فناوری اطلاعات}: در سیستم‌های صنعتی به هم رسیدن فناوری اطلاعات و فناوری عملیات نیازمندی‌های جدیدی را ایجاد کرده است. با توجه به اینکه سیستم‌های آفلاین
می‌تواند باعث از دست رفتن سود تجارت یا آزار مشتریان شود، سیستم‌های سایبر فیزیکال معاصر می‌بایست به صورت پیوسته و امن عملیاتی باشند. بنابراین به روزرسانی‌های نرم‌افزاری و سخت‌افزاری باعث ایجاد نگرانی می‌گردد.
\فقره \متن‌سیاه{ارتباط منقطع}: وقتی دستگاه‌هایی در اینترنت اشیا از ارتباط منقطع استفاده می‌کنند فراهم آوردن یک سرویس بدون قطعی ابری برای آن‌ها دشوار خواهد بود.
\فقره \متن‌سیاه{توزیع جغرافیایی}: بسیاری از اشیا مشتمل بر سرویس‌هایی از منابع پردازشی و ذخیره‌سازی هستند، که در مناطق جفرافیایی بزرگی پراکنده هستند. بنابراین قراردادن آن‌ها در محلی که بتواند به نیازهای اینترنت اشیا پاسخ دهد
دشوار خواهد بود.
\فقره \متن‌سیاه{حساس به متن}: داده‌هایی با زمینه محلی می‌بایست برای برنامه‌های اینترنت اشیا در دسترس و قابل پردازش باشد که برای آن فاصله فیزیکی میان اشیا و ابر مرکزی مشکل‌زا است.
\فقره \متن‌سیاه{امنیت و حریم‌خصوصی}: راه‌کارهای امنیت سایبری حاضر نشان داده‌اند که برای مدیریت برنامه‌های کاربردی اینترنت اشیا با توجه به رشد چالش‌های امنیتی پاسخگو نیستند.
\پایان{فقرات}
\مرجع{Angel2021}

در پردازش مِه از یک معماری خاص پیروی نمی‌شود و هدف سوهی پردازش به سمت لبه است. انتظار می‌رود با استفاده از لبه در پردازش حجم داده‌ای که نیاز به نگهداری و پردازش دارد به میزان قابل مدیریت برسد.
از سوی دیگر پردازش در لبه می‌تواند تاخیر را کاهش داده و دسترسی‌پذیری را افزایش دهد. پردازش در لبه در خانه‌های هوشمند سال‌ها است که مورد استفاده قرار گرفته است اما برای پی بردن به ویژگی‌های پردازش در لبه
نیاز است که آن در کاربردها پیچیده‌تر مانند شهر هوشمند ارزیابی کرد.
\مرجع{Perera2017}

در پردازش مه، داده‌های مختلف در اینترنت اشیا می‌توانند به محل‌های مناسب با توجه به نیازمندی‌های کارایی، جهت پردازش بیشتر هدایت شوند. مثلا داده‌هایی با الویت بالا که نیاز است به موقع به آن‌ها رسیدگی شود،
می‌توانند در نودهای لبه پردازش شوند که نزدیک‌ترین به دستگاه‌هایی هستند که این داده‌ها را تولید کرده است. داده‌هایی با الویت پایین که به تاخیر حساس نیستند می‌توانند در جهت پردازش بیشتر به نودهای تجمع‌کننده
برای پردازش بیشتر ارسال شوند.
یکی از چالش‌های استفاده از پردازش مه، چگونگی مدیریت و تخصیص منابع پردازش لبه است. از سوی دیگر چگونگی تخصیص این منابع به اشیا نیز خود چالش دیگری است.
\مرجع{Lin2017}

مساله ارتباطه منابع پردازش لبه و اشیا، باعث رضایت‌مندی کاربران می‌شود بنابراین در پردازش لبه یک مساله بیشینه‌سازی رضایت‌مندی کلی با توجه به محدودیت منابع پردازشی خواهد بود
که می‌توان آن را در قالب یک مساله بهینه‌سازی مطرح کرد. در این مساله نیاز است که ارتباطی میان رضایت‌مندی کاربر و منابع تخصیص‌یافته بدست آید که پژوهش \مرجع{Lin2017} از مدل
لگاریتیمی استفاده کرده است.
از سوی دیگر، همانطور که بیان شد چالش دیگر پردازش مه، مربوط به مدیریت منابع است. در پردازش مه ممکن است لبه‌، پردازش‌ها را با توجه به کمبود منابع به سمت لبه‌های همسایه هدایت کنند.
در این شیوه هدف کاهش هزینه (تاخیر یا \نقاط‌خ) است. برای این چالش نیز می‌توان از یک مساله بهینه‌سازی استفاده کرد.
\مرجع{Lin2017}

\زیرقسمت{\متن‌لاتین{Mobile Computing}}

پردازش همراه، پردازشی است که روی تلفن‌های همراه، تبلت‌ها و یا لپتاب‌ها صورت می‌گیرد. دستگاه‌های همراه مزایای زیادی را برای کاربرانشان به همراه دارند اما هنوز
محدودیت‌هایی به خاطر توان پردازشی پایین، باتری یا حافظه‌شان، که از اندازه کوچکشان و فعالیت آن‌ها با شبکه روشن یا خاموش منشا می‌گیرد، دارند.
این مشکلات باعث شده است که پردازش همراه برای برنامه‌هایی که نیاز به تاخیر پایین و قابلیت اطمینان آن هم هنگامی که حجم زیادی داده برای ذخیره‌سازی و پردازش دارند، کافی نباشد.
\مرجع{Angel2021}

\زیرقسمت{\متن‌لاتین{Mobile Cloud Computing}}

پردازش همراه ابری یا اختصارا \متن‌لاتین{MCC} بیان می‌کند که پردازش ابری در ترکیب با برنامه‌های موبایل قرار گرفته و ماژول‌های پردازشی پیچیده در ابر پردازش شوند.
در این شیوه پردازش و ذخیره‌سازی به دور از موبایل صورت می‌گیرد نه تنها صاحبان تلفن‌های هوشمند بلکه مشترکان زیادی از موبایل از آن بهره می‌برند.
با انجام تسک‌های محاسباتی بزرگ بر روی ابر با استفاده از \متن‌لاتین{MCC} مصرف باتری دستگاه‌ها بهبود میابد.
با این وجود با استفاده از تکنولوژی‌های ارتباط سلولی، انتقال اطلاعات در فواصل طولانی به/از شبکه هسته، افزایش تاخیر، تغییرات تاخیر و سربار شبکه را در پی دارد.
برای غلبه به این مساله، پردازش و فیلترینگ داده می‌بایست در نزدیکی منبع آن صورت بگیرد، بحثی که در پردازش مه و پردازش همراه لبه (\متن‌لاتین{MEC}) صورت می‌پذیرد.
\مرجع{Angel2021}

\زیرقسمت{\متن‌لاتین{Cloud of Things}}

در ابر اشیا، دستگاه‌های اینترنت اشیا در کنار یکدیگر یک ساختار ابر مجازی را شکل می‌دهند. در این حالت پردازش روی یک مجموعه‌ای از منابع که ساخته شده از اشیا است، صورت می‌پذیرد.
این امر با پردازش غباری متفاوت است چرا که در آن پردازش روی مجموعه‌ای از اشیا صورت می‌پذیرد.
\مرجع{Angel2021}

\زیرقسمت{\متن‌لاتین{Mist Computing}}

پردازش غباری اولین لایه محاسباتی است که اجازه محاسبه، ذخیره‌سازی و شبکه‌سازی را از اشیا تا مِه می‌دهد. در پردازش غباری ما محدود به دستگاه‌های همراه نیستیم بلکه اجزای پیرامونی (حسگرها و عملگرها)
ظرفیت پیش‌پردازش داده‌ها قبل از ارسال آن‌ها به ابر را دارند. پردازش غباری در سیستم‌های بزرگ اینترنت اشیا بسیار کمک می‌کند و می‌تواند کارایی محاسباتی در لبه را بهبود بخشد.
\مرجع{Angel2021}

\زیرقسمت{\متن‌لاتین{Edge Computing}}

همانطور که بیان شد پردازش لبه تکنولوژی‌هایی را مشخص می‌کند که پردازش در لبه شبکه را برای داده‌های ارسال از سرویس‌های ابری و داده‌های ارسالی از سرویس‌های اینترنت اشیا ممکن می‌سازند.
این مدل پردازشی، شبکه‌ی ابری را به واسطه‌ی اضافه کردن پردازش، شبکه و منابع در لبه شبکه در نزدیکی منبع داده‌ها، برای برآورده ساختن نیازهای حیاتی سرویس‌های همزمان، برنامه‌های هوشمند،
امنیت و حریم خصوصی در کنار نیازمندی‌های شبکه‌ای چون تاخیر کم و پهنای باند بالا، گسترش می‌دهد.
\مرجع{Angel2021}

در مقابل پردازش ابری، پردازش در لبه برای پردازش‌های هوشمند و همزمان در مقیاس کوچک کارایی بیشتری دارد. این نوع پردازش ریسک‌های انتقال بر بستر شبکه را نداشته و می‌تواند امنیت داده‌ها تضمین کند.
ارتباط میان لبه و ابر عموما در یک مدل سه‌لایه‌ای مشتمل بر نودهای انتهایی یا \متن‌لاتین{Terminal} (سنسورها، عملگرها و تلفن‌های هوشمند)،
لبه یا \متن‌لاتین{Edge} (\متن‌لاتین{Base Station}ها، \متن‌لاتین{Gateway}ها، سوئیچ‌ها، مسیریاب‌ها و \متن‌لاتین{Access Point}ها)
و ابر یا \متن‌لاتین{Cloud} مشخص می‌گردد.
\مرجع{Angel2021}

\زیرقسمت{\متن‌لاتین{Multi-Access Edge Computing}}

استاندارد \متن‌لاتین{Mobile Edge Computing} یا اختصارا \متن‌لاتین{MEC} به وسیله‌ی \متن‌لاتین{ETSI} برای استفاده از پردازش ابری و کارکردهای فناوری اطلاعات در شبکه‌ی دسترسی رادیویی
در حیطه‌ی مشترکین تلفن همراه مطرح شد. در سال ۲۰۱۷ با توجه به استقبال اپراتورهای غیرسلولی، \متن‌لاتین{ETSI} نام این استاندارد را به \متن‌لاتین{Multi-Access Edge Computing}
تغییر داد. بنابراین \متن‌لاتین{MEC} ادامه پردازش موبایل به وسیله‌ی پردازش لبه برای رساندن منابع پردازشی و ذخیره‌سازی به دستگاه‌های موبایل با منابع و توان محدود است.
\مرجع{Angel2021}

در \متن‌لاتین{MEC} یک سرور ابری در \متن‌لاتین{Base Station} شبکه سلولی، مستقر می‌شود. این سرور کارهایی نظیر بهبود کارایی برنامه‌های کاربردی
با نصب در \متن‌لاتین{Base Station}ها، \متن‌لاتین{MEC} می‌تواند تاخیر کم، نزدیکی به کاربر، اطلاع از موقعیت و توزیع جغرافیایی را بدست آورد.
اما در این شیوه به یک سرور اختصاصی برای \متن‌لاتین{MEC} نیاز است.
\مرجع{Angel2021}

\قسمت{\متن‌لاتین{Network Calculus}}

\متن‌لاتین{Network Calculus} مجموعه‌ای از پیشرفت‌های اخیر است که دید عمیقی در مساله‌های جریان در شبکه‌ها ایجاد می‌کند. پایه \متن‌لاتین{Network Calculus} در تئوری ریاضی \متن‌لاتین{Dioid}ها و مشخصا \متن‌لاتین{Min-Plus dioid} نهفته است.
در ادامه به مرور مفاهیم اصلی این حوزه می‌پردازیم.

\زیرقسمت{منحنی ورودی}

جریان با تابع تجمعی $R(t)$، دارای $\alpha$ به عنوان جریان ورودی (بیشین) است اگر:

\[
  R(t) - R(s) \le \alpha(t - s) \forall t,s \ge 0
\]

که در آن $\alpha$ یک تابع صعودی است. به عنوان مثال اگر فرض کنیم جریان ورودی با الگوریتم \متن‌لاتین{Leaky Bucket} با پارامترهای $r$ و $b$، محدود شده است داریم:

\[
  \alpha(t) = rt + b
\]

جریان‌های ورودی را می‌توان با یکدیگر جمع کرد.

\زیرقسمت{پیچش \متن‌لاتین{Min-Plus}}

پیچش دو جریان $f_{1}$ و $f_{2}$ در جبر \متن‌لاتین{Min-Plus} به شکل زیر تعریف می‌شوند:

\[
  f(t) = \inf_{s \ge 0}(f_{1}(s) + f_{2}(t-s))
\]
\[
  f = f_{1} \otimes f_{2}
\]

این پیچش، ویژگی‌ها خوب پیچیش معمول را دارد:

\[
  (f_{1} \otimes f_{2}) \otimes f_{3} = f_{1} \otimes (f_{2} \otimes f_{3})
\]
\[
  f_{1} \otimes f_{2} = f_{2} \otimes f_{1}
\]

با توجه به این تعریف می‌توان گفت $\alpha$ یک منحنی ورودی برای $R$ خواهد بود اگر و تنها اگر

\[
  R \le R \otimes \alpha
\]
