\فصل{مفاهیم}

\قسمت{مقدمه}
در سال‌های اخیر با کاهش قسمت حسگرهای و عملگرها، تعداد دستگاه‌های اینترنت اشیا به سرعت در حال گسترش است و به سرعت در حال تبدیل کردن خود به یکی از اجزا زندگی ما می‌باشند.
در نتیجه رد پای فاحش اینترنت اشیا امروزه در همه جا قابل مشاهده است.
\مرجع{Mishra2021}

\قسمت{نیازمندی‌های شبکه‌های \متن‌لاتین{LPWAN}}

یکی از بحث‌های اصلی این رساله شبکه‌های با گستره بالا و توان پایین است. این شبکه نیازمندی‌های خاصی دارند و تکنولوژی‌های پیشنهاد شده در این حوزه
هر یک روش‌هاص خاص خود را برای پاسخ به این نیازمندی‌های دنبال کرده‌اند.

\زیرقسمت{نیازمندی‌های ترافیک}

در شبکه‌های اینترنت اشیا، حسگرها و سنسورها همگی از رفتار ترافیکی یکسانی پیروی نمی‌کنند. برخی نیاز دارند که پیام‌هایشان بلافاصله به مقصد برسد و این در حالی است که برخی می‌توانند تاخیر را تحمل کنند.
از سوی دیگر با افزایش تعداد اشیا برای فرآهم آوردن کیفیت سرویس نیاز به یک زمان‌بند با الویت است. پیام‌های ارسالی خود می‌توانند تعداد یا نرخ و اندازه‌های متفاوتی داشته باشند.
در نتیجه شبکه‌های اینترنت اشیا می‌بایست ظرفیت این تنوع ترافیکی را داشته باشند.
\مرجع{Chaudhari2020}

به صورت پایه‌ای شبکه‌های می‌بایست نوعی از مدیریت ترافیک کاربران و کنترل پذیرش را داشته باشند.
این امر البته به معماری شبکه دسترسی هم وابسته است، برخی از پروتکل‌های ارتباطی بدون هیچ نوعی از کنترل پذیرش
داده‌ها را ارسال یا دریافت می‌کنند و در برخی نیاز به نوعی پذیرش وجود دارد.
در شبکه‌هایی که از اشیا متنوع پشتیبانی می‌کنند ممکن است این نیازمندی گسترش پیدا کند، به طور مثال
شبکه ممکن است پذیرش درخواست‌ها را بر پایه اولویت انجام دهد یا شرایط اذحام را مدیریت کند.
\مرجع{Chaudhari2020}

\زیرقسمت{ظرفیت و چگالی}

یکی از نیازمندی‌ها پشتیبانی از تعداد بالا اشیا است. شبکه‌ها می‌بایست بتوانند بدون اختلال در فعالیت نودهای کنونی نودهای جدیدی را به شبکه اضافه کنند.
با توجه به سادگی نودها بار اصلی این کار بر عهده \متن‌لاتین{Gateway}ها و نقاط دسترسی است.
\مرجع{Chaudhari2020}

در شبکه‌های بزرگ نیاز به یک شناسه‌ی یکتا جهانی وجود دارد.
در این شبکه‌ها با توجه به تعداد زیاد اشیا لینک‌های ارتباطی می‌بایست قابل اطمینان و کارا باشند.
از سوی دیگر تداخل در این شبکه‌ها می‌تواند بسیار باشد و نیاز است با شیوه‌هایی همچون استفاده از کانال‌های مختلف،
ارسال‌های تکراری و استفاده از روش‌های تطبیقی با این امر مقابله کرد.
برای روش‌های تطبیقی نیاز به ذخیره‌سازی وضعیت ارتباطی دستگاه‌ها در ایستگاه‌های پایه است و این امر با توجه به تعداد زیاد
اشیا نیاز به بهینه‌سازی دارد.
\مرجع{Chaudhari2020}

\زیرقسمت{توان مصرفی}

این شبکه‌ها نیاز دارند توان مصرفی پایینی داشته باشند چرا که بیشتر نودها با باتری فعالیت می‌کنند
و در برخی موارد جایگزینی باتری هم عمل سختی است.
\مرجع{Chaudhari2020}

دستگاه‌ها در این شبکه‌ها نرخ ارسال پایین دارند و برای ارسال نیز داده‌ها بسیار کوچک هستند
برای همین استفاده از مدهای خواب مناسب و غیر فعال کردن ماژول‌های پر مصرف مثل ماژول
ارتباطی برای زمان‌هایی که از آن‌ها استفاده نمی‌شود می‌تواند به میزان زیادی توان مصرفی را کاهش دهد.
\مرجع{Chaudhari2020}

اشیا در این شبکه‌ها می‌توانند از منابعی مانند باد یا انرژی خورشیدی برای شارژ کردن باتری خود استفاده کنند.
این بهبودها سربار دارند و می‌بایست بین این سربار و افزایش طول عمر باتری مصالحه نبود.
\مرجع{Chaudhari2020}

به صورت کلی مصرف توان برای پردازش‌های دستگاه بسیار کمتر از مصرف توان در ارسال و دریافت داده‌ها می‌باشد
بنابراین ساختاربندی داده‌ها پیش از ارسال می‌تواند به کاهش توان مصرفی کمک کند. از سوی دیگر می‌توان پیچیدگی‌های
ارسال را با در نظر گرفتن لایه‌ی مدیریت دسترسی همزمان ساده‌تر کاهش داد.
\مرجع{Chaudhari2020}

\زیرقسمت{پوشش‌دهی}

پوشش این شبکه‌ها در مناطق غیر شهری بین ۱۰ تا ۴۰ کیلومتر و
در مناطق شهری بین ۱ تا ۵ کیلومتر است.
این شبکه گاهاً نیاز دارند در مناطقی با دسترسی سخت یا داخل ساختمان‌ها
عملیاتی شوند. این شبکه‌ها با استفاده از باندهای زیرگیگاهرتز تلاش می‌کنند تا پوشش بیشتری را با توان کمتری بدست بیاورند.
\مرجع{Chaudhari2020}

مفهوم پوشش‌دهی خود می‌تواند به افزایش پوشش جفرافیایی، دسترسی به محیط‌های اطراف موانع و پوشش محیط‌های داخلی اشاره کند.
تکنیک‌های غلبه بر تضعیف سیگنال مانند افزایش توان ارسال، افزایش حساسیت آنتن‌ها، ارسال چندباره یک بسته و کاهش نرخ ماژولیشن اینجا کمک کننده هستند.
\مرجع{Chaudhari2020}

\زیرقسمت{موقعیت‌یابی}

یکی از نیازمندی‌ها دنبال کردن اشیا یا تشخیص رویدادهایی همچون تغییر موقعیت مکانی آن‌ها است.
برای این امر می‌توان از سیستم \متن‌لاتین{GPS} یا زیر ساخت شبکه استفاده کرد و دقت‌های مختلفی
از سانتی‌متر تا متر بدست آورد.
\مرجع{Chaudhari2020}

برای موقعیت یابی می‌توان از اطلاعات زمانی مربوط به زمان رسیدن سیگنال‌ها در جهت مکان‌یابی استفاده کرد. این روش
یکی از کم هزینه‌ترین روش‌ها است. در زمانی که از شبکه‌های سلولی استفاده می‌کنیم می‌توان از اطلاعات سلول نیز
در مکان‌یابی استفاده کرد. راهکارهای مبتنی بر ماهواره در کاربردهایی که حساس به توان مصرفی نباشد می‌تواند استفاده شود.
\مرجع{Chaudhari2020}

\زیرقسمت{امنیت و حریم خصوصی}

امنیت یکی از مسائل مهم در این شبکه‌ها است و می‌بایست مسائل پایه‌ای
مثل احراز هویت، سطوح دسترسی، اطمینان، محرمانگی، امنیت داده‌ها و عدم همسان‌سازی را در نظر بگیرد.
از سوی دیگری مسائلی چون حملات توزیع شده جلوگیری از دسترسی یا تزریق کد مخرب به شبکه و \نقاط‌خ
نیز وجود دارند که باید برای آن‌ها چاره اندیشی شود.
داده‌های می‌بایست در ارسال و دریافت رمزگذاری شوند تا امنیت آن‌ها و حریم خصوصی کاربران تضمین شود.
\مرجع{Chaudhari2020}

\زیرقسمت{هزینه مناسب}

از آنجایی که تعداد زیادی از اشیا در این شبکه دخیل هستند، هزینه نگهداری از آن‌ها و شبکه می‌بایست کم باشد.
از سوی دیگر به روزرسانی‌های نرم‌افزاری از ویژگی‌های مهم اشیا است که باعث کاهش هزینه‌ها می‌شود.
\مرجع{Chaudhari2020}

استفاده از سخت‌افزار ساده‌تر یکی از مهمترین گام‌ها در کاهش هزینه‌ها می‌باشد. از سوی دیگر به روزرسانی‌های
نرم‌افزاری که باعث کاهش تعداد به روزرسانی‌های سخت‌افزاری شوند می‌توانند به کاهش هزینه‌ها کمک کنند.
کارهای پردازشی و ارتباطی نیاز دارند که ساده باشند. سربار لایه‌های مختلف می‌بایست به گونه‌ای طراحی شوند
که کمینه باشند.
\مرجع{Chaudhari2020}

\زیرقسمت{اشیا با سخت‌افزارهایی با پیچیدگی کم}

از آنجایی که قصد داریم تعداد زیادی از اشیا را در برد بلندی و با هزینه پایین پوشش دهیم طراحی دستگاه‌ها با پیچیدگی کم
و ابعاد کوچک از نیازمندی‌های اساسی به شمار می‌آید. این اشیا نیازی به توان پردازشی بالا ندارند، معماری شبکه‌ای و پروتکل‌های
ساده‌ای را می‌بایست پشتیبانی کنند. دستگاه‌های ارتباطی آن‌ها ساده بوده و باید بتوانند به صورت نرم‌افزاری تنظیم شوند.
\مرجع{Chaudhari2020}

استفاده از تکنیک‌های رادیوهای نرم‌افزار بنیان می‌تواند اینجا بسیار کمک کننده باشد
البته باید در نظر داشت که استفاده از سخت‌افزارهای ساده و تکنیک‌های نرم‌افزاری
باعث ایجاد فرکانس‌های رادیویی غیر ایده‌آل می‌شود که برای رفع آن‌ها نیاز به استفاده
از پردازش‌های نرم‌افزاری سنگین است. بنابراین در اینجا مصالحه‌ای برای پیاده‌سازی
این تکنیک‌های ساده وجود دارد.
\مرجع{Chaudhari2020}

\زیرقسمت{گستردگی راه‌حل‌ها}

اشیا می‌بایست از شبکه‌های داری لایسنس و بدون لایسنس پشتیبانی کنند.
این اشیا می‌بایست از همبندی‌های متفاوت شبکه مانند \متن‌لاتین{Mesh}، \متن‌لاتین{Tree} و \متن‌لاتین{Star} پشتیبانی کنند.
\مرجع{Chaudhari2020}

راه‌های کارهای زیادی برای پیاده‌سازی شبکه وجود دارد بنابراین نودها می‌بایست چندین حالت و فرکانس را پشتیبانی کنند.
حالت‌های مختلف به نودها اجازه می‌دهد که در شبکه‌های مختلف که هر یک ویژگی‌های منحصر به فرد خود را دارند
فعالیت کند و از سوی دیگر فرکانس‌های مختلف به نود اجازه می‌دهد در یک تکنولوژی از چندین فرکانس مختلف استفاده کند.
\مرجع{Chaudhari2020}

\زیرقسمت{عملکرد، ارتباطات و روابط بین شبکه‌ای}

امروز شبکه‌های مختلفی با ویژگی‌های متفاوت وجود دارند اما با گسترگی \متن‌لاتین{IP} انتخاب آن به عنوان
یک استاندارد ارتباطی مطرح است. شبکه‌ها تلاش می‌کنند تا به شبکه‌های \متن‌لاتین{IP} و پروتکل‌هایی مانند
\متن‌لاتین{CoAP} متصل شوند و آن‌ها را پشتیبانی کنند.
در شبکه‌های \متن‌لاتین{LPWAN} اندازه داده‌ها کوچک است و از این رو برای ارسال بسته‌های \متن‌لاتین{IP} و
به خصوص \متن‌لاتین{IPv6} نیاز به فشرده‌سازی وجود دارد.
\مرجع{Chaudhari2020}

\قسمت{همبندی‌های شبکه‌های \متن‌لاتین{LPWAN}}

به صورت کلی دو همبندی در این شبکه‌ها مطرح است. همبندی \متن‌لاتین{Mesh} یا توری
و همبندی \متن‌لاتین{Start} یا ستاره.
در همبندی \متن‌لاتین{Mesh} همه نودها به یکدیگر متصل هستند از این رو زمانی که هدف
افزایش پوشش‌دهی و کاهش توان مصرفی است این همبندی ترجیح داده نمی‌شود. همبندی
\متن‌لاتین{Star} همبندی انتخابی در شبکه‌های \متن‌لاتین{LPWAN} می‌باشد.
در این همبندی با استفاده از یک نود می‌توان به تعداد زیادی نود دسترسی داشت که
خود باعث کاهش هزینه است.
\مرجع{Chaudhari2020}

در شبکه‌های \متن‌لاتین{Star} نود به صورت مستقیم با یک یا چند \متن‌لاتین{Gateway} در ارتباط هستند
و با یکدیگر ارتباطی ندارند. در این شبکه‌ها اطلاعات توسط \متن‌لاتین{Gateway} برای لایه‌های بالاتر ارسال می‌شود.
خود \متن‌لاتین{Gateway} ارتباطی با یکدیگر ندارند و از این رو این شبکه‌ها بسیار از شبکه‌های \متن‌لاتین{Mesh} ساده‌تر هستند.
در این شبکه‌ها نودها دارای ایراد به سادگی شناسایی می‌شوند ولی در صورت ایراد \متن‌لاتین{Gateway} تمام نودهای
متصل به آن دست می‌روند.
\مرجع{Chaudhari2020}

در شبکه‌های \متن‌لاتین{Mesh} کامل همه نودها به یکدیگر متصل هستند ولی در شبکه‌های \متن‌لاتین{Mesh} جزئی یا \متن‌لاتین{Partial Mesh} نودها
همه با یکدیگر ارتباط ندارند و با نودهایی که بیشترین پیام را رد و بدل کرده‌اند در ارتباط هستند.
مزیت این شبکه‌ها در وجود چندین راه ارتباطی بین نودها، امکان ارسال و دریافت همزمان داده‌ها از مسیرهای متفاوت،
گسترش‌پذیری آسان و قابلیت بهبود خودکار است. از سوی دیگر معایب آن شامل افزایش تاخیر به خاطر مسیرهایی با چند گام و
افزایش هزینه و پیچیدگی است.
\مرجع{Chaudhari2020}

\قسمت{معماری شبکه‌های \متن‌لاتین{LPWAN}}

معماری معمول شبکه‌های \متن‌لاتین{LPWAN} در شکل \رجوع{شکل: معماری معمول شبکه‌های LPWAN} آمده است.
کارکرد پایه یک دستگاه \متن‌لاتین{LPWAN} جمع‌اوری داده و پاسخ به ورودی‌های دریافتی از شبکه است.
داده‌های جمع‌اوری شده در یک لینک رادیویی مشخص برای ایستگاه دسترسی بی‌سیم ارسال می‌گردد.
ایستگاه بی‌سیم لینک رادیویی برای مدیریت دستگاه و تبادل اطلاعات فراهم می‌اورد.
این ایستگاه در ارتباط با \متن‌لاتین{Gateway} یا \متن‌لاتین{Concentrator} قرار دارد که در برخی از موارد هسته نامیده می‌شود.
هسته وظیفه پشتبانی از لایه‌های کاربر و کنترل را دارد. وظیفه تبدیل پروتکل‌های قابل فهم برای شبکه و برنامه‌های کاربردی را دارد.
\مرجع{Chaudhari2020}

به خاطر نزدیکی \متن‌لاتین{Gateway} به اشیا در برخی از موارد از آن برای پردازش در لبه در کاربردهای همزمان استفاده می‌گردد.
از سوی دیگر پردازش و ذخیره‌سازی در لبه می‌توان بار پردازش ابری را کاهش دهد. در برخی از تکنولوژی‌های
از \متن‌لاتین{Gateway} برای کنترل درخواست و اولویت‌دهی نیز استفاده می‌گردد.
\مرجع{Chaudhari2020}

\شروع{شکل}
\درج‌تصویر[width=\textwidth]{./img/lpwan-arch.png}
\تنظیم‌ازوسط
\برچسب{شکل: معماری معمول شبکه‌های LPWAN}
\شرح{معماری معمول شبکه‌های \متن‌لاتین{LPWAN} \مرجع{Chaudhari2020}}
\پایان{شکل}

در معماری مرسوم دستگاه به طور مستقیم با شبکه‌ی \متن‌لاتین{LPWAN} در ارتباط است.
تنظیمات دسترسی دیگری را نیز می‌توان در نظر گرفت. دو نمونه از معروف‌ترین آن‌ها
در ادامه آمده است.
معماری اول حالتی است که ارتباط اشیا از طریق تکنولوژی‌هایی مانند \متن‌لاتین{Zigbee}،
\متن‌لاتین{WiFi} و \نقاط‌خ فراهم می‌آید. \متن‌لاتین{Gateway} متناظر این اشیا از طریق
شبکه \متن‌لاتین{LPWAN} متصل می‌شود.
معماری دوم حالتی است که اشیا از چند شبکه \متن‌لاتین{LPWAN} به صورت همزمان استفاده می‌کنند.
\مرجع{Chaudhari2020}

\قسمت{زیرساخت‌های اینترنت اشیا}

حجم داده و پردازش مورد نیاز در اینترنت اشیا بسیار زیاد است و برای همین نیاز به زیرساختی گسترش وجود دارد.
برای پاسخ به این نیاز پردازش ابری با گسترش‌پذیری و ظرفیت بالا، بهترین گزینه است.
در پردازش ابری عموما برنامه‌ها به صورت مجازی‌سازی شده اجرا می‌شوند که این امر کارایی، امنیت، گسترش‌پذیری و کاهش هزینه
را نسبت به اجرای روی \متن‌لاتین{bare-metal} به ارمغان می‌آورد.
\مرجع{Botez2021}

استفاده از کانتینرها برای اجرای برنامه‌ها بر بسترهای ابری به جای استفاده از مجازی‌سازی می‌تواند گسترش‌پذیری
بیشتری را به همراه داشته باشد.
\مرجع{Botez2021}

در شبکه‌های \متن‌لاتین{5G} بحث استفاده از کارکردهای مجازی شبکه مطرح شده است و هدف حذف کارکردهای فیزیکی
و استفاده از پیاده‌سازی‌های نرم‌افزاری آن‌ها است. اما هنوز مشکلاتی وجود دارد،
این پیاده‌سازی‌ها نرم‌افزاری در قالب ماشین‌های مجازی سربار زیادی دارند
و به سادگی نمی‌توانند آن‌ها را گسترش داد، مدیریت کرد یا هماهنگ نمود.
از این روی \متن‌لاتین{Cloud-Native Network Functions}ها مطرح می‌شوند که پیاده‌سازی کارکردها به صورت ابرزی بوده
و می‌توان برای مدیریت آن‌ها به زیرساخت‌هایی چون \متن‌لاتین{Kubernetes} استفاده کرد که خود بسترهایی برای
گسترش خودکار، خطاپذیری و \نقاط‌خ را فراهم می‌آورد.
\مرجع{Botez2021}

\قسمت{ارتباطات و شبکه‌ها}
نیاز کاربردهای اینترنت اشیا روز به روز به تکنولوژی‌هایی که می‌توانند عملکرد توان پایین داشته باشند
و دستگاه‌های انتهایی که بتوانند ارتباط بی‌سیتم در مسافت‌های طولانی را با هزینه و پیچیدگی پایین برقرار کنند بیشتر می‌شود.
در بیشتر کاربردها، دستگاه‌های انتهایی اینترنت اشیا حسگرهایی با باتری می‌باشند، که پروفایل مصرف توان آن‌ها در جهت افزایش طول عمر
باتریشان می‌بایست با دقت طراحی شده باشد.
برد ارتباطی نیاز دارد از چند صد متر تا چندین کیلومتر را شامل شود چرا که دستگاه‌های ارتباطی در محیط عملیاتی بزرگی گسترده‌اند.
با نظر گرفتن همه ویژگی‌های نامبرده، این امر تنها با استفاده از تکنولوژی‌های حوزه شبکه‌های توان پایین با برد بالا\پانویس{LPWAN} ممکن است.
\مرجع{sensors-18-03995}

تکنولوژی‌های بسیاری در حوزه \متن‌لاتین{LPWAN} به بازار عرضه شده‌اند که از جمله‌ی آن‌ها می‌توان به \متن‌لاتین{SigFox}، \متن‌لاتین{NB-IoT}، \متن‌لاتین{LTE Cat-M} و \متن‌لاتین{LoRaWAN}
اشاره کرد.
\متن‌لاتین{SigFox} قصد دارد یک پوشش جهانی را در قالب یک اپراتور شبکه که در کشورهای مختلف با استفاده از شرکت‌های تابعه اجرا می‌شود، فراهم آورد.
این شبکه به صورت کامل از سخت‌افزار تا لایه شبکه در انحصار همین شرکت است و همکاری با آن تنها راه برای عملیاتی کردن این شبکه است.
\متن‌لاتین{NB-IoT} توسط شرکت‌های مخابراطی به عنوان یک جایگزین در ارتباطات اینترنت اشیا، نسبت به تکنولوژی‌های زیرگیگاهرتز \متن‌لاتین{LPWAN} ارائه می‌شود.
از آنجایی \متن‌لاتین{NB-IoT} در طیف فرکانسی دارای لایسنس فعالیت می‌کند، می‌تواند قابلیت اطمینان بیشتری در ترافیک نسبت به سایر تکنولوژی‌های زیرگیگاهرتز ارائه دهد.
برخلاف \متن‌لاتین{SigFox} و \متن‌لاتین{NB-IoT}، متن‌لاتین{LoRaWAN} قابلیت ارائه به صورت شبکه‌های خصوصی و ادغام آسان با پلتفرم‌های شبکه‌ای جهانی مانند \متن‌لاتین{The Things Network} را فراهم می‌آورد.
به همین دلیل و از سوی دیگر باز بودن استاندارد، \متن‌لاتین{LoRaWAN} توجه جامعه محققان را از اولین نمود خود در بازار جلب کرده است.
\مرجع{sensors-18-03995}
\مرجع{Mekki2019}

\begin{table}
\caption{مقایسه تکنولوژی‌های \متن‌لاتین{LPWAN} \مرجع{SanchezIborra2016} \مرجع{Mekki2019} \مرجع{Naik2018}}
\begin{latin}\begin{tabularx}
  {\textwidth}
  {|*{6}{X|}}
  \toprule
  &
  LoRaWAN &
  Sigfox &
  NB-IoT &
  Ingenu &
  Telensa \\
  \midrule
  Band &
  433/868/ 780/915 MHz &
  868/915 MHz &
  Cellular &
  2.4 GHz &
  868/915 MHz \\
  \midrule
  Data Rate &
  50 kbps &
  100 bps &
  200 kbps &
  19 kbps &
  346 Mbps \\
  \midrule
  Range &
  5 km &
  10 km &
  35 km &
  15 km &
  1 km \\
  \midrule
  Number of Channels &
  6 &
  333 &
  --- &
  --- &
  --- \\
  \midrule
  MAC &
  ALOHA &
  none &
  Non-Access Stratum &
  --- &
  --- \\
  \midrule
  Topology &
  Star-of-Stars &
  Star &
  Star &
  Star / Tree &
  Star / Tree \\
  \midrule
  Adaptive Data Rate &
  Yes &
  No &
  Yes &
  --- &
  --- \\
  \midrule
  Payload Length &
  256 B &
  12 B &
  1600 B &
  10 kB &
  65 kB \\
  \midrule
  Handover &
  No &
  No &
  Yes &
  --- &
  --- \\
  \midrule
  Authentication / Encryption &
  AES 128 &
  No &
  LTE Encryption &
  --- &
  --- \\
  \midrule
  Over the air update &
  --- &
  --- &
  --- &
  --- &
  --- \\
  \midrule
  Battery life &
  10Y+ &
  10Y+ &
  --- &
  --- &
  10Y+ \\
  Bi-Directional &
  Yes &
  Yes &
  Yes &
  Yes &
  Yes \\
  \bottomrule
\end{tabularx}\end{latin}
\end{table}

شبکه‌ها \متن‌لاتین{Sigfox} پهنای باند بسیار کمی داشته و محدودیت‌های زیادی برای اندازه بسته و تعداد بسته‌ها در نظر گرفته است. با توجه به مدل تجاری خاص آن که پیشتر
به آن پرداخته شد، توجه‌ها بیشتر به سوی \متن‌لاتین{LoRaWAN} معطوف شده است.
\مرجع{Adelantado2017}

مستقل از ارتباط رادیویی که برای شکل دادن شبکه‌ی \متن‌لاتین{M2M} از آن استفاده شده است، دستگاه انتهایی یا ماشین می‌بایست داده خود را از طریق اینرتنت قابل دسترسی کنند.
دستگاه اینترنت اشیا عموما منابع محدودی دارند و این به آن معناست که باید با حافظه، توان پردازشی، توان شبکه‌ای و باتری محدودی فعالیت کنند.
بنابراین کارایی ارتباط ماشین به ماشین وابستگی زیادی به پروتکل زیرین مورد استفاده در اپلیکشن اینرتنت اشیا دارد.
\مرجع{Mishra2021}

پروتکل‌های ارتباطی زیادی در حوزه اینترنت اشیا مطرح است که می‌توان از بین آن‌ها به \متن‌لاتین{MQTT}، \متن‌لاتین{CoAP}، \متن‌لاتین{AMQP} و \متن‌لاتین{HTTP} اشاره کرد.
\مرجع{Mishra2021}

\زیرقسمت{\متن‌لاتین{NB-IoT}}

همانطور که اشاره شد شبکه \متن‌لاتین{NB-IoT} از باند دارای لایسنس استفاده می‌کند. سه مد عملیاتی برای شبکه‌های \متن‌لاتین{NB-IoT} وجود دارد.
در مد اول یا \متن‌لاتین{Stand-Alone} از باندهای فرکانسی \متن‌لاتین{GSM} استفاده می‌شود.
در مد دوم یا \متن‌لاتین{Guard-Band} از منابع استفاده نشده در باند محافظ \متن‌لاتین{LTE} استفاده می‌شود.
در مد سوم یا \متن‌لاتین{In-Band} از منابع داخلی \متن‌لاتین{LTE} استفاده می‌شود.
\مرجع{Mekki2019}

نودها در شبکه‌ی \متن‌لاتین{NB-IoT} می‌توانند دو حالت \متن‌لاتین{eDRX} و \متن‌لاتین{PSM} را برای صرفه‌جویی انتخاب کنند.
در مد \متن‌لاتین{eDRX} دستگاه برای مدت تا ۱۷۵ دقیقه مودم خود را خاموش می‌کند.
در \متن‌لاتین{DRX} که پیشتر هم در شبکه‌های سلولی وجود داشته است همین رویه برای بازه‌ی کوتاه ۲.۵۶ ثانیه خاموش می‌شده است
و تفاوت \متن‌لاتین{eDRX} در همین مدت زمان است.
در نظر داشته باشید که مساله زمان از این جهت مطرح است که در صورت خاموش بودن مودم پاسخ \متن‌لاتین{downlink} با تاخیر مواجه می‌شود.
در روش \متن‌لاتین{PSM} مودم برای مدتهای طولانی مانند چندین ماه خاموش می‌شود.
این حالت برای سنسورهایی که صرفا در شرایط مشخصی \متن‌لاتین{uplink} دارند، کاربرد دارد.
\مرجع{Lee2017}

\زیرقسمت{\متن‌لاتین{LoRa}}

لایه‌ی فیزیکی \متن‌لاتین{LoRa} که در \متن‌لاتین{LoRaWAN} استفاده می‌شود، در سال ۲۰۱۴ توسط \متن‌لاتین{Semtech} به ثبت رسید
و بنابراین برای بررسی‌ها کاملا باز نیست. مطالبی که در ادامه می‌آید بخشی بر اساس قسمت‌های باز استاندارد و بخشی بر اساس آزمایش‌های
تجربی بدست آمده‌اند.
از ویژگی‌های \متن‌لاتین{LoRa} می‌توان به توان عملیاتی پایین، نرخ پایین داده و برند ارتباطی بالا اشاره کرد.
\مرجع{sensors-18-03995}
\مرجع{Adelantado2017}

از سال ۲۰۱۵ جامعه تحقیقاتی شروع به مطالعه در رابطه با کارآیی و ویژگی‌های مختلف تکنولوژی‌های \متن‌لاتین{LoRa} و \متن‌لاتین{LoRaWAN} کرد.
از آن تاریخ مقلات متعددی در ژورنال‌ها و کنفرانس‌های عملی در سراسر دنیا چاپ و ارائه شده‌اند.
\مرجع{sensors-18-03995}

ماژولیشن آن بر پایه \متن‌لاتین{Chirp Spread Spectrum} بوده و به صورت دوره‌ای سیگنال‌های \متن‌لاتین{chirp}ای تولید می‌کنند که همه آن‌ها بازه زمانی یکسانی دارند.
\متن‌لاتین{chirp} یک سیگنال سینوسی است که فرکانس آن با زمان به صورت خطی افزایش یا کاهش پیدا می‌کند.
یک \متن‌لاتین{chirp} به وسیله‌ی پروفایل زمانی فرکانس لحظه‌ی آن که در بازه‌ی زمانی \متن‌لاتین{T} از فرکانس $f_0$ به فرکانس $f_1$
تغییر می‌کند شناخته می‌شود.
در \متن‌لاتین{LoRa} دو نوع \متن‌لاتین{chirp} تعریف شده است. \متن‌لاتین{chirp} پایه که فرکانس پروفایل زمانی آن با فرکانس مینیمال
\(f_{\min} = -\frac{BW}{2}\)
شروع شده و با فرکانس ماکسیمال
\(f_{\max} = \frac{BW}{2}\)
خاتمه می‌یابد.
برای ورودی‌های دیجیتال مختلف، یک ماژولاتور \متن‌لاتین{chirp}های مختلفی تولید می‌کند که نسبت به \متن‌لاتین{chirp} پایه شیف زمانی خورده‌اند.
\مرجع{sensors-18-03995}

\شروع{شکل}
\درج‌تصویر[width=\textwidth]{./img/lora-mod.png}
\تنظیم‌ازوسط
\شرح{ماژولیشن \متن‌لاتین{LoRa}}
\پایان{شکل}

\شروع{شکل}
\درج‌تصویر[width=\textwidth]{./img/lora-chirp-sf.png}
\تنظیم‌ازوسط
\شرح{\متن‌لاتین{chirp}های پایه}
\پایان{شکل}

\متن‌لاتین{LoRa} از باند فرکانسی بدون مجوز استفاده می‌کند بنابراین برای راه‌اندازی شبکه‌ی آن نیاز به تهیه هیچ مجوزی نیست. البته باید در نظر داشته که نرخ پیام در این باندهای بدون مجوز توسط قانون‌گذاران محدود شده است.
یکی از محدودیت‌های مهم در شبکه‌های \متن‌لاتین{LoRa} محدودیت \متن‌لاتین{Duty Cycle} است که استفاده از کانال را محدود می‌کند. این محدودیت بیان می‌کند برای استفاده از کانال به مدت $T_{a}$ شی می‌بایست
حداقل به اندازه $T_{s}$ که از رابطه \رجوع{معادله: چرخه وظیفه} بدست می‌آید، ارسالی نداشته باشد.
\مرجع{Cruz2021}
\مرجع{Adelantado2017}

\begin{align}
  \label{معادله: چرخه وظیفه}
  T_{s} = T_{a}\left( \frac{1}{d} - 1 \right)
\end{align}

لایه فیزیکی \متن‌لاتین{LoRa} با توجه به ویژگی‌های گسترده‌ای که دارد در راهکارهای دیگری به جز \متن‌لاتین{LoRaWAN} نیز استفاده شده است که از جمله‌ی آن می‌توان به \متن‌لاتین{Meshed LoRa} اشاره کرد.
\مرجع{Beltramelli2021}

پارامترهای فاکتور گسترش یا به اختصار \متن‌لاتین{SF}، پهنای باند و نرخ‌کدگذاری قابل تنظیم می‌باشند و می‌توانند روی زمان ارسال بسته، نرخ ارسال، مصرف انرژی و برد ارتباطی تاثیر داشته باشند.
در ادامه به مرور این پارامترها و تاثیرشان می‌پردازیم.

به صورت غیر رسمی فاکتور گسترش لگاریتم مبنای ۲ تعداد \متن‌لاتین{chirp}ها در هر علامت است. مقدار فاکتور گسترش بین ۷ تا ۱۲ است.
با افزایش فاکتور گسترش پوشش‌دهی بیشتر می‌شود اما بهای آن کاهش نرخ بیت و افزایش زمان ارسال\پانویس{Time on Air} است (معادله \رجوع{معادله: نرخ داده در LoRa}).
\مرجع{Augustin2016}

در بسته‌های \متن‌لاتین{LoRa} از تصحیح خطا جلورونده یا مختصرا \متن‌لاتین{FEC} استفاده می‌شود.
در این فرآیند بیت‌های تصحیح خطا به داده‌های ارسال اضافه می‌شوند.
این بیت‌های اضافه شده کمک می‌کنند تا داده‌های از دست رفته به خاطر تداخل بازگردانی شوند.
بیت‌های بیشتر این پروسه بازگردانی را ساده‌تر می‌کنند اما باعث هدر رفت پهنای باند و عمر باتری می‌شوند.
در \متن‌لاتین{LoRa} ما نرخ‌های کدگذاری $4/5$، $4/6$، $4/7$ و $4/8$ را داریم.

\begin{table}
\caption{توانایی \متن‌لاتین{LoRa} در تشخیص و تصحیح خطا \مرجع{Pham2020}}
\begin{latin}\begin{tabularx}
  {\textwidth}
  {|*{3}{X|}}
  \toprule
  Coding rates &
  Error detection (bits) &
  Error correction (bits) \\
  \midrule
  $4/5$ &
  0 &
  0 \\
  \midrule
  $4/6$ &
  1 &
  0 \\
  \midrule
  $4/7$ &
  2 &
  1 \\
  \midrule
  $4/8$ &
  3 &
  1 \\
  \bottomrule
\end{tabularx}\end{latin}
\end{table}

پهنانی باند در \متن‌لاتین{LoRa} می‌توان بین ۱۲۵ تا ۵۰۰ کیلوهرتز باشد و با توجه به استفاده از باند بدون لایسنس این پهنای باند وابسته به پارامتر‌های منطقه‌ای و فاکتور گسترش می‌باشد.
به طور مثال در باند فرکانسی ۸۶۸ مگاهرتز ۸ کانال متفاوت وجود دارد که ۷ کانال ابتدایی تنها با پهنای باند ۱۲۵ کیلوهرتز کار می‌کنند و کانال آخر می‌تواند با پهنای باند‌های
۱۲۵، ۲۵۰ و ۵۰۰ کیلوهرتز کار کند.

\شروع{شکل}
\درج‌تصویر[width=\textwidth]{./img/lora-868-channels.jpg}
\تنظیم‌ازوسط
\شرح{کانال‌های \متن‌لاتین{LoRa} در باند فرکانسی ۸۶۸ مگاهرتز}
\پایان{شکل}

در متن‌لاتین{LoRa} نرخ باد یا نرخ علائم از رابطه‌ی زیر محاسبه می‌گردد:

\begin{align}
  \label{معادله: نرخ باد یا علائم در LoRa}
  R_{s} = BW / 2^{SF}
\end{align}

که در آن \متن‌لاتین{BW} پهنای باند و \متن‌لاتین{SF} فاکتور گسترش می‌باشد.
\مرجع{Augustin2016}

در ادامه نرخ داده‌ی ارسالی را می‌توان با استفاده از رابطه زیر محاسبه کرد:

\begin{align}
  \label{معادله: نرخ داده در LoRa}
  R_{b} = SF \times \frac{BW}{2^{SF}} \times CR
\end{align}

در این رابطه \متن‌لاتین{CR} نرخ کدگذاری، \متن‌لاتین{SF} فاکتور گسترش و \متن‌لاتین{BW} پهنای باند می‌باشد.
\مرجع{Augustin2016}

\شروع{شکل}
\درج‌تصویر[width=.5\textwidth]{./img/lora-packet.png}
\تنظیم‌ازوسط
\شرح{ساختار بسته \متن‌لاتین{LoRa} \مرجع{Augustin2016}}
\پایان{شکل}

رابطه زیر مشخص می‌کند برای ارسال یک داده به چه تعداد علامت نیاز داریم. این پارامتر با $n_{s}$ نمایش داده می‌شود.

\begin{align}
  \label{معادله: تعداد علائم مورد نیاز در LoRa}
  n_{s} = 8 + \max\left( \left\lceil \frac{8PL - 4SF + 8 + CRC + H}{4 \times (SF - DE)} \right\rceil \times \frac{4}{CR}, 0 \right)
\end{align}

در این رابطه در صورت فعال بودن \متن‌لاتین{CRC} مقدار آن برابر ۱۶ و در غیر این صورت برابر صفر است.
\متن‌لاتین{CR} نرخ کدگذاری،
\متن‌لاتین{PL} اندازه داده،
\متن‌لاتین{SF} فاکتور گسترش است.
در این رابطه \متن‌لاتین{H} اندازه سرآیند بوده که در صورت فعال بودن برابر ۲۰ و در غیر این صورت صفر است.
در این رابطه \متن‌لاتین{DE} در صورت فعال بودن حالت نرخ داده پایین یا \متن‌لاتین{low data rate} برابر ۲ و در غیر این صورت برابر صفر است.
\مرجع{Augustin2016}
\مرجع{Pham2020}

همانطور که محاسبات دیده می‌شود استفاده از مقدارهای بالاتر برای فاکتور گسترش زمان ارسال را بیشتر کرده و تاثیر \متن‌لاتین{Duty Cycle} را بیشتر می‌کند.
و پژوهش \مرجع{Adelantado2017} بیان می‌کند احتمال استفاده از فاکتورهای گسترش بالاتر بیشتر است.

یکی از تکنیک‌ها در شبکه‌های بی‌سیم استفاده از \متن‌لاتین{Frequency Hopping} است. در این تکنیک با هماهنگی در میان ارسال کننده و گیرنده فرکانس‌های ارسال در زمان
تغییر می‌کند. پیاده‌سازی این شیوه در شبکه‌های \متن‌لاتین{LoRa} در قالب \متن‌لاتین{LR-FHSS} یا
\متن‌لاتین{Frequency Hopping Spread Spectrum}
صورت می‌پذیرد. در این روش هر کانال به تعدادی زیرکانال شکسته شده و سرآینده بسته روی همه این زیرکانال‌ها ارسال می‌شود.
خود داده اما قطعه قطعه شده و هر قطعه به وسیله‌ی یک زیرکانال ارسال می‌گردد.
از آنجایی که \متن‌لاتین{Gateway} روی همه‌ی این کانال‌ها گوش می‌دهد می‌تواند بسته را دوباره بازسازی کند.

\زیرقسمت{\متن‌لاتین{LoRaWAN}}

\متن‌لاتین{LoRaWAN} پروتکل لایه لینک و شبکه می‌باشد که شامل پروتکل کنترل دسترسی چندگانه\پانویس{MAC} نیز می‌باشد. این پروتکل اجازه می‌دهد تا دستگاه‌های \متن‌لاتین{LoRa} با برنامه‌های کاربردی ارتباط برقرار کنند.
این پروتکل توسط \متن‌لاتین{LoRa Alliance} توسعه پیدا کرده و برای همگان قابل استفاده است.
این پروتکل برای ارتباط دستگاه به دستگاه ایجاد نشده است و تنها هدف آن ارتباط اشیا با \متن‌لاتین{Gateway} و \متن‌لاتین{Network Server} است.
در صورت نیاز به ارتباط بین دستگاه‌ها می‌بایست از \متن‌لاتین{Gateway} و \متن‌لاتین{Network Server} استفاده کرد یا اینکه
تنها لایه‌ی فیزیکی \متن‌لاتین{LoRa} را مورد استفاده قرار داد.
\مرجع{Cruz2021}
\مرجع{Augustin2016}

یک شبکه‌ی \متن‌لاتین{LoRaWAN} در ساده‌ترین شکل از اجزای زیر تشکیل شده است:

\شروع{شمارش}
\فقره یک دستگاه حسگر یا عملگر که توان و محاسبات محدودی دارد.
\فقره یک \متن‌لاتین{Gateway} که عنصر شبکه‌ای برای دریافت و ارسال اطلاعات از و به دستگاه‌ها را برعهده دارد.
\فقره سرور شبکه که پیام‌های دریافت شده از یک مجموعه \متن‌لاتین{Gateway}ها را به برنامه‌های کاربردی می‌رساند و برعکس
\فقره برنامه کاربردی که می‌تواند در بستر اینترنت قرار داشته باشد و داده‌ها را از طریق سرور شبکه برای اشیا ارسال و دریافت کند.
\پایان{شمارش}

\شروع{شکل}
\درج‌تصویر[width=\textwidth]{./img/nrm-home.png}
\تنظیم‌ازوسط
\شرح{مدل مرجع شبکه \متن‌لاتین{LoRaWAN} - شبکه‌ی خانگی}
\پایان{شکل}

\شروع{شکل}
\درج‌تصویر[width=\textwidth]{./img/nrm-roaming.png}
\تنظیم‌ازوسط
\شرح{مدل مرجع شبکه \متن‌لاتین{LoRaWAN} - شبکه‌ی فراگرد}
\پایان{شکل}

\شروع{شکل}
\درج‌تصویر[width=\textwidth]{./img/lora-architecture-osi.png}
\تنظیم‌ازوسط
\شرح{معماری شبکه \متن‌لاتین{LoRaWAN} از نگاه مدل لایه‌ای \متن‌لاتین{OSI} \مرجع{Ertrk2019}}
\پایان{شکل}

در حوزه امنیت \متن‌لاتین{LoRaWAN} دولایه از امنیت را تعریف می‌کند. لایه اول امنیت میان شی و شبکه است در حالی که لایه دوم میان شی و برنامه کاربردی می‌باشد.
به این صورت می‌توان مطمئن شد که تنها برنامه کاربردی است که می‌تواند داده‌های ارسالی توسط دستگاه را رمزگشایی کند.
\مرجع{Cruz2021}

در ضمن \متن‌لاتین{LoRaWAN} ویژگی‌های دیگری مانند نرخ داده تطبیقی\پانویس{ADR} را اضافه می‌کند. در نرخ داده تطبیقی شبکه با دستگاه در رابطه با پارامترهای لایه‌ی فیزیکی \متن‌لاتین{LoRa} مذاکره می‌کند
که در نتجیه آن کارآیی مصرف بهینه می‌شود. شکل \رجوع{شکل:لایه‌های لورا} مدل لایه‌ای \متن‌لاتین{LoRa} و \متن‌لاتین{LoRaWAN} را نمایش می‌دهد.
\مرجع{Cruz2021}

\شروع{شکل}
\درج‌تصویر[width=\textwidth]{./img/lora-layers.png}
\تنظیم‌ازوسط
\برچسب{شکل:لایه‌های لورا}
\شرح{مدل لایه‌ای \متن‌لاتین{LoRa} و \متن‌لاتین{LoRaWAN} \مرجع{Cruz2021}}
\پایان{شکل}

در شبکه‌های \متن‌لاتین{LoRaWAN} سه کلاس کاری می‌توان برای اشیا در نظر گرفت.

\شروع{فقرات}
\فقره در کلاس ($A$ یا \متن‌لاتین{All}) شی هر زمان که به خواهد شروع به ارسال داده کرده و دو پریود متوالی آینده را برای دریافت \متن‌لاتین{Downlink} خواهد داشت. این کلاس پایین‌ترین مصرف انرژی را دارد چرا که شی تنها در زمان‌هایی که لازم است
روشن می‌شود و می‌تواند دوباره خاموش شود. با توجه به ساختار دریافت \متن‌لاتین{Downlink} در این کلاس می‌توان به سادگی برای پیام‌ها \متن‌لاتین{Ack} دریافت کرد اما برای سایر پیام‌های \متن‌لاتین{Downlink} می‌بایست تا تصمیم شی برای ارسال داده صبر کرد.
\فقره در کلاس ($B$ یا \متن‌لاتین{Beacon}) نود به صورت همگام می‌تواند \متن‌لاتین{Downlink} دریافت کند برای اینکار نود به جز دو بازه دریافت که در کلاس $A$ تعریف شده بود یک بازه دریافت قابل پیش‌بینی نیز دارد.
این کلاس مصرف بالاتری دارد چرا که نیاز است یک بازه دوره‌ای برای دریافت \متن‌لاتین{Downlink} روشن شود. مزیت این کلاس قابلیت دریافت \متن‌لاتین{Downlink} حتی در زمان‌هایی که ارسالی ندارد می‌باشد.
\فقره در کلاس ($C$ یا \متن‌لاتین{Continues}) بیشترین مصرف توان را داشته و شی در هر زمان می‌تواند داده دریافت کند.
\پایان{فقرات}

پیشتر به ساختار بسته‌ها در لایه‌ی فیزیکی \متن‌لاتین{LoRa} پرداختیم. در \متن‌لاتین{LoRaWAN} برای پیام‌های \متن‌لاتین{uplink} استفاده از سرآیند و \متن‌لاتین{CRC} اجباری است
و نمی‌توان در \متن‌لاتین{LoRaWAN} از فاکتور گسترش ۶ استفاده کرد.
نرخ کدگذاری در \متن‌لاتین{LoRaWAN} مشخص نشده و قابل تغییر است.
\مرجع{Augustin2016}

\شروع{شکل}
\درج‌تصویر[width=\textwidth]{./img/lorawan-packet.png}
\تنظیم‌ازوسط
\برچسب{شکل:بسته‌های LoRaWAN}
\شرح{بسته‌های پروتکل \متن‌لاتین{LoRaWAN} \مرجع{Augustin2016}}
\پایان{شکل}

ساختار پیام‌های \متن‌لاتین{LoRaWAN} در شکل رجوع{شکل:بسته‌های LoRaWAN} آورده شده است. \متن‌لاتین{DevAddr} آدرس دستگاه است.
\متن‌لاتین{FPort} پورت برای مالتی‌پلکس است و در صورتی که مقدار آن صفر باشد نشان‌دهنده‌ی این موضوع است که بسته تنها شامل
دستورات لایه‌ی \متن‌لاتین{MAC} است.
\مرجع{Augustin2016}

در \متن‌لاتین{LoRaWAN} از الگوریتم کنترل دسترسی همزمان \متن‌لاتین{ALOHA} استفاده میشود. در این الگوریتم بسته‌ها می‌توانند اندازه‌های متغیر داشته باشند، نودها هر زمان که قصد داشته باشند داده ارسال کنند و نیازی به همگام‌سازی زمانی ندارد.
مشکل اصلی این روش در تعداد برخوردهای بالا در زمانی است که شبکه نودهای زیادی دارد، از این رو روش‌های ``گوش‌دادن پیش از حرف‌زدن'' مانند \متن‌لاتین{CSMA/CA} کارایی بهتری دارند.
چنین الگوریتم‌هایی در مقابل نیاز به همگام‌سازی دارند به این معنی که می‌بایست اشیا یک \متن‌لاتین{clock} محلی مشترک داشته باشند.
\مرجع{Beltramelli2021}

به صورت کلی می‌توان الگوریتم‌های دسترسی همزمان را در سه گروه اصلی قرار داد:

\شروع{فقرات}
\فقره الگوریتم‌های قطعه‌بندی کانال
\فقره الگوریتم‌های دسترسی تصادفی
\فقره الگوریتم‌های نوبت‌دهی
\پایان{فقرات}

که الگوریتم \متن‌لاتین{ALOHA} یا \متن‌لاتین{CSMA} در دسته الگوریتم‌های دسترسی تصادفی است. روش \متن‌لاتین{ALOHA} خود به دو گونه \متن‌لاتین{Pure ALOHA} و \متن‌لاتین{Slotted ALOHA}
قابل پیاده‌سازی است. در روش \متن‌لاتین{Pure ALOHA} که در \متن‌لاتین{LoRaWAN} نیز استفاده می‌شود، نودها هر زمان که داده‌ای داشته باشند می‌توانند آن را ارسال کنند و در صورت رخ دادن تصادم
با احتمال $p$ داده را باز ارسال می‌کنند. در روش \متن‌لاتین{Slotted ALOHA} نیاز است که نودها با یکدیگری همگام بوده و در ابتدای بازه‌های مشخصی شروع به ارسال کنند و در صورت
وقوع تصادم بعد از صبر کردن در ابتدای بازه‌ی بعدی با احتمال $p$ باز ارسال را شروع می‌کنند. کارایی روش \متن‌لاتین{Slotted ALOHA} از روش \متن‌لاتین{ALOHA} بیشتر بوده است ولی نیاز به همگام‌سازی نودها دارد
که خود موضوعی هزینه‌بر است. پژوهش‌های متعددی روش‌های همگام‌سازی برای \متن‌لاتین{LoRaWAN} در جهت استفاده از \متن‌لاتین{Slotted ALOHA} پیشنهاد داده‌اند.
می‌توان نشان داد کارایی پروتکل \متن‌لاتین{ALOHA} برابر $\frac{1}{2e}$ است و از سوی دیگر کارایی پروتکل \متن‌لاتین{Slotted ALOHA} تقریبا دو برابر بوده و برابر $\frac{1}{e}$ است.

ارسال اطلاعات همگام‌سازی در صورتی که از طریق همان بستر بی‌سیم رخ بدهد به آن همگام‌سازی داخل باند گفته می‌شود. در این روش نیاز است که به محدودیت‌های بستر رادیویی احترام گذاشت.
در این شیوه ممکن است کیفیت ارتباط رادیویی به علت ارسال همین اطلاعات به میزان غیرقابل قبولی افت کند. برای حل این مشکل می‌توان از راه‌حل‌های خارج از باند استفاده کرد.
\مرجع{Beltramelli2021}

پروتکل \متن‌لاتین{LoRaWAN} یک پروتکل تک گام است و از مسیریابی یا ارسال چند گامه پیام پشتیبانی نمی‌کند. یکی از مهم‌ترین ویژگی‌های این پروتکل نرخ داده تطبیق‌پذیر و تشخیص
موقعیت مکانی بدون نیاز به \متن‌لاتین{GPS} و تنها با ۳ \متن‌لاتین{Gateway} است.
\مرجع{Ertrk2019}

اولین نسخه از استاندارد \متن‌لاتین{LoRaWAN} در سال ۲۰۱۵ منتشر شد. در طی این سال‌ها بهبودهایی در آن حاصل شد که مهمترین نسخه‌های آن ۱.۰.۳ و ۱.۱ است.
\مرجع{Ertrk2019}

\شروع{شکل}
\درج‌تصویر[width=\textwidth]{./img/lorawan-gps.png}
\تنظیم‌ازوسط
\شرح{تشخیص موقعیت مکانی در \متن‌لاتین{LoRaWAN} \مرجع{Ertrk2019}}
\پایان{شکل}

\زیرقسمت{\متن‌لاتین{WiFi 7}}

اندکی پس از انتشار \متن‌لاتین{WiFi 6} کارگروه \متن‌لاتین{IEEE 802.11} به همراه \متن‌لاتین{WiFi Alliance} شروع به طراحی نسل بعدی آن در شبکه‌های بی‌سیم محلی با نام \متن‌لاتین{WiFi 7} کردند.
یکی از اجزای \متن‌لاتین{WiFi 7}، \متن‌لاتین{IEEE 802.11be} می‌باشد و قرار است در این نسل از \متن‌لاتین{Time-Sensitive Networking} یا \متن‌لاتین{TSN} برای ارتباط‌هایی با تاخیر کم و قابلیت
اطمینان بالا پشتیبانی شود.
\مرجع{Adame2021}

\متن‌لاتین{TSN} در ابتدا برای شبکه‌ها اترنت (\متن‌لاتین{IEEE 802.3}) طراحی شده بود اما به آرامی راه خود را به شبکه‌های بی‌سیم باز می‌کند. در \متن‌لاتین{TSN} سعی می‌شود
هیچ بسته‌ای به خاطر ازدحام بافرها از دست نرود، بسته‌های کمی در خرابی تجهیزات از دست بروند و تاخیر انتها به انتها گارانتی شده باشد.
کارگروه \متن‌لاتین{IEEE P802.11be} برای طراحی لایه \متن‌لاتین{MAC} و \متن‌لاتین{PHY} در می ۲۰۱۹ شکل گرفت. یکی از اهداف \متن‌لاتین{WiFi 7} کاهش بدترین حالت تاخیر و \متن‌لاتین{Jitter} می‌باشد
که برای آن، کارگروه در حال بررسی استانداردهای \متن‌لاتین{TSN} می‌باشد.
\مرجع{Adame2021}

با وجود اینکه هرگز \متن‌لاتین{WiFi} نخواهد توانست تاخیر محدودی را با توجه به ماهیت خود در استفاده از باندهای فرکانسی بدون مجوز، ارائه دهد اما استفاده از مفاهیم \متن‌لاتین{TSN}
می‌تواند آن را در زمره فناوری‌های پیشرو در \متن‌لاتین{6G} قرار دهد.
\مرجع{Adame2021}

به صورت سنتی \متن‌لاتین{WiFi} برای مدیریت دسترسی همزمان از \متن‌لاتین{Distributed Coordination Function} یا مختصرا \متن‌لاتین{DCF} استفاده می‌کند.
این شیوه بر پایه حس حامل و عقب‌نشینی نمایی عمل می‌کند. از مشکلات اصلی آن می‌توان به عدم قابلیت برای اولویت‌دهی ترافیک و از سوی دیگر غیرقابل پیش‌بینی بودن
آن اشاره کرد. در واقع در \متن‌لاتین{DCF} چند ایستگاه می‌توانند باعث اشباع شدن کانل شده و بنابراین نمی‌توان گارانتی از نظر زمانی برای داده‌ها ارائه داد.
\مرجع{Adame2021}

برای حل این مشکل روش \متن‌لاتین{EDCF} یا \متن‌لاتین{Enhanced DCF} در \متن‌لاتین{IEEE 802.11e} پیشنهاد شد. در این روش امکان اولویت‌دهی بر پایه
کاتالوگ‌های دسترسی اضافه شد. در ادامه این شیوده در \متن‌لاتین{IEEE 802.11aa} برای ارتباطات صدا و تصویر بهبود بیشتری یافت.
با این حال هیچ یک از این استانداردها کیفیت سرویس را در شرایطی که \متن‌لاتین{WiFi} دارای بار اضافه است، گارانتی نمی‌کنند.
\مرجع{Adame2021}

در لایه انتقال وجود بافر در پروتکل \متن‌لاتین{TCP} باعث تاخیرهای زیادی می‌شود و این امر کار برای انتقال جریان‌های ترافیکی \متن‌لاتین{TCP}
با استانداردهای \متن‌لاتین{TSN} سخت می‌کند. از سوی دیگر تکنیک‌های شبکه‌های سیمی مانند روش‌های نوین مدیریت صف و \نقاط‌خ در اینجا
کارایی زیادی ندارد.
\مرجع{Adame2021}

در استاندارد \متن‌لاتین{IEEE 802.11be} حالت عملیاتی چند کاناله وجود دارد. با استفاده از این حالت امکان افزایش بهره‌وری با ارسال همزمان
روی چند کانال به وجود می‌آید و از سوی دیگر می‌توان یک بسته یکسان را در چند کانال ارسال کرده تا از رسیدن آن مطمئن شد. در نهایت ارسال‌کننده
می‌تواند کانال با تاخیر کمتر را انتخاب کرده و تاخیر را کاهش دهد. این حالت عملیاتی خود می‌تواند در دو حالت همزمان و غیرهمزمان استفاده شود.
در حالت همزمان بعد از ارسال از کانال اصلی یک مدتی صبر شده و بعد می‌توان از کانال ثانویه استفاده کرد این در حالتی است که در حالت غیرهمزمان
هر دو کانال می‌توانند همزمان استفاده شوند ولی امکان تداخل میان آن‌ها وجود دارد.
\مرجع{Adame2021}

\قسمت{پروتکل‌های ارتباطی}

\زیرقسمت{\متن‌لاتین{MQTT}}

پروتکل‌های اینترنت اشیا امروزه قلب اصلی ارتباط‌های ماشین به ماشین (\متن‌لاتین{M2M}) را تشکیل می‌دهند. فارغ از تکنولوژی رادیویی که برای پیاده‌سازی شبکه‌ی اینتنرت اشیا و ماشین به ماشین استفاده می‌شود، همه داده‌هایی که توسط سنسورها و عملگرهای اینترنت اشیا
تولید می‌شوند وابستگی زیادی به پروتکل ارتباطی که برای ارتباط ماشین به ماشین در اپلیکیشن اینترنت اشیا استفاده شده است، دارند.
با افزایش تقاضا برای سرویس‌های مبتنی بر اینترنت اشیا، نیاز برای کاهش توان دستگاه‌ها و سرویس‌های اینترنت اشیا نیز در جهت محیط زیست پایدار برای نسل‌های آینده، افزایش یافته است.

پروتکل \متن‌لاتین{Messaging Queue Telemetry Transport} که مختصرا \متن‌لاتین{MQTT} نامیده می‌شود یکی از پروتکل‌ها پر استفاده در اینترنت اشیا می‌باشد.
این پروتکل یک پروتکل با معماری انتشار و اشتراک است که توان مصرفی پایینی دارد.
\متن‌لاتین{MQTT} یک پروتکل لایه کاربرد است که برای لایه انتقال از \متن‌لاتین{TCP/IP} و پورت‌های ۱۸۸۳ و ۸۸۸۳ (به ترتیب برای ارتباط رمز شده و ارتباط رمز نشده) استفاده می‌کند. البته پژوهش‌هایی چون \مرجع{Fernndez2021} سعی در تغییر لایه انتقال به \متن‌لاتین{UDP/Quic} داشته‌اند.
\مرجع{Mishra2021}

سه نقش در معماری \متن‌لاتین{MQTT} تعریف شده است. نقش اول کلاینت تولید کننده داده می‌باشد که به آن \متن‌لاتین{Producer} گفته می‌شود. نقش دوم سرور دلال پیام می‌باشد و نقش سوم کلاینت دریافت کننده داده است که به آن \متن‌لاتین{Subscriber} گفته می‌شود.
از \متن‌لاتین{Topic} برای مشخص کردن جریان‌های داده‌ای استفاده می‌شود و در تولید کننده داده می‌بایست برای داده‌ی خود \متن‌لاتین{Topic} داشته باشد و هر دریافت کننده داده روی \متن‌لاتین{Topic} خاصی مشترک می‌شود.
این \متن‌لاتین{Topic}ها می‌توانند به صورت سلسله مراتبی نیز می‌تواند تشکیل شود.
\مرجع{Mishra2021}

در پروتکل \متن‌لاتین{‌MQTT} سه سطح مختلف از کیفیت سرویس تعریف می‌شود. در کیفیت سرویس \متن‌لاتین{QoS0} پیام‌ها به صورت ارسال و فراموش کردن ارسال می‌شوند و هیچ تضمینی برای موفیت این ارسال وجود ندارد.
در کیفیت سرویس \متن‌لاتین{QoS1} پیام‌ها حداقل یکبار تحویل داده خواهند شد، در این حالت کلاینت ارسال‌کننده پس از ارسال منتظر پیام \متن‌لاتین{PUBACK} می‌ماند و در صورت عدم دریافت آن فرآیند ارسال را دوباره تکرار می‌کند.
در کیفیت سرویس \متن‌لاتین{QoS2} پیام‌ها دقیقا یکبار تحویل داده می‌شوند، این بالاترین کیفیت سرویس بوده و منابع زیادی را مصرف می‌کند.
\مرجع{Mishra2021}

در نظر داشته باشید که کیفیت سرویس پروتکل \متن‌لاتین{MQTT} به صورت انتها به انتها نیست و به ارتباط میان دلال پیام و کلاینت‌ها وابسته است.
کیفیت سرویس پیام دریافت شده از سوی \متن‌لاتین{Subscriber} وابسته کیفیت سرویس عملیات انتشار و اشتراک است. اگر کلاینت الف عملیات انتشار را با کیفیت سرویس بالاتری
نسبت به عملیات اشتراک در کلاینت ب انجام دهد کیفیت سرویسی که سرور پیام را به دست کلاینت ب می‌رساند کیفیت سرویسی است که کلاینت ب در عملیات اشتراک استفاده کرده است.
اگر کلاینت الف عملیات انتشار را با کیفیت سرویس پایین‌تری نسبت به عملیات اشتراک در کلاینت ب انجام دهد کیفیت سرویسی که سرور پیام را به دست کلاینت ب می‌رساند کیفیت سرویسی است که
کلاینت الف در عملیات انتشار استفاده کرده است.
\مرجع{MQTTQoS}

\شروع{لوح}
\شرح{چگونگی محاسبه کیفیت سرویس پیام دریافتی در پروتکل \متن‌لاتین{MQTT}\مرجع{MQTTQoS}}
\فضای‌و{5mm}
\begin{tabularx}{\textwidth}{|X|X|X|}
\خط‌پر
کیفیت سرویس عملیات انتشار & کیفیت سرویس عملیات اشتراک & کیفیت سرویس پیام دریافتی \\
\خط‌پر
۰ & ۰ & ۰ \\
\خط‌پر
۰ & ۱ & ۰ \\
\خط‌پر
۰ & ۲ & ۰ \\
\خط‌پر
۱ & ۰ & ۰ \\
\خط‌پر
۱ & ۱ & ۱ \\
\خط‌پر
۱ & ۲ & ۱ \\
\خط‌پر
۲ & ۰ & ۰ \\
\خط‌پر
۲ & ۱ & ۱ \\
\خط‌پر
۲ & ۲ & ۲ \\
\خط‌پر
\end{tabularx}
\پایان{لوح}

\شروع{شکل}
\تنظیم‌ازوسط
\درج‌تصویر[width=\textwidth]{./img/mqtt-qos.png}
\فضای‌و{5mm}
\شرح{سطوح مختلف کیفیت سرویس در پروتکل \متن‌لاتین{MQTT}\مرجع{Mishra2021}}
\پایان{شکل}

به صورت کلی می‌توان دلال‌های پیام در پروتکل \متن‌لاتین{MQTT} را به دو دسته کلی تقسم کرد. دلال‌هایی که از تعداد مشخصی نخ استفاده می‌کنند و نمی‌توانند در صورت لزوم از همه منابع سیستم استفاده کنند و دلال‌هایی که به صورت چند پروسه‌ای یا چند نخی طراحی شده‌اند
و می‌توانند در صورت لزوم تمام منابع سیستم را مصرف کنند. دسته اول برای زمانی که سیستم منابع زیادی ندارد یا در لبه می‌توانند کاربرد داشته باشند و دسته دوم عموما در زیرساخت‌های بزرگ و ابری می‌توانند مورد استفاده قرار بگیرند.
\مرجع{Mishra2021}

همانطور که اشاره شد پروتکل \متن‌لاتین{MQTT} در کاربردهای اینترنت اشیا می‌تواند برای ارتباط مستقیم میان اشیا و اپلیکیشن مورد استفاده قرار بگیرد. از سوی دیگر این پروتکل با توجه به ماهیت غیرهمزمانی که دارد یکی از راه‌های شناخته شده برای ارتباط میان سرور شبکه \متن‌لاتین{LoRaWAN}
و برنامه‌های کاربردی و \متن‌لاتین{Gateway} می‌باشد. این امر در معماری سامانه \متن‌لاتین{Chirpstack} (شکل \رجوع{شکل:معماری سامانه Chripstack}) که یکی از بسترهای شناخته شده و متن باز برای \متن‌لاتین{LoRa} می‌باشد مشهود است.

\شروع{شکل}
\درج‌تصویر[width=\textwidth]{./img/chirpstack-architecture.png}
\تنظیم‌ازوسط
\برچسب{شکل:معماری سامانه Chripstack}
\شرح{معماری \متن‌لاتین{LoRaWAN} سرور متن باز \متن‌لاتین{Chirpstack}}
\پایان{شکل}

\زیرقسمت{\متن‌لاتین{QUIC}}

پروتکل \متن‌لاتین{QUIC} توسط گوگل در سال ۲۰۱۳ پیشنهاد شد و ۳ سال بعد کاگروهی در \متن‌لاتین{IETF} برای استانداردسازی آن شکل گرفت. این پروتکل در لایه کاربر بوده و بر پایه \متن‌لاتین{UDP} کار می‌کند.
هدف این پروتکل جایگزین کردن پشته سابق \متن‌لاتین{HTTP2}، \متن‌لاتین{TLS} و \متن‌لاتین{TCP} است.
پیاده‌سازی این پروتکل در لایه کاربر اجازه توسعه و سازگاری آن را ساده می‌کند و از سوی دیگر استفاده از \متن‌لاتین{UDP} اجازه می‌دهد به سادگی بر بستر شبکه‌های حاضر فعالیت کند.

این پروتکل با پشتیبانی ذاتی رمزنگاری سربار دست‌دادها را کاهش داده است و از سوی دیگر با استفاده از جریان‌ها باعث می‌شود تا بسته‌ها به صورت موازی و بدون تاخیر در یک ارتباط ارسال شوند.
\مرجع{10.1145/3098822.3098842}

\قسمت{پردازش در لبه}

با گسترش اینترنت اشیا و مطرح شدن داده‌های حجیم در چند سال اخیر مشخص شده است که استفاده تنها از سرورهای ابری نمی‌تواند برای این پردازش مفید باشد و باعث هدر رفت منابعی چون پهنای باند و \نقاط‌خ می‌گردد.
برای پاسخ به این مشکل، بحث پردازش مِه یا \متن‌لاتین{Fog Computing} پیشنهاد شد که در آن پیشنهاد می‌شود تا پردازش‌ها به جای انجام شدن در سرورهای ابری در لبه شبکه انجام شوند.
اگر بخواهیم بهتر بیان کنیم داده‌ّهای جمع‌آوری شده به جای ارسال و پردازش در ابر در همان نودهای لبه پردازش شوند.
\مرجع{Perera2017}

در پردازش مِه از یک معماری خاص پیروی نمی‌شود و هدف سودهی پردازش به سمت لبه است. انتظار می‌رود با استفاده از لبه در پردازش حجم داده‌ای که نیاز به نگهداری و پردازش دارد به میزان قابل مدیریت برسد.
از سوی دیگر پردازش در لبه می‌تواند تاخیر را کاهش داده و دسترسی‌پذیری را افزایش دهد. پردازش در لبه در خانه‌های هوشمند سال‌ها است که مورد استفاده قرار گرفته است اما برای پی بردن به ویژگی‌های پردازش در لبه
نیاز است که آن در کاربردها پیچیده‌تر مانند شهر هوشمند ارزیابی کرد.
\مرجع{Perera2017}

\قسمت{\متن‌لاتین{Network Calculus}}

\متن‌لاتین{Network Calculus} مجموعه‌ای از پیشرفت‌های اخیر است که دید عمیقی در مساله‌های جریان در شبکه‌ها ایجاد می‌کند. پایه \متن‌لاتین{Network Calculus} در تئوری ریاضی \متن‌لاتین{Dioid}ها و مشخصا \متن‌لاتین{Min-Plus dioid} نهفته است.
در ادامه به مرور مفاهیم اصلی این حوزه می‌پردازیم.

\زیرقسمت{منحنی ورودی}

جریان با تابع تجمعی $R(t)$، دارای $\alpha$ به عنوان جریان ورودی (بیشین) است اگر:

\[
  R(t) - R(s) \le \alpha(t - s) \forall t,s \ge 0
\]

که در آن $\alpha$ یک تابع صعودی است. به عنوان مثال اگر فرض کنیم جریان ورودی با الگوریتم \متن‌لاتین{Leaky Bucket} با پارامترهای $r$ و $b$، محدود شده است داریم:

\[
  \alpha(t) = rt + b
\]

جریان‌های ورودی را می‌توان با یکدیگر جمع کرد.

\زیرقسمت{پیچش \متن‌لاتین{Min-Plus}}

پیچش دو جریان $f_{1}$ و $f_{2}$ در جبر \متن‌لاتین{Min-Plus} به شکل زیر تعریف می‌شوند:

\[
  f(t) = \inf_{s \ge 0}(f_{1}(s) + f_{2}(t-s))
\]
\[
  f = f_{1} \otimes f_{2}
\]

این پیچش، ویژگی‌ها خوب پیچیش معمول را دارد:

\[
  (f_{1} \otimes f_{2}) \otimes f_{3} = f_{1} \otimes (f_{2} \otimes f_{3})
\]
\[
  f_{1} \otimes f_{2} = f_{2} \otimes f_{1}
\]

با توجه به این تعریف می‌توان گفت $\alpha$ یک منحنی ورودی برای $R$ خواهد بود اگر و تنها اگر

\[
  R \le R \otimes \alpha
\]
