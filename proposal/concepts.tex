\فصل{مفاهیم}

\قسمت{مقدمه}
در سال‌های اخیر با کاهش قسمت حسگرهای و عملگرها، تعداد دستگاه‌های اینترنت اشیا به سرعت در حال گسترش است و به سرعت در حال تبدیل کردن خود به یکی از اجزا زندگی ما می‌باشند.
در نتیجه رد پای فاحش اینترنت اشیا امروزه در همه جا قابل مشاهده است.
\مرجع{Mishra2021}

نیاز کاربردهای اینترنت اشیا روز به روز به تکنولوژی‌هایی که می‌توانند عملکرد توان پایین داشته باشند
و دستگاه‌های انتهایی که بتوانند ارتباط بی‌سیتم در مسافت‌های طولانی را با هزینه و پیچیدگی پایین برقرار کنند بیشتر می‌شود.
در بیشتر کاربردها، دستگاه‌های انتهایی اینترنت اشیا حسگرهایی با باتری می‌باشند، که پروفایل مصرف توان آن‌ها در جهت افزایش طول عمر
باتریشان می‌بایست با دقت طراحی شده باشد.
برد ارتباطی نیاز دارد از چند صد متر تا چندین کیلومتر را شامل شود چرا که دستگاه‌های ارتباطی در محیط عملیاتی بزرگی گسترده‌اند.
با نظر گرفتن همه ویژگی‌های نامبرده، این امر تنها با استفاده از تکنولوژی‌های حوزه شبکه‌های توان پایین با برد بالا\پانویس{LPWAN} ممکن است.
\مرجع{sensors-18-03995}

تکنولوژی‌های بسیاری در حوزه \متن‌لاتین{LPWAN} به بازار عرضه شده‌اند که از جمله‌ی آن‌ها می‌توان به \متن‌لاتین{SigFox}، \متن‌لاتین{NB-IoT} و \متن‌لاتین{LoRaWAN}
اشاره کرد.
\متن‌لاتین{SigFox} قصد دارد یک پوشش جهانی را در قالب یک اپراتور شبکه که در کشورهای مختلف با استفاده از شرکت‌های تابعه اجرا می‌شود، فراهم آورد.
\متن‌لاتین{NB-IoT} توسط شرکت‌های مخابراطی به عنوان یک جایگزین در ارتباطات اینترنت اشیا، نسبت به تکنولوژی‌های زیرگیگاهرتز \متن‌لاتین{LPWAN} ارائه می‌شود.
از آنجایی \متن‌لاتین{NB-IoT} در طیف فرکانسی دارای لایسنس فعالیت می‌کند می‌تواند قابلیت اطمینان بیشتری در ترافیک نسبت به سایر تکنولوژی‌های زیرگیگاهرتز ارائه دهد.
برخلاف \متن‌لاتین{SigFox} و \متن‌لاتین{NB-IoT}، متن‌لاتین{LoRaWAN} قابلیت ارائه به صورت شبکه‌های خصوصی و ادغام آسان با پلتفرم‌های شبکه‌ای جهانی مانند \متن‌لاتین{The Things Network} را فراهم می‌آورد.
به همین دلیل و از سوی دیگر باز بودن استاندارد، \متن‌لاتین{LoRaWAN} توجه جامعه محققان را از اولین نمود خود در بازار جلب کرده است.
\مرجع{sensors-18-03995}

از سال ۲۰۱۵ جامعه تحقیقاتی شروع به مطالعه در رابطه با کارآیی و ویژگی‌های مختلف تکنولوژی‌های \متن‌لاتین{LoRa} و \متن‌لاتین{LoRaWAN} کرد.
از آن تاریخ مقلات متعددی در ژورنال‌ها و کنفرانس‌های عملی در سراسر دنیا چاپ و ارائه شده‌اند.
\مرجع{sensors-18-03995}

مستقل از ارتباط رادیویی که برای شکل دادن شبکه‌ی \متن‌لاتین{M2M} از آن استفاده شده است، دستگاه انتهایی یا ماشین می‌بایست داده خود را از طریق اینرتنت قابل دسترسی کنند.
دستگاه اینترنت اشیا عموما منابع محدودی دارند و این به آن معناست که باید با حافظه، توان پردازشی، توان شبکه‌ای و باتری محدودی فعالیت کنند.
بنابراین کارایی ارتباط ماشین به ماشین وابستگی زیادی به پروتکل زیرین مورد استفاده در اپلیکشن اینرتنت اشیا دارد.
\مرجع{Mishra2021}

پروتکل‌های ارتباطی زیادی در حوزه اینترنت اشیا مطرح است که می‌توان از بین آن‌ها به \متن‌لاتین{MQTT}، \متن‌لاتین{CoAP}، \متن‌لاتین{AMQP} و \متن‌لاتین{HTTP} اشاره کرد.
\مرجع{Mishra2021}

\قسمت{\متن‌لاتین{LoRa}}

لایه‌ی فیزیکی \متن‌لاتین{LoRa} که در \متن‌لاتین{LoRaWAN} استفاده می‌شود، در سال ۲۰۱۴ توسط \متن‌لاتین{Semtech} به ثبت رسید.
از ویژگی‌های \متن‌لاتین{LoRa} می‌توان به توان عملیاتی پایین، نرخ پایین داده و برند ارتباطی بالا اشاره کرد.
\مرجع{sensors-18-03995}
\مرجع{Adelantado2017}


ماژولیشن آن بر پایه \متن‌لاتین{Chirp Spread Spectrum} بوده و به صورت دوره‌ای سیگنال‌های \متن‌لاتین{chirp}ای تولید می‌کنند که همه آن‌ها بازه زمانی یکسانی دارند.
\متن‌لاتین{chirp} یک سیگنال سینوسی است که فرکانس آن با زمان افزایش یا کاهش پیدا می‌کند.
یک \متن‌لاتین{chirp} به وسیله‌ی پروفایل زمانی فرکانس لحظه‌ی آن که در بازه‌ی زمانی \متن‌لاتین{T} از فرکانس $f_0$ به فرکانس $f_1$
تغییر می‌کند شناخته می‌شود.
در \متن‌لاتین{LoRa} دو نوع \متن‌لاتین{chirp} تعریف شده است. \متن‌لاتین{chirp} پایه که فرکانس پروفایل زمانی آن با فرکانس مینیمال
\(f_{\min} = -\frac{BW}{2}\)
شروع شده و با فرکانس ماکسیمال
\(f_{\max} = \frac{BW}{2}\)
خاتمه می‌یابد.
برای ورودی‌های دیجیتال مختلف، یک ماژولاتور \متن‌لاتین{chirp}های مختلفی تولید می‌کند که نسبت به \متن‌لاتین{chirp} پایه شیف زمانی خورده‌اند.
\مرجع{sensors-18-03995}

\متن‌لاتین{LoRa} از باند فرکانسی بدون مجوز استفاده می‌کند بنابراین برای راه‌اندازی شبکه‌ی آن نیاز به تهیه هیچ مجوزی نیست. البته باید در نظر داشته که نرخ پیام در این باندهای بدون مجوز توسط قانون‌گذاران محدود شده است.
\مرجع{Cruz2021}

\قسمت{\متن‌لاتین{LoRaWAN}}

\متن‌لاتین{LoRaWAN} پروتکل لایه لینک و شبکه می‌باشد که شامل پروتکل کنترل دسترسی چندگانه\پانویس{MAC} نیز می‌باشد. این پروتکل اجازه می‌دهد تا دستگاه‌های \متن‌لاتین{LoRa} با برنامه‌های کاربردی ارتباط برقرار کنند.
این پروتکل توسط \متن‌لاتین{LoRa Alliance} توسعه پیدا کرده و برای همگان قابل استفاده است.
\مرجع{Cruz2021}

یک شبکه‌ی \متن‌لاتین{LoRaWAN} در ساده‌ترین شکل از اجزای زیر تشکیل شده است:

\شروع{شمارش}
\فقره یک دستگاه حسگر یا عملگر که توان و محاسبات محدودی دارد.
\فقره یک \متن‌لاتین{Gateway} که عنصر شبکه‌ای برای دریافت و ارسال اطلاعات از و به دستگاه‌ها را برعهده دارد.
\فقره سرور شبکه که پیام‌های دریافت شده از یک مجموعه \متن‌لاتین{Gateway}ها را به برنامه‌های کاربردی می‌رساند و برعکس
\فقره برنامه کاربردی که می‌تواند در بستر اینترنت قرار داشته باشد و داده‌ها را از طریق سرور شبکه برای اشیا ارسال و دریافت کند.
\پایان{شمارش}

\شروع{شکل}
\درج‌تصویر[width=\textwidth]{./img/nrm-home.png}
\تنظیم‌ازوسط
\شرح{مدل مرجع شبکه \متن‌لاتین{LoRaWAN} - شبکه‌ی خانگی}
\پایان{شکل}

\شروع{شکل}
\درج‌تصویر[width=\textwidth]{./img/nrm-roaming.png}
\تنظیم‌ازوسط
\شرح{مدل مرجع شبکه \متن‌لاتین{LoRaWAN} - شبکه‌ی فراگرد}
\پایان{شکل}

در حوزه امنیت \متن‌لاتین{LoRaWAN} دولایه از امنیت را تعریف می‌کند. لایه اول امنیت میان شی و شبکه است در حالی که لایه دوم میان شی و برنامه کاربردی می‌باشد.
به این صورت می‌توان مطمئن شد که تنها برنامه کاربردی است که می‌تواند داده‌های ارسالی توسط دستگاه را رمزگشایی کند.
\مرجع{Cruz2021}

در ضمن \متن‌لاتین{LoRaWAN} ویژگی‌های دیگری مانند نرخ داده تطبیقی\پانویس{ADR} را اضافه می‌کند. در نرخ داده تطبیقی شبکه با دستگاه در رابطه با پارامترهای لایه‌ی فیزیکی \متن‌لاتین{LoRa} مذاکره می‌کند
که در نتجیه آن کارآیی مصرف بهینه می‌شود. شکل \رجوع{شکل:لایه‌های لورا} مدل لایه‌ای \متن‌لاتین{LoRa} و \متن‌لاتین{LoRaWAN} را نمایش می‌دهد.
\مرجع{Cruz2021}

\شروع{شکل}
\درج‌تصویر[width=\textwidth]{./img/lora-layers.png}
\تنظیم‌ازوسط
\برچسب{شکل:لایه‌های لورا}
\شرح{مدل لایه‌ای \متن‌لاتین{LoRa} و \متن‌لاتین{LoRaWAN} \مرجع{Cruz2021}}
\پایان{شکل}

\قسمت{\متن‌لاتین{MQTT}}

پروتکل‌های اینترنت اشیا امروزه قلب اصلی ارتباط‌های ماشین به ماشین (\متن‌لاتین{M2M}) را تشکیل می‌دهند. فارغ از تکنولوژی رادیویی که برای پیاده‌سازی شبکه‌ی اینتنرت اشیا و ماشین به ماشین استفاده می‌شود، همه داده‌هایی که توسط سنسورها و عملگرهای اینترنت اشیا
تولید می‌شوند وابستگی زیادی به پروتکل ارتباطی که برای ارتباط ماشین به ماشین در اپلیکیشن اینترنت اشیا استفاده شده است، دارند.
با افزایش تقاضا برای سرویس‌های مبتنی بر اینترنت اشیا، نیاز برای کاهش توان دستگاه‌ها و سرویس‌های اینترنت اشیا نیز در جهت محیط زیست پایدار برای نسل‌های آینده، افزایش یافته است.

پروتکل \متن‌لاتین{Messaging Queue Telemetry Transport} که مختصرا \متن‌لاتین{MQTT} نامیده می‌شود یکی از پروتکل‌ها پر استفاده در اینترنت اشیا می‌باشد.
این پروتکل یک پروتکل با معماری انتشار و اشتراک است که توان مصرفی پایینی دارد.
\متن‌لاتین{MQTT} یک پروتکل لایه کاربرد است که برای لایه انتقال از \متن‌لاتین{TCP/IP} و پورت‌های ۱۸۸۳ و ۸۸۸۳ (به ترتیب برای ارتباط رمز شده و ارتباط رمز نشده) استفاده می‌کند. البته پژوهش‌هایی چون \مرجع{Fernndez2021} سعی در تغییر لایه انتقال به \متن‌لاتین{UDP/Quic} داشته‌اند.
\مرجع{Mishra2021}

سه نقش در معماری \متن‌لاتین{MQTT} تعریف شده است. نقش اول کلاینت تولید کننده داده می‌باشد که به آن \متن‌لاتین{Producer} گفته می‌شود. نقش دوم سرور دلال پیام می‌باشد و نقش سوم کلاینت دریافت کننده داده است که به آن \متن‌لاتین{Subscriber} گفته می‌شود.
از \متن‌لاتین{Topic} برای مشخص کردن جریان‌های داده‌ای استفاده می‌شود و در تولید کننده داده می‌بایست برای داده‌ی خود \متن‌لاتین{Topic} داشته باشد و هر دریافت کننده داده روی \متن‌لاتین{Topic} خاصی مشترک می‌شود.
این \متن‌لاتین{Topic}ها می‌توانند به صورت سلسله مراتبی نیز می‌تواند تشکیل شود.
\مرجع{Mishra2021}

در پروتکل \متن‌لاتین{‌MQTT} سه سطح مختلف از کیفیت سرویس تعریف می‌شود. در کیفیت سرویس \متن‌لاتین{QoS0} پیام‌ها به صورت ارسال و فراموش کردن ارسال می‌شوند و هیچ تضمینی برای موفیت این ارسال وجود ندارد.
در کیفیت سرویس \متن‌لاتین{QoS1} پیام‌ها حداقل یکبار تحویل داده خواهند شد، در این حالت کلاینت ارسال‌کننده پس از ارسال منتظر پیام \متن‌لاتین{PUBACK} می‌ماند و در صورت عدم دریافت آن فرآیند ارسال را دوباره تکرار می‌کند.
در کیفیت سرویس \متن‌لاتین{QoS2} پیام‌ها دقیقا یکبار تحویل داده می‌شوند، این بالاترین کیفیت سرویس بوده و منابع زیادی را مصرف می‌کند.
\مرجع{Mishra2021}

در نظر داشته باشید که کیفیت سرویس پروتکل \متن‌لاتین{MQTT} به صورت انتها به انتها نیست و به ارتباط میان دلال پیام و کلاینت‌ها وابسته است.
کیفیت سرویس پیام دریافت شده از سوی \متن‌لاتین{Subscriber} وابسته کیفیت سرویس عملیات انتشار و اشتراک است. اگر کلاینت الف عملیات انتشار را با کیفیت سرویس بالاتری
نسبت به عملیات اشتراک در کلاینت ب انجام دهد کیفیت سرویسی که سرور پیام را به دست کلاینت ب می‌رساند کیفیت سرویسی است که کلاینت ب در عملیات اشتراک استفاده کرده است.
اگر کلاینت الف عملیات انتشار را با کیفیت سرویس پایین‌تری نسبت به عملیات اشتراک در کلاینت ب انجام دهد کیفیت سرویسی که سرور پیام را به دست کلاینت ب می‌رساند کیفیت سرویسی است که
کلاینت الف در عملیات انتشار استفاده کرده است.
\مرجع{MQTTQoS}

\شروع{لوح}
\شرح{چگونگی محاسبه کیفیت سرویس پیام دریافتی در پروتکل \متن‌لاتین{MQTT}\مرجع{MQTTQoS}}
\فضای‌و{5mm}
\شروع{جدول}{|c|c|c|}
\خط‌پر
کیفیت سرویس عملیات انتشار & کیفیت سرویس عملیات اشتراک & کیفیت سرویس پیام دریافتی \\
\خط‌پر
۰ & ۰ & ۰ \\
\خط‌پر
۰ & ۱ & ۰ \\
\خط‌پر
۰ & ۲ & ۰ \\
\خط‌پر
۱ & ۰ & ۰ \\
\خط‌پر
۱ & ۱ & ۱ \\
\خط‌پر
۱ & ۲ & ۱ \\
\خط‌پر
۲ & ۰ & ۰ \\
\خط‌پر
۲ & ۱ & ۱ \\
\خط‌پر
۲ & ۲ & ۲ \\
\خط‌پر
\پایان{جدول}
\پایان{لوح}

\شروع{شکل}
\تنظیم‌ازوسط
\درج‌تصویر[width=\textwidth]{./img/mqtt-qos.png}
\فضای‌و{5mm}
\شرح{سطوح مختلف کیفیت سرویس در پروتکل \متن‌لاتین{MQTT}\مرجع{Mishra2021}}
\پایان{شکل}


به صورت کلی می‌توان دلال‌های پیام در پروتکل \متن‌لاتین{MQTT} را به دو دسته کلی تقسم کرد. دلال‌هایی که از تعداد مشخصی نخ استفاده می‌کنند و نمی‌توانند در صورت لزوم از همه منابع سیستم استفاده کنند و دلال‌هایی که به صورت چند پروسه‌ای یا چند نخی طراحی شده‌اند
و می‌توانند در صورت لزوم تمام منابع سیستم را مصرف کنند.
\مرجع{Mishra2021}

همانطور که اشاره شد پروتکل \متن‌لاتین{MQTT} در کاربردهای اینترنت اشیا می‌تواند برای ارتباط مستقیم میان اشیا و اپلیکیشن مورد استفاده قرار بگیرد. از سوی دیگر این پروتکل با توجه به ماهیت غیرهمزمانی که دارد یکی از راه‌های شناخته شده برای ارتباط میان سرور شبکه \متن‌لاتین{LoRaWAN}
و برنامه‌های کاربردی می‌باشد.
