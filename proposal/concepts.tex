\فصل{مفاهیم}

\قسمت{مقدمه}
در سال‌های اخیر با کاهش قسمت حسگرهای و عملگرها، تعداد دستگاه‌های اینترنت اشیا به سرعت در حال گسترش است و به سرعت در حال تبدیل کردن خود به یکی از اجزا زندگی ما می‌باشند.
در نتیجه رد پای فاحش اینترنت اشیا امروزه در همه جا قابل مشاهده است.
\مرجع{Mishra2021}

نیاز کاربردهای اینترنت اشیا روز به روز به تکنولوژی‌هایی که می‌توانند عملکرد توان پایین داشته باشند
و دستگاه‌های انتهایی که بتوانند ارتباط بی‌سیتم در مسافت‌های طولانی را با هزینه و پیچیدگی پایین برقرار کنند بیشتر می‌شود.
در بیشتر کاربردها، دستگاه‌های انتهایی اینترنت اشیا حسگرهایی با باتری می‌باشند، که پروفایل مصرف توان آن‌ها در جهت افزایش طول عمر
باتریشان می‌بایست با دقت طراحی شده باشد.
برد ارتباطی نیاز دارد از چند صد متر تا چندین کیلومتر را شامل شود چرا که دستگاه‌های ارتباطی در محیط عملیاتی بزرگی گسترده‌اند.
با نظر گرفتن همه ویژگی‌های نامبرده، این امر تنها با استفاده از تکنولوژی‌های حوزه شبکه‌های توان پایین با برد بالا\پانویس{LPWAN} ممکن است.
\مرجع{sensors-18-03995}

تکنولوژی‌های بسیاری در حوزه \متن‌لاتین{LPWAN} به بازار عرضه شده‌اند که از جمله‌ی آن‌ها می‌توان به \متن‌لاتین{SigFox}، \متن‌لاتین{NB-IoT} و \متن‌لاتین{LoRaWAN}
اشاره کرد.
\متن‌لاتین{SigFox} قصد دارد یک پوشش جهانی را در قالب یک اپراتور شبکه که در کشورهای مختلف با استفاده از شرکت‌های تابعه اجرا می‌شود، فراهم آورد.
\متن‌لاتین{NB-IoT} توسط شرکت‌های مخابراطی به عنوان یک جایگزین در ارتباطات اینترنت اشیا، نسبت به تکنولوژی‌های زیرگیگاهرتز \متن‌لاتین{LPWAN} ارائه می‌شود.
از آنجایی \متن‌لاتین{NB-IoT} در طیف فرکانسی دارای لایسنس فعالیت می‌کند می‌تواند قابلیت اطمینان بیشتری در ترافیک نسبت به سایر تکنولوژی‌های زیرگیگاهرتز ارائه دهد.
برخلاف \متن‌لاتین{SigFox} و \متن‌لاتین{NB-IoT}، متن‌لاتین{LoRaWAN} قابلیت ارائه به صورت شبکه‌های خصوصی و ادغام آسان با پلتفرم‌های شبکه‌ای جهانی مانند \متن‌لاتین{The Things Network} را فراهم می‌آورد.
به همین دلیل و از سوی دیگر باز بودن استاندارد، \متن‌لاتین{LoRaWAN} توجه جامعه محققان را از اولین نمود خود در بازار جلب کرده است.
\مرجع{sensors-18-03995}

از سال ۲۰۱۵ جامعه تحقیقاتی شروع به مطالعه در رابطه با کارآیی و ویژگی‌های مختلف تکنولوژی‌های \متن‌لاتین{LoRa} و \متن‌لاتین{LoRaWAN} کرد.
از آن تاریخ مقلات متعددی در ژورنال‌ها و کنفرانس‌های عملی در سراسر دنیا چاپ و ارائه شده‌اند.
\مرجع{sensors-18-03995}

مستقل از ارتباط رادیویی که برای شکل دادن شبکه‌ی \متن‌لاتین{M2M} از آن استفاده شده است، دستگاه انتهایی یا ماشین می‌بایست داده خود را از طریق اینرتنت قابل دسترسی کنند.
دستگاه اینترنت اشیا عموما منابع محدودی دارند و این به آن معناست که باید با حافظه، توان پردازشی، توان شبکه‌ای و باتری محدودی فعالیت کنند.
بنابراین کارایی ارتباط ماشین به ماشین وابستگی زیادی به پروتکل زیرین مورد استفاده در اپلیکشن اینرتنت اشیا دارد.
\مرجع{Mishra2021}

پروتکل‌های ارتباطی زیادی در حوزه اینترنت اشیا مطرح است که می‌توان از بین آن‌ها به \متن‌لاتین{MQTT}، \متن‌لاتین{CoAP}، \متن‌لاتین{AMQP} و \متن‌لاتین{HTTP} اشاره کرد.

\قسمت{\متن‌لاتین{LoRa}}
لایه‌ی فیزیکی \متن‌لاتین{LoRa} که در \متن‌لاتین{LoRaWAN} استفاده می‌شود، در سال ۲۰۱۴ توسط \متن‌لاتین{Semtech} به ثبت رسید.
از ویژگی‌های \متن‌لاتین{LoRa} می‌توان به توان عملیاتی پایین، نرخ پایین داده و برند ارتباطی بالا اشاره کرد.
\مرجع{sensors-18-03995}
\مرجع{Adelantado2017}


ماژولیشن آن بر پایه \متن‌لاتین{Chirp Spread Spectrum} بوده و به صورت دوره‌ای سیگنال‌های \متن‌لاتین{chirp}ای تولید می‌کنند که همه آن‌ها بازه زمانی یکسانی دارند.
\متن‌لاتین{chirp} یک سیگنال سینوسی است که فرکانس آن با زمان افزایش یا کاهش پیدا می‌کند.
یک \متن‌لاتین{chirp} به وسیله‌ی پروفایل زمانی فرکانس لحظه‌ی آن که در بازه‌ی زمانی \متن‌لاتین{T} از فرکانس $f_0$ به فرکانس $f_1$
تغییر می‌کند شناخته می‌شود.
در \متن‌لاتین{LoRa} دو نوع \متن‌لاتین{chirp} تعریف شده است. \متن‌لاتین{chirp} پایه که فرکانس پروفایل زمانی آن با فرکانس مینیمال
\(f_{\min} = -\frac{BW}{2}\)
شروع شده و با فرکانس ماکسیمال
\(f_{\max} = \frac{BW}{2}\)
خاتمه می‌یابد.
برای ورودی‌های دیجیتال مختلف، یک ماژولاتور \متن‌لاتین{chirp}های مختلفی تولید می‌کند که نسبت به \متن‌لاتین{chirp} پایه شیف زمانی خورده‌اند.


\قسمت{\متن‌لاتین{MQTT}}

پروتکل‌های اینترنت اشیا امروزه قلب اصلی ارتباط‌های ماشین به ماشین (\متن‌لاتین{M2M}) را تشکیل می‌دهند. فارغ از تکنولوژی رادیویی که برای پیاده‌سازی شبکه‌ی اینتنرت اشیا و ماشین به ماشین استفاده می‌شود، همه داده‌هایی که توسط سنسورها و عملگرهای اینترنت اشیا
تولید می‌شوند وابستگی زیادی به پروتکل ارتباطی که برای ارتباط ماشین به ماشین در اپلیکیشن اینترنت اشیا استفاده شده است، دارند.
با افزایش تقاضا برای سرویس‌های مبتنی بر اینترنت اشیا، نیاز برای کاهش توان دستگاه‌ها و سرویس‌های اینترنت اشیا نیز در جهت محیط زیست پایدار برای نسل‌های آینده، افزایش یافته است.

پروتکل \متن‌لاتین{Messaging Queue Telemetry Transport} که مختصرا \متن‌لاتین{MQTT} نامیده می‌شود یکی از پروتکل‌ها پر استفاده در اینترنت اشیا می‌باشد.
این پروتکل یک پروتکل با معماری انتشار و اشتراک است که توان مصرفی پایینی دارد.
\مرجع{Mishra2021}
