\فصل{بیان مسائل}

در این فصل به مرور مسائلی می‌پردازیم که از چتر مساله‌ی اصلی این رساله خارج بوده اما ارزش بررسی و ارزیابی را دارند.

\قسمت*{جایگزینی شبکه‌ی دسترسی}

\زیرقسمت*{ارزیابی شبکه \متن‌لاتین{IEEE 802.11be}}

\متن‌لاتین{WiFi 7} و \متن‌لاتین{IEEE 802.11be} فناوری‌های جدیدی هستند که به دنبال پشتیبانی از کاربردهای حساس به زمان هستند. هنوز این فناوری‌ها در عمل مورد سنجش قرار نگرفته‌اند.
سنجش آن‌ها می‌تواند نقاط ضعف و قوت آن‌ها را بهتر مشخص کند. استاندارد نهایی برای سال ۲۰۲۴ خواهد بود ولی اولین نسخه‌ی آن در سال ۲۰۲۲ منتشر خواهد شد. برای شبیه‌سازی و ارزیابی در این
مساله می‌بایست از نرم‌افزارهای شبیه‌ساز استفاده کرد.

با توجه به ماهیت این شبکه بحث ارزیابی کران بالای تاخیر در آن‌ها بسیار ارزشمند بوده و می‌توان از تکنیک‌های \متن‌لاتین{Network Calculus} برای این امر استفاده کرد.

\زیرقسمت*{ارزیابی شبکه \متن‌لاتین{NB-IoT}}

شبکه‌های \متن‌لاتین{NB-IoT} با زیرساخت فعلی شبکه‌های سلولی فعالیت می‌کنند، از این رو برای فراهم آوردن کیفیت سرویس مناسب نیاز دارند تا بین ترافیک کاربران عادی
و ترافیک اینترنت اشیا تفاوت قائل شوند. در اینجا بحث تقسیم‌بندی شبکه و استفاده از کارکردهای مجازی شده پیش می‌آید که در در این حوزه می‌توان ارزیابی صورت داد.
از سوی دیگر مفاهیمی همچون کارکردهای شبکه‌ای ابرزی در اینجا می‌تواند نقش پررنگی در شبکه \متن‌لاتین{Backhaul} داشته باشد.

\زیرقسمت*{استفاده از روش \متن‌لاتین{Federated Learning} برای مکانیزم نرخ داده تطبیق‌پذیر}

همانطور که پیشتر اشاره شد، مکانیزم نرخ داده تطبیق‌پذیر در سرور شبکه پیاده‌سازی شده است و یکی از پارامترهای مهم در آن نرخ همگرایی است.
در صورتی که بتوان این الگوریتم را در دروازه پیاده‌سازی کرد، می‌توان پاسخ را بهبود داده و در نتیجه زودتر به همگرایی رسید.
با توجه به اینکه توان پردازشی در دروازهها نمی‌تواند خیلی زیاد باشد و از سوی دیگر بحث حریم خصوصی کاربران نیز می‌تواند مطرح باشد،
ایده \متن‌لاتین{Federated Learning} به وسیله‌ی یک سرور خارجی (که می‌تواند همان سرور شبکه باشد) و دروازه می‌تواند پیاده‌سازی شده
و ارزیابی شود.

\قسمت*{کارگروه‌های \متن‌لاتین{IETF} مرتبط}

\شروع{فقرات}
\فقره \متن‌لاتین{A Semantic Definition Format for Data and Interactions of Things} یا اختصارا \متن‌لاتین{asdf} در حوزه \متن‌لاتین{Apps \& Runtime}
\فقره \متن‌لاتین{Concise Binary Object Representation Maintenance and Extensions} یا اختصارا \متن‌لاتین{cbor} در حوزه \متن‌لاتین{Apps \& Runtime}
\فقره \متن‌لاتین{Constrained RESTful Environments} یا اختصارا \متن‌لاتین{core} در حوزه \متن‌لاتین{Apps \& Runtime}
\فقره \متن‌لاتین{IPv6 over Low Power Wide-Area Networks} یا اختصارا \متن‌لاتین{lpwan} در حوزه \متن‌لاتین{Internet}
\پایان{فقرات}

\قسمت*{پروژه‌های کارشناسی}

از آنجایی که در مساله پیشنهادی نیاز به پیاده‌سازی استانداردها و پروتکل‌هایی وجود دارد، این موارد می‌توانند در قالب پروژه کارشناسی تعریف شوند.

\شروع{فقرات}
\فقره پیاده‌سازی کتابخانه‌ی \متن‌لاتین{SenML} با زبان \متن‌لاتین{C} یا \متن‌لاتین{C++} بر پایه پروتکل \متن‌لاتین{CBOR}
\فقره پیاده‌سازی یک زیرساخت لبه با قابلیت مدیریت چندین گره لبه از دور
\پایان{فقرات}
