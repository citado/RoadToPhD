% در این فایل، دستورها و تنظیمات مورد نیاز، آورده شده است.
%-------------------------------------------------------------------------------------------------------------------
% در ورژن جدید زی‌پرشین برای تایپ متن‌های ریاضی، این سه بسته، حتماً باید فراخوانی شود.
\usepackage{amsthm,amssymb,amsmath,amsfonts}
% بسته‌ای برای تنطیم حاشیه‌های بالا، پایین، چپ و راست صفحه
\usepackage[top=30mm, bottom=30mm, left=25mm, right=30mm]{geometry}
% بسته‌‌ای برای ظاهر شدن شکل‌ها و تصاویر متن
\usepackage{graphicx}
\usepackage{color}
\usepackage{titleps}
\usepackage{titletoc}
\usepackage{tocloft}
\usepackage{multirow}
\usepackage{tabularx}
\usepackage{datatool}
\usepackage{booktabs}
\usepackage{enumitem}
%\usepackage{titlesec}
% بسته‌ و دستوراتی برای ایجاد لینک‌های رنگی با امکان جهش
\usepackage[pagebackref=false,colorlinks,linkcolor=blue,citecolor=red]{hyperref}
\usepackage[nameinlink]{cleveref}%capitalize,,noabbrev

\AtBeginDocument{%
  \crefname{equation}{برابری}{equations}%
  \crefname{chapter}{فصل}{chapters}%
  \crefname{section}{بخش}{sections}%
  \crefname{appendix}{پیوست}{appendices}%
  \crefname{enumi}{مورد}{items}%
  \crefname{footnote}{زیرنویس}{footnotes}%
  \crefname{figure}{شکل}{figures}%
  \crefname{table}{جدول}{tables}%
  \crefname{theorem}{قضیه}{theorems}%
  \crefname{lemma}{لم}{lemmas}%
  \crefname{corollary}{نتیجه}{corollaries}%
  \crefname{proposition}{گزاره}{propositions}%
  \crefname{definition}{تعریف}{definitions}%
  \crefname{result}{نتیجه}{results}%
  \crefname{example}{مثال}{examples}%
  \crefname{remark}{نکته}{remarks}%
  \crefname{note}{یادداشت}{notes}%
}
% چنانچه قصد پرینت گرفتن نوشته خود را دارید، خط بالا را غیرفعال و  از دستور زیر استفاده کنید چون در صورت استفاده از دستور زیر‌‌،
% لینک‌ها به رنگ سیاه ظاهر خواهند شد که برای پرینت گرفتن، مناسب‌تر است
%\usepackage[pagebackref=false]{hyperref}
% بسته‌ لازم برای تنظیم سربرگ‌ها
\usepackage{fancyhdr}
% بسته‌ای برای ظاهر شدن «مراجع»  در فهرست مطالب
\usepackage[nottoc]{tocbibind}
% دستورات مربوط به ایجاد نمایه
\usepackage{makeidx,multicol}
\setlength{\columnsep}{1.5cm}
% rotating tables and figures
\usepackage{rotating}
%\usepackage{tensor}
\usepackage[all]{xy}
\usepackage{tikz}
\usetikzlibrary{arrows}
%%%%%%%%%%%%%%%%%%%%%%%%%%
\usepackage{verbatim}
\makeindex
% \usepackage{sectsty}
% فراخوانی بسته زی‌پرشین و تعریف قلم فارسی و انگلیسی
\usepackage[%
  localise,%
  fontsize={10,12},%
  % perpagefootnote=on,%
]{xepersian}
\SepMark{-}
%حتماً از تک لایو 2014 استفاده کنید.
\settextfont[
  Path = fonts/,
  UprightFont = *-Regular,
  BoldFont = *-Bold,
  ItalicFont = *-Variable
]{Vazir}

\setlatintextfont[
  Path = fonts/,
  UprightFont = *-Regular,
  BoldFont = *-Bold,
  ItalicFont = *-Italic
]{Neuton}

\renewcommand{\baselinestretch}{1.5}
\renewcommand{\labelitemi}{$\bullet$}
%%%%%%%%%%%%%%%%%%%%%%%%%%
%\setmathdigitfont{IRXLotus}[Scale=1.1]
%%%%%%%%%%%%%%%%%%%%%%%%%%
% تعریف قلم‌های فارسی اضافی برای استفاده در بعضی از قسمت‌های متن
\defpersianfont\nastaliq[Scale=2,Path=fonts/]{IranNastaliq}
\defpersianfont\chapternumber[
  Scale=3,
  Path=fonts/,
  UprightFont = *-Regular,
  BoldFont = *-Bold,
  ItalicFont = *-Variable
]{Vazir}
%%%%%%%%%%%%%%%%%%%%%%%%%%
% دستوری برای تغییر نام کلمه «اثبات» به «برهان»
\renewcommand\proofname{\textbf{برهان}}

% دستوری برای تغییر نام کلمه «کتاب‌نامه» به «منابع و مراجع«
\renewcommand{\bibname}{منابع و مراجع}


% Headings for every page of ToC, LoF and Lot
\setlength{\cftbeforetoctitleskip}{-1.2em}
\setlength{\cftbeforelottitleskip}{-1.2em}
\setlength{\cftbeforeloftitleskip}{-1.2em}
\setlength{\cftaftertoctitleskip}{-1em}
\setlength{\cftafterlottitleskip}{-1em}
\setlength{\cftafterloftitleskip}{-1em}
%%\makeatletter
%%%%\renewcommand{\l@chapter}{\@dottedtocline{1}{1em\bfseries}{1em}}
%%%%\renewcommand{\l@section}{\@dottedtocline{2}{2em}{2em}}
%%%%\renewcommand{\l@subsection}{\@dottedtocline{3}{3em}{3em}}
%%%%\renewcommand{\l@subsubsection}{\@dottedtocline{4}{4em}{4em}}
%%%%\makeatother

\newcommand\tocheading{\par عنوان\hfill صفحه \par}
\newcommand\lofheading{\hspace*{.5cm}\figurename\hfill صفحه \par}
\newcommand\lotheading{\hspace*{.5cm}\tablename\hfill صفحه \par}

\renewcommand{\cftchapleader}{\cftdotfill{\cftdotsep}}
\renewcommand{\cfttoctitlefont}{\hspace*{\fill}\LARGE\bfseries}%\Large
\renewcommand{\cftaftertoctitle}{\hspace*{\fill}}
\renewcommand{\cftlottitlefont}{\hspace*{\fill}\LARGE\bfseries}%\Large
\renewcommand{\cftafterlottitle}{\hspace*{\fill}}
\renewcommand{\cftloftitlefont}{\hspace*{\fill}\LARGE\bfseries}
\renewcommand{\cftafterloftitle}{\hspace*{\fill}}

%%%%%%%%%%%%%%%%%%%%%%%%%%
% تعریف و نحوه ظاهر شدن عنوان قضیه‌ها، تعریف‌ها، مثال‌ها و ...
%برای شماره گذاری سه تایی قضیه ها
\theoremstyle{definition}
\newtheorem{definition}{تعریف}[section]
\newtheorem{remark}[definition]{نکته}
\newtheorem{note}[definition]{یادداشت}
\newtheorem{example}[definition]{نمونه}
\newtheorem{question}[definition]{سوال}
\newtheorem{remember}[definition]{یاداوری}
%\theoremstyle{theorem}
\newtheorem{theorem}[definition]{قضیه}
\newtheorem{lemma}[definition]{لم}
\newtheorem{proposition}[definition]{گزاره}
\newtheorem{corollary}[definition]{نتیجه}
%%%%%%%%%%%%%%%%%%%%%%%%
%%%%%%%%%%%%%%%%%%%
%%% برای شماره گذاری چهارتایی قضیه ها و ...
%%\newtheorem{definition1}[subsubsection]{تعریف}
%%\newtheorem{theorem1}[subsubsection]{قضیه}
%%\newtheorem{lemma1}[subsubsection]{لم}
%%\newtheorem{proposition1}[subsubsection]{گزاره}
%%\newtheorem{corollary1}[subsubsection]{نتیجه}
%%\newtheorem{remark1}[subsubsection]{نکته}
%%\newtheorem{example1}[subsubsection]{مثال}
%%\newtheorem{question1}[subsubsection]{سوال}

%%%%%%%%%%%%%%%%%%%%%%%%%%%%

% دستورهایی برای سفارشی کردن صفحات اول فصل‌ها
\makeatletter
\newcommand\mycustomraggedright{%
 \if@RTL\raggedleft%
 \else\raggedright%
 \fi}
\def\@makechapterhead#1{%
\thispagestyle{main}
\vspace*{20\p@}%
{\parindent \z@ \mycustomraggedright %\@mycustomfont
\ifnum \c@secnumdepth >\m@ne
\if@mainmatter

\bfseries{\Huge \@chapapp}\small\space {\chapternumber\thechapter}
\par\nobreak
\vskip 0\p@
\fi
\fi
\interlinepenalty\@M
\Huge \bfseries #1\par\nobreak
\vskip 120\p@

}

%\thispagestyle{empty}
\newpage}
\bidi@patchcmd{\@makechapterhead}{\thechapter}{\tartibi{chapter}}{}{}
\bidi@patchcmd{\chaptermark}{\thechapter}{\tartibi{chapter}}{}{}
\makeatother

\pagestyle{fancy}
\renewcommand{\chaptermark}[1]{\markboth{\chaptername~\tartibi{chapter}: #1}{}}

\setlength{\headheight}{13pt}

\fancypagestyle{main}{%
  \fancyhf{}
  \fancyfoot[c]{\thepage}
  \fancyhead[R]{\leftmark}
  \renewcommand{\headrulewidth}{1.2pt}
}

\fancypagestyle{version}{
  \fancyhf{}
  \fancyfoot[C]{\thepage}
  \renewcommand{\headrulewidth}{0pt}
}

\fancypagestyle{abstract}{
  \fancyhf{}
  \fancyhead[R]{چکیده}
  \fancyfoot[C]{\thepage}
  \renewcommand{\headrulewidth}{1.2pt}
}

\fancypagestyle{style3}{%
  \fancyhf{}%
  \fancyhead[R]{فهرست نمادها}
  \fancyfoot[C]{\thepage}
  \renewcommand{\headrulewidth}{1.2pt}
}

\fancypagestyle{list-of-tables}{%
  \fancyhf{}
  \fancyhead[R]{فهرست جداول}
  \fancyfoot[C]{\thepage}
  \renewcommand{\headrulewidth}{1.2pt}
}

\fancypagestyle{list-of-figures}{
  \fancyhf{}
  \fancyhead[R]{فهرست اشکال}
  \fancyfoot[C]{\thepage}
  \renewcommand{\headrulewidth}{1.2pt}
}

\fancypagestyle{table-of-contents}{
  \fancyhf{}
  \fancyhead[R]{فهرست مطالب}
  \fancyfoot[C]{\thepage}
  \renewcommand{\headrulewidth}{1.2pt}
}

\fancypagestyle{index}{
  \fancyhf{}
  \fancyhead[R]{نمایه}
  \fancyfoot[C]{\thepage}
  \renewcommand{\headrulewidth}{1.2pt}
}

\fancypagestyle{references}{
  \fancyhf{}
  \fancyhead[R]{منابع و مراجع}
  \fancyfoot[C]{\thepage}
  \renewcommand{\headrulewidth}{1.2pt}
}

%دستور حذف نام لیست تصاویر و لیست جداول از فهرست مطالب
\newcommand*{\BeginNoToc}{%
  \addtocontents{toc}{%
    \edef\protect\SavedTocDepth{\protect\the\protect\value{tocdepth}}%
  }%
  \addtocontents{toc}{%
    \protect\setcounter{tocdepth}{-10}%
  }%
}
\newcommand*{\EndNoToc}{%
  \addtocontents{toc}{%
    \protect\setcounter{tocdepth}{\protect\SavedTocDepth}%
  }%
}
\newcounter{savepage}
\renewcommand{\listfigurename}{فهرست اشکال}
\renewcommand{\listtablename}{فهرست جداول}
%\renewcommand\cftsecleader{\cftdotfill{\cftdotsep}}
