\فصل{مقدمه}

اینترنت اشیا اولین بار توسط \متن‌لاتین{Kevin Ashton} در سال ۱۹۹۹ پیشنهاد شد، اما تولد حقیقی آن
با توجه به تخمین \متن‌لاتین{Cisco} در بازه‌ای بین سال‌های ۲۰۰۸ تا ۲۰۰۹ بازمی‌گردد که برای اولین بار تعداد اشیا
متصل از جمعیت جهان بیشتر شد. در سال ۲۰۱۰ تعداد اشیا متصل تقریبا دو برابر جمعیت جهان در آن سال شد و تقریبا به عدد ۱۲/۵ بیلیون رسید.
از آن سال‌ها به لطف پیشرفت‌های تکنولوژی و سرمایه‌گذاری‌های قابل توجه شرکت‌ها، اینترنت اشیا در حال گسترش در زندگی روزمره است
\مرجع{Lombardi2021}.

در حال حاضر بیشتر شبکه‌های اینترنت اشیا از بسترهای \متن‌لاتین{WSN} مانند \متن‌لاتین{Zigbee}، \متن‌لاتین{Bluetooth} یا \متن‌لاتین{WiFi} ساخته شده‌اند.
این تکنولوژی‌ها با گسترش روزافزون اینترنت اشیا همخوانی ندارند، در آینده ما نیاز داریم تا هزینه هر واحد را کاهش دهیم، پوشش شبکه را گسترده‌تر کنیم، توان مصرفی نودهای در لبه را کاهش دهیم و شبکه‌های گسترش‌پذیر داشته باشیم.
الگو جدیدی تحت عنوان \متن‌لاتین{LPWAN} یا \متن‌لاتین{Low-Power Wide Area Network} برای پوشش دادن این شکاف به وجود آمده است \مرجع{SanchezIborra2016}.

به صورت کلی تکنولوژی‌های سنتی در ارتباطات اینترنت اشیا را می‌توان به دو دسته برد بلند و برد کوتاه تقسیم کرد. در ارتباطات برد کوتاه مشکل اصلی عدم توانایی شبکه برای پوشش گسترده می‌باشد و در صورتی که قصد چنین کاری را داشته باشیم
باید شبکه‌ها را به وسیله‌ی واسطه‌هایی مانند \متن‌لاتین{IP} به یکدیگر متصل کنیم. در شبکه‌های برد بلند به صورت سنتی از تکنولوژی‌هایی مانند \متن‌لاتین{3G/4G} یا \متن‌لاتین{GSM} یا \نقاط‌خ استفاده می‌شود.
این تکنولوژی‌ها برد بالایی را پشتیبانی می‌کنند ولی هزینه زیادی دارند و مصرف آن‌ها زیاد است و همین امر دقیقا برای ارتباطات ماهواره‌ای نیز مطرح است. اما در \متن‌لاتین{LPWAN}ها پوشش گسترده با توان مصرفی و پیچیدگی پایین مدنظر است.
البته استفاده از \متن‌لاتین{LPWAN}ها خالی از ایراد نبوده و بحث چگالی یا تعداد نودها در این شبکه‌ها، کیفیت سرویس و مدیریت منابع رادیویی و \نقاط‌خ در این شبکه‌ها مطرح است.

در این سال‌های شبکه‌های مختلفی تحت عنوان \متن‌لاتین{LPWAN} پیشنهاد شده‌اند که از جمله‌ی آن‌ها می‌توان به \متن‌لاتین{LoRa}، \متن‌لاتین{Sigfox} و \متن‌لاتین{NB-IoT} اشاره کرد.
با توجه به استفاده از باند فرکانسی بدون مجوز در \متن‌لاتین{LoRa} و امکان راه‌اندازی شبکه‌های \متن‌لاتین{LoRa} توسط اشخاص ثالث پژوهش‌های زیادی در این سال‌ها به این نوع شبکه‌ها پرداخته‌اند.

\متن‌لاتین{LoRa} توسط \متن‌لاتین{Semtech} ثبت شده است و برای همگان انتشار نیافته است اما \متن‌لاتین{LoRaWAN} که یک لایه پیوند داده بر پایه لایه فیزیکی \متن‌لاتین{LoRa}
است، یک استاندارد رایگان بوده و در اختیار همگان است. شبکه‌های \متن‌لاتین{LoRaWAN} در قالب \متن‌لاتین{Star of Stars} فعالیت می‌کنند و هر نود با یک گام به شبکه متصل می‌شود.
در \متن‌لاتین{LoRaWAN} ارتباطی میان \متن‌لاتین{Gateway} و نودها وجود ندارد و داده‌ی ارسالی توسط هر نود می‌تواند به وسیله‌ی یک یا چند \متن‌لاتین{Gateway} دریافت شود.
سرور شبکه داده‌های این \متن‌لاتین{Gateway}ها را دریافت می‌کند و عملیات شناسایی، رمزگشایی و حذف بسته‌های تکراری را انجام می‌دهد.

هدف این رساله ارزیابی انتها به انتها شبکه‌های \متن‌لاتین{LoRa} است. این شبکه‌های اجزای مختلفی دارند و انتخاب‌های زیادی را پیش روی معماران قرار می‌دهند که هر یک از این تصمیمات
تاثیر منحصر به فردی در کارایی انتها به انتها سیستم دارد.
