\فصل{مقدمه}

\قسمت{مقدمه}

\قسمت{اینترنت اشیا}

\قسمت{کاربردهای اینترنت اشیا}

\زیرقسمت{شهر هوشمند}

\زیرزیرقسمت{مدیریت هوشمند پسماند}

با گسترش مصرف و بزرگتر شدن شهرها یکی از موارد مهم در شهرهای هوشمند مدیریت پسماند می‌باشد. مدیریت هوشمند پسماند در ساده‌ترین سطح شامل نظارت بر میزان پر بودن سطل‌های آشغال
و کمک به مسیریابی بهینه ماشین جمع‌اوری زباله می‌باشد. بدین ترتیب جلوی خالی کردن زود هنگام سطل‌ها و از سوی دیگر دیرکرد در خالی کردن سطل‌هایی که بیش از اندازه پر شده‌اند گرفته خواهد شد.

در کنار میزان پر بودن می‌توان دما سطل‌ها یا میزان گاز $CO_{2}$ آن‌ها را نیز اندازه‌گیری کرد. از سوی دیگر با استفاده از الگوریتم‌های هوش‌مصنوعی می‌توان فرآیند جمع‌اوری زباله‌ها را بهینه‌سازی نمود.
البته گاز کربن دی اکسید تنها گازی نیست که از سطل‌های عمومی زباله منتشر می‌گردد بلکه گازهای دیگری نیز وجود دارند که می‌توانند باعث بوی بد سطح شوند.

یکی از مسائل مهم در این حوزه تکنولوژی ارتباطی است، برای ارتباط می‌توان از تکنولوژی‌های سلولی چون \متن‌لاتین{GSM} یا \متن‌لاتین{GPRS} استفاده کرد که البته هزینه‌ی زیادی دارد.
از سوی دیگر گزینه استفاده از \متن‌لاتین{WiFi} با هزینه پایین وجود دارد که البته برد آن نیز بسیار کم خواهد بود.
