\فصل{مقدمه}

اینترنت اشیا اولین بار توسط \متن‌لاتین{Kevin Ashton} در سال ۱۹۹۹ پیشنهاد شد، اما تولد حقیقی آن
با توجه به تخمین \متن‌لاتین{Cisco} به بازه‌ای بین سال‌های ۲۰۰۸ تا ۲۰۰۹ بازمی‌گردد که برای اولین بار تعداد اشیا
متصل از جمعیت جهان در آن سال‌ها بیشتر شد. در سال ۲۰۱۰ تعداد اشیا متصل تقریبا دو برابر جمعیت جهان در آن سال شد و تقریبا به عدد ۱۲/۵ بیلیون رسید.
از آن سال‌ها به لطف پیشرفت‌های تکنولوژی و سرمایه‌گذاری‌های قابل توجه شرکت‌ها، اینترنت اشیا در حال گسترش در زندگی روزمره است
\مرجع{Lombardi2021}.

در حال حاضر بیشتر شبکه‌های اینترنت اشیا از بسترهای \متن‌لاتین{WSN} مانند \متن‌لاتین{Zigbee}، \متن‌لاتین{Bluetooth} یا \متن‌لاتین{WiFi} ساخته شده‌اند.
این تکنولوژی‌ها با گسترش روزافزون اینترنت اشیا همخوانی ندارند، در آینده ما نیاز داریم تا هزینه هر واحد را کاهش دهیم، پوشش شبکه را گسترده‌تر کنیم، توان مصرفی نودهای در لبه را کاهش دهیم و شبکه‌های گسترش‌پذیر داشته باشیم.
الگو جدیدی تحت عنوان \متن‌لاتین{LPWAN} یا \متن‌لاتین{Low-Power Wide Area Network} برای پوشش دادن این شکاف به وجود آمده است \مرجع{SanchezIborra2016}.

به صورت کلی تکنولوژی‌های سنتی در ارتباطات اینترنت اشیا را می‌توان به دو دسته برد بلند و برد کوتاه تقسیم کرد. در ارتباطات برد کوتاه مشکل اصلی عدم توانایی شبکه برای پوشش گسترده است و در صورتی که قصد چنین کاری را داشته باشیم
باید شبکه‌ها را به وسیله‌ی واسطه‌هایی مانند \متن‌لاتین{IP} به یکدیگر متصل کنیم. در شبکه‌های برد بلند به صورت سنتی از تکنولوژی‌هایی مانند \متن‌لاتین{3G/4G} یا \متن‌لاتین{GSM} یا \نقاط‌خ استفاده می‌شود.
این تکنولوژی‌ها برد بالایی را پشتیبانی می‌کنند ولی هزینه زیادی دارند و مصرف آن‌ها زیاد است و همین امر دقیقا برای ارتباطات ماهواره‌ای نیز مطرح است. اما در \متن‌لاتین{LPWAN}ها پوشش گسترده با توان مصرفی و پیچیدگی پایین مدنظر است.
البته استفاده از \متن‌لاتین{LPWAN}ها خالی از ایراد نبوده و بحث چگالی یا تعداد نودها در این شبکه‌ها، کیفیت سرویس و مدیریت منابع رادیویی و \نقاط‌خ در این شبکه‌ها مطرح است.

در این سال‌های شبکه‌های مختلفی تحت عنوان \متن‌لاتین{LPWAN} پیشنهاد شده‌اند که از جمله‌ی آن‌ها می‌توان به \متن‌لاتین{LoRa}، \متن‌لاتین{Sigfox} و \متن‌لاتین{NB-IoT} اشاره کرد.
با توجه به استفاده از باند فرکانسی بدون مجوز در \متن‌لاتین{LoRa} و امکان راه‌اندازی شبکه‌های \متن‌لاتین{LoRa} توسط اشخاص ثالث پژوهش‌های زیادی در این سال‌ها به این نوع شبکه‌ها پرداخته‌اند.
\متن‌لاتین{LoRa} بیشترین تکنولوژی استفاده شده در کاربردهای اینترنت اشیا با برد بالا، نرخ داده کم و توان مصرفی پایین است \مرجع{Almojamed2021}.
از سوی دیگر \متن‌لاتین{LoRaWAN} نیز از بین تکنولوژی‌های \متن‌لاتین{LPWAN} بیشترین انطباق را برای اینرتنت اشیا دارد \مرجع{Fujdiak2022}.

\متن‌لاتین{LoRa} توسط \متن‌لاتین{Semtech} ثبت شده است و برای همگان انتشار نیافته است اما \متن‌لاتین{LoRaWAN} که یک لایه پیوند داده بر پایه لایه فیزیکی \متن‌لاتین{LoRa}
است، یک استاندارد رایگان (منتشر شده از سوی \متن‌لاتین{LoRa Alliance} در سال ۲۰۱۵) بوده و در اختیار همگان است.
شبکه‌های \متن‌لاتین{LoRaWAN} با همبندی \متن‌لاتین{Star of Stars} فعالیت می‌کنند و هر نود با یک گام به شبکه متصل می‌شود.
در \متن‌لاتین{LoRaWAN} ارتباطی میان \متن‌لاتین{Gateway} و نودها وجود ندارد و داده‌ی ارسالی توسط هر نود می‌تواند به وسیله‌ی یک یا چند \متن‌لاتین{Gateway} دریافت شود.
\متن‌لاتین{Gateway}ها وظیفه ارسال ترافیک دستگاه‌های \متن‌لاتین{LoRa} به سرور شبکه و برعکس را دارا است.
عملیات احراز هویت، شناسایی و حذف بسته‌های تکراری توسط سرور شبکه صورت می‌پذیرد که در عین حال وظیفه کنترل
تنظیمات شبکه به وسیله‌ی مکانیزم نرخ داده تطبیق‌پذیر را نیز انجام می‌دهد \مرجع{Almojamed2021}.

پارامترهای زیادی برای مشخص کردن ماژولیشن \متن‌لاتین{LoRa} وجود دارد. تنظیم کردن این پارامترها
مصالحه‌ای میان توان مصرفی، برد ارتباطی و میزان داده‌ی انتقال یافته است.
این پارامترها شامل پهنای باند، فاکتور گسترش (\متن‌لاتین{SF})، نرخ کدگذاری (\متن‌لاتین{CR}) و توان ارسال است.
پهنای باند می‌تواند ۵۰۰، ۲۵۰ یا ۱۲۵ کیلوهرتز باشد.
فاکتور گسترش نسبت بین نرخ علامت و نرخ \متن‌لاتین{chirp} است که بر تعداد بیت‌های کد شده در هر علامت دلالت می‌کند و می‌تواند مقداری بین ۷ تا ۱۲ داشته باشد.
فاکتورهای گسترش بالاتر زمان ارسال بسته را افزایش می‌دهند که باعث می‌شود توان مصرفی بیشتر شده و نرخ داده کاهش پیدا کند و البته برد ارسال را افزایش می‌دهند.
نرخ کدگذاری توسط دستگاه دریافت کننده \متن‌لاتین{LoRa} مورد استفاده قرار می‌گیرد تا به واسطه انجام تصحیح خطای رو به جلو، تاثیر نویز بر داده‌های دریافتی را از بین ببرد.
نرخ کدگذاری می‌تواند مقدارهای $4/5$، $4/6$، $4/7$ و $4/8$ را اختیار کند
\مرجع{Almojamed2021}.

در \متن‌لاتین{LoRaWAN} برای مدیریت دسترسی همزمان از \متن‌لاتین{ALOHA} استفاده می‌شود.
\متن‌لاتین{ALOHA} به نود اجازه می‌دهد به محض بیدار شدن داده‌ی خود را ارسال کرده و در صورت وقوع تصادم
از عقب‌نشینی استفاده می‌شود
\مرجع{Kufakunesu2020}.

دستگاه‌های انتهایی در \متن‌لاتین{LoRaWAN} می‌توانند در سه کلاس عملیاتی فعالیت کنند. کلاس $A$ که در آن مصرف توان کمینه بوده و بیشترین تاخیر
برای دریافت \متن‌لاتین{downlink} وجود دارد. کلاس $B$ با افزایش توان مصرفی اجازه هماهنگی دریافت \متن‌لاتین{downlink} را فراهم می‌آورد
و در نهایت کلاس $C$ که بیشترین توان مصرفی را داشته و کمترین تاخیر در دریافت \متن‌لاتین{downlink} وجود دارد.

پژوهش‌های صورت گرفته در حوزه‌ی \متن‌لاتین{LoRa} را می‌توان به چند دسته کلی تقسیم کرد. دسته اول پژوهش‌هایی هستند که به ارزیابی این شبکه با تنظیمات مشخصی
به صورت شبیه‌سازی یا عملیاتی پرداخته‌اند. این پژوهش‌ها که از جمله‌ی آن‌ها می‌توان به \مرجع{Almojamed2021}، \مرجع{Baldo2021}، \مرجع{sensors-20-02078}
و \نقاط‌خ اشاره کرد، یک استقرار از شبکه‌ی \متن‌لاتین{LoRa}
را با پارامترهای گوناگون نظیر فاکتور گسترش، نرخ ارسال اشیا، جابجایی اشیا، پروتکل لایه کاربرد، تعداد اشیا و \نقاط‌خ مورد ارزیابی قرار داده و با استفاده از آن گزارش‌های مختلفی را
برای پارامترهایی چون نرخ دریافت صحیح، حداکثر برد ارتباطی و \نقاط‌خ تهیه می‌کنند.
دسته دوم پژوهش‌هایی هستند که به مساله دسترسی همزمان در شبکه‌های \متن‌لاتین{LoRaWAN} می‌پردازند. این پژوهش‌ها که از جمله‌ی آن‌ها می‌توان به \مرجع{Beltramelli2021}، \مرجع{Lee2021}، \مرجع{Polonelli2019} و \نقاط‌خ
اشاره کرد با پیشنهاد الگوریتم‌های دسترسی همزمان مختلف سعی می‌کنند پارامترهایی مانند نرخ دریافت صحیح، مصرف توان و \نقاط‌خ را بهینه کنند. روش پیشنهادی این پژوهش‌ها گاه با ساختار فعلی
شبکه‌های \متن‌لاتین{LoRaWAN} سازگار بوده و یا قابلیت پیاده‌سازی دارند که به صورت عملی مورد ارزیابی قرار می‌گیرند و گاه با توجه به پیچیدگی پیاده‌سازی آن به صورت شبیه‌سازی ارزیابی می‌شوند.
دسته سوم پژوهش‌هایی هستند که الگوریتم‌های متفاوتی را برای نرخ داده تطبیق‌پذیر پیشنهاد و ارزیابی می‌کنند. از جمله این پژوهش‌ها می‌توان به \مرجع{sensors-20-03061-v2} اشاره کرد.
دسته چهارم پژهش‌هایی هستند که با در نظرگرفتن لایه فیزیکی \متن‌لاتین{LoRa} سعی در طراحی یک شبکه \متن‌لاتین{Mesh} می‌نمایند. در این شبکه اشیا می‌توانند
به صورت مستقیم برای یکدیگر داده ارسال کنند و با تکیه بر یکدیگر داده‌ها را تا \متن‌لاتین{Gateway} جابجا کنند. از جمله این پژوهش‌ها می‌توان به \مرجع{Marahatta2021}، \مرجع{Famaey2018} و \نقاط‌خ اشاره کرد.
گاه این پژوهش‌ها در کنار \متن‌لاتین{LoRa} از پروتکل‌های دیگری نیز استفاده می‌کنند و شبکه‌ای با چند پروتکل را تشکیل می‌دهند.
دسته پنجم از پژوهش‌ها به بحث پردازش در لبه برای شبکه‌های \متن‌لاتین{LoRa} می‌پردازند. پردازش در لبه با نزدیک کردن منابع پردازشی به منبع اصلی داده‌ها زمان پاسخ و پردازش را کاهش می‌دهد.
از جمله‌ی این پژوهش‌ها می‌توان به \مرجع{Taleb2017} اشاره کرد.

هدف این رساله ارزیابی انتها به انتها شبکه‌های \متن‌لاتین{LoRa} است. این شبکه‌های اجزای مختلفی دارند و انتخاب‌های زیادی را پیش روی معماران قرار می‌دهند که هر یک از این تصمیمات
تاثیر منحصر به فردی در کارایی انتها به انتها سیستم دارد.

در ادامه این رساله، ابتدا به مرور مفاهیم مورد استفاده می‌پردازیم و سعی می‌کنیم مفاهیمی چون اینترنت اشیا، شبکه‌های برد بالا با توان پایین و \نقاط‌خ را بررسی نماییم. در ادامه
کارهای مرتبط با حوزه شبکه‌های برد بالا و توان پایین را با تمرکز بر شبکه‌های \متن‌لاتین{LoRa} مرور می‌کنیم و در نهایت بیان مساله می‌پردازیم.
