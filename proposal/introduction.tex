\فصل{مقدمه}

\قسمت{اینترنت اشیا}

با آمدن اینترنت اشیا و ارتباطات ماشین به ماشین انتظار می‌رود به زودی افزایش زیادی در تعداد نودها دیده شود. پیش‌بینی می‌شود تا سال ۲۰۲۵ بیش از ۷۵ بیلیون نود اینترنت اشیا داشته باشیم.
این نودها شامل ماشین‌ها، حسگرها، شمارشگرها، دستگاه‌های فروش و \نقاط‌خ می‌باشند.
\مرجع{Chaudhari2020}

در حال حاضر بیشتر شبکه‌های اینترنت اشیا از بسترهای \متن‌لاتین{WSN} مانند \متن‌لاتین{Zigbee}، \متن‌لاتین{Bluetooth} یا \متن‌لاتین{WiFi} ساخته شده‌اند.
این تکنولوژی‌ها با گسترش روزافزون اینترنت اشیا همخوانی ندارند، در آینده ما نیاز داریم تا هزینه هر واحد را کاهش دهیم، پوشش شبکه را گسترده کرده‌تر کنیم، توان مصرفی نودهای در لبه را کاهش دهیم و شبکه‌های گسترش‌پذیر داشته باشیم.
الگو جدیدی تحت عنوان \متن‌لاتین{LP-WAN} یا \متن‌لاتین{Low-Power Wide Area Network} برای پوشش دادن این شکاف به وجود آمده است. \مرجع{SanchezIborra2016}

به صورت کلی تکنولوژی‌های سنتی در ارتباطات اینترنت اشیا را می‌توان به دو دسته برد بلند و برد کوتاه تقسیم کرد. در ارتباطات برد کوتاه مشکل اصلی عدم توانایی شبکه برای پوشش گسترده می‌باشد و در صورتی که قصد چنین کاری را داشته باشیم
باید شبکه‌ها را به وسیله‌ی واسطه‌هایی مانند \متن‌لاتین{IP} به یکدیگر متصل کنیم. در شبکه‌های برد بلند به صورت سنتی از تکنولوژی‌هایی مانند \متن‌لاتین{3G/4G} یا \متن‌لاتین{GSM} یا \نقاط‌خ استفاده می‌شود.
این تکنولوژی‌ها برد بالایی را پشتیبانی می‌کنند ولی هزینه زیادی دارند و مصرف آن‌ها زیاد است و همین امر دقیقا برای ارتباطات ماهواره‌ای نیز مطرح است.

البته استفاده از \متن‌لاتین{LPWAN}ها خالی از ایراد نبوده و بحث چگالی یا تعداد نودها در این شبکه‌ها، کیفیت سرویس و مدیریت منابع رادیویی و \نقاط‌خ در این شبکه‌ها مطرح است.
در ادامه این رساله در رابطه با این نیازمندی‌ها بیشتر صحبت خواهد شد.
