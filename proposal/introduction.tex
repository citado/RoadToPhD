\فصل{مقدمه}

اینترنت اشیا اولین بار توسط \متن‌لاتین{Kevin Ashton} در سال ۱۹۹۹ پیشنهاد شد، اما تولد حقیقی آن
با توجه به تخمین \متن‌لاتین{Cisco} به بازه‌ای بین سال‌های ۲۰۰۸ تا ۲۰۰۹ بازمی‌گردد که برای اولین بار تعداد اشیا
متصل از جمعیت جهان در آن سال‌ها بیشتر شد. در سال ۲۰۱۰ تعداد اشیا متصل تقریبا دو برابر جمعیت جهان در آن سال شد و تقریبا به عدد ۱۲/۵ بیلیون رسید.
از آن سال‌ها به لطف پیشرفت‌های فناوری و سرمایه‌گذاری‌های قابل توجه شرکت‌ها، اینترنت اشیا در حال گسترش در زندگی روزمره است
\مرجع{Lombardi2021}.

به صورت کلی فناوری‌های ارتباطی اینترنت اشیا به دو دسته برد بلند و برد کوتاه تقسیم می‌شوند. در ارتباطات برد کوتاه مشکل اصلی عدم توانایی شبکه برای پوشش گسترده است
که برای حل این مشکل باید شبکه‌ها را به وسیله‌ی واسطه‌هایی مانند \متن‌لاتین{IP} به یکدیگر متصل کنیم.
در شبکه‌های برد بلند به طور معمول از فناوری‌هایی مانند \متن‌لاتین{3G/4G} یا \متن‌لاتین{GSM} استفاده می‌شود.
این فناوری‌ها برد بالایی را پشتیبانی می‌کنند ولی هزینه زیادی دارند و مصرف آن‌ها زیاد است و همین امر دقیقا برای ارتباطات ماهواره‌ای نیز مطرح است.

در حال حاضر بیشتر شبکه‌های اینترنت اشیا از بسترهای شبکه‌های حسگر بی‌سیم مانند \متن‌لاتین{Zigbee}، \متن‌لاتین{Bluetooth} یا \متن‌لاتین{WiFi} ساخته شده‌اند.
این فناوری‌ها با گسترش روزافزون اینترنت اشیا همخوانی ندارند، در آینده ما نیاز داریم تا هزینه هر واحد را کاهش دهیم،
پوشش شبکه را گسترده‌تر کنیم، توان مصرفی گره‌های در لبه را کاهش دهیم و شبکه‌های گسترش‌پذیر داشته باشیم \مرجع{SanchezIborra2016}.

الگو جدیدی تحت عنوان \متن‌لاتین{LPWAN} یا \متن‌لاتین{Low-Power Wide Area Network} برای پوشش دادن این شکاف به وجود آمده است \مرجع{SanchezIborra2016}.
در \متن‌لاتین{LPWAN}ها پوشش گسترده با توان مصرفی و پیچیدگی پایین مدنظر است.

در این سال‌ها فناوری‌های \متن‌لاتین{LPWAN} متعددی پیشنهاد شده‌اند که از جمله‌ی آن‌ها می‌توان به \متن‌لاتین{LoRa}، \متن‌لاتین{Sigfox} و \متن‌لاتین{NB-IoT} اشاره کرد.
استفاده از \متن‌لاتین{LPWAN}ها خالی از ایراد نبوده و بحث چگالی یا تعداد زیاد گره‌ها در این شبکه‌ها، کیفیت سرویس و مدیریت منابع رادیویی
از جمله چالش‌ها در این شبکه‌ها است.
با توجه به استفاده از باند فرکانسی بدون مجوز در \متن‌لاتین{LoRa} و امکان راه‌اندازی شبکه‌های \متن‌لاتین{LoRa} توسط اشخاص ثالث پژوهش‌های زیادی در این سال‌ها به این نوع شبکه‌ها پرداخته‌اند.
\متن‌لاتین{LoRa} بیشترین فناوری استفاده شده در کاربردهای اینترنت اشیا با برد بالا، نرخ داده کم و توان مصرفی پایین است \مرجع{Almojamed2021}.
از سوی دیگر \متن‌لاتین{LoRaWAN} نیز از بین فناوری‌های \متن‌لاتین{LPWAN} بیشترین انطباق را برای اینرتنت اشیا دارد \مرجع{Fujdiak2022}.

\متن‌لاتین{LoRa} توسط \متن‌لاتین{Semtech} ثبت شده است و برای همگان انتشار نیافته است اما \متن‌لاتین{LoRaWAN} که یک لایه پیوند داده بر پایه لایه فیزیکی \متن‌لاتین{LoRa}
است، یک استاندارد رایگان (منتشر شده از سوی \متن‌لاتین{LoRa Alliance} در سال ۲۰۱۵) بوده و در اختیار همگان است.
شبکه‌های \متن‌لاتین{LoRaWAN} با همبندی مشابه با ستاره فعالیت می‌کنند و هر گره با یک گام به شبکه متصل می‌شود.
در \متن‌لاتین{LoRaWAN} ارتباطی میان دروازه و گرهها وجود ندارد و داده‌ی ارسالی توسط هر گره می‌تواند به وسیله‌ی یک یا چند دروازه دریافت شود.
دروازهها وظیفه ارسال ترافیک دستگاه‌های \متن‌لاتین{LoRa} به سرور شبکه و برعکس را دارا است.
عملیات احراز هویت، شناسایی و حذف بسته‌های تکراری توسط سرور شبکه صورت می‌پذیرد که در عین حال وظیفه کنترل
تنظیمات شبکه به وسیله‌ی مکانیزم نرخ داده تطبیق‌پذیر را نیز انجام می‌دهد \مرجع{Almojamed2021}.

ماژولیشن \متن‌لاتین{LoRa} مبتنی بر روش \متن‌لاتین{Chirp Spread Spectrum} است. تنظیم کردن پارامترهای این ماژولیشن
مصالحه‌ای میان توان مصرفی، برد ارتباطی و نرخ داده‌ی انتقال یافته است.
این پارامترها شامل پهنای باند، فاکتور گسترش (\متن‌لاتین{SF})، نرخ کدگذاری (\متن‌لاتین{CR}) و توان ارسال است.
پهنای باند می‌تواند ۵۰۰، ۲۵۰ یا ۱۲۵ کیلوهرتز باشد.
فاکتور گسترش نسبت بین نرخ علامت و نرخ \متن‌لاتین{chirp} است که بر تعداد بیت‌های کد شده در هر علامت دلالت می‌کند و می‌تواند مقداری بین ۷ تا ۱۲ داشته باشد.
فاکتورهای گسترش بالاتر زمان ارسال بسته را افزایش می‌دهند که باعث می‌شود توان مصرفی بیشتر شده و نرخ داده کاهش پیدا کند و البته برد ارسال را افزایش می‌دهند.
نرخ کدگذاری توسط دستگاه دریافت کننده \متن‌لاتین{LoRa} مورد استفاده قرار می‌گیرد تا به واسطه انجام تصحیح خطای رو به جلو، تاثیر نویز بر داده‌های دریافتی را از بین ببرد.
نرخ کدگذاری می‌تواند مقدارهای $4/5$، $4/6$، $4/7$ و $4/8$ را اختیار کند
\مرجع{Almojamed2021}.

در \متن‌لاتین{LoRaWAN} برای مدیریت دسترسی همزمان از \متن‌لاتین{ALOHA} استفاده می‌شود.
\متن‌لاتین{ALOHA} به گره اجازه می‌دهد به محض بیدار شدن داده‌ی خود را ارسال کرده و در صورت وقوع تصادم
از عقب‌نشینی استفاده می‌شود
\مرجع{Kufakunesu2020}.

پژوهش‌های صورت گرفته در حوزه‌ی \متن‌لاتین{LoRa} را می‌توان به چند دسته کلی تقسیم کرد. دسته اول پژوهش‌هایی هستند که به ارزیابی این شبکه با تنظیمات مشخصی
به صورت شبیه‌سازی یا عملیاتی پرداخته‌اند. این پژوهش‌ها که از جمله‌ی آن‌ها می‌توان به \مرجع{Almojamed2021}، \مرجع{Baldo2021} و \مرجع{sensors-20-02078}
اشاره کرد، عملکرد یک استقرار از شبکه‌ی \متن‌لاتین{LoRa}
را با پارامترهای گوناگون نظیر فاکتور گسترش، نرخ ارسال اشیا، جابجایی اشیا، پروتکل لایه کاربرد و تعداد اشیا
از نظر پارامترهایی چون نرخ دریافت صحیح داده، حداکثر برد ارتباطی و تاخیر انتها به انتها
مورد ارزیابی قرار داده‌اند.
دسته دوم پژوهش‌هایی هستند که به مساله دسترسی همزمان در شبکه‌های \متن‌لاتین{LoRaWAN} می‌پردازند. این پژوهش‌ها که از جمله‌ی آن‌ها می‌توان به \مرجع{Beltramelli2021}، \مرجع{Lee2021} و \مرجع{Polonelli2019}
اشاره کرد با پیشنهاد الگوریتم‌های دسترسی همزمان مختلف سعی می‌کنند پارامترهایی مانند نرخ دریافت صحیح و مصرف توان را بهینه کنند. روش پیشنهادی این پژوهش‌ها گاه با ساختار فعلی
شبکه‌های \متن‌لاتین{LoRaWAN} سازگار بوده و یا قابلیت پیاده‌سازی دارند که به صورت عملی مورد ارزیابی قرار می‌گیرند و گاه با توجه به پیچیدگی پیاده‌سازی آن به صورت شبیه‌سازی ارزیابی می‌شوند.
دسته سوم پژوهش‌هایی هستند که الگوریتم‌های متفاوتی را برای نرخ داده تطبیق‌پذیر پیشنهاد و ارزیابی می‌کنند. از جمله این پژوهش‌ها می‌توان به \مرجع{sensors-20-03061-v2} اشاره کرد.
دسته چهارم پژهش‌هایی هستند که با در نظرگرفتن لایه فیزیکی \متن‌لاتین{LoRa} سعی در طراحی یک شبکه \متن‌لاتین{Mesh} می‌نمایند. در این شبکه اشیا می‌توانند
به صورت مستقیم برای یکدیگر داده ارسال کنند و با تکیه بر یکدیگر داده‌ها را تا دروازه جابجا کنند. از جمله این پژوهش‌ها می‌توان به \مرجع{Marahatta2021} و \مرجع{Famaey2018}
اشاره کرد.
گاه این پژوهش‌ها در کنار \متن‌لاتین{LoRa} از پروتکل‌های دیگری نیز استفاده می‌کنند و شبکه‌ای با چند پروتکل را تشکیل می‌دهند.
دسته پنجم از پژوهش‌ها به بحث پردازش در لبه برای شبکه‌های \متن‌لاتین{LoRa} می‌پردازند. پردازش در لبه با نزدیک کردن منابع پردازشی به منبع اصلی داده‌ها زمان پاسخ و پردازش را کاهش می‌دهد.
از جمله‌ی این پژوهش‌ها می‌توان به \مرجع{Taleb2017} اشاره کرد.

ارزیابی‌های صورت پذیرفته در این حوزه هر یک بخش مجزایی از معماری انتها به انتها \متن‌لاتین{LoRaWAN} را هدف قرار داده‌اند و از سوی دیگر شبکه‌ی هسته \متن‌لاتین{LoRaWAN}
که از آن به عنوان زیرساخت \متن‌لاتین{IP} نیز یاد می‌کنند
و از اهمیت بالایی برخوردار است چرا که قسمت مهمی از پردازش شبکه‌ی \متن‌لاتین{LoRaWAN} را صورت می‌دهد، کمتر مورد توجه و ارزیابی قرار گرفته است.
مواردی از جمله تاثیر تعداد اشیا و نرخ ارسال داده بر پارامترهای کیفیت سرویس مانند تاخیر انتها به انتها نیاز به ارزیابی دارند.
شبکه‌های \متن‌لاتین{LoRa} اجزای مختلفی دارند و انتخاب‌های زیادی را پیش روی معماران قرار می‌دهند
چرا که استانداردها در حوزه زیرساخت \متن‌لاتین{IP} بازتر هستند و
هر یک از این تصمیمات نیز
تاثیر منحصر به فردی در کارایی انتها به انتها سیستم دارد.

هدف این پیشنهاد رساله ارزیابی انتها به انتها زیرساخت \متن‌لاتین{IP} شبکه‌های \متن‌لاتین{LoRa}
و تاثیر ساختار داده لایه کاربر بر کارایی شبکه
است.
در این ارزیابی پارامترهایی از جمله نرخ ارسال داده، تعداد اشیا، جابجایی اشیا، اندازه داده‌ی ارسالی، تاخیر انتها به انتها و
نرخ دریافت صحیح اطلاعات مورد ارزیابی قرار خواهند گرفت.

در ادامه این پیشنهاد رساله، ابتدا به مرور مفاهیم مورد استفاده می‌پردازیم و سعی می‌کنیم مفاهیمی چون اینترنت اشیا، شبکه‌های برد بالا با توان پایین و
پروتکل‌های لایه کاربرد را بررسی نماییم. در ادامه
کارهای مرتبط با حوزه شبکه‌های برد بالا و توان پایین را با تمرکز بر شبکه‌های \متن‌لاتین{LoRa} مرور می‌کنیم و در نهایت بیان مساله می‌پردازیم.
