\فصل{مقدمه}

\قسمت{مقدمه}

\قسمت{اینترنت اشیا}

در حال حاضر بیشتر شبکه‌های اینترنت اشیا از بسترهای \متن‌لاتین{WSN} مانند \متن‌لاتین{Zigbee}، \متن‌لاتین{Bluetooth} یا \متن‌لاتین{WiFi} ساخته شده‌اند.
این تکنولوژی‌ها با گسترش روزافزون اینترنت اشیا همخوانی ندارند، در آینده ما نیاز داریم تا هزینه هر واحد را کاهش دهیم، پوشش شبکه را گسترده کرده‌تر کنیم، توان مصرفی نودهای در لبه را کاهش دهیم و شبکه‌های گسترش‌پذیر داشته باشیم.
الگو جدیدی تحت عنوان \متن‌لاتین{LP-WAN} یا \متن‌لاتین{Low-Power Wide Area Network} برای پوشش دادن این شکاف به وجود آمده است. \مرجع{SanchezIborra2016}

به صورت کلی تکنولوژی‌های سنتی در ارتباطات اینترنت اشیا را می‌توان به دو دسته برد بلند و برد کوتاه تقسیم کرد. در ارتباطات برد کوتاه مشکل اصلی عدم توانایی شبکه برای پوشش گسترده می‌باشد و در صورتی که قصد چنین کاری را داشته باشیم
باید شبکه‌ها را به وسیله‌ی واسطه‌هایی مانند \متن‌لاتین{IP} به یکدیگر متصل کنیم. در شبکه‌های برد بلند به صورت سنتی از تکنولوژی‌هایی مانند \متن‌لاتین{3G/4G} یا \متن‌لاتین{GSM} یا \نقاط‌خ استفاده می‌شود.
این تکنولوژی‌ها برد بالایی را پشتیبانی می‌کنند ولی هزینه زیادی دارند و مصرف آن‌ها زیاد است و همین امر دقیقا برای ارتباطات ماهواره‌ای نیز مطرح است.

\قسمت{کاربردهای اینترنت اشیا}

\زیرقسمت{کشاورزی هوشمند}

در کشاورزی هوشمند هدف افزایش کیفیت و حجم محصولات با استفاده از مدیریت هوشمند آبیاری، نظارت هوشمند و \نقاط‌خ می‌باشد.
در ساده‌ترین سطح کشاورزی هوشمند قرار دادن سنسورهایی برای رطوبت خاک، رسانایی و \نقاط‌خ در زمین کشاورزی می‌باشد. در سطوح بالاتر در کنار حسگرها، عملگرهایی برای آبیاری، سم‌پاشی و \نقاط‌خ تعریف می‌شوند.
امروزه \متن‌لاتین{UAV}ها در کشورهای پیشرفته جزئی از صنعت کشاورزی شده‌اند و می‌توانند به کشاورزان برای آبیاری، سم‌پاشی و نظارت بر محصولاتشان کمک کنند. مدیریت و برنامه‌ریزی این \متن‌لاتین{UAV}ها خود می‌تواند
در کشاورزی هوشمند صورت بپذیرد. در نهایت بحث کشاورزی دقیق نیز مطرح است که می‌توان از عکس‌های ماهواره‌ای و سنجش از دور برای تخمین محصول برداشتی، شناسایی آفات و \نقاط‌خ بهره جست.

با توجه به قرارگیری زمین‌های کشاورزی در مناطق غیرشهری و دورافتاده بحث نحوه ارتباط از چالش‌های مهم این حوزه می‌باشد، چرا که ارتباط‌های سلولی لزوما در این مناطق فعال نیستند.
از سوی دیگر ارتباط با \متن‌لاتین{UAV}ها با توجه به ماهیت متحرکی که دارند از دیگر چالش‌های این حوزه می‌باشد. در نهایت می‌توان به توان پردازشی مورد نیاز برای اجرای الگوریتم‌های پیش‌بینی
و \نقاط‌خ روی داده‌های جمع‌آوری شده اشاره کرد.

\زیرقسمت{صحن هوشمند دانشگاه}

\زیرقسمت{شهر هوشمند}

\زیرزیرقسمت{مدیریت هوشمند پسماند}

با گسترش مصرف و بزرگتر شدن شهرها یکی از موارد مهم در شهرهای هوشمند مدیریت پسماند می‌باشد. مدیریت هوشمند پسماند در ساده‌ترین سطح شامل نظارت بر میزان پر بودن سطل‌های آشغال
و کمک به مسیریابی بهینه ماشین جمع‌اوری زباله می‌باشد. بدین ترتیب جلوی خالی کردن زود هنگام سطل‌ها و از سوی دیگر دیرکرد در خالی کردن سطل‌هایی که بیش از اندازه پر شده‌اند گرفته خواهد شد.

در کنار میزان پر بودن می‌توان دما سطل‌ها یا میزان گاز $CO_{2}$ آن‌ها را نیز اندازه‌گیری کرد. از سوی دیگر با استفاده از الگوریتم‌های هوش‌مصنوعی می‌توان فرآیند جمع‌اوری زباله‌ها را بهینه‌سازی نمود.
البته گاز کربن دی اکسید تنها گازی نیست که از سطل‌های عمومی زباله منتشر می‌گردد بلکه گازهای دیگری نیز وجود دارند که می‌توانند باعث بوی بد سطح شوند.

یکی از مسائل مهم در این حوزه تکنولوژی ارتباطی است، برای ارتباط می‌توان از تکنولوژی‌های سلولی چون \متن‌لاتین{GSM} یا \متن‌لاتین{GPRS} استفاده کرد که البته هزینه‌ی زیادی دارد.
از سوی دیگر گزینه استفاده از \متن‌لاتین{WiFi} با هزینه پایین وجود دارد که البته برد آن نیز بسیار کم خواهد بود.

سطل‌های زباله همواره در معرض آسیب‌های عمدی و غیرعمدی قرار دارند و در صورتی که بتوان با هزینه کمی نسبت به هزینه ساخت آن‌ها از آن‌ها محافظت کرد، این هزینه حتما ارزش خواهد شد.
این امر یکی از محرک‌های بحث سطل‌های هوشمند در مدیریت هوشمند پسماند می‌باشد.
