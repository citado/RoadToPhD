\فصل{کارهای مرتبط}

\قسمت{مقدمه}

\قسمت{کارهای مرتبط}

\شروع{لوح}
\شرح{دسته‌بندی پژوهش‌های حوزه ارزیابی شبکه‌های توان پایین برد بلند}
\فضای‌و{5mm}
\کوچک
\تنظیم‌ازوسط
\begin{tabularx}
  {1.2\textwidth}
  {|*{13}{X|}}
  \خط‌پر
  مرجع &
  تحرک &
  پوشش‌دهی &
  شبکه &
  لایه انتقال &
  لایه اپلیکشن &
  اندازه بسته &
  \متن‌لاتین{Coding Rate} &
  تاخیر &
  محیط &
  \متن‌لاتین{LoRa Mesh} &
  توان مصرفی &
  شبیه‌سازی \\
  \خط‌پر
  \مرجع{sensors-18-00772-v3} &
  سیار &
  گزارش شده &
  \متن‌لاتین{LoRa} &
  ندارد &
  ندارد &
  گزارش شده &
  گزارش شده &
  گزارش نشده &
  باز &
  ندارد &
  گزارش نشده &
  واقعی / \متن‌لاتین{cloudRF} \\
  \خط‌پر
  \مرجع{sensors-19-00007} &
  ثابت &
  گزارش شده &
  \متن‌لاتین{NB-IoT} &
  \متن‌لاتین{TCP / UDP} &
  \متن‌لاتین{MQTT / CoAP} &
  گزارش شده &
  گزارش نشده &
  گزارش شده &
  باز &
  ندارد &
  گزارش نشده &
  \متن‌لاتین{Ericsson inter. sim.} \\
  \خط‌پر
  \مرجع{sensors-20-03061-v2} &
  ثابت &
  گزارش شده &
  \متن‌لاتین{LoRa} &
  ندارد &
  ندارد &
  گزارش شده &
  گزارش شده &
  گزارش شده &
  باز / بسته &
  ندارد &
  گزارش نشده &
  \متن‌لاتین{ns-3} \\
  \خط‌پر
  \مرجع{sensors-20-00280-v2} &
  ثابت &
  گزارش شده &
  \متن‌لاتین{LoRa} &
  \متن‌لاتین{UDP ov IPv6} &
  \متن‌لاتین{CoAP} &
  گزارش شده &
  گزارش شده &
  گزارش شده &
  باز / بسته &
  ندارد &
  گزارش شده &
  واقعی \\
  \خط‌پر
  \مرجع{sensors-20-06721} &
  ثابت &
  گزارش شده &
  \متن‌لاتین{LoRa} &
  ندارد &
  ندارد &
  گزارش شده &
  گزارش شده &
  گزارش شده &
  بسته &
  ندارد &
  گزارش شده &
  واقعی \\
  \خط‌پر
  \مرجع{sustainability-12-08443} &
  متحرک &
  گزارش شده &
  \متن‌لاتین{LoRa \ NB-IoT} &
  \متن‌لاتین{UDP/TCP ov IPv6} &
  \متن‌لاتین{CoAP / ReST} &
  گزارش شده &
  گزارش شده &
  گزارش شده &
  باز &
  ندارد &
  گزارش شده &
  واقعی \\
  \خط‌پر
  \مرجع{Lee2018} &
  ثابت &
  گزارش شده &
  \متن‌لاتین{LoRa} &
  ندارد &
  ندارد &
  گزارش نشده &
  گزارش شده &
  گزارش شده &
  باز / بسته &
  دارد &
  گزارش نشده &
  واقعی \\
  \خط‌پر
  \مرجع{Marahatta2021} &
  ثابت &
  گزارش شده &
  \متن‌لاتین{LoRa} &
  ندارد &
  ندارد &
  گزارش شده &
  گزارش شده &
  گزارش شده &
  باز / بسته &
  دارد &
  گزارش نشده &
  \متن‌لاتین{ns-2} \\
  \خط‌پر
\end{tabularx}
\پایان{لوح}

\زیرقسمت{ارزیابی کارایی}

\زیرزیرقسمت{مقدمه}

در حوزه اینترنت اشیا پروتکل‌ها و معماری‌های مختلفی وجود دارد که می‌توان از آن‌ها استفاده کرد. هر یک از پروتکل‌ها یا معماری‌ها در شرایط خاصی کارآیی خوبی دارند بنابراین پژوهش‌های زیادی برای ارزیابی کارایی آن‌ها صورت پذیرفته است.
در این ارزیابی پروتکل‌ها و معماری‌ها بدون هیچ تغییر یا بهبودی ارزیابی می‌شوند.
این ارزیابی‌ها به صورت کلی در دو دسته واقعی یا شبیه‌سازی می‌باشند. برخی از آن‌ها در یک لایه به خصوص مانند لایه دسترسی یا لایه هسته فعالیت کرده‌اند و برخی یک راه‌حل انتها به انتها اینترنت اشیا را ارزیابی کرده‌اند.
پارامترهای متنوعی مورد ارزیابی قرار می‌گیرند که از عمده آن‌ها می‌توان توان مصرفی، نرخ داده، جابجای اشیا و \نقاط‌خ را نام برد.

\زیرزیرقسمت{مرجع \مرجع{Baldo2021}}

در پژوهش \مرجع{Baldo2021} محققان بیان می‌کنند معماری‌های پیشنهادی تا به امروز بر سیستم‌های تک نقش متمرکز بودند و پژوهش حاضر قصد دارد سیستمی متشکل از سرویس‌هایی با توپولوژی‌ها متفاوت را
برای مدیریت هوشمند پسماند ارائه دهد. این معماری کلاس‌های مختلف \متن‌لاتین{LoRaWAN} را در برگرفته و شامل نودهای با پیچیدگی‌های رو به رشد است.
این معماری با سطل‌های ساده هوشمند شروع شده، شامل دورریزهایی است که با کاربران مراوده می‌کنند و دوربین‌هایی که سطل‌ها را برای جلوگیری از حریق نظارت می‌کنند.

این پژوهش بیان می‌کنند داده‌های ساده‌ی سنسورهایی که میزان پر بودن سطل‌های زباله را نشان می‌دهند زمانی که در بستر هوش مصنوعی قرار می‌گیرند می‌توانند ارزش بالایی خلق کنند.

معماری پیشنهادی بر پایه شبکه‌های \متن‌لاتین{LoRaWAN} از سه نود اصلی تشکیل شده است. دسته اول نودهایی هستند که در سطل زباله‌های عمومی نصب شده و پارامترهایی مانند
میزان پر بودن، دما و جهت قرار گرفتن سطل را گزارش می‌کنند. نودهای دسته دوم پیچیدگی بیشتری داشته و در دوریزها نصب می‌شوند. یکی از وظایف این گروه از نودها تعامل با کاربران است.
در نهایت دسته سوم نودهای نظارت ویدیویی هستند که برای مراقبت و نظارت بر دوریزها و سطل‌های زباله نصب می‌شوند. برای حفظ حریم خصوصی این دوربین‌ها هیچ تصویری را ارسال یا ذخیره نمی‌کنند
بلکه بر پایه \متن‌لاتین{Fog Computing} و توان پردازشی بالایی که دارند این تصاویر را پردازش می‌کنند.

در نظر داشته باشید دوربین‌ها تنها برای پیشگیری از مواردی مانند آتش‌سوزی یا آسیب عمدی تعبیه شده‌اند اما در نهایت با جمع‌اوری داده‌های این سه گروه از اشیا می‌تواند تصمیم بهتری گرفته و دسته جدیدی از دادگان را خلق نمود.

برای نظارت بر سطح پر بودن سطل‌ها راه‌های متفاوتی وجود دارد که یکی از آن‌ها استفاده از سنسور فراصوت می‌باشد. این سنسور جهت عملکرد مناسب می‌بایست به صورت ضد آب جایگذاری شود اما با این وجود
این سنسورها یکی از بهترین و مطمئن‌ترین راه‌ها برای ارزیابی سطح پر بودن سطل زباله هستند. در کنار این سنسورها می‌توان از سیستم‌های نظارت ویدیویی نیز بهره جست که در نهایت ترکیب آن با داده‌های سنسور
از سطل‌ها توانایی تصمیم‌گیری بهتری می‌دهد.


\زیرزیرقسمت{مرجع \مرجع{sensors-19-00007}}

پژوهش \مرجع{sensors-19-00007} قصد ارزیابی بین پروتکل‌های \متن‌لاتین{MQTT} و \متن‌لاتین{CoAP} که به ترتیب بر بسترهای \متن‌لاتین{UDP} و \متن‌لاتین{TCP} فعالیت می‌کنند، را دارد.
ارزیابی بر شبکه زیرساخت \متن‌لاتین{NB-IoT} صورت می‌گیرد که با فعالیت روی باند دارای لایسنس و نبود \متن‌لاتین{duty-cycle}، امکان اجرای پروتکل \متن‌لاتین{tcp} را نیز فراهم می‌آورد.
این پژوهش در نهایت نشان می‌دهد پروتکل \متن‌لاتین{MQTT} نسبت به پروتکل \متن‌لاتین{CoAP} کارآیی کمتری در معیارهای تاخیر، پوشش و ظرفیت سیستم دارد.

در این پژوهش با وجود اهمیت بحث‌های امنیتی بر پروتکل‌های اینترنت اشیا از آن‌ها صرف نظر شده است. برای امنیت پروتکل \متن‌لاتین{CoAP} می‌بایست از \متن‌لاتین{DTLS} و
برای امنیت پروتکل \متن‌لاتین{MQTT} از \متن‌لاتین{TLS} استفاده کرد.
در ضمن این پژوهش تنها به نسخه‌های مشهور این پروتکل‌های اکتفا کرده است و این در حالی است که نسخه‌های دیگری مانند \متن‌لاتین{MQTT-SN}، \متن‌لاتین{MQTT-over-QUIC} و \نقاط‌خ
که گاها برای راهکارهای اینترنت اشیا بهینه شده‌اند، نیازمند ارزیابی می‌باشند.

ارزیابی این پژوهش بر بستر \متن‌لاتین{NB-IoT} به صورت شبیه‌سازی و بر پایه \متن‌لاتین{Ericsson internet event-based radio network system simulator} انجام داده است.

\زیرزیرقسمت{مرجع \مرجع{sensors-20-02078}}

پژوهشگران در \مرجع{sensors-20-02078} بیان می‌کنند که کارهای زیادی در کشاورزی دقیق با هدف مصرف توان پایین پیشنهاد شده‌اند اما تعداد کمی از آن‌ها در عمل تست شده‌اند.
این پژوهش به دنبال تست عملیاتی مصرف توان در کشاورزی و کاهش آن است.

در ابتدا از بین تکنولوژی‌های ارتباطی موجود بیان می‌شود زمانی که داده‌ی زیادی برای ارسال موجود نیست مانند سنسورهای کشاورزی، استفاده از \متن‌لاتین{LoRa} گزینه خوبی است.
در ضمن پوشش \متن‌لاتین{LTE} در نواحی غیرشهری و کشاورزی کافی نیست بنابراین استفاده از \متن‌لاتین{NB-IoT} گزینه خوبی نیست.

این پژوهش حسگرهای صنعتی کشاورزی را بررسی کرده و برای ارسال داده‌ها نرخ منطقی ۳۰ دقیقه را محاسبه می‌کند.

\زیرزیرقسمت{مرجع \مرجع{sensors-18-00772-v3}}

در \مرجع{sensors-18-00772-v3} پژوهشگران برای ارزیابی شبکه \متن‌لاتین{LoRaWAN} در ابتدا یک ارزیابی رادیویی بر پایه مدل انتشار \متن‌لاتین{Okumura-Hata} صورت داده
و در ادامه با استفاده از محیط‌های واقعی شهری، نیمه‌شهری و غیرشهری دست به ارزیابی عملیاتی زده‌اند. ارزیابی اولیه به محققان کمک کرده است تا بتوانند محل خوبی را برای
تنها \متن‌لاتین{Gateway} این آزمایش پیدا کنند.

در ارزیابی عملیاتی از یک نود متحرک که پارامترهای ارسالش با زمان تغییر می‌کند، استفاده شده است. این نود برای بازه‌های زمانی ثابت می‌ماند تا تاثیر حرکت در پارامترها حذف شود.
در انتها این پژوهش بیان می‌کند برای اجرای یک زیرساخت \متن‌لاتین{LoRaWAN} باید مصالحه‌ای بین کیفیت لینک، نرخ داده‌ی انتقالی و سیار بودن نود برقرار شود.

\زیرزیرقسمت{مرجع \مرجع{sensors-20-00280-v2}}

پژوهش \مرجع{sensors-20-00280-v2}
اگلوریتم \متن‌لاتین{Static Context Header Compression (SCHC)} را برای \متن‌لاتین{IPv6} پیاده‌سازی کرده است.
این پژوهش از این پیاده‌سازی برای انتقال پروتکل \متن‌لاتین{CoAP} بر بستر \متن‌لاتین{UDP} و \متن‌لاتین{IPv6} استفاده کرده است.
هدف این پژوهش ارزیابی این الگوریتم بوده است.

این پژوهش بیان می‌کند منابع مصرفی در جهت استفاده از \متن‌لاتین{IPv6} نسبت به سود حاصل از آن بسیار کم است. در مقابل کارایی انرژی و داده در قطعه‌بندی کم است.
از سوی دیگر این پژوهش بیان می‌کند برای استفاده از قطعه‌بندی پیشنهادی \متن‌لاتین{IETF} نیاز است که ترتیب بسته‌ها حفظ شود.

از مزایای \متن‌لاتین{IPv6} می‌توان به زمانی اشاره کرد که یک نود بین \متن‌لاتین{gateway}ها جابجا می‌شود در صورت لزوم پروسه پیوستن به \متن‌لاتین{NS} را انجام می‌دهد که صورت استفاده از یک آدرس \متن‌لاتین{IPv6} ثابت اتصال تضمین خواهد شد.

در نظر گرفتن نودهای متحرک و سناریوهای انتها به انتها چیزی که در این پژوهش به آن پرداخته نشده است. اهمیت اصلی استفاده از پروتکل‌های استاندارد اینترنت در شبکه‌های اینترنت اشیا
در واقع ارزش خود را زمانی نشان خواهد داد که به صورت انتها به انتها استفاده شده و لایه‌های دیگر بتوانند در صورت نیاز با شیوه‌های استاندارد با اشیا ارتباط برقرار کنند.


\زیرزیرقسمت{مرجع‌های \مرجع{sustainability-12-08443} و \مرجع{Santa2020}}

هر دو پژوهش \مرجع{sustainability-12-08443} و \مرجع{Santa2020} توسط یک تیم انجام شده و تکمیل شده یکدیگر می‌باشند. این پروژه‌ها وسایل حمل و نقل نوظهور مانند دوچرخه و اسکوترهای برقی را هدف قرار می‌دهند.
هدف طراحی یک سیستم \متن‌لاتین{OBU} یا \متن‌لاتین{on-board unit} می‌باشد که بتوان از آن‌ها برای گردآوری داده از این وسایل استفاده کرد.

این پژوهش‌ها بیان می‌کنند از بین راه‌کارهای توان پایین با برد بالا بهتر از برای گردآوری داده‌های غیرحیاتی از \متن‌لاتین{LoRaWAN} یا \متن‌لاتین{NB-IoT} استفاده کرد
و تکنولوژی‌های سلولی مانند \متن‌لاتین{GSM} یا \متن‌لاتین{LTE} برای ارتباط‌های حیاتی حفظ کرد.

این پژوهش‌ها یک دستگاه \متن‌لاتین{OBU} با هر دو تکنولوژی \متن‌لاتین{LoRaWAN} و \متن‌لاتین{NB-IoT} پیاده‌سازی کرده و آن را در عمل ارزیابی می‌کنند.
برای ارزیابی \متن‌لاتین{NB-IoT} از زیرساخت عملیاتی \متن‌لاتین{Vodafone} استفاده شده است.

\زیرزیرقسمت{مرجع \مرجع{Islam2021}}

پژوهش \مرجع{Islam2021} به بحث استفاده از \متن‌لاتین{UAV}ها، ارتباطات \متن‌لاتین{LoRaWAN} و ارتباطات ماهواره‌ای در کشاوری پرداخته است.
این پژوهش بیان می‌کند که محدودیت‌های ارتباطی در این حوزه به قدر کافی مورد توجه قرار نگرفته است و تلاش دارد چالش‌های ارتباطی موجود در کشاورزی هوشمند را مرور کند.
این محدودیت‌های ارتباطی در کنار حسگرها، برای \متن‌لاتین{UAV}ها نیز مطرح هستند.

این پژوهش به عنوان یک راهکار در بحث مشکلات ارتباطی بحث \متن‌لاتین{Mesh LoRa} را مطرح می‌کند و مشکلات زیر را برای آن برمی‌شمارد.
\شروع{فقرات}
\فقره کارآیی و قابلیت بسیار پایین است.
\فقره نیاز به نگهداری از راه دور برای دروازه‌ها وجود دارد.
\فقره نیاز به پردازش لبه در دروازه‌ها وجود دارد.
\پایان{فقرات}

این پژوهش راهکاری برای این مشکلات ارائه نمی‌دهد و بحث \متن‌لاتین{Mesh LoRa} بسیار کلی ذکر می‌کند و ساختار مشخصی برای آن برنمی‌شمارد.

\زیرزیرقسمت{مرجع \مرجع{Mishra2021}}

در \مرجع{Mishra2021} پژوهشگران دست به ارزیابی کارایی دلال‌های پیام برای پروتکل \متن‌لاتین{MQTT} زده‌اند. در این پژوهش معیارهای نرخ پیام، مصرف \متن‌لاتین{CPU} و تاخیر مورد نظر بوده‌اند.
در این میان تاخیر مدت زمانی است که از ارسال پیام در \متن‌لاتین{Publisher} تا دریافت آن در \متن‌لاتین{Subscriber} طول می‌کشد.
دلال‌های پیامی که برای این پژوهش بررسی شده‌اند در جدول \رجوع{جدول:دلال‌های پیام مورد ارزیابی در Mishra2021} آورده شده‌اند.

\شروع{لوح}
\شرح{دلال‌های پیام مورد ارزبابی در پروژه \مرجع{Mishra2021}}
\برچسب{جدول:دلال‌های پیام مورد ارزیابی در Mishra2021}
\فضای‌و{5mm}
\begin{tabularx}
  {\textwidth}
  {p{3cm}*6{X}}
\خط‌پر
دلال‌پیام \متن‌لاتین{MQTT} & \متن‌لاتین{Mosquitto} & \متن‌لاتین{Bevywise MQTT Route} & \متن‌لاتین{ActiveMQ} & \متن‌لاتین{HiveMQ CE} & \متن‌لاتین{VerneMQ} & \متن‌لاتین{EMQ X} \\
\خط‌پر
متن‌باز & است & نیست & است & است & است & است \\
\خط‌پر
زبان برنامه‌نویسی اصلی & \متن‌لاتین{C} & \متن‌لاتین{C} و \متن‌لاتین{Python} & \متن‌لاتین{Java} & \متن‌لاتین{Java} & \متن‌لاتین{Erlang} & \متن‌لاتین{Erlang} \\
\خط‌پر
نسخه پروتکل \متن‌لاتین{MQTT} & نسخه ۳ و ۵ & نسخه ۳ و ۵ & نسخه ۳ & نسخه ۳ و ۵ & نسخه ۳ و ۵ & نسخه ۳ \\
\خط‌پر
کیفیت‌سرویس‌های پشتیبانی شده & ۰، ۱ و ۲ & ۰، ۱ و ۲ & ۰، ۱ و ۲ & ۰، ۱ و ۲ & ۰، ۱ و ۲ & ۰، ۱ و ۲ \\
\خط‌پر
سیستم‌عامل‌های پشتبیانی شده & \متن‌لاتین{Linux, Mac, Windws} & \متن‌لاتین{Windows, Linux, Mac, Raspberry Pi} & \متن‌لاتین{Windows, Linux} & \متن‌لاتین{Windows, Mac, Linux} & \متن‌لاتین{Linux, Mac} & \متن‌لاتین{Linux, Mac, Windows} \\
\خط‌پر
\end{tabularx}
\پایان{لوح}

برای تست از دو سناریو مختلف استفاده شده است. در هر دو سناریو یک انتشاردهنده، یک مشترک و یک سرور قرار دارد. در سناریو اول برای شبیه‌سازی از لپ‌تاب‌های شخصی استفاده شده است
تا شرایط لبه شبیه‌سازی شود و در سناریو دوم از ماشین‌های مجازی زیرساخت ابری \متن‌لاتین{Google Cloud Platform} استفاده شده است.
در هر دو این سناریو مساله تعداد ارتباط‌های همزمان مورد بحث قرار نگرفته است که با توجه به تعداد زیاد اشیا می‌تواند چالش بزرگی برای منابع سیستم دلال پیام باشد.
مساله دیگر در این ارزیابی‌ها عدم استفاده از تکنولوژی‌های ابری به روز مانند کانتینرها و \متن‌لاتین{Kubernetes} می‌باشد که امروز جز جدانشدنی از پیاده‌سازی سیستم‌های اینترنت اشیا حتی در لبه می‌باشند.
در کنار این دو مورد استفاده از رمزنگاری در پروتکل \متن‌لاتین{MQTT} ممکن است، که این امر خود می‌تواند باعثث تغییر در کارایی دلال پیام شود و از این رو نیازمند ارزبابی است که این پژوهش به آن نپرداخته است.

در نهایت این پروژهش به این جمع‌بندی میرسد که دلال‌های پیام غیرگسترش‌پذیر مانند \متن‌لاتین{Mosquitto} که از تعداد مشخصی نخ استفاده می‌کنند برای محیط‌های محدود مناسب‌تر هستند.
از سوی دیگر از میان دلال‌های پیام گسترش‌پذیر \متن‌لاتین{ActiveMQ} کارآیی بالایی داشته و \متن‌لاتین{EMQ X}، \متن‌لاتین{VerneMQ} و \متن‌لاتین{HiveMQ} کارآیی مناسبی دارند.

\زیرزیرقسمت{مرجع \مرجع{Cruz2021}}

پژوهشگران در \مرجع{Cruz2021} استفاده آزمایشی از \متن‌لاتین{LoRa} و \متن‌لاتین{LoRaWAN} به عنوان زیرساخت مدیریت پسماند در شهر لیسبون را گزارش می‌دهند.
در حال حاضر پسماند شهر لیسبون با استفاده از زیرساخت \متن‌لاتین{GPRS} فعالیت می‌کند. مدیران شهری قصد دارند این زیرساخت را به یک زیرساخت \متن‌لاتین{LPWAN} تغییر دهند.
در حال حاضر تکنولوژی‌های متنوعی مانند \متن‌لاتین{LoRaWAN}، \متن‌لاتین{Sigfox} و \متن‌لاتین{NB-IoT} در بازار وجود دارند و پژوهشگران قصد دارند کارآیی \متن‌لاتین{LoRaWAN} را در
سناریو مدیریت پسماند به صورت عملی بررسی کنند.

در واقع مشارکت اصلی این پژوهش در ارزیابی واقعی سنسورها، شبکه، کارایی انرژی و تکنولوژی‌های ارتباطی است که در قالب یک مدیریت هوشمند پسماند شهری رخ می‌دهد.
آزمایش‌های یک پژوهش در دو سطح رخ می‌دهند. در سطح اول هدف ارزیابی پوشش شبکه‌ای \متن‌لاتین{LoRa} برای مانیتورینگ مخازن پسماند سطحی و زیرزمینی است.
در سطح دوم هدف بررسی ظرفیت شبکه برای ارسال داده‌های مورد نیاز اپلیکشن‌ها می‌باشد.

این آزمایش‌های همگی توسط تجهیزات تجاری صورت گرفته است. برای \متن‌لاتین{Gateway}ها از دو \متن‌لاتین{Gateway} تجاری متفاوت استفاده شده است.
برای سنسورهای سطح پسماند نیز از دو سنسور تجاری متفاوت که قیمت و ویژگی‌های متفاوتی (حداکثر ارتفاع قابل سنج، سنسورهای ثانویه، باتری و \نقاط‌خ) دارند استفاده شده است.
سنسورهای سطح پسماند با نصب شدن بر درب مخزن و با استفاده از فراصوت فاصله خود تا پسماند را اندازه‌گیری می‌کنند. این روش می‌تواند خطا داشته باشد
و مکان سنسور و توان پردازشی آن تاثیر به سزایی در این امر دارد.

سه آزمایش کلی برای پوشش انجام شده است. پوشش کوتاه (نزدیک به ۱۰۰ متر)، میانی (نزدیک به یک کیلومتر) و طولانی (نزدیک به ۵ کیلومتر)، که در هر یک یک نود پسماند قرار دارد.
در آزمایش پوشش کوتاه مشکلی ایجاد نشده و همه اطلاعات دریافت می‌شوند ولی آزمایش برد طولانی عملا هیچ داده‌ای را منتقل نکرده است.
در آزمایش برد میانی نود پسماند یک نود زیرزمینی بوده است و چالش اصلی طراحی نود بوده است.

اندازه‌ی بسته‌های ارسالی از دو حسگر مقدارهای متفاوت ۴ و ۸ بایت بوده است. در شروع تست‌ها از مکانیزم نرخ داده تطبیق‌پذیر استفاده شده است و این مورد در ادامه غیرفعال شده است.

در نهایت این پژوهش بیان می‌کند که می‌توان از \متن‌لاتین{LoRa} برای زیرساخت مدیریت هوشمند پسماند شهری استفاده کرد.
البته این پژوهش در رابطه با تعداد سنسورهایی که می‌توان در شبکه داشت و تداخل آن‌ها مطالعه‌ای انجام نداده است.

\زیرزیرقسمت{مرجع \مرجع{sensors-20-06721}}

پژوهشگران \مرجع{sensors-20-06721} تجربه بیش از دو سال نگهداری از شبکه‌ی سنسورهای فضای بسته دانشگاه \متن‌لاتین{oulu} کشور فلاند مبتنی بر \متن‌لاتین{LoRaWAN} در این پژوهش مرور می‌کنند.
این پژوهش بار زیادی داشته و این تنها مقاله‌ای نیست که از آن به چاپ رسیده است. در این تجربه ۳۳۱ سنسور در سقف در ریل‌های چراغ‌ها با فاصله‌های یک و نیم‌متری در یک محل اجتماعات در دانشگاه نصب شدند.
برای جمع‌آوری داده از یک \متن‌لاتین{Gateway} با دید مستقیم استفاده شده است. پیش از استقرار سنسورها یک تخمین برای پارامتر \متن‌لاتین{SF} در شبکه \متن‌لاتین{LoRaWAN} نیز انجام شده است.

در این پژوهش هیچ استفاده‌از \متن‌لاتین{ADR}، بسته‌های \متن‌لاتین{downlink} و \متن‌لاتین{ACK} نشده است. نرخ ارسال سنسورها ۱۵ دقیقه‌ای بوده و اندازه بسته‌ی آنها مشخص است.
هر نود در واقع شامل پنج سنسور دما، رطوبت، شدت‌نور، تشخیص حرکت و سطح $CO_{2}$ است. این نودها برای اتصال به شبکه از فعال‌سازی \متن‌لاتین{OTAA} استفاده می‌کنند.

یکی از موارد مهمی که در این پژوهش به آن اشاره می‌شود، از دست رفتن بسته‌ها به جز در شبکه‌ی دسترسی و در \متن‌لاتین{Backend} است.
منظور از شبکه \متن‌لاتین{Backend} زیرساخت ارتباط میان \متن‌لاتین{NS} و سرور پلتفرم می‌باشد.
این بازه‌های از دست رفتن بیش از ۵۰ درصد بسته‌ها در \متن‌لاتین{Backend} به صورت دوره‌های ۱.۵ ماه رخ می‌دادند. این پژوهش به بررسی بیشتر این موضوع نپرداخته است و دلیلی ارائه نمی‌دهد.

\زیرقسمت{کنترل دسترسی همزمان}

\زیرزیرقسمت{مقدمه}
همانطور که بیان شد، \متن‌لاتین{LoRaWAN} از پروتکل کنترل دسترسی همزمان \متن‌لاتین{ALOHA} استفاده می‌کند. این پروتکل سربار کمی دارد ولی در شبکه‌های شلوغ به خوبی عمل نمی‌کند.
از این رو پژوهش‌های زیادی الگوریتم‌های جدید پیشهاد کرده و آن‌ها را به صورت شبیه‌سازی یا واقعی ارزیابی می‌کنند.

\زیرزیرقسمت{مرجع \مرجع{Beltramelli2021}}

در \مرجع{Beltramelli2021} پژوهشگران یک لایه \متن‌لاتین{MAC} جدید برای شبکه‌های \متن‌لاتین{LoRaWAN} پیشنهاد داده‌اند. در این پروژه در ابتدا کارهای پیشین در زمینه بهبود کنترل دسترسی همزمان
مورد بحث قرار گرفته است. این پژوهش خود قصد دارد از شیوه‌ی \متن‌لاتین{Slotted ALOHA} یا اختصارا \متن‌لاتین{S-ALOHA} با همگام‌سازی خارج از باند استفاده کند.
در نهایت در این پروژهش این شیوه در شبیه‌سازی و شرایط واقعی شهری آزمایش می‌شود. در نظر داشته باشید که روش پیشنهادی صرفا برای برنامه‌های نظارتی بوده که پیام‌های \متن‌لاتین{downlink} ندارند.

روش‌های همگام‌سازی خارج از باند این پژوهش عبارتند از:

\شروع{فقرات}
\فقره \متن‌لاتین{GNSS}: \متن‌لاتین{GPS} امروزه یکی از گسترش‌یافته‌ترین \متن‌لاتین{GNSS}ها بوده و می‌تواند با دقت خوبی همگام‌سازی زمان را انجام دهد اما هزینه و توان مصرفی آن برای بیشتر راهکارهای اینترنت اشیا زیاد است.
\فقره \متن‌لاتین{RCCs}:‌ در این سیستم از موج‌های بلند در بازه $40kHz$ تا $80kHz$ استفاده می‌شود. این امواج در مرکز اروپا در دسترس بوده و می‌توان از آن‌ها برای همگامی زمانی تا دقت $0.1$ ثانیه استفاده کرد.
\فقره \متن‌لاتین{FM-RDS}: در این سیستم داده‌های همگام‌سازی از موج \متن‌لاتین{FM} استخراج می‌شود اما لزوما همه‌ی پایگاه‌های \متن‌لاتین{FM} این داده را ارسال نمی‌کنند و لزوما نیز همه پایگاه‌ها خود زمان صحیح ندارند.
\پایان{فقرات}

از بین این روش‌ها، پژوهش حاضر از \متن‌لاتین{FM-RDS} استفاده می‌کند. ارزیابی واقعی به وسیله دو دستگاه صورت گرفته است و برای ارزیابی با تعداد اشیا بالا از شبیه‌سازی \متن‌لاتین{LoRaEnergySim} به زبان \متن‌لاتین{Python} و به صورت تغییریافته استفاده شده است.

\زیرزیرقسمت{مرجع \مرجع{Lee2021}}

پژوهش \مرجع{Lee2021} از جمله مقالاتی است که در بهبود لایه \متن‌لاتین{MAC} برای بهبود شبکه‌های \متن‌لاتین{LoRa} در مقیاس بزرگ کار کرده است.
این پژوهش ساختار جدیدی برای لایه \متن‌لاتین{MAC} ارائه داده و بر این ساختار بحث ارسال گروهی پیام‌های \متن‌لاتین{ACK} را مطرح می‌کند.
عملکرد این لایه پیشنهادی بر پایه ارسال متناوب \متن‌لاتین{Beacon}ها برای همگام‌سازی و مشخص شدن ساختار \متن‌لاتین{Frame}ها می‌باشد.

هر \متن‌لاتین{Frame} خود از \متن‌لاتین{Subframe}هایی تشکیل شده است که هر یک در \متن‌لاتین{slot}های مشخصی به نودها
اجازه می‌دهد تا \متن‌لاتین{uplink} و \متن‌لاتین{downlink} داشته باشند.

در این ساختار، پژوهش قصد دارد به صورت گروهی \متن‌لاتین{Ack} ارسال کند. برای ارسال گروهی این \متن‌لاتین{Ack}ها یک پیام مشخص ارسال می‌شود.
این پیام مشخص می‌تواند با توجه به اندازه‌ای که دارد شامل تعداد مختلفی از \متن‌لاتین{Ack}ها باشد. همانطور که پیشتر اشاره شد این اندازه بسته خود وابسته به
نرخ ارسال داده و \متن‌لاتین{SF}ها است. از آنجایی که \متن‌لاتین{SF}ها به صورت شبه عمود می‌باشند، \متن‌لاتین{Gateway}ها می‌توانند از \متن‌لاتین{SF}ها به صورت همزمان
استفاده کنند اما با توجه به تغییر نرخ ارسال، آن‌ها به تعداد \متن‌لاتین{slot}ها متفاوتی نیاز خواهند داشت و بنابراین مساله تخصیص بهینه \متن‌لاتین{SF}ها وجود خواهد داشت که این پژوهش به
آن نیز پرداخته است.

\زیرقسمت{شبکه‌های \متن‌لاتین{Mesh}}

\زیرزیرقسمت{مقدمه}

یکی از راه‌ها افزایش کارایی شبکه‌های \متن‌لاتین{LoRa} استفاده از لایه‌ی فیزیکی \متن‌لاتین{LoRa} و تشکیل یک شبکه \متن‌لاتین{Mesh} می‌باشد. در این شبکه نودها با کمک یکدیگر داده‌ها را ارسال کرده و به دست \متن‌لاتین{Gateway} می‌رسانند.
البته در چنین شبکه‌هایی چالش‌های جدیدتری مانند چگونگی اضافه شدن یا حذف شدن یک نود، چگونگی ساخته شدن شبکه و \نقاط‌خ مطرح می‌باشد. دسته‌ای از پژوهش‌ها به پیاده‌سازی و ارزیابی چنین شبکه‌هایی پرداخته‌اند.

\زیرزیرقسمت{مرجع \مرجع{Lee2018}}

در \مرجع{Lee2018} پژوهشگران دست به پیاده‌سازی ۱۹ نود شبکه‌ی \متن‌لاتین{Mesh LoRa} در محیط دانشگاه با ارسال داده‌ها در بازه‌های یک دقیقه‌ای زده‌اند.
پژوهشگران ادعا می‌کنند این اولین کاری است که تجربه واقعی در پیاده‌سازی \متن‌لاتین{Mesh LoRa} داشته است و در آن بیان می‌شود با این روش نیاز به افرایش تعداد \متن‌لاتین{Gateway}ها از بین می‌رود.
نویسندگان معتقد هستند که استفاده از \متن‌لاتین{ALOHA} توانایی \متن‌لاتین{LoRa} در هندل کردن تعداد زیادی از اشیا را از بین برده است.
سیستم طراحی شده در این پژوهش به جای \متن‌لاتین{LoRaWAN} بر پایه لایه‌ی فیزیکی \متن‌لاتین{LoRa} می‌باشد.

نودها به صورت خودمختار برای انتخاب پدر خود در زمان پیوستن به شبکه‌ی \متن‌لاتین{Mesh} تصمیم می‌گیرند، آن‌ها برای این امر از پارامترهای پیام‌های داده‌ای یا \متن‌لاتین{Beacon}های نودهای متصل به شکبه، که به آن‌ها می‌رسد، استفاده می‌کنند.
در این ساختار هر نود لیستی از فرزندان خود نگهداری کرده و آن را در اختیار \متن‌لاتین{‌Gateway} هم قرار می‌دهد.
این شبکه برای ارتباطات بین نودها بهینه نشده است و بیشتر هدف آن ارتباط میان \متن‌لاتین{Gateway} و نودها می‌باشد.
بازه ارسال داده‌ها برای نودها در این شبکه مشخص است و بر پایه عدم دریافت پیام در این بازه می‌توانند وضعیت شبکه‌ای خود را ارزیابی و پدر خود را تغییر دهند.
این پژوهش در مورد مقدار بهینه پارامترهای \متن‌لاتین{RSSI}، \متن‌لاتین{PDR} و \نقاط‌خ صحبت می‌کند و همانطور که بیان شد، ادعا می‌کند که برای رسیدن به این مقدارهای بهینه تنها دو راه افزایش تعداد \متن‌لاتین{Gateway}ها یا استفاده از شبکه‌ی \متن‌لاتین{Mesh} وجود دارد.

در نهایت می‌توان گفت موارد زیر در این پژوهش دیده نشده‌اند:
\شروع{فقرات}
\فقره در این پروژه \متن‌لاتین{Gateway} برای دریافت داده‌ها به نودها درخواست می‌دهد و در مورد ارسال خودکار داده‌ها توسط اشیا که می‌تواند منجر به \متن‌لاتین{Collusion} شود بحث نشده است.
\فقره در نظر نگرفتن کلاس‌های کاری و توان مصرفی شبکه‌ی \متن‌لاتین{LoRa}
\پایان{فقرات}

\زیرزیرقسمت{مرجع \مرجع{Marahatta2021}}

پژوهش \مرجع{Marahatta2021} قصد دارد استفاده از شبکه‌های \متن‌لاتین{LoRa} را در مناطق دور افتاده برای کنترل مصرف انرژی ارزیابی کند.
یکی از مسائل پیش رو در این پژوهش برای استفاده از لایه لینک \متن‌لاتین{LoRaWAN} نیاز آن به تعداد \متن‌لاتین{Gateway}های بالا می‌باشد.
از این رو این پژوهش به استفاده از شبکه‌ی \متن‌لاتین{Mesh LoRa} روی می‌آورد که در آن هر نود به عنوان تکرار کننده برای نودهای همسایه خود عمل کرده
و تعداد \متن‌لاتین{Gateway}های مورد نیاز را کاهش می‌دهد.

این پژوهش روش پیشنهادی خود برای استفاده از \متن‌لاتین{Mesh} را در قالب شبیه‌سازی با \متن‌لاتین{ns-2} در محیط‌های شهری و غیرشهری ارزیابی می‌کند.
در صورت لزوم با توجه به پارامترهای ارتباطی، شبکه به زیرشبکه‌هایی شکسته شده است.
این پژوهش با توجه به زمینه خود در \متن‌لاتین{Smart Grid} از پارامتر جمع‌آوری استاندارد داده در این حوزه برای نرخ ارسال داده استفاده کرده است.


\زیرزیرقسمت{مرجع \مرجع{Famaey2018}}

پژوهش \مرجع{Famaey2018} شبکه‌ی یکپارچه‌ای مبتنی بر تکنولوژی‌های ارتباطی متنوع را تعریف می‌کند. این شبکه در واقع توسط اپراتور مجازی اینترنت اشیا
ساخته می‌شود که زیرساخت ارتباطی خود را از اپراتورهای مختلفی می‌گیرد. این زیرساخت به برنامه‌های کاربردی اجازه می‌دهد تا بتوانند با اشیایی در شبکه‌های ارتباطی مختلف ارتباط داشته باشند.
این اپراتور مجازی برای لایه شبکه از \متن‌لاتین{IPv6} و برای لایه کاربرد از پروتکل \متن‌لاتین{CoAP} استفاده می‌کند.

چالش‌های مختلفی می‌توان برای این اپراتورهای مجازی در نظر گرفت مانند مدل داده‌ای، ساختار داده‌ها و \نقاط‌خ که این پژوهش به آن‌ها نپرداخته است.

\زیرقسمت{نرخ‌داده تطبیق پذیر}

\زیرزیرقسمت{مقدمه}

یکی از بحث‌های در شکبه‌های \متن‌لاتین{LoRaWAN} قابلیت لایه‌ی لینک این شبکه برای تغییر پارامترهای ارتباطی در جهت بهبود کیفیت ارتباط می‌باشد. در استاندارد \متن‌لاتین{LoRaWAN} یک روش پیشنهادی ساده مطرح شده است
اما پژوهش‌های زیادی دست به بهبود و ارزیابی آن در شرایط متفاوت زده‌اند. از سوی دیگر پارامتر \متن‌لاتین{SF} در این شبکه‌ها به صورت شبه عمود بوده و اجازه ارتباط همزمان را می‌دهد بنابراین پژوهش‌های زیادی دست به طرح مساله
برای تخصص این پارامتر زده‌اند.

\زیرزیرقسمت{مرجع \مرجع{sensors-20-03061-v2}}

در پژوهش \مرجع{sensors-20-03061-v2} از شبه عمود بودن \متن‌لاتین{SF}ها در شبکه‌های \متن‌لاتین{LoRaWAN} برای انتقال همزمان پیام‌های دورسنجی و اخطار استفاده می‌شود.
این پژوهش دو استراتژی برای تخصیص \متن‌لاتین{SF}های متمایز برای داده‌های اخطاری و دورسنجی پیشنهاد می‌دهد.
آزمایش عملی این پژوهش به دلیل نیاز به تعداد بالای نود و نیاز به تغییر رویه تخصص \متن‌لاتین{SF}ها امکان‌پذیر نیست و بنابراین آزمایش در محیط شبیه‌سازی \متن‌لاتین{ns-3} صورت می‌گیرد.
برای شبیه‌سازی از سه محیط مختلف استفاده شده است، یک محیط بسته و دو محیط باز که به ترتیب یک و چهار \متن‌لاتین{Gateway} دارند.

در نهایت می‌توان نشان داد جدا کردن پیام‌های اخطار از دورسنجی می‌تواند به فراهم آوردن یک کران بالا برای تاخیر نیز کمک کند. پژوهشگران قصد دارند در پژوهش‌های آتی ارزیابی عملی از این الگوریتم‌ها داشته باشند
و مدل ریاضی برای موفقیت ارسال بسته‌ها بدست بیاورند. چهارچوب \متن‌لاتین{Network Calculus} می‌تواند به این پژوهش در بدست آوردن یک کران بالا برای تاخیر کمک کند.
