\فصل{کارهای مرتبط}

\قسمت{مقدمه}


\قسمت{کارهای مرتبط}

\زیرقسمت{ارزیابی کارایی}

\زیرزیرقسمت{مقدمه}

در حوزه اینترنت اشیا پروتکل‌ها و معماری‌های مختلفی وجود دارد که می‌توان از آن‌ها استفاده کرد. هر یک از پروتکل‌ها یا معماری‌ها در شرایط خاصی کارآیی خوبی دارند بنابراین پژوه‌ش‌های زیادی برای ارزیابی کارایی آن‌ها صورت پذیرفته است.
این ارزیابی‌ها به صورت کلی در دو دسته واقعی یا شبیه‌سازی می‌باشند. برخی از آن‌ها در یک لایه به خصوص مانند لایه دسترسی یا لایه هسته فعالیت کرده‌اند و برخی یک راه‌حل انتها به انتها اینترنت اشیا را ارزیابی کرده‌اند.
پارامترهای متنوعی مورد ارزیابی قرار می‌گیرند که از عمده آن‌ها می‌توان توان مصرفی، نرخ داده، جابجای اشیا و \نقاط‌خ را نام برد.

\زیرزیرقسمت{مرجع \مرجع{Mishra2021}}

در \مرجع{Mishra2021} پژوهشگران دست به ارزیابی کارایی دلال‌های پیام برای پروتکل \متن‌لاتین{MQTT} زده‌اند. در این پژوهش معیارهای نرخ پیام، مصرف \متن‌لاتین{CPU} و تاخیر مورد نظر بوده‌اند.
در این میان تاخیر مدت زمانی است که از ارسال پیام در \متن‌لاتین{Publisher} تا دریافت آن در \متن‌لاتین{Subscriber} طول می‌کشد.
دلال‌های پیامی که برای این پژوهش بررسی شده‌اند در جدول \رجوع{جدول:دلال‌های پیام مورد ارزیابی در Mishra2021} آورده شده‌اند.

\شروع{لوح}
\شرح{دلال‌های پیام مورد ارزبابی در پروژه \مرجع{Mishra2021}}
\برچسب{جدول:دلال‌های پیام مورد ارزیابی در Mishra2021}
\فضای‌و{5mm}
\begin{tabularx}
  {\textwidth}
  {p{3cm}*6{X}}
\خط‌پر
دلال‌پیام \متن‌لاتین{MQTT} & \متن‌لاتین{Mosquitto} & \متن‌لاتین{Bevywise MQTT Route} & \متن‌لاتین{ActiveMQ} & \متن‌لاتین{HiveMQ CE} & \متن‌لاتین{VerneMQ} & \متن‌لاتین{EMQ X} \\
\خط‌پر
متن‌باز & است & نیست & است & است & است & است \\
\خط‌پر
زبان برنامه‌نویسی اصلی & \متن‌لاتین{C} & \متن‌لاتین{C} و \متن‌لاتین{Python} & \متن‌لاتین{Java} & \متن‌لاتین{Java} & \متن‌لاتین{Erlang} & \متن‌لاتین{Erlang} \\
\خط‌پر
نسخه پروتکل \متن‌لاتین{MQTT} & نسخه ۳ و ۵ & نسخه ۳ و ۵ & نسخه ۳ & نسخه ۳ و ۵ & نسخه ۳ و ۵ & نسخه ۳ \\
\خط‌پر
کیفیت‌سرویس‌های پشتیبانی شده & ۰، ۱ و ۲ & ۰، ۱ و ۲ & ۰، ۱ و ۲ & ۰، ۱ و ۲ & ۰، ۱ و ۲ & ۰، ۱ و ۲ \\
\خط‌پر
سیستم‌عامل‌های پشتبیانی شده & \متن‌لاتین{Linux, Mac, Windws} & \متن‌لاتین{Windows, Linux, Mac, Raspberry Pi} & \متن‌لاتین{Windows, Linux} & \متن‌لاتین{Windows, Mac, Linux} & \متن‌لاتین{Linux, Mac} & \متن‌لاتین{Linux, Mac, Windows} \\
\خط‌پر
\end{tabularx}
\پایان{لوح}

برای تست از دو سناریو مختلف استفاده شده است. در هر دو سناریو یک انتشاردهنده، یک مشترک و یک سرور قرار دارد. در سناریو اول برای شبیه‌سازی از لپ‌تاب‌های شخصی استفاده شده است
تا شرایط لبه شبیه‌سازی شود و در سناریو دوم از ماشین‌های مجازی زیرساخت ابری \متن‌لاتین{Google Cloud Platform} استفاده شده است.
در هر دو این سناریو مساله تعداد ارتباط‌های همزمان مورد بحث قرار نگرفته است که با توجه به تعداد زیاد اشیا می‌تواند چالش بزرگی برای منابع سیستم دلال پیام باشد.
مساله دیگر در این ارزیابی‌ها عدم استفاده از تکنولوژی‌های ابری به روز مانند کانتینرها و \متن‌لاتین{Kubernetes} می‌باشد که امروز جز جدانشدنی از پیاده‌سازی سیستم‌های اینترنت اشیا حتی در لبه می‌باشند.
در کنار این دو مورد استفاده از رمزنگاری در پروتکل \متن‌لاتین{MQTT} ممکن است، که این امر خود می‌تواند باعثث تغییر در کارایی دلال پیام شود و از این رو نیازمند ارزبابی است که این پژوهش به آن نپرداخته است.

در نهایت این پروژهش به این جمع‌بندی میرسد که دلال‌های پیام غیرگسترش‌پذیر مانند \متن‌لاتین{Mosquitto} که از تعداد مشخصی نخ استفاده می‌کنند برای محیط‌های محدود مناسب‌تر هستند.
از سوی دیگر از میان دلال‌های پیام گسترش‌پذیر \متن‌لاتین{ActiveMQ} کارآیی بالایی داشته و \متن‌لاتین{EMQ X}، \متن‌لاتین{VerneMQ} و \متن‌لاتین{HiveMQ} کارآیی مناسبی دارند.


\زیرزیرقسمت{مرجع \مرجع{Cruz2021}}

پژوهشگران در \مرجع{Cruz2021} استفاده آزمایشی از \متن‌لاتین{LoRa} و \متن‌لاتین{LoRaWAN} به عنوان زیرساخت مدیریت پسماند در شهر لیسبون را گزارش می‌دهند.
در حال حاضر پسماند شهر لیسبون با استفاده از زیرساخت \متن‌لاتین{GPRS} فعالیت می‌کند. مدیران شهری قصد دارند این زیرساخت را به یک زیرساخت \متن‌لاتین{LPWAN} تغییر دهند.
در حال حاضر تکنولوژی‌های متنوعی مانند \متن‌لاتین{LoRaWAN}، \متن‌لاتین{Sigfox} و \متن‌لاتین{NB-IoT} در بازار وجود دارند و پژوهشگران قصد دارند کارآیی \متن‌لاتین{LoRaWAN} را در
سناریو مدیریت پسماند به صورت عملی بررسی کنند.

در واقع مشارکت اصلی این پژوهش در ارزیابی واقعی سنسورها، شبکه، کارایی انرژی و تکنولوژی‌های ارتباطی است که در قالب یک مدیریت هوشمند پسماند شهری رخ می‌دهد.
آزمایش‌های یک پژوهش در دو سطح رخ می‌دهند. در سطح اول هدف ارزیابی پوشش شبکه‌ای \متن‌لاتین{LoRa} برای مانیتورینگ مخازن پسماند سطحی و زیرزمینی است.
در سطح دوم هدف بررسی ظرفیت شبکه برای ارسال داده‌های مورد نیاز اپلیکشن‌ها می‌باشد.

این آزمایش‌های همگی توسط تجهیزات تجاری صورت گرفته است. برای \متن‌لاتین{Gateway}ها از دو \متن‌لاتین{Gateway} تجاری متفاوت استفاده شده است.
برای سنسورهای سطح پسماند نیز از دو سنسور تجاری متفاوت که قیمت و ویژگی‌های متفاوتی (حداکثر ارتفاع قابل سنج، سنسورهای ثانویه، باتری و \نقاط‌خ) دارند استفاده شده است.
سنسورهای سطح پسماند با نصب شدن بر درب مخزن و با استفاده از فراصوت فاصله خود تا پسماند را اندازه‌گیری می‌کنند. این روش می‌تواند خطا داشته باشد
و مکان سنسور و توان پردازشی آن تاثیر به سزایی در این امر دارد.

سه آزمایش کلی برای پوشش انجام شده است. پوشش کوتاه (نزدیک به ۱۰۰ متر)، میانی (نزدیک به یک کیلومتر) و طولانی (نزدیک به ۵ کیلومتر)، که در هر یک یک نود پسماند قرار دارد.
در آزمایش پوشش کوتاه مشکلی ایجاد نشده و همه اطلاعات دریافت می‌شوند ولی آزمایش برد طولانی عملا هیچ داده‌ای را منتقل نکرده است.
در آزمایش برد میانی نود پسماند یک نود زیرزمینی بوده است و چالش اصلی طراحی نود بوده است.

اندازه‌ی بسته‌های ارسالی از دو حسگر مقدارهای متفاوت ۴ و ۸ بایت بوده است. در شروع تست‌ها از مکانیزم نرخ داده تطبیق‌پذیر استفاده شده است و این مورد در ادامه غیرفعال شده است.

در نهایت این پژوهش بیان می‌کند که می‌توان از \متن‌لاتین{LoRa} برای زیرساخت مدیریت هوشمند پسماند شهری استفاده کرد.
البته این پژوهش در رابطه با تعداد سنسورهایی که می‌توان در شبکه داشت و تداخل آن‌ها مطالعه‌ای انجام نداده است.


\زیرقسمت{کنترل دسترسی همزمان}

\زیرزیرقسمت{مقدمه}
همانطور که بیان شد، \متن‌لاتین{LoRaWAN} از پروتکل کنترل دسترسی همزمان \متن‌لاتین{ALOHA} استفاده می‌کند. این پروتکل سربار کمی دارد ولی در شبکه‌های شلوغ به خوبی عمل نمی‌کند.
از این رو پژوهش‌های زیادی الگوریتم‌های جدید پیشهاد کرده و آن‌ها را به صورت شبیه‌سازی یا واقعی ارزیابی می‌کنند.

\زیرزیرقسمت{مرجع \مرجع{Beltramelli2021}}

در \مرجع{Beltramelli2021} پژوهشگران یک لایه \متن‌لاتین{MAC} جدید برای شبکه‌های \متن‌لاتین{LoRaWAN} پیشنهاد داده‌اند. در این پروژه در ابتدا کارهای پیشین در زمینه بهبود کنترل دسترسی همزمان
مورد بحث قرار گرفته است. این پژوهش خود قصد دارد از شیوه‌ی \متن‌لاتین{Slotted ALOHA} یا اختصارا \متن‌لاتین{S-ALOHA} با همگام‌سازی خارج از باند استفاده کند.
در نهایت در این پروژهش این شیوه در شبیه‌سازی و شرایط واقعی شهری آزمایش می‌شود.
