\فصل{کارهای مرتبط}

\قسمت{مقدمه}


\قسمت{کارهای مرتبط}

\زیرقسمت{ارزیابی کارایی}

در \مرجع{Mishra2021} پژوهشگران دست به ارزیابی کارایی دلال‌های پیام برای پروتکل \متن‌لاتین{MQTT} زده‌اند. در این پژوهش معیارهای نرخ پیام، مصرف \متن‌لاتین{CPU} و تاخیر مورد نظر بوده‌اند.
در این میان تاخیر مدت زمانی است که از ارسال پیام در \متن‌لاتین{Publisher} تا دریافت آن در \متن‌لاتین{Subscriber} طول می‌کشد.
دلال‌های پیامی که برای این پژوهش بررسی شده‌اند در جدول \رجوع{} آورده شده‌اند.

\شروع{لوح}
\شرح{دلال‌های پیام مورد ارزبابی در پروژه \مرجع{Mishra2021}}
\فضای‌و{5mm}
\شروع{جدول}{ *7{p{2cm} } }
\خط‌پر
دلال‌پیام \متن‌لاتین{MQTT} & \متن‌لاتین{Mosquitto} & \متن‌لاتین{Bevywise MQTT Route} & \متن‌لاتین{ActiveMQ} & \متن‌لاتین{HiveMQ CE} & \متن‌لاتین{VerneMQ} & \متن‌لاتین{EMQ X} \\
\خط‌پر
متن‌باز & است & نیست & است & است & است & است \\
\خط‌پر
زبان برنامه‌نویسی اصلی & \متن‌لاتین{C} & \متن‌لاتین{C} و \متن‌لاتین{Python} & \متن‌لاتین{Java} & \متن‌لاتین{Java} & \متن‌لاتین{Erlang} & \متن‌لاتین{Erlang} \\
\خط‌پر
نسخه پروتکل \متن‌لاتین{MQTT} & نسخه ۳ و ۵ & نسخه ۳ و ۵ & نسخه ۳ & نسخه ۳ و ۵ & نسخه ۳ و ۵ & نسخه ۳ \\
\خط‌پر
کیفیت‌سرویس‌های پشتیبانی شده & ۰، ۱ و ۲ & ۰، ۱ و ۲ & ۰، ۱ و ۲ & ۰، ۱ و ۲ & ۰، ۱ و ۲ & ۰، ۱ و ۲ \\
\خط‌پر
سیستم‌عامل‌های پشتبیانی شده & \متن‌لاتین{Linux, Mac, Windws} & \متن‌لاتین{Windows, Linux, Mac, Raspberry Pi} & \متن‌لاتین{Windows, Linux} & \متن‌لاتین{Windows, Mac, Linux} & \متن‌لاتین{Linux, Mac} & \متن‌لاتین{Linux, Mac, Windows} \\
\خط‌پر
\پایان{جدول}
\پایان{لوح}


برای تست از دو سناریو مختلف استفاده شده است. در هر دو سناریو یک انتشاردهنده، یک مشترک و یک سرور قرار دارد. در سناریو اول برای شبیه‌سازی از لپ‌تاب‌های شخصی استفاده شده است
تا شرایط لبه شبیه‌سازی شود و در سناریو دوم از ماشین‌های مجازی زیرساخت ابری \متن‌لاتین{Google Cloud Platform} استفاده شده است.
در هر دو این سناریو مساله تعداد ارتباط‌های همزمان مورد بحث قرار نگرفته است که با توجه به تعداد زیاد اشیا می‌تواند چالش بزرگی برای منابع سیستم دلال پیام باشد.
مساله دیگر در این ارزیابی‌ها عدم استفاده از تکنولوژی‌های ابری به روز مانند کانتینرها و \متن‌لاتین{Kubernetes} می‌باشد که امروز جز جدانشدنی از پیاده‌سازی سیستم‌های اینترنت اشیا حتی در لبه می‌باشند.
