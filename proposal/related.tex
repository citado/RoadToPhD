\فصل{کارهای مرتبط}

\شروع{sidewaystable}
\شرح{دسته‌بندی پژوهش‌های حوزه ارزیابی شبکه‌های توان پایین برد بلند}
\فضای‌و{5mm}
\کوچک
\تنظیم‌ازوسط
\begin{tabularx}
  {\textwidth}
  {|*{13}{X|}}
  \خط‌پر
  مرجع &
  تحرک &
  پوشش‌دهی &
  شبکه &
  لایه انتقال &
  لایه اپلیکشن &
  اندازه بسته &
  \متن‌لاتین{Coding Rate} &
  تاخیر &
  محیط &
  \متن‌لاتین{LoRa Mesh} &
  توان مصرفی &
  شبیه‌سازی \\
  \خط‌پر
  \مرجع{sensors-18-00772-v3} &
  سیار &
  گزارش شده &
  \متن‌لاتین{LoRa} &
  ندارد &
  ندارد &
  گزارش شده &
  گزارش شده &
  گزارش نشده &
  باز &
  ندارد &
  گزارش نشده &
  واقعی / \متن‌لاتین{cloudRF} \\
  \خط‌پر
  \مرجع{sensors-19-00007} &
  ثابت &
  گزارش شده &
  \متن‌لاتین{NB-IoT} &
  \متن‌لاتین{TCP / UDP} &
  \متن‌لاتین{MQTT / CoAP} &
  گزارش شده &
  گزارش نشده &
  گزارش شده &
  باز &
  ندارد &
  گزارش نشده &
  \متن‌لاتین{Ericsson inter. sim.} \\
  \خط‌پر
  \مرجع{sensors-20-03061-v2} &
  ثابت &
  گزارش شده &
  \متن‌لاتین{LoRa} &
  ندارد &
  ندارد &
  گزارش شده &
  گزارش شده &
  گزارش شده &
  باز / بسته &
  ندارد &
  گزارش نشده &
  \متن‌لاتین{ns-3} \\
  \خط‌پر
  \مرجع{sensors-20-00280-v2} &
  ثابت &
  گزارش شده &
  \متن‌لاتین{LoRa} &
  \متن‌لاتین{UDP ov IPv6} &
  \متن‌لاتین{CoAP} &
  گزارش شده &
  گزارش شده &
  گزارش شده &
  باز / بسته &
  ندارد &
  گزارش شده &
  واقعی \\
  \خط‌پر
  \مرجع{sensors-20-06721} &
  ثابت &
  گزارش شده &
  \متن‌لاتین{LoRa} &
  ندارد &
  ندارد &
  گزارش شده &
  گزارش شده &
  گزارش شده &
  بسته &
  ندارد &
  گزارش شده &
  واقعی \\
  \خط‌پر
  \مرجع{SanchezIborra2020} &
  متحرک &
  گزارش شده &
  \متن‌لاتین{LoRa \ NB-IoT} &
  \متن‌لاتین{UDP/TCP ov IPv6} &
  \متن‌لاتین{CoAP / ReST} &
  گزارش شده &
  گزارش شده &
  گزارش شده &
  باز &
  ندارد &
  گزارش شده &
  واقعی \\
  \خط‌پر
  \مرجع{Lee2018} &
  ثابت &
  گزارش شده &
  \متن‌لاتین{LoRa} &
  ندارد &
  ندارد &
  گزارش نشده &
  گزارش شده &
  گزارش شده &
  باز / بسته &
  دارد &
  گزارش نشده &
  واقعی \\
  \خط‌پر
  \مرجع{Marahatta2021} &
  ثابت &
  گزارش شده &
  \متن‌لاتین{LoRa} &
  ندارد &
  ندارد &
  گزارش شده &
  گزارش شده &
  گزارش شده &
  باز / بسته &
  دارد &
  گزارش نشده &
  \متن‌لاتین{ns-2} \\
  \خط‌پر
\end{tabularx}
\پایان{sidewaystable}

\قسمت{ارزیابی کارایی}

در حوزه اینترنت اشیا پروتکل‌ها و معماری‌های مختلفی وجود دارد که می‌توان از آن‌ها استفاده کرد. هر یک از پروتکل‌ها یا معماری‌ها در شرایط خاصی کارآیی خوبی دارند بنابراین پژوهش‌های زیادی برای ارزیابی کارایی آن‌ها صورت پذیرفته است.
در این ارزیابی پروتکل‌ها و معماری‌ها بدون هیچ تغییر یا بهبودی ارزیابی می‌شوند.
این ارزیابی‌ها به صورت کلی در دو دسته واقعی یا شبیه‌سازی می‌باشند. برخی از آن‌ها در یک لایه به خصوص مانند لایه دسترسی یا لایه هسته فعالیت کرده‌اند و برخی یک راه‌حل انتها به انتها اینترنت اشیا را ارزیابی کرده‌اند.

پارامترهای متنوعی مورد ارزیابی قرار می‌گیرند که از عمده آن‌ها می‌توان توان مصرفی، نرخ داده، جابجای اشیا و \نقاط‌خ را نام برد.
در کنار این پارامترها پژوهش‌هایی در لایه فیزیکی \متن‌لاتین{LoRa} بحث تاثیر تداخل سایر پروتکل‌هایی که از باند \متن‌لاتین{ISM} استفاده می‌کنند، تاثیر پارامترهای منطقه‌ای و محیط عملیاتی، تاثیر پارامتر‌های جوی مانند باران
را بررسی کرده‌اند.
از سوی دیگر پژوهش‌هایی در لایه پیوند داده \متن‌لاتین{LoRaWAN} بحث استفاده از \متن‌لاتین{IPv6}، توزیع ترافیک، ارتباط با شبکه‌های دیگر مانند \متن‌لاتین{WiFi} و \نقاط‌خ را بررسی کرده‌اند.

\زیرقسمت{مرجع \مرجع{Almojamed2021}}

پژوهش \مرجع{Almojamed2021} به دنبال بررسی جابجایی در شبکه‌های \متن‌لاتین{LoRaWAN} است و از همین رو با استفاده از شبیه‌سازی با نرم‌افزار \متن‌لاتین{OMNET++}
دو مدل جابجایی شناخته شده را برای تاثیر جابجایی بر کارایی شبکه‌ی \متن‌لاتین{LoRaWAN} مورد تحقیق قرار می‌دهد. این پژوهش بیان می‌کند در نظر گرفتن جابجایی در شبکه‌های
\متن‌لاتین{LoRaWAN} ایده‌ی جدیدی نیست اما پژوهش‌های حاضر مدل‌های جابجایی کمی را در نظر گرفته‌اند و سرعت و تعداد اشیای آن‌ها محدود بوده است.
علاوه بر این، برخی از این پژوهش‌ها بر روی جابجایی از منظر فراگرد بین شبکه‌های مختلف تمرکز کرده‌اند یا جابجایی را تنها برای بخشی از اشیا در نظر گرفته‌اند.

نوآوری‌های این پژوهش شامل مواردی است که در ادامه آورده شده است:

\شروع{فقرات}
\فقره ادغام کردن \متن‌لاتین{Framework for LoRa} یا مختصرا \متن‌لاتین{FLoRa} با مدل‌های جابجایی مختلف در شبیه‌ساز \متن‌لاتین{OMNet++}،
\فقره ارزیابی کارایی \متن‌لاتین{LoRaWAN} بر پایه مدل‌های جابجایی مختلف (مدل‌های ثابت، گاوس--مارکو و \متن‌لاتین{RWP})
\فقره انجام یک ارزیابی جامع روی سه مدل جابجایی با در نظر گرفتن شبکه‌ای با بیش از ۵۰۰۰ دستگاه انتهایی که با سرعت‌های مختلف (تا ۲۵ متر بر ثانیه) جابجا می‌شوند و در عین حال از تعداد
\متن‌لاتین{Gateway}های مختلفی استفاده می‌کنند.
\فقره در نظر گرفتن تاثیر اندازه داده بر کارایی شبکه \متن‌لاتین{LoRaWAN} با جابجایی
\پایان{فقرات}



\زیرقسمت{مرجع \مرجع{FerrndezPastor2018}}

در پژوهش \مرجع{FerrndezPastor2018} تلاش برای طراحی معماری در کشاورزی دقیق بوده است. این پژوهش بیان می‌کند که کشاورزان مهارت زیادی در کشاورزی بدست آورده‌اند اما آشنایی کمی با سیستم‌های اینترنت اشیا دارند
بنابراین کاربران اینترنت اشیا می‌بایست در بهبود استفاده و ترکیب آن مشارکت کنند. طراحی با مرکزیت کاربر یا اختصارا \متن‌لاتین{UCD}، متد توسعه‌ای است که گارانتی می‌کند محصول، نرم‌افزار و یا وب‌سایت به سادگی قابل استفاده باشند.
در بحث کشاورزی دقیق، طراحی با محوریت کاربر پروسه طراحی را تعریف می‌کند که کشاورزان بر چگونگی شکل‌گیری طراحی تاثیر می‌گزارند. اصول یک طراحی با مرکزیت کاربر را می‌توان به شرح زیر برشمرد:

\شروع{فقرات}
\فقره \متن‌سیاه{جمع‌آوری نیازمندی‌ها}: درک و مشخص کردن محیط استفاده
\فقره \متن‌سیاه{مشخص کردن نیازمندی‌ها}: مشخص کردن نیازمندی‌های کاربر و سازمان
\فقره \متن‌سیاه{طراحی}: تولید نمونه و طراحی
\فقره \متن‌سیاه{ارزیابی}: انجام ارزیابی‌های مبتنی بر کاربر
\پایان{فقرات}

این پژوهش کشاورزی دقیق را مشتمل بر نظارت بر پارامترهای مزرعه، مانیتورینگ، نظارت بر زمین‌ها و نظارت بر انبار می‌داند. پژوهشگران سامانه \متن‌لاتین{SmartFarmNet} را طراحی کرده‌اند که پژوهش حاضر بر پایه تجربیات آن شکل گرفته است.
بستر \متن‌لاتین{SmartFarmNet} میتواند با هر سنسور، دوربین، ایستگاه‌های هواشناسی و \نقاط‌خ کار کرده و داده‌های آن‌ها را ذخیره کند. این ذخیره‌سازی در ابر صورت می‌گیرد و از آن برای ارزیابی کارایی و ایجاد توصیه‌ها استفاد می‌گردد.
پژوهشگران مدعی هستند این سامانه اولین و بزرگترین (با توجه به تعداد سنسورهای متصل، کاربران فعال و محصولات تحت نظارت) در نوع خود می‌باشد.

پژوهشگران برای فهم دقیق نیازمندی‌های به گلخانه‌های هوشمند سر زده‌اند و به نظر می‌رسد سیستم‌های زیر در این گلخانه‌های ضروری است. سیستم‌های آبیاری و مواد مغذی و سیستم‌های خنک‌کننده و تهویه.
همانطور که پیشتر هم بیان شده بود این پژوهش قصد دارد از تجربه و دانش کشاورزان بهره ببرد و از همین رو با مصاحبه با کشاورزان پروسه‌ها را استخراج کرده و کار را آغاز می‌کنند.
در اولین گام پس از مصاحبه اشیا، روابط‌شان و سرویس‌های احتمالی مشخص می‌شوند. زمانی که سرویس‌ها و اشیا تشخیص داده شدند، آن‌ها می‌بایست به وسیله‌ی پروتکل‌های اینترنت اشیا متصل شوند.
این پروتکل‌ها می‌بایست استاندارد و همکنش‌پذیر باشند تا بتوان از برنامه‌های کاربردی آزاد و باز استفاده‌پذیر بهره برد.
رابط‌های انسانی تنظیم شوند.
قوانین خبره و سرویس‌های هوشمند آنالیز می‌شوند و در نهایت رویه‌ی نصب، مراقبت و راه‌اندازی مشخص می‌گردد.

این پژوهش با توجه به مواردی که ادامه می‌آید پروتکل \متن‌لاتین{MQTT} را برای لایه اپلیکشن انتخاب می‌کند.

\شروع{فقرات}
\فقره پروتکل \متن‌لاتین{MQTT} یک پروتکل اشتراک و انتشار است که برای دستگاه‌هایی با منابع محدود طراحی شده است. مدلی که توسط شرکت‌های بزرگ به صورت جهانی اجرا شده است
و می‌تواند با سیستم‌های قدیمی نیز کار کند.
\فقره تمام پیام‌های موضوعاتی که از کلماتی تشکیل شده است که با ``/'' از یکدیگر جدا شده‌اند. یک فرمت مرسوم \متن‌لاتین{/place/device-type/device-id/measurement-type/status}
است. مشترکین می‌توانند روی اندازه‌گیری‌هایی که از یک کلاس خاص از اشیا می‌آید، مشترک شوند.
\فقره پهنای باند لازم برای پروتکل \متن‌لاتین{MQTT} بسیار کم بوده و ماهیت آن به گونه‌ای است که از منظر مصرف انرژی بسیار کارا است.
\فقره رابط برنامه نویسی آن بسیار ساده است و در سمت کلاینت حافظه کمی مصرف می‌کند. این باعث می‌شود که برای سیستم‌های نهفته انتخاب مناسبی باشد.
\فقره سطوح مختلف کیفیت سرویس در این پروتکل می‌تواند عملیات‌های قابل اطمینان فراهم آورد.
\پایان{فقرات}

این پژوهش بیان می‌کند که در صورت وجود تاسیسات در مزرعه، این تاسیسات می‌بایست با حفظ کارکرد قبلی وارد چرخه هوشمند‌سازی شوند. از این را با اتصال تعدادی \متن‌لاتین{Edge-Node} به این تاسیسات می‌توان
کارکرد سابق آن‌ها را حفظ کرده و از آن‌ها برای یک پروسه آموزش با نظارت استفاده کرد. این اتفاق در شکل \رجوع{شکل: استفاده از Fog-Node و Edge-Node در تاسیسات موجود کشاورزی دقیق} آورده شده است.

\شروع{شکل}
\درج‌تصویر[width=\textwidth]{./img/precision-agriculture-fog-edge-nodes.png}
\تنظیم‌ازوسط
\شرح{استفاده از \متن‌لاتین{Fog-Node} و \متن‌لاتین{Edge-Node} در تاسیسات موجود کشاورزی دقیق \مرجع{FerrndezPastor2018}}
\برچسب{شکل: استفاده از Fog-Node و Edge-Node در تاسیسات موجود کشاورزی دقیق}
\پایان{شکل}

یکی از موارد مهمی که این پژوهش مطرح می‌کند چگونگی تست و ارزیابی مدل یادگیری ماشین است. این پژوهش مراحلی را در جهت جمع‌اوری داده و تشکیل مجموعه‌های داده‌ای تست، ارزیابی و آموزش معرفی می‌کند.

برای ارزیابی از یک گلخانه هوشمند استفاده شده است. در این گلخانه آب، خاک، اقلیم و انرژی مدیریت و مانیتور می‌شوند. همانطور که پیشتر بیان شد از پروتکل \متن‌لاتین{MQTT} برای اشیا استفاده می‌شود و دلال پیام آن
در \متن‌لاتین{Fog Node}ها مستقر شده است. در این پیاده‌سازی عملی از دو \متن‌لاتین{Edge Node} و یک \متن‌لاتین{Fog Node} استفاده شده است. الگوریتم‌های یادگیری ماشین و هوش مصنوعی نیز به زبان پایتون و متن‌باز
توسعه پیدا کرده‌اند که در \متن‌لاتین{Fog Node} اجرا می‌شوند. در نهایت در این پیاده‌سازی یک قسمت ابری نیز وجود دارد که شامل داشبردها و محل نگهداری داده‌ها است. در نهایت زیرساخت ارتباطی با توجه به فضای محدود گلخانه شبکه
\متن‌لاتین{Zigbee} بوده است.

\زیرقسمت{مرجع \مرجع{Weber2016}}

پژوهش \مرجع{Weber2016} پیش از ارائه استانداردهایی مانند \متن‌لاتین{Static Context Header Compression} یا مختصرا \متن‌لاتین{SCHC} از \متن‌لاتین{IETF} در حوزه عملیاتی کردن \متن‌لاتین{IPv6} روی \متن‌لاتین{LoRaWAN}
با ارائه \متن‌لاتین{6LoRaWAN} کار کرده است و در آن زمان پژوهشگران این پژوهش در همکاری با \متن‌لاتین{IETF} قصد تهیه یک نسخه استاندارد از کارشان را داشته‌اند. این پژوهش روش پیشنهادی را به صورت عملی پیاده‌سازی و در عمل ارزیابی کرده است.

این پژوهش به دو مساله در ارتباط میان \متن‌لاتین{IPv6} و \متن‌لاتین{LoRaWAN} اشاره می‌کند. مساله اول حداقل واحد انتقالی (\متن‌لاتین{MTU}) در پروتکل \متن‌لاتین{IPv6} است که مقدار آن برابر با ۱۲۸۰ بایت است.
مساله بعدی اندازه متغیر بسته‌های \متن‌لاتین{LoRaWAN} بر پایه مقادیر مختلف برای فاکتور گسترش است. این پژوهش بیان می‌کند روش پیشنهادی \متن‌لاتین{6LoRaWAN}
دقیقا مشابه با \متن‌لاتین{6LoWPAN} یک پروتکل تطبیقی است که امکان ارتباط میان پروتکل \متن‌لاتین{IPv6} در لایه شبکه با دو لایه پیوند داده و فیزیکی \متن‌لاتین{LoRaWAN} و \متن‌لاتین{LoRa} را فراهم می‌آورد.

این پژوهش از فشرده‌سازی سرآیند برای رفع چالش اندازه بسته‌های \متن‌لاتین{IPv6} استفاده می‌کند و بیان می‌کند برای تبدیل پروتکل از \متن‌لاتین{6LoRaWAN} می‌توان از \متن‌لاتین{Gateway} یا سرور شبکه استفاده کرد.
معماری پیشنهادی بر پایه استفاده از \متن‌لاتین{Gateway} در قالب مدل لایه‌ای \متن‌لاتین{OSI} در شکل \رجوع{شکل: مدل لایه‌ای 6LoRaWAN} آورده شده است.

برای فشرده‌سازی سرآیند این پژوهش تنها سرآیند \متن‌لاتین{IPv6} را در نظر می‌گیرد و سرآیند پروتکل‌هایی چون \متن‌لاتین{UDP} را مورد بحث قرار نمی‌دهد.
در فرآیند فشرده‌سازی نیاز است اطلاعاتی مانند \متن‌لاتین{DevAddr} میان مبدل پروتکل و دستگاه هماهنگ شده باشند، برای همین نیاز است که این مبدل اطلاعات پروسه‌های عضویت \متن‌لاتین{ABP} و \متن‌لاتین{OTAA}
را داشته باشد. اطلاعات \متن‌لاتین{ABP} به صورت ایستا موجود هستند اما اطلاعات \متن‌لاتین{OTAA} نیاز است که پس تولید در فرآیند عضویت در این مبدل ذخیره شوند.
در نهایت این پژوهش روش پیشنهادی را پیاده‌سازی کرده و آن در یک مقیاس بسیار کوچک به صورت عملیاتی مورد ارزیابی قرار می‌دهد، در این ارزیابی فقط کارکرد صحیح مدنظر بوده است.

\شروع{شکل}
\درج‌تصویر[width=\textwidth]{./img/6lorawan.png}
\تنظیم‌ازوسط
\شرح{مدل لایه‌ای \متن‌لاتین{OSI} روش پیشنهادی \متن‌لاتین{6LoRaWAN} \مرجع{Weber2016}}
\برچسب{شکل: مدل لایه‌ای 6LoRaWAN}
\پایان{شکل}

\زیرقسمت{مرجع \مرجع{Augustin2016}}

پژوهش \مرجع{Augustin2016} با هدف ارزیابی توانایی گیرنده \متن‌لاتین{LoRa} آزمایش عملی را انجام داده است. در این آزمایش \متن‌لاتین{Gateway} در یک فضای بسته قرار گرفته است و فرستنده
به صورت متحرک در فضای شهری حرکت کرده است. پهنای باند مورد استفاده ۱۲۵ کیلوهرتز، نرخ کدگذاری $4/5$ و کمترین توان ارسالی مورد استفاده قرار گرفته است.

در آزمایش دیگری این پژوهش قصد ارزیابی پوشش شبکه \متن‌لاتین{LoRa} را داشته است و از این رو در فضای شهری پاریس \متن‌لاتین{Gateway} را در طبقه دوم یک ساختمان مستقر کرده و فرستنده
درون یک ماشین در موقعیت‌های مشخصی قرار گرفته است. این آزمایش نشان داده است در فاصله ۳۴۰۰ متری با استفاده از فاکتور گسترش ۱۲ تا ۴۰ درصد بسته‌ها به گیرنده رسیده‌اند.
آز آنجایی که آزمایش در فضای شهری بوده است در نقاط مورد آزمایش گاه ساختمان‌های بلندی نیز وجود داشته‌اند که در نتیجه آن کارایی فاکتور گسترش ۱۲ برای افزایش برد به خوبی مشهود است.
این آزمایش پروتکل \متن‌لاتین{LoRa} را هدف قرار داده است و از بازارسال، \متن‌لاتین{Ack} یا \متن‌لاتین{ADR} استفاده‌ای نشده است.

پس از ارزیابی لایه فیزیکی و پروتکل \متن‌لاتین{LoRa} این پژوهش لایه \متن‌لاتین{LoRaWAN} را مورد ارزیابی قرار می‌دهد. در اولین ارزیابی این پژوهش قصد دارد بیشترین گذردهی یک دستگاه
با استفاده از \متن‌لاتین{LoRaWAN} را محاسبه کند. همانطور که انتظار می‌رود این گذردهی بیشتر وابسته به لایه فیزیکی بوده است و با لایه پیوند داده ارتباطی پیدا نمی‌کند اما یک دید کلی از
کارایی در هنگام استفاده از \متن‌لاتین{LoRaWAN} می‌دهد. در نتیجه این پژوهش دیده می‌شود در حالتی که اندازه بسته بسیار کوچک است \متن‌لاتین{Duty Cycle} محدود کننده نیست بلکه
محدودیت از سمت پنجره‌های دریافت است که باعث می‌شود ارسال بسته‌ی بعدی به تاخیر بیافتد. در استاندارد فعلی \متن‌لاتین{LoRaWAN} راهی برای شکستن بسته‌ها و ارسال آن‌ها در قطعات دیده نشده است.

در ادامه این پژوهش \متن‌لاتین{LoRaWAN} را برای لود بالا ارزیابی می‌کند در نتیجه این ارزیابی مشخص می‌شود که پروتکل \متن‌لاتین{LoRaWAN} نسبت به افزایش نرخ بسته‌ها حساس است و نمی‌تواند برای لود بالا کارایی داشته باشد.
این پژوهش پیشنهاد تغییر لایه دسترسی همزمان از \متن‌لاتین{ALOHA} را مطرح می‌کند و از سوی دیگر بیان می‌کند این پروتکل برای مصارف حساس به تاخیر طراحی نشده است. یک راه حل پیشنهادی برای افزایش گذردهی ارسال یک بسته
بیش از یکبار است که البته باید خطر افزایش تصادم را نیز در نظر گرفت.

\زیرقسمت{مرجع \مرجع{Bharadwaj2016}}

پژوهش \مرجع{Bharadwaj2016} در هندوستان انجام شده است و با توجه به حجم بالای زباله‌های شهری قصد دارد یک شیوه‌ی هوشمند برای مدیریت پسماند ارائه کند که در ابعاد شهری و عملیاتی قابل استفاده باشد.
سیستم پیشنهادی بر پایه شبکه ارتباطی \متن‌لاتین{LoRaWAN} و پروتکل \متن‌لاتین{MQTT} کار می‌کند. این پژوهش بیان می‌کند استفاده از زیرساخت ارتباطی \متن‌لاتین{GSM} با توجه به نیاز آن به سیم کارت قابلیت
استفاده در همه سطل زباله‌ها را ندارد و شیوه‌هایی که نیز بر پایه اهراز هویت اشخاص برای سطل‌های زباله کار می‌کنند در ابعاد بزرگ عملیاتی نیستند.

پارامترهای اندازه‌گیری شده در سطل‌های زباله مشابه با سایر کارها در این حوزه متشکل شده از وزن سطل، میزان پر بودن سطل و گازهای سمی داخل است.

در لایه کاربرد در این معماری سرویس‌های پنل کاربری، سرویس فراهم آوری سناریو و پردازش داده و پایگاه داده‌ای \متن‌لاتین{NoSQL} قرار گرفته‌اند که با پروتکل \متن‌لاتین{MQTT} داده‌ها را دریافت می‌کنند.
نودها داده‌ها را به وسیله‌ی زیرساخت \متن‌لاتین{LoRaWAN} برای \متن‌لاتین{Gateway} ارسال می‌کنند و داده‌ها از \متن‌لاتین{Gateway} به وسیله‌ی \متن‌لاتین{MQTT} برای سرورهای ابری ارسال می‌شود.

این پژوهش برای مدیریت سطل‌های زباله شهر را به مربع‌هایی تقسیم کرده است و برای هر مربع یک مدیر در نظر گرفته شده است که با استفاده از داشبرد خود می‌تواند سطل‌های زباله تحت مدیریت خود را نظارت کند.
از سوی دیگر برنامه موبایل نیز برای رانندگان ماشین‌های حمل زباله تعبیه شده است که به واسطه‌ی آن بهترین راه برای رسیدن به سطل‌های زباله پر شده برای آن‌ها پیدا می‌شود.

\زیرقسمت{مرجع \مرجع{Chen2019}}

هدف در پژوهش \مرجع{Chen2019} ارائه‌ی یک معماری عملیاتی برای شبکه‌ی شناختی می‌باشد. این پژوهش با بررسی تکنولوژی‌های مختلف ارتباطی در شبکه‌های اینترنت اشیا به این تقطه می‌رسد که می‌توان
با استفاده از شبکه شناختی از مجموع ویژگی‌های این ارتباط‌ها بهره‌برداری کرد. این معماری پیشنهادی که از تکنولوژی‌های ارتباطی مختلفی برای کاربردهای مختلف استفاده می‌کند در شکل \رجوع{شکل: معماری شبکه شناختی}
آورده شده است.

\شروع{شکل}
\درج‌تصویر[width=\textwidth]{./img/cognitive-lpwa.png}
\تنظیم‌ازوسط
\شرح{معماری شبکه شناختی پیشنهادی در پژوهش \مرجع{Chen2019}}
\برچسب{شکل: معماری شبکه شناختی}
\پایان{شکل}

در این شبکه شناختی نود در ابتدا با یک تکنولوژی و سپس به واسطه تکنولوژی‌های سلولی به لبه و سرورهای ابری متصل می‌شود. این پژوهش معماری پیشنهادی را به صورت عملی مورد ارزیابی قرار داده است.

\زیرقسمت{مرجع \مرجع{Tseng2021}}

پژوهش \متن‌لاتین{Tseng2021} قصد دارد مهندسی عمران را با مهندسی برق ترکیب کند تا بتوان یک دانشگاه هوشمند و سبز بسازد. این پژوهش هدف اصلی خود را تاثیر پوشش سقف بتونی با صفحات خورشیدی
بر دمای محیط عنوان می‌کند و برای این ارزیابی از سنسورهای \متن‌لاتین{LoRa} با توجه به مصرف کم و برد بالا استفاده می‌کند. این پژوهش بیان می‌کند تکنولوژی اینترنت اشیا سبز می‌تواند به نظارت که امری مهم
در دانش مهندسی است کمک کند.

با توجه به شرایط واقعی این پژوهش، از سنسورهای دما و رطوبت برای اندازه‌گیری رطوبت و دما تجربه شده توسط انسان در محیط دانشگاه استفاده شده است.
سنسور مورد استفاده \متن‌لاتین{SHT10} بوده است و شبکه‌ی مورد استفاده نیز همانطور که ذکر شد، شبکه‌ی \متن‌لاتین{LoRa} بوده است. در این شبکه داده‌ها
به \متن‌لاتین{Gateway} مخابره شده و از آنجا با اترنت برای ذخیره‌سازی به دیتابیس ارسال می‌گردند.

این پژوهش از \متن‌لاتین{LoRa} به صورت مستقیم برای ارسال اطلاعات استفاده کرده است و نیازی به استفاده از کلاس‌های \متن‌لاتین{LoRaWAN} نداشته است.
در نهایت هدف اصلی این پژوهش ارزیابی روش‌های ساخت یک دانشگاه سبز و هوشمند بوده است و نتایج و جمع‌بندی در همین حوزه هستند.

\زیرقسمت{مرجع \مرجع{BertrandMartinez2020}}

در پژهش \مرجع{BertrandMartinez2020} پژوهشگران قصد استفاده از یک روش ساختارمند برای ارزیابی دلال‌های پیام \متن‌لاتین{MQTT} را دارند.
این پژوهش بیان می‌کند که در حوزه ارزیابی دلال‌های پیام \متن‌لاتین{MQTT} سه بحث کلی وجود دارد:

\شروع{فقرات}
\فقره ارزیابی کمی
\فقره ارزیابی کیفی
\فقره ساختار ارزیابی
\پایان{فقرات}

در پژوهش‌های پیشین به بحث‌های کمی بسیار پرداخته شده است اما بحث‌های کیفی مانند قابلیت اطمینان نیز اهمیت فراوانی دارند
که کارهای کمتری به آن‌ها پرداخته‌اند. هدف از این پژوهش ارائه یک ساختار برای ارزیابی است که در آینده بتوان بر پایه آن ارزیابی‌هایی
را با اهداف دلخواه ولی بدون از دست دادن پارامترهای مهم و تاثیرگذار صورت داد. مراحل این متد به شرح زیر است:

\شروع{شمارش}
\فقره بیان اهداف: اولین گام در ارزیابی یک سیستم تعریف اهداف مشخص و به دور از بایاس است. این اهداف ممکن است
در طول ارزیابی و با درک بهتر مساله تغییر کنند.
\فقره تعریف سیستم: دومین گام تعریف مرزهای سیستم مورد ارزیابی است. این مرزها می‌توانند معیارهای ارزیابی
سیستم را تعریف کنند.
\فقره تعریف متریک‌ها: سومین گام مشخص کردن متریک‌هایی است که قرار است استفاده شوند. این متریک‌ها شامل
متریک‌های کارایی و متریک‌های ویژگی‌ها هستند. این متریک‌ها می‌بایست همه‌ی مدهای عملیاتی سیستم را شامل شوند.
\فقره تعریف سناریو ارزیابی: در چهارمین گام سناریویی که در آن آزمایش‌ها صورت می‌پذیرد تعریف می‌گردد. از جمله آنچه در این سناریو تعریف می‌شود
می‌توان به لود، متغیرهای تاثیرگذار و همبندی منطقی سیستم اشاره کرد.
\فقره نصب و ارزیابی: بر پایه سناریویی که در گام چهارم مشخص شد ارزیابی در گام پنجم صورت می‌گیرد. در این گام نیاز است تعداد تکرار آزمایش برای
رسیدن به اطمینان در نتایج مشخص شود.
\فقره آنالیز نتایج: در ششمین گام نتایج گام پیش به صورت کامل مورد ارزیابی و بررسی قرار می‌گیرد و مشخص می‌شود آیا برای ارائه نتیجه نهایی و جمع‌بندی
قانع کننده هستند یا خیر، در صورت نیاز ممکن است گام پنجم با تغییراتی برای رسیدن به نتایج مورد نیاز تکرار شود.
\فقره ارائه نتایج: اگر نتایج گام پیشین دید مشخصی از معیارهای مورد بحث می‌دهند، ارائه نتایج و جمع‌بندی می‌تواند به انتخاب بهتر و گسترش دانش کمک کند.
\پایان{شمارش}

این پژوهش متد ذکر شده را برای ارزیابی ۱۲ بستر \متن‌لاتین{MQTT} متن‌باز استفاده می‌کند. برای این ارزیابی سه سطح کیفیت سرویس در پروتکل \متن‌لاتین{MQTT}
مدنظر است و از سوی دیگر نیازمندی‌های غیرکارکردی مانند کیفیت مستندات، گستره تنظیمات، کارکرد بر سیستم‌عامل‌های مختلف و \نقاط‌خ. در نهایت سه بستر انتخاب
با معیارهای کیفی برای مقایسه کارایی استفاده می‌شوند.

سیستم در این پژوهش تمامیت بستر \متن‌لاتین{MQTT} به عنوان یک جعبه سیاه است. در این پروژهش با اجزای داخل سیستم توجهی نشده و پارامترهایی که بر کارایی
کلاینت‌های تاثیر دارند از این پژوهش خارج هستند.

این پژوهش معیارهای غیرکارکردی را به دو دسته کلی تقسیم می‌کند متریک‌های تکنیکال و متریک‌های اپراتور. متریک‌های اپراتور متریک‌هایی است که اپراتور استفاده کننده از
پلتفرم به صورت روزانه با آن سر و کار دارد مانند کیفیت مستندات، گستردگی تنظیمات، سادگی اعمال تنظیمات و \نقاط‌خ. متریک‌های تکنیکال بحث‌هایی مانند پشتیبانی از سطوح
کیفیت سرویس متنوع، سیستم‌های مختلف و \نقاط‌خ قرار می‌گیرد.

معیارهای کارایی این پژوهش خود به سه دسته متریک‌های بهره‌وری، متریک‌های قابلیت اطمینان و دسترسی‌پذیری تقسیم می‌شوند.
بر اساس ارزیابی صورت پذیرفته از بین ۱۲ بستر متن‌باز اولیه تنها سه بستر \متن‌لاتین{RabbitMQ}، \متن‌لاتین{Mosquitto} و \متن‌لاتین{HiveMQ Community Edition}
برای ارزیابی کارایی انتخاب شده‌اند.

در نهایت این پژوهش برای کارایی این بسترها
از ترافیک زیادی استفاده نکرده است و می‌توان ارزیابی با داده‌های بیشتری نیز انجام داد و در ادامه تعداد کانکشن‌ها و توزیع‌شدگی این بسترها در نظر گرفته نشده است.

\زیرقسمت{مرجع \مرجع{Palmese2021}}

در پژوهش \مرجع{Palmese2021} پژوهشگران قصد مقایسه دو پروتکل انتشار و اشتراک \متن‌لاتین{CoAP} و \متن‌لاتین{MQTT-SN} را به صورت عملیاتی دارند.
پروتکل \متن‌لاتین{CoAP} به تازگی در پیش‌نویسی که توسط \متن‌لاتین{IETF} منتشر شده است از مدل انتشار و اشتراک پشتیبانی می‌کند و \متن‌لاتین{MQTT-SN}
مدل تغییر یافته پروتکل \متن‌لاتین{MQTT} برای شبکه‌های سنسوری می‌باشد. هر دو این پروتکل‌ها بر پایه \متن‌لاتین{UDP} بوده و مقایسه آن‌ها منصفانه به نظر می‌رسد.
این پژوهش بیان می‌کند پروتکل \متن‌لاتین{CoAP} برای شبکه‌هایی با پویایی بالا انتخاب عاقلانه‌ای به نظر می‌رسد.

می‌توان گفت که امروز در بحث پروتکل‌های اینترنت اشیا، یک تصمیم‌گیری مهم بین استفاده از حالت انتشار و اشتراک یا استفاده از حالت تقاضا و درخواست می‌باشد.
پروتکل‌های \متن‌لاتین{MQTT} و \متن‌لاتین{CoAP} به ترتیب از پروتکل‌های شناخته‌شده در حوزه انتشار و اشتراک و تقاضا و درخواست هستند.
این پژوهش در واقع قصد دارد آخرین بهبودهای این پروتکل‌ها در حوزه اینترنت اشیا را با یکدیگر مقایسه کند.

در نهایت جمع‌بندی این پژوهش به این شرح است. زمانی ماهیت شبکه تحت ثاثیر پارامترهایی مانند \متن‌لاتین{Duty Cycle} بوده
و اشیا نمی‌توانند یک ارتباط دائم داشته باشند استفاده از \متن‌لاتین{CoAP} گزینه‌ی بهتری است. از سوی دیگر پشتیبانی پروتکل
\متن‌لاتین{CoAP} از قطعه‌بندی آن را برای کاربردهایی با پیام‌های بزرگ مناسب می‌کند.

\زیرقسمت{مرجع \مرجع{Botez2021}}

در پژوهش \مرجع{Botez2021} هدف ارزیابی بحث‌های پردازش در ابر، پردازش در لبه و قسمت‌بندی شبکه در شبکه‌های \متن‌لاتین{5G} است.
به این منظور دو آزمایش صورت پذیرفته است. آزمایش اول استفاده از بستر ابری و لبه برای اجرای پلتفرم اینترنت اشیا
به صورت کانتینرسازی شده و ارزیابی منابع مصرفی آن است. برای بستر لبه از \متن‌لاتین{balenaCloud} و برای بستر ابری از سرویس‌های \متن‌لاتین{AWS}
استفاده شده است.
آزمایش دوم در شبکه‌ی ارتباطی \متن‌لاتین{NB-IoT} و بر بستر
پلتفرم \متن‌لاتین{Kubernetes} که بر پایه \متن‌لاتین{OpenStack} نصب شده است، صورت پذیرفته است.
هدف آزمایش دوم قسمت‌بندی شبکه برای پشتیبانی از انواع مختلف ترافیک بوده است.

در آزمایش اول، سناریو اول زیر ساخت لبه \متن‌لاتین{balenaCloud} که با استفاده از \متن‌لاتین{docker}
کار می‌کند، مورد استفاده است. پردازش روی \متن‌لاتین{RPi} نسخه ۴ صورت می‌گیرد. نتیجه پردازش داخل پایگاه داده‌ای
\متن‌لاتین{InfluxDB} ذخیره می‌شوند. در این سناریو سنسور دما \متن‌لاتین{DHT11} مستقیما
به \متن‌لاتین{RPi} نسخه ۴ متصل است.
در دو سناریو بعدی از زیرساخت ابری \متن‌لاتین{AWS} متعلق به شرکت آمازون مورد استفاده قرار گرفته است.
در این سناریوها داده از \متن‌لاتین{ESP32} به زیرساخت ابری ارسال می‌شود. در این سناریوها به ترتیب
از پروتکل‌های \متن‌لاتین{HTTP} و \متن‌لاتین{MQTT} استفاده شده است.
در زمان استفاده از \متن‌لاتین{MQTT} تاپیک‌های مختلفی برای کیفیت سرویس‌های متفاوت تخصیص داده
شده‌اند و در نهایت کارایی بهتری حاصل شده است.

در آزمایش دوم زیرساخت \متن‌لاتین{NB-IoT} در یک شبکه‌ی نسل چهار با تجهیزات شرکت نوکیا پیاده‌سازی شده است.
در این زیرساخت از \متن‌لاتین{OpenStack} استفاده شده است و بر پایه‌ی آن \متن‌لاتین{Kubernetes} فراهم شده است
تا سرویس‌های شبکه‌ی \متن‌لاتین{Backhaul} در قالب کانتینر بالا بیایند.
این سرویس‌ها \متن‌لاتین{Cloud-Native Network Function}ها نام دارند و سربار بسیار کمتری نسبت ماشین‌های مجازی
و آنچه در \متن‌لاتین{NFV MANO} بحث شده است دارند.

عملا این پژوهش ارزیابی را در حوزه مختلف انجام می‌دهد که وجه مشترکشان استفاده از سرویس‌های ابرزی و بسترهای ابری است.
این دو موضوع بحث تقسیم‌بندی در شبکه‌های نسل جدید و استفاده از پردازش در لبه می‌باشند.

\زیرقسمت{مرجع \مرجع{Jurva2020}}

پژوهش \مرجع{Jurva2020} به مساله پیچیدگی در مدیریت صحن هوشمند دانشگاه می‌پردازد. این پژوهش با معرفی مدل میکرو اپراتورها و ارزیابی آن در دانشگاه \متن‌لاتین{Oulu} فلاند تلاش می‌کند این مساله را حل کند.
میکرو اپراتور یک سرویس دهنده محلی می‌باشد که زیرساخت دیجیتال و سرویس‌های دانشگاه را مدیریت می‌کند.

این پژوهش دانشگاه هوشمند را به سه موجودیت تابعی می‌شکند. یک عملیات‌های اصلی دانشگاه، سرویس‌های صحن دانشگاه و محیط صحن دانشگاه.
در عملیات‌های اصلی دانشگاه، دیجیتال‌سازی و شبکه‌ها به آموزش بهتر کمک می‌کنند. دانشگاه می‌تواند از این زیرساخت برای ارائه آموزش در محدوده وسیع‌تر، واقعیت مجازی و عملیاتی ساختن تحقیقات استفاده کند.
سرویس‌های صحن دانشگاه تقریبا مشابه با سرویس‌هایی است که در مغازه‌ها، مرکزهای خرید و \نقاط‌خ در شهر هوشمند ارائه می‌شوند. از جمله این سرویس‌ها می‌توان به سرویس راهنمایی، پارک و شارژ وسایل نقلیه و \نقاط‌خ اشاره کرد.
سرویس‌های شهر هوشمند مانند حمل و نقل می‌تواند با سرویس‌های صحن دانشگاه ارتباط ترکیب شده و سرویس‌های هوشمند‌تری را به وجود بیاورند.

برای پیاده‌سازی دانشگاه هوشمند این پژوهش یک مدل (شکل \رجوع{شکل: چهارچوب دانشگاه هوشمند}) متشکل از لایه‌های افقی و عمودی را پیشنهاد می‌دهد. لایه‌ها عمودی موجودیت‌های تابعی را نمایندگی می‌کنند که پیشتر به آن پرداخته شد و لایه‌های
افقی پلتفرم‌های هستند که دسترسی را یکسان‌سازی می‌کنند. در چهارچوب پیشنهادی مدیریت زیرساخت و شبکه دانشگاه هوشمند توسط میکرو اپراتور صورت می‌پذیرد. این اپراتور می‌بایست با سیاست‌گذاران برای استفاده از پهنای باند در فضای دانشگاه
به توافق برسد، از زیرساخت‌های اپراتورهای فعلی برای سرویس خود بهره بگیرد، برای کسب مجوزهای لازم با مدیران دانشگاه در ارتباط باشد و \نقاط‌خ

\شروع{شکل}
\درج‌تصویر[width=\textwidth]{./img/smart-campus-framework.png}
\تنظیم‌ازوسط
\شرح{چهارجوب دانشگاه هوشمند \مرجع{Jurva2020}}
\برچسب{شکل: چهارچوب دانشگاه هوشمند}
\پایان{شکل}

\زیرقسمت{مرجع \مرجع{Baldo2021}}

در پژوهش \مرجع{Baldo2021} محققان بیان می‌کنند معماری‌های پیشنهادی تا به امروز بر سیستم‌های تک نقش متمرکز بودند و پژوهش حاضر قصد دارد سیستمی متشکل از سرویس‌هایی با توپولوژی‌ها متفاوت را
برای مدیریت هوشمند پسماند ارائه دهد. این معماری کلاس‌های مختلف \متن‌لاتین{LoRaWAN} را پیشتر کمتر به آن‌ها پرداخته شده بود، در برگرفته و شامل نودهای با پیچیدگی‌های رو به رشد است.
این معماری با سطل‌های ساده هوشمند شروع شده، شامل دورریزهایی است که با کاربران مراوده می‌کنند و دوربین‌هایی که سطل‌ها را برای جلوگیری از حریق نظارت می‌کنند.

معماری پیشنهادی بر پایه شبکه‌های \متن‌لاتین{LoRaWAN} از سه نود اصلی تشکیل شده است.
دسته اول نودهایی هستند که در سطل زباله‌های عمومی نصب شده و پارامترهایی مانند
میزان پر بودن، دما و جهت قرار گرفتن سطل را گزارش می‌کنند. این گروه از نودها از کلاس $A$ از \متن‌لاتین{LoRaWAN} استفاده می‌کنند چرا که نیازی به \متن‌لاتین{Downlink} ندارند.
این نودها با باتری فعالیت کرده و بازه نمونه‌گیری آن‌ها هر نیم ساعت می‌باشد.
نودهای دسته دوم پیچیدگی بیشتری داشته و در دورریزها نصب می‌شوند. یکی از وظایف این گروه از نودها تعامل با کاربران است.
با توجه به این موارد این دسته از نودها با پنل‌های \متن‌لاتین{Photo-Voltaic} کار کرده و
از کلاس $B$ از \متن‌لاتین{LoRaWAN} استفاده می‌کنند.
در نهایت دسته سوم نودهای نظارت ویدیویی هستند که برای مراقبت و نظارت بر دوریزها و سطل‌های زباله نصب می‌شوند. برای حفظ حریم خصوصی این دوربین‌ها هیچ تصویری را ارسال یا ذخیره نمی‌کنند
بلکه بر پایه \متن‌لاتین{Fog Computing} و توان پردازشی بالایی که دارند این تصاویر را پردازش می‌کنند.
این نودها با توجه به پردازش بالا و دارا بودن دوربین مصرف زیادی داشته و می‌توانند با هر دو کلاس $A$ و $B$ از \متن‌لاتین{LoRaWAN} فعالیت کنند.

تمامی نودها از \متن‌لاتین{SF}، هفت استفاده می‌کنند تا بتوانند برد بالایی را با توان مصرفی پایینی پشتیبانی کنند. پهنای باند مورد استفاده ۱۲۵ کیلوهرتز است و از نظر کدگذاری $4/5$ استفاده می‌شود.
داده‌ها توسط \متن‌لاتین{Gateway}ها جمع‌اوری شده و از آنجا توسط پروتکل \متن‌لاتین{MQTT} بدست \متن‌لاتین{Network Server} و \متن‌لاتین{Application Server} می‌رسند.
این پژوهش تمامی قطعات و چگونگی ساخت این سه نود را شرح داده است.

این پژوهش بیان می‌کنند داده‌های ساده‌ی سنسورهایی که میزان پر بودن سطل‌های زباله را نشان می‌دهند زمانی که در بستر هوش مصنوعی قرار می‌گیرند می‌توانند ارزش بالایی خلق کنند.
در نظر داشته باشید دوربین‌ها تنها برای پیشگیری از مواردی مانند آتش‌سوزی یا آسیب عمدی تعبیه شده‌اند اما در نهایت با جمع‌اوری داده‌های این سه گروه از اشیا می‌تواند تصمیم بهتری گرفته و دسته جدیدی از دادگان را خلق نمود.

برای نظارت بر سطح پر بودن سطل‌ها راه‌های متفاوتی وجود دارد که یکی از آن‌ها استفاده از سنسور فراصوت می‌باشد. این سنسور جهت عملکرد مناسب می‌بایست به صورت ضد آب جایگذاری شود اما با این وجود
این سنسورها یکی از بهترین و مطمئن‌ترین راه‌ها برای ارزیابی سطح پر بودن سطل زباله هستند. در کنار این سنسورها می‌توان از سیستم‌های نظارت ویدیویی نیز بهره جست که در نهایت ترکیب آن با داده‌های سنسور
از سطل‌ها توانایی تصمیم‌گیری بهتری می‌دهد. استفاده از سیستم‌های نظارت ویدیویی اولین دغدغه‌ای که به وجود می‌اورد بحث حفظ حریم خصوصی می‌باشد که این پژوهش به وسیله‌ی پردازش در لبه به آن پاسخ داده است.

نودهای قرارگرفته روی دورریزها از \متن‌لاتین{RFID} برای احراز هویت کاربران استفاده می‌کنند. آن‌ها در جهت تایید دسترسی نیاز دارند شناسه خوانده شده از \متن‌لاتین{RFID} را با سرور بررسی کنند
از این رو نیاز دارند که از ارتباط کلاس $B$ استفاده کنند تا در صورت وقوع تاخیر در سرور باز هم قابلیت دریافت پاسخ را داشته باشند. در ضمن استفاده از کلاس $B$ به سرور اجازه می‌دهد تا داده‌ها را
در صورت نیاز با درخواست از نود، دریافت کند.

\زیرقسمت{مرجع \مرجع{sensors-19-00007}}

پژوهش \مرجع{sensors-19-00007} قصد ارزیابی بین پروتکل‌های \متن‌لاتین{MQTT} و \متن‌لاتین{CoAP} که به ترتیب بر بسترهای \متن‌لاتین{UDP} و \متن‌لاتین{TCP} فعالیت می‌کنند، را دارد.
ارزیابی بر شبکه زیرساخت \متن‌لاتین{NB-IoT} صورت می‌گیرد که با فعالیت روی باند دارای لایسنس و نبود \متن‌لاتین{duty-cycle}، امکان اجرای پروتکل \متن‌لاتین{tcp} را نیز فراهم می‌آورد.
این پژوهش در نهایت نشان می‌دهد پروتکل \متن‌لاتین{MQTT} نسبت به پروتکل \متن‌لاتین{CoAP} کارآیی کمتری در معیارهای تاخیر، پوشش و ظرفیت سیستم دارد.

در این پژوهش با وجود اهمیت بحث‌های امنیتی بر پروتکل‌های اینترنت اشیا از آن‌ها صرف نظر شده است. برای امنیت پروتکل \متن‌لاتین{CoAP} می‌بایست از \متن‌لاتین{DTLS} و
برای امنیت پروتکل \متن‌لاتین{MQTT} از \متن‌لاتین{TLS} استفاده کرد.
در ضمن این پژوهش تنها به نسخه‌های مشهور این پروتکل‌های اکتفا کرده است و این در حالی است که نسخه‌های دیگری مانند \متن‌لاتین{MQTT-SN}، \متن‌لاتین{MQTT-over-QUIC} و \نقاط‌خ
که گاها برای راهکارهای اینترنت اشیا بهینه شده‌اند، نیازمند ارزیابی می‌باشند.

ارزیابی این پژوهش بر بستر \متن‌لاتین{NB-IoT} به صورت شبیه‌سازی و بر پایه \متن‌لاتین{Ericsson internet event-based radio network system simulator} انجام داده است.

\زیرقسمت{مرجع \مرجع{sensors-20-02078}}

پژوهشگران در \مرجع{sensors-20-02078} بیان می‌کنند که کارهای زیادی در کشاورزی دقیق با هدف مصرف توان پایین پیشنهاد شده‌اند اما تعداد کمی از آن‌ها در عمل تست شده‌اند.
این پژوهش به دنبال تست عملیاتی مصرف توان در کشاورزی و کاهش آن است.

در ابتدا از بین تکنولوژی‌های ارتباطی موجود بیان می‌شود زمانی که داده‌ی زیادی برای ارسال موجود نیست مانند سنسورهای کشاورزی، استفاده از \متن‌لاتین{LoRa} گزینه خوبی است.
در ضمن پوشش \متن‌لاتین{LTE} در نواحی غیرشهری و کشاورزی کافی نیست بنابراین استفاده از \متن‌لاتین{NB-IoT} گزینه خوبی نیست.

این پژوهش حسگرهای صنعتی کشاورزی را بررسی کرده و برای ارسال داده‌ها نرخ منطقی ۳۰ دقیقه را محاسبه می‌کند.

\زیرقسمت{مرجع \مرجع{sensors-18-00772-v3}}

در \مرجع{sensors-18-00772-v3} پژوهشگران برای ارزیابی شبکه \متن‌لاتین{LoRaWAN} در ابتدا یک ارزیابی رادیویی بر پایه مدل انتشار \متن‌لاتین{Okumura-Hata} صورت داده
و در ادامه با استفاده از محیط‌های واقعی شهری، نیمه‌شهری و غیرشهری دست به ارزیابی عملیاتی زده‌اند. ارزیابی اولیه به محققان کمک کرده است تا بتوانند محل خوبی را برای
تنها \متن‌لاتین{Gateway} این آزمایش پیدا کنند.

در ارزیابی عملیاتی از یک نود متحرک که پارامترهای ارسالش با زمان تغییر می‌کند، استفاده شده است. این نود برای بازه‌های زمانی ثابت می‌ماند تا تاثیر حرکت در پارامترها حذف شود.
در انتها این پژوهش بیان می‌کند برای اجرای یک زیرساخت \متن‌لاتین{LoRaWAN} باید مصالحه‌ای بین کیفیت لینک، نرخ داده‌ی انتقالی و سیار بودن نود برقرار شود.

مدل \متن‌لاتین{Okumura-Hata} در پیاده‌سازی‌هایی معتبر است که \متن‌لاتین{Gateway} نسبت به ساختمان‌ها و اطراف در ارتفاع بالاتری قرار داشته باشد.
\متن‌لاتین{Pham2020}

\زیرقسمت{مرجع \مرجع{sensors-20-00280-v2}}

پژوهش \مرجع{sensors-20-00280-v2}
اگلوریتم \متن‌لاتین{Static Context Header Compression (SCHC)} را برای \متن‌لاتین{IPv6} پیاده‌سازی کرده است.
این پژوهش از این پیاده‌سازی برای انتقال پروتکل \متن‌لاتین{CoAP} بر بستر \متن‌لاتین{UDP} و \متن‌لاتین{IPv6} استفاده کرده است.
هدف این پژوهش ارزیابی این الگوریتم بوده است.

این پژوهش بیان می‌کند منابع مصرفی در جهت استفاده از \متن‌لاتین{IPv6} نسبت به سود حاصل از آن بسیار کم است. در مقابل کارایی انرژی و داده در قطعه‌بندی کم است.
از سوی دیگر این پژوهش بیان می‌کند برای استفاده از قطعه‌بندی پیشنهادی \متن‌لاتین{IETF} نیاز است که ترتیب بسته‌ها حفظ شود.

از مزایای \متن‌لاتین{IPv6} می‌توان به زمانی اشاره کرد که یک نود بین \متن‌لاتین{gateway}ها جابجا می‌شود در صورت لزوم پروسه پیوستن به \متن‌لاتین{NS} را انجام می‌دهد که صورت استفاده از یک آدرس \متن‌لاتین{IPv6} ثابت اتصال تضمین خواهد شد.

در نظر گرفتن نودهای متحرک و سناریوهای انتها به انتها چیزی که در این پژوهش به آن پرداخته نشده است. اهمیت اصلی استفاده از پروتکل‌های استاندارد اینترنت در شبکه‌های اینترنت اشیا
در واقع ارزش خود را زمانی نشان خواهد داد که به صورت انتها به انتها استفاده شده و لایه‌های دیگر بتوانند در صورت نیاز با شیوه‌های استاندارد با اشیا ارتباط برقرار کنند.


\زیرقسمت{مرجع‌های \مرجع{SanchezIborra2020} و \مرجع{Santa2020}}

هر دو پژوهش \مرجع{SanchezIborra2020} و \مرجع{Santa2020} توسط یک تیم انجام شده و تکمیل شده یکدیگر می‌باشند. این پروژه‌ها وسایل حمل و نقل نوظهور مانند دوچرخه و اسکوترهای برقی را هدف قرار می‌دهند.
هدف طراحی یک سیستم \متن‌لاتین{OBU} یا \متن‌لاتین{on-board unit} می‌باشد که بتوان از آن‌ها برای گردآوری داده از این وسایل استفاده کرد.

این پژوهش‌ها بیان می‌کنند از بین راه‌کارهای توان پایین با برد بالا بهتر از برای گردآوری داده‌های غیرحیاتی از \متن‌لاتین{LoRaWAN} یا \متن‌لاتین{NB-IoT} استفاده کرد
و تکنولوژی‌های سلولی مانند \متن‌لاتین{GSM} یا \متن‌لاتین{LTE} برای ارتباط‌های حیاتی حفظ کرد.

این پژوهش‌ها یک دستگاه \متن‌لاتین{OBU} با هر دو تکنولوژی \متن‌لاتین{LoRaWAN} و \متن‌لاتین{NB-IoT} پیاده‌سازی کرده و آن را در عمل ارزیابی می‌کنند.
برای ارزیابی \متن‌لاتین{NB-IoT} از زیرساخت عملیاتی \متن‌لاتین{Vodafone} استفاده شده است.

\زیرقسمت{مرجع \مرجع{Islam2021}}

پژوهش \مرجع{Islam2021} به بحث استفاده از \متن‌لاتین{UAV}ها، ارتباطات \متن‌لاتین{LoRaWAN} و ارتباطات ماهواره‌ای در کشاوری پرداخته است.
این پژوهش بیان می‌کند که محدودیت‌های ارتباطی در این حوزه به قدر کافی مورد توجه قرار نگرفته است و تلاش دارد چالش‌های ارتباطی موجود در کشاورزی هوشمند را مرور کند.
این محدودیت‌های ارتباطی در کنار حسگرها، برای \متن‌لاتین{UAV}ها نیز مطرح هستند.

این پژوهش به عنوان یک راهکار در بحث مشکلات ارتباطی بحث \متن‌لاتین{Mesh LoRa} را مطرح می‌کند و مشکلات زیر را برای آن برمی‌شمارد.
\شروع{فقرات}
\فقره کارآیی و قابلیت بسیار پایین است.
\فقره نیاز به نگهداری از راه دور برای دروازه‌ها وجود دارد.
\فقره نیاز به پردازش لبه در دروازه‌ها وجود دارد.
\پایان{فقرات}

این پژوهش راهکاری برای این مشکلات ارائه نمی‌دهد و بحث \متن‌لاتین{Mesh LoRa} بسیار کلی ذکر می‌کند و ساختار مشخصی برای آن برنمی‌شمارد.

\زیرقسمت{مرجع \مرجع{Mishra2021}}

در \مرجع{Mishra2021} پژوهشگران دست به ارزیابی کارایی دلال‌های پیام برای پروتکل \متن‌لاتین{MQTT} زده‌اند. در این پژوهش معیارهای نرخ پیام، مصرف \متن‌لاتین{CPU} و تاخیر مورد نظر بوده‌اند.
در این میان تاخیر مدت زمانی است که از ارسال پیام در \متن‌لاتین{Publisher} تا دریافت آن در \متن‌لاتین{Subscriber} طول می‌کشد.
دلال‌های پیامی که برای این پژوهش بررسی شده‌اند در جدول \رجوع{جدول:دلال‌های پیام مورد ارزیابی در Mishra2021} آورده شده‌اند.

\شروع{لوح}
\شرح{دلال‌های پیام مورد ارزبابی در پروژه \مرجع{Mishra2021}}
\برچسب{جدول:دلال‌های پیام مورد ارزیابی در Mishra2021}
\فضای‌و{5mm}
\begin{tabularx}
  {\textwidth}
  {p{3cm}*6{X}}
\خط‌پر
دلال‌پیام \متن‌لاتین{MQTT} & \متن‌لاتین{Mosquitto} & \متن‌لاتین{Bevywise MQTT Route} & \متن‌لاتین{ActiveMQ} & \متن‌لاتین{HiveMQ CE} & \متن‌لاتین{VerneMQ} & \متن‌لاتین{EMQ X} \\
\خط‌پر
متن‌باز & است & نیست & است & است & است & است \\
\خط‌پر
زبان برنامه‌نویسی اصلی & \متن‌لاتین{C} & \متن‌لاتین{C} و \متن‌لاتین{Python} & \متن‌لاتین{Java} & \متن‌لاتین{Java} & \متن‌لاتین{Erlang} & \متن‌لاتین{Erlang} \\
\خط‌پر
نسخه پروتکل \متن‌لاتین{MQTT} & نسخه ۳ و ۵ & نسخه ۳ و ۵ & نسخه ۳ & نسخه ۳ و ۵ & نسخه ۳ و ۵ & نسخه ۳ \\
\خط‌پر
کیفیت‌سرویس‌های پشتیبانی شده & ۰، ۱ و ۲ & ۰، ۱ و ۲ & ۰، ۱ و ۲ & ۰، ۱ و ۲ & ۰، ۱ و ۲ & ۰، ۱ و ۲ \\
\خط‌پر
سیستم‌عامل‌های پشتبیانی شده & \متن‌لاتین{Linux, Mac, Windws} & \متن‌لاتین{Windows, Linux, Mac, Raspberry Pi} & \متن‌لاتین{Windows, Linux} & \متن‌لاتین{Windows, Mac, Linux} & \متن‌لاتین{Linux, Mac} & \متن‌لاتین{Linux, Mac, Windows} \\
\خط‌پر
\end{tabularx}
\پایان{لوح}

برای تست از دو سناریو مختلف استفاده شده است. در هر دو سناریو یک انتشاردهنده، یک مشترک و یک سرور قرار دارد. در سناریو اول برای شبیه‌سازی از لپ‌تاب‌های شخصی استفاده شده است
تا شرایط لبه شبیه‌سازی شود و در سناریو دوم از ماشین‌های مجازی زیرساخت ابری \متن‌لاتین{Google Cloud Platform} استفاده شده است.
در هر دو این سناریو مساله تعداد ارتباط‌های همزمان مورد بحث قرار نگرفته است که با توجه به تعداد زیاد اشیا می‌تواند چالش بزرگی برای منابع سیستم دلال پیام باشد.
مساله دیگر در این ارزیابی‌ها عدم استفاده از تکنولوژی‌های ابری به روز مانند کانتینرها و \متن‌لاتین{Kubernetes} می‌باشد که امروز جز جدانشدنی از پیاده‌سازی سیستم‌های اینترنت اشیا حتی در لبه می‌باشند.
در کنار این دو مورد استفاده از رمزنگاری در پروتکل \متن‌لاتین{MQTT} ممکن است، که این امر خود می‌تواند باعثث تغییر در کارایی دلال پیام شود و از این رو نیازمند ارزبابی است که این پژوهش به آن نپرداخته است.

در نهایت این پروژهش به این جمع‌بندی میرسد که دلال‌های پیام غیرگسترش‌پذیر مانند \متن‌لاتین{Mosquitto} که از تعداد مشخصی نخ استفاده می‌کنند برای محیط‌های محدود مناسب‌تر هستند.
از سوی دیگر از میان دلال‌های پیام گسترش‌پذیر \متن‌لاتین{ActiveMQ} کارآیی بالایی داشته و \متن‌لاتین{EMQ X}، \متن‌لاتین{VerneMQ} و \متن‌لاتین{HiveMQ} کارآیی مناسبی دارند.

\زیرقسمت{مرجع \مرجع{Cruz2021}}

پژوهشگران در \مرجع{Cruz2021} استفاده آزمایشی از \متن‌لاتین{LoRa} و \متن‌لاتین{LoRaWAN} به عنوان زیرساخت مدیریت پسماند در شهر لیسبون را گزارش می‌دهند.
در حال حاضر پسماند شهر لیسبون با استفاده از زیرساخت \متن‌لاتین{GPRS} فعالیت می‌کند. مدیران شهری قصد دارند این زیرساخت را به یک زیرساخت \متن‌لاتین{LPWAN} تغییر دهند.
در حال حاضر تکنولوژی‌های متنوعی مانند \متن‌لاتین{LoRaWAN}، \متن‌لاتین{Sigfox} و \متن‌لاتین{NB-IoT} در بازار وجود دارند و پژوهشگران قصد دارند کارآیی \متن‌لاتین{LoRaWAN} را در
سناریو مدیریت پسماند به صورت عملی بررسی کنند.

در واقع مشارکت اصلی این پژوهش در ارزیابی واقعی سنسورها، شبکه، کارایی انرژی و تکنولوژی‌های ارتباطی است که در قالب یک مدیریت هوشمند پسماند شهری رخ می‌دهد.
آزمایش‌های یک پژوهش در دو سطح رخ می‌دهند. در سطح اول هدف ارزیابی پوشش شبکه‌ای \متن‌لاتین{LoRa} برای مانیتورینگ مخازن پسماند سطحی و زیرزمینی است.
در سطح دوم هدف بررسی ظرفیت شبکه برای ارسال داده‌های مورد نیاز اپلیکشن‌ها می‌باشد.

این آزمایش‌های همگی توسط تجهیزات تجاری صورت گرفته است. برای \متن‌لاتین{Gateway}ها از دو \متن‌لاتین{Gateway} تجاری متفاوت استفاده شده است.
برای سنسورهای سطح پسماند نیز از دو سنسور تجاری متفاوت که قیمت و ویژگی‌های متفاوتی (حداکثر ارتفاع قابل سنج، سنسورهای ثانویه، باتری و \نقاط‌خ) دارند استفاده شده است.
سنسورهای سطح پسماند با نصب شدن بر درب مخزن و با استفاده از فراصوت فاصله خود تا پسماند را اندازه‌گیری می‌کنند. این روش می‌تواند خطا داشته باشد
و مکان سنسور و توان پردازشی آن تاثیر به سزایی در این امر دارد.

سه آزمایش کلی برای پوشش انجام شده است. پوشش کوتاه (نزدیک به ۱۰۰ متر)، میانی (نزدیک به یک کیلومتر) و طولانی (نزدیک به ۵ کیلومتر)، که در هر یک یک نود پسماند قرار دارد.
در آزمایش پوشش کوتاه مشکلی ایجاد نشده و همه اطلاعات دریافت می‌شوند ولی آزمایش برد طولانی عملا هیچ داده‌ای را منتقل نکرده است.
در آزمایش برد میانی نود پسماند یک نود زیرزمینی بوده است و چالش اصلی طراحی نود بوده است.

اندازه‌ی بسته‌های ارسالی از دو حسگر مقدارهای متفاوت ۴ و ۸ بایت بوده است. در شروع تست‌ها از مکانیزم نرخ داده تطبیق‌پذیر استفاده شده است و این مورد در ادامه غیرفعال شده است.

در نهایت این پژوهش بیان می‌کند که می‌توان از \متن‌لاتین{LoRa} برای زیرساخت مدیریت هوشمند پسماند شهری استفاده کرد.
البته این پژوهش در رابطه با تعداد سنسورهایی که می‌توان در شبکه داشت و تداخل آن‌ها مطالعه‌ای انجام نداده است.

\زیرقسمت{مرجع \مرجع{sensors-20-06721}}

پژوهشگران \مرجع{sensors-20-06721} تجربه بیش از دو سال نگهداری از شبکه‌ی سنسورهای فضای بسته دانشگاه \متن‌لاتین{oulu} کشور فلاند مبتنی بر \متن‌لاتین{LoRaWAN} در این پژوهش مرور می‌کنند.
این پژوهش بار زیادی داشته و این تنها مقاله‌ای نیست که از آن به چاپ رسیده است. در این تجربه ۳۳۱ سنسور در سقف در ریل‌های چراغ‌ها با فاصله‌های یک و نیم‌متری در یک محل اجتماعات در دانشگاه نصب شدند.
برای جمع‌آوری داده از یک \متن‌لاتین{Gateway} با دید مستقیم استفاده شده است. پیش از استقرار سنسورها یک تخمین برای پارامتر \متن‌لاتین{SF} در شبکه \متن‌لاتین{LoRaWAN} نیز انجام شده است.

در این پژوهش هیچ استفاده‌از \متن‌لاتین{ADR}، بسته‌های \متن‌لاتین{downlink} و \متن‌لاتین{ACK} نشده است. نرخ ارسال سنسورها ۱۵ دقیقه‌ای بوده و اندازه بسته‌ی آنها مشخص است.
هر نود در واقع شامل پنج سنسور دما، رطوبت، شدت‌نور، تشخیص حرکت و سطح $CO_{2}$ است. این نودها برای اتصال به شبکه از فعال‌سازی \متن‌لاتین{OTAA} استفاده می‌کنند.

یکی از موارد مهمی که در این پژوهش به آن اشاره می‌شود، از دست رفتن بسته‌ها به جز در شبکه‌ی دسترسی و در \متن‌لاتین{Backend} است.
منظور از شبکه \متن‌لاتین{Backend} زیرساخت ارتباط میان \متن‌لاتین{NS} و سرور پلتفرم می‌باشد.
این بازه‌های از دست رفتن بیش از ۵۰ درصد بسته‌ها در \متن‌لاتین{Backend} به صورت دوره‌های ۱.۵ ماه رخ می‌دادند. این پژوهش به بررسی بیشتر این موضوع نپرداخته است و دلیلی ارائه نمی‌دهد.

\قسمت{کنترل دسترسی همزمان}

مساله دسترسی همزمان سال‌ها است که در شبکه‌ها و به خصوص شبکه‌های بی‌سیم مطرح است. راه‌های زیادی برای تخصیص منابع و کنترل دسترسی همزمان وجود دارد
و هر یک از پروتکل‌های \متن‌لاتین{LPWAN} با توجه به ساختار خود، چالش‌های منحصر به فردی در این حوزه دارند.
همانطور که بیان شد، \متن‌لاتین{LoRaWAN} از پروتکل کنترل دسترسی همزمان \متن‌لاتین{ALOHA} استفاده می‌کند. این پروتکل سربار کمی دارد ولی در شبکه‌های شلوغ به خوبی عمل نمی‌کند.
از این رو پژوهش‌های زیادی الگوریتم‌های جدید پیشهاد کرده و آن‌ها را به صورت شبیه‌سازی یا واقعی ارزیابی می‌کنند.

در \متن‌لاتین{NB-IoT} دسترسی همزمان تنها در زمان درخواست منابع وجود دارد
اما پروسه تخصیص منابع خود مصرف توان بالایی دارد و از همین رو پژوهش‌هایی به آن پرداخته‌اند.

البته هدف اصلی این رساله شبکه‌های \متن‌لاتین{LoRa} هستند و بررسی پژوهش‌هایی در حوزه شبکه‌های دیگری چون \متن‌لاتین{NB-IoT} از بابت الگوریتم‌های پیشنهادی صورت پذیرفته است.

\زیرقسمت{مرجع \مرجع{Beltramelli2021}}

در \مرجع{Beltramelli2021} پژوهشگران یک لایه \متن‌لاتین{MAC} جدید برای شبکه‌های \متن‌لاتین{LoRaWAN} پیشنهاد داده‌اند. در این پروژه در ابتدا کارهای پیشین در زمینه بهبود کنترل دسترسی همزمان
مورد بحث قرار گرفته است. این پژوهش خود قصد دارد از شیوه‌ی \متن‌لاتین{Slotted ALOHA} یا اختصارا \متن‌لاتین{S-ALOHA} با همگام‌سازی خارج از باند استفاده کند.
در نهایت در این پروژهش این شیوه در شبیه‌سازی و شرایط واقعی شهری آزمایش می‌شود. در نظر داشته باشید که روش پیشنهادی صرفا برای برنامه‌های نظارتی بوده که پیام‌های \متن‌لاتین{downlink} ندارند.

روش‌های همگام‌سازی خارج از باند این پژوهش عبارتند از:

\شروع{فقرات}
\فقره \متن‌لاتین{GNSS}: \متن‌لاتین{GPS} امروزه یکی از گسترش‌یافته‌ترین \متن‌لاتین{GNSS}ها بوده و می‌تواند با دقت خوبی همگام‌سازی زمان را انجام دهد اما هزینه و توان مصرفی آن برای بیشتر راهکارهای اینترنت اشیا زیاد است.
\فقره \متن‌لاتین{RCCs}:‌ در این سیستم از موج‌های بلند در بازه $40kHz$ تا $80kHz$ استفاده می‌شود. این امواج در مرکز اروپا در دسترس بوده و می‌توان از آن‌ها برای همگامی زمانی تا دقت $0.1$ ثانیه استفاده کرد.
\فقره \متن‌لاتین{FM-RDS}: در این سیستم داده‌های همگام‌سازی از موج \متن‌لاتین{FM} استخراج می‌شود اما لزوما همه‌ی پایگاه‌های \متن‌لاتین{FM} این داده را ارسال نمی‌کنند و لزوما نیز همه پایگاه‌ها خود زمان صحیح ندارند.
\پایان{فقرات}

از بین این روش‌ها، پژوهش حاضر از \متن‌لاتین{FM-RDS} استفاده می‌کند. ارزیابی واقعی به وسیله دو دستگاه صورت گرفته است و برای ارزیابی با تعداد اشیا بالا از شبیه‌سازی \متن‌لاتین{LoRaEnergySim} به زبان \متن‌لاتین{Python} و به صورت تغییریافته استفاده شده است.

\زیرقسمت{مرجع \مرجع{Lee2021}}

پژوهش \مرجع{Lee2021} از جمله مقالاتی است که در بهبود لایه \متن‌لاتین{MAC} برای بهبود شبکه‌های \متن‌لاتین{LoRa} در مقیاس بزرگ کار کرده است.
این پژوهش ساختار جدیدی برای لایه \متن‌لاتین{MAC} ارائه داده و بر این ساختار بحث ارسال گروهی پیام‌های \متن‌لاتین{ACK} را مطرح می‌کند.
عملکرد این لایه پیشنهادی بر پایه ارسال متناوب \متن‌لاتین{Beacon}ها برای همگام‌سازی و مشخص شدن ساختار \متن‌لاتین{Frame}ها می‌باشد.

هر \متن‌لاتین{Frame} خود از \متن‌لاتین{Subframe}هایی تشکیل شده است که هر یک در \متن‌لاتین{slot}های مشخصی به نودها
اجازه می‌دهد تا \متن‌لاتین{uplink} و \متن‌لاتین{downlink} داشته باشند.

در این ساختار، پژوهش قصد دارد به صورت گروهی \متن‌لاتین{Ack} ارسال کند. برای ارسال گروهی این \متن‌لاتین{Ack}ها یک پیام مشخص ارسال می‌شود.
این پیام مشخص می‌تواند با توجه به اندازه‌ای که دارد شامل تعداد مختلفی از \متن‌لاتین{Ack}ها باشد. همانطور که پیشتر اشاره شد این اندازه بسته خود وابسته به
نرخ ارسال داده و \متن‌لاتین{SF}ها است. از آنجایی که \متن‌لاتین{SF}ها به صورت شبه عمود می‌باشند، \متن‌لاتین{Gateway}ها می‌توانند از \متن‌لاتین{SF}ها به صورت همزمان
استفاده کنند اما با توجه به تغییر نرخ ارسال، آن‌ها به تعداد \متن‌لاتین{slot}ها متفاوتی نیاز خواهند داشت و بنابراین مساله تخصیص بهینه \متن‌لاتین{SF}ها وجود خواهد داشت که این پژوهش به
آن نیز پرداخته است.

\زیرقسمت{مرجع \مرجع{Polonelli2019}}

پژوهش \مرجع{Polonelli2019} روش \متن‌لاتین{Slotted ALOHA} برای دسترسی همزمان در شبکه‌های \متن‌لاتین{LoRaWAN} به جای استفاده از \متن‌لاتین{Pure ALOHA} ارائه می‌دهد.
این روش بر روی روش \متن‌لاتین{Pure ALOHA} که روش استاندارد است ارائه شده و از این رو نیازی به تغییر در کتابخانه‌های فعلی نیست.
این پژوهش بیان می‌کند که این روش برای نودهای کلاس $A$ ارائه می‌شود چرا که نودهای کلاس‌های $B$ و $C$ محدودیت مصرفی کمی دارند.

روش \متن‌لاتین{Slotted ALOHA} یا اختصارا \متن‌لاتین{S-ALOHA} از دهه هفتاد در شبکه‌های محلی بی‌سیم شناخته شده است. در این روش زمان به قسمت‌های مشخصی تقسیم می‌شود
و هر دستگاه تنها در ابتدای هر زمان می‌تواند شروع به ارسال کند و در صورت تداخل ارسال را متوقف می‌کند. بهره‌وری کانال در این شیوه به صورت تئوری ۳۷ درصد می‌باشد.

در ادامه این پژوهش برای ارزیابی کارایی روش پیشنهادی شروع به محاسبه پارامترهای لازم می‌کند. اولین پارامتر $T$ مدت زمان لازم برای ارسال موفقیت آمیز یک بسته است. در \متن‌لاتین{LoRaWAN}
برای ارسال یک بسته و سپس دریافت \متن‌لاتین{ACK} آن نیاز به صبر کردن تا رسیدن برای اولین بازه دریافت است، این باز تاخیر \متن‌لاتین{RX1 Delay} نامیده می‌شود و با توجه به زمان بالای ارسال
به خصوص در \متن‌لاتین{SF}های بالا عملا این بازه نباید برای ارسال سایر نودها استفاده شود. با توجه به این موضوع خواهیم داشت:

\[
  T_{LoRaWAN} = T_{ack} + T + RX1\_Delay \approx 2.22T
\]

از همین زمان مشخص می‌شود که در صورت استفاده از ارتباط نیمه دو طرفه (درخواست \متن‌لاتین{ACK} برای پیام‌های ارسالی) در \متن‌لاتین{LoRaWAN} در صورت استفاده از \متن‌لاتین{Pure ALOHA}
یا انحصارا \متن‌لاتین{P-ALOHA} گذردهی به شدت افت می‌کند. پژوهش حاضر این را با استفاده از شبیه‌سازی در \متن‌لاتین{MATLAB} نیز به نمایش گذاشته است.(شکل \رجوع{شکل: گذردهی شبکه LoRaWAN})

\شروع{شکل}
\درج‌تصویر[width=\textwidth]{./img/lorawan-throughput.png}
\تنظیم‌ازوسط
\شرح{شبیه‌سازی گذردهی \متن‌لاتین{LoRaWAN} در زمان استفاده از بسته‌های با تایید تحت فاکتورهای گسترش مختلف \مرجع{Polonelli2019}}
\برچسب{شکل: گذردهی شبکه LoRaWAN}
\پایان{شکل}

در ادامه با در نظر گرفتن همین زمان ارسال و استفاده از \متن‌لاتین{S-ALOHA} گذردهی تا دو برابر افزایش پیدا می‌کند. یکی از مسائل مهم در استفاده از نسخه \متن‌لاتین{S-ALOHA} مساله همگام‌سازی نودها است
در این پژوهش برای این امر از بسته‌های \متن‌لاتین{ACK} استفاده می‌شود و به این ترتیب نودها با در نظر گرفتن زمان رسیدن \متن‌لاتین{ACK} زمان خود را با \متن‌لاتین{Gateway} همگام می‌کنند.

در این پژوهش فرض شده است که از کانال ۶ در پروتکل \متن‌لاتین{LoRa} استفاده می‌شود و برای جلوگیری از پیچیدگی محاسباتی تغییر کانال توسط نودها لحاظ نشده است. بازه‌های زمانی در \متن‌لاتین{S-ALOHA}
پیشنهادی به جز زمان $T_{LoRaWAN}$ محاسبه شده یک بازه اطمینان برای طولانی شدن پروسه همگام‌سازی نیز دارند. در نظر داشته باشید که برای اجرای رویه همگام‌سازی نیاز است که در ابتدا از سوی شی ارسال
صورت بپذیرد.

همانطور که اشاره شد پروتکل پیشنهادی در این پژوهش کاملا با پیاده‌سازی فعلی \متن‌لاتین{LoRaWAN} همگام است و از این رو این پژوهش ارزیابی عملیاتی از پروتکل پیشنهادی داشته است.
در کنار ارزیابی تئوری نتایج ارزیابی عملیاتی حتی بهتر نیز بوده‌اند. در نهایت این پروتکل پیشنهادی تنها با افزودن یک سربار برای ارسال زمان در پیام‌های \متن‌لاتین{ACK} گذردهی به مراتب بهتر از دو برابر
نسبت به نسخه‌ی اصلی بدست می‌آورد. تنها مساله باقی‌مانده در همگام‌سازی شروع توسط نودها است که در صورت عدم ارسال داده برای بازه‌ی طولانی و استفاده از کریستال‌های ارزان قیمت می‌تواند
اختلاف زمان زیادی بین نودها به وجود بیاورد.

\زیرقسمت{مرجع \مرجع{Lee2017}}

در پژوهش \مرجع{Lee2017} هدف ارائه یک روش تخصیص منابع مبتنی بر پیش‌بینی برای شبکه \متن‌لاتین{NB-IoT} است.
این پژوهش بیان می‌کند کارهای پیشین در این حوزه به بحث پیش‌بینی ترافیک اینترنت اشیا و ارتباط میان ترافیک‌های \متن‌لاتین{uplink} و \متن‌لاتین{downlink} نپرداخته‌اند.

حتی بعد از زمانی که ارتباط حامل رادیویی برقرار شده است و \متن‌لاتین{RRC} در وضعیت اتصال است، در \متن‌لاتین{NB-IoT} نمی‌توان بسته‌های \متن‌لاتین{uplink} را بدون
رویه درخواست زمان‌بندی ارسال کرد. مساله این پژوهش مشکل مصرف توان در این پروسه درخواست زمان‌بندی است.

مشابه با شبکه‌های \متن‌لاتین{LTE}، در \متن‌لاتین{NB-IoT} منابع رادیویی برای ارسال بسته‌ها بین دستگاه‌های کاربران (\متن‌لاتین{UE}ها) مشترک است و
زمان‌بند \متن‌لاتین{eNodeB} به صورت پویا این منابع را بر پایه سیاست زمان‌بندی به هر دستگاه کاربر تخصیص می‌دهد.
برخلاف \متن‌لاتین{LTE}، \متن‌لاتین{NB-IoT} منابع رادیویی اختصاصی برای درخواست‌های زمان‌بندی ندارد و بنابراین این درخواست‌ها تنها ممکن است از طریق روش‌های دسترسی تصادفی ارسال شوند.
از آنجایی که این رویه با سایر دستگاه‌ها رقابت می‌کند، ممکن تصادم به وجود بیاید و در این شرایط دستگاه کاربر می‌بایست بعد از یک زمان مشخص دوباره دسترسی تصادفی را تکرار کند.
در پاسخ به این درخواست \متن‌لاتین{eNodeB} پاسخی را ارسال می‌کند که شامل اطلاعات پیشگیری از تصادم و دستور زمان‌بندی است.
از آنجایی که زمان‌بند \متن‌لاتین{eNodeB} هیچ اطلاعی از حجم اطلاعات در بافر دستگاه کاربر ندارد، یک منبع کوچک رادیویی برای دریافت وضعیت بافر دستگاه کاربر در نظر می‌گیرد.
دستگاه کاربر پس از دریافت دستور زمان‌بندی \متن‌لاتین{uplink}، گزارش وضعیت بافر خود را ارسال کرده و \متن‌لاتین{eNodeB} بعد از دریافت آن به دستگاه کاربر اطلاع داده و به صورت پیوسته
منابع رادیویی \متن‌لاتین{uplink} را تخصیص می‌دهد تا همه‌ی اطلاعات بافر به صورت کامل منتقل شوند.
این مراحل در شکل \رجوع{شکل: زمان‌بندی در NB-IoT} نمایش داده شده است.

\شروع{شکل}
\تنظیم‌ازوسط
\درج‌تصویر[height=\textwidth]{img/nbiot-scheduling.png}
\شرح{رویه درخواست زمان‌بندی به صورت دسترسی تصادفی \مرجع{Lee2017}}
\برچسب{شکل: زمان‌بندی در NB-IoT}
\پایان{شکل}

این پژوهش بیان می‌کند در کنار دست‌دادهایی که در \متن‌لاتین{NB-IoT} صورت می‌گیرد، مساله مصرف توان برای دست‌دادهایی که پروتکل‌های لایه انتقال
استفاده می‌شوند مانند \متن‌لاتین{CoAP}، \متن‌لاتین{DTLS} و \نقاط‌خ هم وجود دارد.
در واقع اگر مساله انتقال یک بسته \متن‌لاتین{uplink} باشد سربار پروسه درخواست زمان‌بندی زیاد نخواهد بود اما اگر یک دست‌داد میان شبکه و دستگاه
کاربر لازم باشد، سربار این رویه بیشتر خواهد بود و مصرف توان آن نیز افزایش پیدا می‌کند.. شکل \رجوع{شکل: دست‌داد میان شبکه و دستگاه کاربر در NB-IoT} این مورد را بهتر نمایش می‌دهد.

\شروع{شکل}
\تنظیم‌ازوسط
\درج‌تصویر[height=.8\textwidth]{img/nbiot-handshake.png}
\شرح{دست‌داد میان شبکه و دستگاه کاربر در \متن‌لاتین{NB-IoT} \مرجع{Lee2017}}
\برچسب{شکل: دست‌داد میان شبکه و دستگاه کاربر در NB-IoT}
\پایان{شکل}

روش پیشنهادی این پژوهش که بر پایه پیش‌بینی عمل می‌کند، \متن‌لاتین{Prediction-Based Energy Saving Mechanism} یا مختصرا \متن‌لاتین{PBESM} نام دارد.
این روش تخصیص منابع را مبتنی بر پیش‌بینی صورت می‌دهد و باعث می‌شود مصرف انرژی به خاطر رویه‌های درخواست زمان‌بندی کاهش پیدا کند.
در این روش با استفاده از رخ داد \متن‌لاتین{uplink}ها و تاخیر پردازش که از بررسی بسته‌ها بدست می‌آید، \متن‌لاتین{eNodeB}
منابع رادیویی را برای انتقال بسته \متن‌لاتین{uplink} از پیش تخصیص می‌دهد و به این ترتیب اجازه می‌دهد تا بسته‌های \متن‌لاتین{uplink}
بدون رویه درخواست زمان‌بندی امکان ارسال داشته باشند.

همانطور که بیان شد این الگوریتم بسته‌ها را پردازش می‌کند و از این رو در صورتی که بخواهیم زمان پاسخ برنامه را در این الگوریتم محاسبه کنیم این الگوریتم نیاز دارد تا پروتکل لایه کاربرد
موردنظر پشتیبانی کند. در پیاده‌سازی پژوهشگران از این الگوریتم پروتکل‌های \متن‌لاتین{RLC}، \متن‌لاتین{RRC}، \متن‌لاتین{NAS} و \نقاط‌خ صورت پذیرفته است.
برای پیش‌بینی زمان پردازش مقدار ابتدایی با توجه به شبیه‌سازی‌ها پر شده و الگوریتم از میانگین متحرک و صحت پیش‌بینی صورت گرفته، استفاده می‌کند.
در سمت دستگاه کاربر پیام‌های از پیش تخصیص یافتن منابع ذخیره می‌شود و در صورتی که کاربر داده‌ای برای ارسال داشته باشد تا فرارسیدن این زمان بازه از پیش تخصیص یافته
صبر می‌شود اما اگر پیامی مبنی بر از پیش تخصیص یافتن منابع وجود نداشته باشد رویه درخواست سنتی زمان‌بندی صورت می‌گیرد. اگر منابعی از پیش تخصیص یافته باشد اما دستگاه
در آن داده‌ای برای ارسال نداشته باشد، هیچ پیامی ارسال نمی‌کند و این به منزله پیش‌بینی غلط است.

در نهایت این پژوهش بیان می‌کند با استفاده از \متن‌لاتین{PBESM} فاز دسترسی تصادفی حذف می‌شود و از همین رو مصرف توان کاهش پیدا می‌کند.
مشکل اصلی در روش \متن‌لاتین{PBESM} تاخیری است که داده ممکن است تا رسیدن به شروع بازه از پیش تخصیص یافته تحمل کند.
این تاخیر قابل کنترل است که با افزایش آن خطاهای پیش‌بینی کاهش پیدا می‌کند و مصرف توان بهتر خواهد شد ولی تاخیر روی داده‌ها بیشتر خواهد بود.
این پارامتر مصالحه‌ای میان تاخیر و توان مصرفی است. برای ارزیابی این روش از شبیه‌سازی در سطح سیستم استفاده شده است.
در این شبیه‌سازی از ۵ وضعیت متفاوت استفاده شده است:

\شروع{شمارش}
\فقره \متن‌سیاه{سناریو گزارش‌دهی دستگاه کاربر}، مانند سرویس اندازه‌گیری دوره‌ای، که در آن دستگاه اینترنت اشیا مقدار حسگر را
به شبکه با بسته‌های غیر \متن‌لاتین{IP} ارسال کرده و با دریافت تاییدیه از شبکه نشست کامل می‌شود.
\فقره در این سناریو برنامه کاربردی وضعیت شی و مقدار حسگر آن را بررسی می‌کند. مانند حالت اول، از بسته‌های غیر \متن‌لاتین{IP} استفاده می‌شود اما
نشست از سمت شبکه آغاز شده و با پاسخ دستگاه کاربر و تاییدیه‌اش خاتمه می‌یابد.
\فقره این سناریو مشابه با وضعیت ۲ است با این تفاوت که از بسته‌های \متن‌لاتین{IP} در آن استفاده می‌شود. پروتکل مورد استفاده \متن‌لاتین{UDP/DTLS/CoAP} بوده و فرض می‌شود نشست \متن‌لاتین{DTLS} می‌بایست ایجاد شود.
\فقره این سناریو مشابه با وضعیت ۲ است با این تفاوت که از بسته‌های \متن‌لاتین{IP} در آن استفاده می‌شود. پروتکل مورد استفاده \متن‌لاتین{UDP/DTLS/CoAP} بوده و فرض می‌شود نشست \متن‌لاتین{DTLS} وجود داشته و در صورت نیاز تمدید می‌شود.
\فقره این سناریو مشابه با وضعیت ۲ است با این تفاوت که از بسته‌های \متن‌لاتین{IP} در آن استفاده می‌شود. پروتکل مورد استفاده \متن‌لاتین{TCP} بوده که سرآیند به واسطه یک الگوریتم کاهش اندازه سرآیند قدرتمند کاهش پیدا کرده است.
\پایان{شمارش}

در نهایت برای شبیه‌سازی زمان پاسخ از توزیع \متن‌لاتین{Pareto} استفاده شده است. در انتها این پژوهش کاهش توان مصرفی را به صورت ریاضی و بر پایه احتمال موفیت آمیز بودن پیش‌بینی فرمول‌بندی می‌کند.

\قسمت{شبکه‌های \متن‌لاتین{Mesh}}

یکی از راه‌ها افزایش کارایی شبکه‌های \متن‌لاتین{LoRa} استفاده از لایه‌ی فیزیکی \متن‌لاتین{LoRa} و تشکیل یک شبکه \متن‌لاتین{Mesh} می‌باشد. در این شبکه نودها با کمک یکدیگر داده‌ها را ارسال کرده و به دست \متن‌لاتین{Gateway} می‌رسانند.
البته در چنین شبکه‌هایی چالش‌های جدیدتری مانند چگونگی اضافه شدن یا حذف شدن یک نود، چگونگی ساخته شدن شبکه و \نقاط‌خ مطرح می‌باشد. دسته‌ای از پژوهش‌ها به پیاده‌سازی و ارزیابی چنین شبکه‌هایی پرداخته‌اند.

\زیرقسمت{مرجع \مرجع{Lee2018}}

در \مرجع{Lee2018} پژوهشگران دست به پیاده‌سازی ۱۹ نود شبکه‌ی \متن‌لاتین{Mesh LoRa} در محیط دانشگاه با ارسال داده‌ها در بازه‌های یک دقیقه‌ای زده‌اند.
پژوهشگران ادعا می‌کنند این اولین کاری است که تجربه واقعی در پیاده‌سازی \متن‌لاتین{Mesh LoRa} داشته است و در آن بیان می‌شود با این روش نیاز به افرایش تعداد \متن‌لاتین{Gateway}ها از بین می‌رود.
نویسندگان معتقد هستند که استفاده از \متن‌لاتین{ALOHA} توانایی \متن‌لاتین{LoRa} در هندل کردن تعداد زیادی از اشیا را از بین برده است.
سیستم طراحی شده در این پژوهش به جای \متن‌لاتین{LoRaWAN} بر پایه لایه‌ی فیزیکی \متن‌لاتین{LoRa} می‌باشد.

نودها به صورت خودمختار برای انتخاب پدر خود در زمان پیوستن به شبکه‌ی \متن‌لاتین{Mesh} تصمیم می‌گیرند، آن‌ها برای این امر از پارامترهای پیام‌های داده‌ای یا \متن‌لاتین{Beacon}های نودهای متصل به شکبه، که به آن‌ها می‌رسد، استفاده می‌کنند.
در این ساختار هر نود لیستی از فرزندان خود نگهداری کرده و آن را در اختیار \متن‌لاتین{‌Gateway} هم قرار می‌دهد.
این شبکه برای ارتباطات بین نودها بهینه نشده است و بیشتر هدف آن ارتباط میان \متن‌لاتین{Gateway} و نودها می‌باشد.
بازه ارسال داده‌ها برای نودها در این شبکه مشخص است و بر پایه عدم دریافت پیام در این بازه می‌توانند وضعیت شبکه‌ای خود را ارزیابی و پدر خود را تغییر دهند.
این پژوهش در مورد مقدار بهینه پارامترهای \متن‌لاتین{RSSI}، \متن‌لاتین{PDR} و \نقاط‌خ صحبت می‌کند و همانطور که بیان شد، ادعا می‌کند که برای رسیدن به این مقدارهای بهینه تنها دو راه افزایش تعداد \متن‌لاتین{Gateway}ها یا استفاده از شبکه‌ی \متن‌لاتین{Mesh} وجود دارد.

در نهایت می‌توان گفت موارد زیر در این پژوهش دیده نشده‌اند:
\شروع{فقرات}
\فقره در این پروژه \متن‌لاتین{Gateway} برای دریافت داده‌ها به نودها درخواست می‌دهد و در مورد ارسال خودکار داده‌ها توسط اشیا که می‌تواند منجر به \متن‌لاتین{Collusion} شود بحث نشده است.
\فقره در نظر نگرفتن کلاس‌های کاری و توان مصرفی شبکه‌ی \متن‌لاتین{LoRa}
\پایان{فقرات}

\زیرقسمت{مرجع \مرجع{Marahatta2021}}

پژوهش \مرجع{Marahatta2021} قصد دارد استفاده از شبکه‌های \متن‌لاتین{LoRa} را در مناطق دور افتاده برای کنترل مصرف انرژی ارزیابی کند.
یکی از مسائل پیش رو در این پژوهش برای استفاده از لایه لینک \متن‌لاتین{LoRaWAN} نیاز آن به تعداد \متن‌لاتین{Gateway}های بالا می‌باشد.
از این رو این پژوهش به استفاده از شبکه‌ی \متن‌لاتین{Mesh LoRa} روی می‌آورد که در آن هر نود به عنوان تکرار کننده برای نودهای همسایه خود عمل کرده
و تعداد \متن‌لاتین{Gateway}های مورد نیاز را کاهش می‌دهد.

این پژوهش روش پیشنهادی خود برای استفاده از \متن‌لاتین{Mesh} را در قالب شبیه‌سازی با \متن‌لاتین{ns-2} در محیط‌های شهری و غیرشهری ارزیابی می‌کند.
در صورت لزوم با توجه به پارامترهای ارتباطی، شبکه به زیرشبکه‌هایی شکسته شده است.
این پژوهش با توجه به زمینه خود در \متن‌لاتین{Smart Grid} از پارامتر جمع‌آوری استاندارد داده در این حوزه برای نرخ ارسال داده استفاده کرده است.


\زیرقسمت{مرجع \مرجع{Famaey2018}}

پژوهش \مرجع{Famaey2018} شبکه‌ی یکپارچه‌ای مبتنی بر تکنولوژی‌های ارتباطی متنوع را تعریف می‌کند. این شبکه در واقع توسط اپراتور مجازی اینترنت اشیا
ساخته می‌شود که زیرساخت ارتباطی خود را از اپراتورهای مختلفی می‌گیرد. این زیرساخت به برنامه‌های کاربردی اجازه می‌دهد تا بتوانند با اشیایی در شبکه‌های ارتباطی مختلف ارتباط داشته باشند.
این اپراتور مجازی برای لایه شبکه از \متن‌لاتین{IPv6} و برای لایه کاربرد از پروتکل \متن‌لاتین{CoAP} استفاده می‌کند.

چالش‌های مختلفی می‌توان برای این اپراتورهای مجازی در نظر گرفت مانند مدل داده‌ای، ساختار داده‌ها و \نقاط‌خ که این پژوهش به آن‌ها نپرداخته است.

\زیرقسمت{مرجع \مرجع{Kim2020}}

پژوهش \مرجع{Kim2020} از لایه‌ی فیزیکی \متن‌لاتین{LoRa} استفاده می‌کند و قصد دارد که با استفاده از آن یک شبکه \متن‌لاتین{Mesh} با هزینه‌ی پایین تشکیل دهد.
این پژوهش قصد دارد که نودها از \متن‌لاتین{SF}های مختلف روی یک کانال فرکانسی استفاده کنند ولی بتوانند بدون نیاز به نودهای گران قیمت با یکدیگر ارتباط بگیرند.
برای این امر نودها با دریافت پیشاید سعی بر رمزگشایی آن با مقدارهای متفاوت \متن‌لاتین{SF} می‌کنند و با این روش می‌توانند \متن‌لاتین{SF} مورد نظر ارسال کننده را پیدا کنند.
فرآیند تشخیص به \متن‌لاتین{SF} پایین آغاز شده و تا \متن‌لاتین{SF} بالا ادامه می‌یابد. برای تطبیق پیشایند با هر یک از این \متن‌لاتین{SF}ها نیاز به زمان است
و این پژوهش بیان می‌کند که رویه شروع با \متن‌لاتین{SF} پایین و سپس رسیدن به \متن‌لاتین{SF}های بالاتر در مجموع زمان کمتری نیاز دارد.
از سوی دیگر این پژوهش بیان می‌کند پیشایند ارسالی می‌بایست به قدر کافی طولانی باشد.

یکی از مسائل پیشرو در این پژوهش عدم اطمینان کامل به فاکتور گسترش تشخیص داده شده است.
این پژوهش بیان می‌کند ممکن است به جای فاکتور گسترش اصلی اشتباه یکی از فاکتورهای گسترش همسایه انتخاب شود و
این امر بین \متن‌لاتین{SF}های ۱۰ تا ۱۲ احتمال بالاتری دارد.

برای حل این مشکل، پژوهش الگوریتم جدیدی را پیشنهاد می‌دهد. در این الگوریتم جدید فرآیند تشخیص سه بار تکرار می‌شود.
در این تکرارها اگر فاکتور گسترش نتیجه شده، ۷ یا ۸ باشد فرآیند متوقف می‌شود.
در صورتی که نتیحه حاصل بین ۹ تا ۱۲ باشد فرآیند تا انتهای پیشایند رویه تشخیص را ادامه می‌دهد.
با این روش احتمال تشخیص غلط تقریبا برابر صفر خواهد بود.

ارزیابی این سیستم به شکل عملی صورت پذیرفته است. در این ارزیابی از چیپ‌های \متن‌لاتین{SX1272} و \متن‌لاتین{SX1301} در باند فرکانسی ۸۶۰ تا ۱۰۲۰ مگاهرتز استفاده شده است.
چیپ \متن‌لاتین{SX1272} به صورت تک کانال بوده و از پهنای باندهای ۱۲۵، ۲۵۰ و ۵۰۰ کیلوهرتز پشتیبانی می‌کند.
چیپ \متن‌لاتین{SX1307} یک ارسال و دریافت کننده قوی می‌باشد. این چیپ می‌تواند کانال \متن‌لاتین{IF8} و کانال‌های \متن‌لاتین{IF0} تا \متن‌لاتین{IF7} را پشتیبانی کند.
پهنای باند کانال‌های \متن‌لاتین{IF0} تا \متن‌لاتین{IF7} می‌تواند تنها ۱۲۵ کیلوهرتز باشد و این در حالی است که کانال \متن‌لاتین{IF08} میتواند پهنای‌باندهای ۱۲۵، ۲۵۰ و ۵۰۰ کیلوهرتز را داشته باشد.

چیپ \متن‌لاتین{SX1272} تنها می‌تواند با یک فاکتور گسترش کار کند این در حالی است که چیپ \متن‌لاتین{SX1307} می‌تواند با فاکتورهای گسترش متفاوتی کار کند.
برای فعالیت بدون ایراد یک شبکه نیاز است که فاکتور گسترش از پیش تعیین شود اما با الگوریتم پیشنهادی این پژوهش این نیاز از بین می‌رود و دستگاه‌ها می‌توانند
این مقدار را در طول فعالیت خود تغییر دهند.

با این روش پژوهش گذردهی انتها به انتها را به وسیله‌ی چیپ‌های ارزان \متن‌لاتین{SX1272} به اندازه چیپ‌های گران قیمت \متن‌لاتین{SX1307} افزایش می‌دهد.

\قسمت{نرخ‌داده تطبیق پذیر}

یکی از بحث‌های در شکبه‌های \متن‌لاتین{LoRaWAN} قابلیت لایه‌ی لینک این شبکه برای تغییر پارامترهای ارتباطی در جهت بهبود کیفیت ارتباط می‌باشد. در استاندارد \متن‌لاتین{LoRaWAN} یک روش پیشنهادی ساده مطرح شده است
اما پژوهش‌های زیادی دست به بهبود و ارزیابی آن در شرایط متفاوت زده‌اند. از سوی دیگر پارامتر \متن‌لاتین{SF} در این شبکه‌ها به صورت شبه عمود بوده و اجازه ارتباط همزمان را می‌دهد بنابراین پژوهش‌های زیادی دست به طرح مساله
برای تخصص این پارامتر زده‌اند.

\زیرقسمت{مرجع \مرجع{sensors-20-03061-v2}}

در پژوهش \مرجع{sensors-20-03061-v2} از شبه عمود بودن \متن‌لاتین{SF}ها در شبکه‌های \متن‌لاتین{LoRaWAN} برای انتقال همزمان پیام‌های دورسنجی و اخطار استفاده می‌شود.
این پژوهش دو استراتژی برای تخصیص \متن‌لاتین{SF}های متمایز برای داده‌های اخطاری و دورسنجی پیشنهاد می‌دهد.
آزمایش عملی این پژوهش به دلیل نیاز به تعداد بالای نود و نیاز به تغییر رویه تخصص \متن‌لاتین{SF}ها امکان‌پذیر نیست و بنابراین آزمایش در محیط شبیه‌سازی \متن‌لاتین{ns-3} صورت می‌گیرد.
برای شبیه‌سازی از سه محیط مختلف استفاده شده است، یک محیط بسته و دو محیط باز که به ترتیب یک و چهار \متن‌لاتین{Gateway} دارند.

در نهایت می‌توان نشان داد جدا کردن پیام‌های اخطار از دورسنجی می‌تواند به فراهم آوردن یک کران بالا برای تاخیر نیز کمک کند. پژوهشگران قصد دارند در پژوهش‌های آتی ارزیابی عملی از این الگوریتم‌ها داشته باشند
و مدل ریاضی برای موفقیت ارسال بسته‌ها بدست بیاورند. چهارچوب \متن‌لاتین{Network Calculus} می‌تواند به این پژوهش در بدست آوردن یک کران بالا برای تاخیر کمک کند.


\قسمت{پردازش در لبه}

با توجه به منابع محدود اشیا در شبکه‌های توان پایین با برد بلند مانند \متن‌لاتین{LoRaWAN} امکان استفاده از پردازش لبه کم به نظر می‌رسد اما پژوهش‌هایی به این امر پرداخته‌اند
چرا که امروز با افزایش تعداد اشیا و حجم داده‌ها نمی‌توان تنها به سرورهای ابری برای پردازش اتکا کرده و از سوی دیگر در برنامه‌های حساس به کارایی و پرخطر امکان استفاده از سرورهای
ابری با تاخیر غیرقابل پیش‌بینی وجود ندارد.
\مرجع{Sarker2019}

از سوی دیگر در صورت قطع بودن اینترنت با استفاده از پردازش در لبه شی یا \متن‌لاتین{Gateway} می‌تواند پردازش‌های اولیه‌ای را انجام داده و تصمیم‌گیری کند.
\متن‌لاتین{Gateway} می‌تواند پیش از ارسال اطلاعات به سرورهای ابری آن‌ها فشرده‌سازی کرده یا اطلاعاتی که به درستی دریافت نشده‌اند با توجه کد تصحیح خطای آن‌ها
در صورت امکان بازیابی کند.
\مرجع{Sarker2019}

در برخی از پژوهش‌ها، لبه برای اجرای الگوریتم‌های یادگیری ماشین در راستای کاهش هوشمندانه ابعاد داده‌ها و \نقاط‌خ مورد استفاده قرار گرفته است.
در نهایت لبه می‌تواند با توجه به معماری مورد استفاده در شبکه \متن‌لاتین{LPWAN} تعریف منحصر به فرد خود را داشته باشد.

\زیرقسمت{مرجع \مرجع{Taleb2017}}

پژوهش \مرجع{Taleb2017} پردازش در لبه مستقل را از تکنولوژی ارتباط مورد استفاده بحث می‌کند. در این پژوهش هدف در نظر گرفتن جابجایی اشیا در پردازش لبه است و پژوهش قصد دارد
با جابجایی لبه بهترین کیفیت از تجربه را ارائه کند. برای این امر پژوهش حاضر از بحث مجازی‌سازی سبک یا همان کانتینرها و مهاجرت آن‌ها استفاده می‌کند.

در این معماری سرورهایی در لبه تعبیه شده‌اند که منابع پردازشی و ذخیره‌سازی را فراهم می‌کنند و وظیفه اجرای کانتینرها را برعهده دارند. از سوی دیگر برای جابجایی این کانتینرها یک حافظه مشترک
میان این سرورها تعبیه شده است.

مساله مهم در این پژوهش بحث جابجایی زنده است. در جابجایی زنده نیاز است که وضعیت حافظه کانتینر پیگیری شده و در جابجایی منتقل شود تا کانتینر در مقصد کاملا مشابه با مبدا اجرا شود.
این روش می‌بایست بدون خطا و سریع باشد اما این پژوهش هنوز ایده جابجایی حافظه را نگه داشته و تنها با کپی کردن آن روی حافظه مشترک سعی در بهبود سرعت دارد اما هنوز مشکل خطا
اجرا در برنامه وجود دارد.

یکی دیگر از چالش‌هایی که این پژوهش به آن پرداخته است جابجایی نود بین فراهم آورندگان مختلف است که ممکن است نتوان در این شرایط جابجایی زنده داشت و از این رو ممکن است
در این شرایط دوباره نیاز به همگام‌سازی با سرورهای ابری باشد.

در نهایت این پژوهش برای ارزیابی به صورت عملیاتی و با استفاده از کانتینرهای \متن‌لاتین{OpenVZ} اقدام کرده است. زیرساخت شبکه هم \متن‌لاتین{IEEE 802.11} در نظر گرفته شده است.
