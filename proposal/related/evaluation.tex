\قسمت{ارزیابی کارایی}

در حوزه اینترنت اشیا پروتکل‌ها و معماری‌های مختلفی وجود دارد که می‌توان از آن‌ها استفاده کرد. هر یک از پروتکل‌ها یا معماری‌ها در شرایط خاصی کارآیی خوبی دارند بنابراین پژوهش‌های زیادی برای ارزیابی کارایی آن‌ها صورت پذیرفته است.
در این ارزیابی پروتکل‌ها و معماری‌ها بدون هیچ تغییر یا بهبودی ارزیابی می‌شوند.
این ارزیابی‌ها به صورت کلی در دو دسته واقعی یا شبیه‌سازی هستند. برخی از آن‌ها در یک لایه به خصوص مانند لایه دسترسی یا لایه هسته فعالیت کرده‌اند و برخی یک راه‌حل انتها به انتها اینترنت اشیا را ارزیابی کرده‌اند.

پارامترهای متنوعی مورد ارزیابی قرار می‌گیرند که از جلمه آن‌ها می‌توان توان مصرفی، نرخ داده و جابجایی اشیا را نام برد.
در کنار این پارامترها پژوهش‌هایی در لایه فیزیکی \متن‌لاتین{LoRa} بحث تاثیر تداخل سایر پروتکل‌هایی که از باند \متن‌لاتین{ISM} استفاده می‌کنند، تاثیر پارامترهای منطقه‌ای و محیط عملیاتی، تاثیر پارامتر‌های جوی مانند باران
را بررسی کرده‌اند.
از سوی دیگر پژوهش‌هایی در لایه پیوند داده \متن‌لاتین{LoRaWAN} بحث استفاده از \متن‌لاتین{IPv6}، توزیع ترافیک، ارتباط با شبکه‌های دیگر مانند \متن‌لاتین{WiFi} را بررسی کرده‌اند.
در نهایت برخی از پژوهش‌ها پارامتر مشخصی در حوزه اینترنت اشیا را ارزیابی نکرده بلکه راهکار پیشنهادی در حوزه تخصصی مساله‌شان را مورد ارزیابی قرار داده‌اند، به طور مثال به دقت ارزیابی سرعت
وسایل نقلیه در یک راه کار انتها به انتها شامل سنسورهای متنوع پرداخته‌اند.

\زیرقسمت{تاخیر انتها به انتها}

پژوهش \مرجع{FernandesCarvalho2019} یک روش برای ارزیابی تاخیر در پیاده‌سازی‌های \متن‌لاتین{LoRaWAN} ارائه می‌کند.
در این پژوهش اجزای شبکه به عنوان جعبه سیاه در نظر گرفته می‌شوند و تمرکز روی جریان داده اصلی صورت می‌گیرد.
این پژوهش معتقد است که \متن‌لاتین{uplink} دغدغه‌ی اصلی بیشتر شبکه‌های \متن‌لاتین{LoRaWAN} بوده و از همین رو
روش پیشنهادی این پژوهش بر روی \متن‌لاتین{uplink} متمرکز است.
در این روش ترافیک کاربر انتهایی به شبکه‌ی \متن‌لاتین{LoRaWAN} تزریق شده و از سرور برنامه کاربردی دریافت می‌شود.


در روش پیشنهادی این پژوهش ابتدا نقاط ورودی و خروجی با توجه به جریان‌داده‌ای که پیشنهاد شد مشخص می‌شوند.
در ادامه تنظیمات نقاط زمانی (نقاطی که زمان آن‌ها به بسته اضافه می‌شود) مشخص می‌شوند.
با توجه به ساختار سرور‌ها و گستره جغرافیایی آن‌ها نیاز به روش‌هایی برای همگام‌سازی زمان برای گارانتی نمودن
یکپارچگی اندازه‌گیری‌ها زمانی وجود دارد.
در نهایت آنالیز و اندازه‌گیری‌ها صورت می‌گیرد. پارامترهای مورد ارزیابی با توجه به هدف آزمایش می‌تواند متغیر باشد.

در پژوهش \مرجع{Carvalho2019}، پژوهشگران بیان می‌کنند که در کارهای گذشته به بحث تاخیر تنها در شبکه‌ی بی‌سیم \متن‌لاتین{LoRaWAN} پرداخته شده است و شبکه \متن‌لاتین{IP} پشتی آن
و ارزیابی انتها به انتها مورد توجه نبوده است. این پژوهش قصد دارد، روشی فرمال در جهت ارزیابی تاخیر در لایه کاربرد ارائه کند.
این پژوهش برای همگام‌سازی زمانی دریافت کننده و ارسال کننده داده، از یک سرور \متن‌لاتین{NTP} استفاده می‌کند
نتایج نشان می‌دهد که گره‌هایی که در شهر محل آزمایش قرار داشتند نتایج نزدیک به هم گزارش کرده‌اند، این در حالی است که گره انتهایی
که در شهر دیگری قرار داشته است به خاطر وضعیت نامناسب اینترنت تغییرات بیشتری را تجربه کرده است.
پژوهشگران برای تعداد کمی از بسته‌ها (حدودا ۱۰ بسته در روز) تاخیر نامتعارف حدود ۴ ثانیه را گزارش کرده‌اند که بیان می‌کنند این امر در سرور \متن‌لاتین{MQTT}
رخ نداده و این تاخیرها در سرورهای شبکه و اپلیکشن \متن‌لاتین{LoRaWAN} بوده است.

در پژوهش \مرجع{Carvalho2018} به ارزیابی تاخیر در شبکه‌های \متن‌لاتین{LoRaWAN} پرداخته و بیان می‌کند تاخیر ناشی از زیرساخت، در کارهای این حوزه مورد توجه قرار نگرفته است
که این پژوهش قصد پرداختن به آن را دارد.
برای ارزیابی این پژوهش، معماری مرجع \متن‌لاتین{Semtech} را در نظر می‌گیرد. این پژوهش دروازه‌ها را به واسطه شبکه اترنت به سرور شبکه متصل کرده است.
در نهایت این ارزیابی نشان می‌دهد بالاترین میانگین تاخیر مربوط به سرور شبکه است که ادعا این پژوهش را نسبت به حجم بالای پردازش در این سرور، تایید می‌کند.

پژوهش \مرجع{Potsch2019} به بررسی تاخیر انتها به انتها در شبکه‌های \متن‌لاتین{LoRaWAN} پرداخته است و برخلاف سایر پژوهش‌ها هدف از این پژوهش
ارزیابی عملی تاخیر و تغییرات تاخیر انتها به انتها، در یک انتقال واقعی داده میان حسگر تا دریافت موفقیت‌آمیز آن به صورت از رمزگشایی شده در برنامه‌ی کاربردی است.
برای ارزیابی عملی از پیاده‌سازی \متن‌لاتین{loraserver.io} استفاده شده است.
مقدار فاکتور گسترش از میان ۷، ۹ و ۱۲ و اندازه بسته نیز از میان ۸، ۲۲ و ۵۰ (اندازه بسته از ۵۱ بایت محدودیت فاکتور گسترش ۱۲ کمتر است) بایت انتخاب می‌شود.

دو سناریو شبیه‌سازی صورت گرفته است، در سناریو اول سرور اپلکیشن و شبکه \متن‌لاتین{LoRaWAN} هر دو بر روی همان \متن‌لاتین{Raspberry Pi}
قرار گرفته‌اند، که دروازه قرار گرفته است. در سناریو دوم سرور اپلکیشن، شبکه و \متن‌لاتین{MQTT} همگی روی ابر مستقر شده و ارتباط
دروازه به واسطه شبکه‌ی سلولی صورت پذیرفته است.


این پژوهش بیان می‌کند با مهاجرت از شبکه‌ی داخلی به شبکه‌ی اینترنت در سناریو دوم نه تنها تاخیر بلکه انحراف معیار افزایش چشم‌گیری
پیدا می‌کنند. در ادامه این پژوهش بیان می‌کند با افزایش مقدار فاکتور گسترش بر خلاف انتظار پارامترهایی به جز زمان ارسال افزایش پیدا می‌کنند
در حالی که پیش‌بینی این مقدارها ثابت باقی بمانند و این مساله نیازمند بررسی بیشتر است.

\زیرقسمت{ارزیابی پروتکل‌های وب}

پژوهش \مرجع{Afzal2022} به ارزیابی استفاده از پروتکل‌های دنیای وب در اینترنت اشیا می‌پردازد و اکوسیستم \متن‌لاتین{Swarm} را هدف قرار می‌دهد که در آن اشیا
منابع خود را در قالب سرویس ارائه می‌دهند. این پژوهش از شِماهایی که برای کاهش لود ارتباطی پیشنهاد شده‌اند مانند \متن‌لاتین{SCHC} یا \متن‌لاتین{CBOR}
استفاده کرده و در نهایت نتجیه می‌گیرد استفاده از آن‌ها نسبت به حالت استاندارد، که در دنیای وب مورد استفاده قرار می‌گیرد، ۹۸ درصد کاهش حجم به ارمغان می‌آورد.

پژوهش \مرجع{Weber2016} پیش از ارائه استانداردهایی مانند \متن‌لاتین{Static Context Header Compression} یا مختصرا \متن‌لاتین{SCHC} از \متن‌لاتین{IETF} در حوزه عملیاتی کردن \متن‌لاتین{IPv6} روی \متن‌لاتین{LoRaWAN}
با ارائه \متن‌لاتین{6LoRaWAN} کار کرده است و در آن زمان پژوهشگران این پژوهش در همکاری با \متن‌لاتین{IETF} قصد تهیه یک نسخه استاندارد از کارشان را داشته‌اند. این پژوهش روش پیشنهادی را به صورت عملی پیاده‌سازی و در عمل ارزیابی کرده است.
این پژوهش به دو مساله در ارتباط میان \متن‌لاتین{IPv6} و \متن‌لاتین{LoRaWAN} اشاره می‌کند. مساله اول حداقل واحد انتقالی (\متن‌لاتین{MTU}) در پروتکل \متن‌لاتین{IPv6} است که مقدار آن برابر با ۱۲۸۰ بایت است.
مساله بعدی اندازه متغیر بسته‌های \متن‌لاتین{LoRaWAN} بر پایه مقادیر مختلف برای فاکتور گسترش است. این پژوهش بیان می‌کند روش پیشنهادی \متن‌لاتین{6LoRaWAN}
دقیقا مشابه با \متن‌لاتین{6LoWPAN} یک پروتکل تطبیقی است که امکان ارتباط میان پروتکل \متن‌لاتین{IPv6} در لایه شبکه با دو لایه پیوند داده و فیزیکی \متن‌لاتین{LoRaWAN} و \متن‌لاتین{LoRa} را فراهم می‌آورد.
این پژوهش از فشرده‌سازی سرآیند برای رفع چالش اندازه بسته‌های \متن‌لاتین{IPv6} استفاده می‌کند و بیان می‌کند برای تبدیل پروتکل از \متن‌لاتین{6LoRaWAN} می‌توان از دروازه یا سرور شبکه استفاده کرد.

پژوهش \مرجع{sensors-20-00280-v2}
اگلوریتم \متن‌لاتین{Static Context Header Compression (SCHC)} را برای \متن‌لاتین{IPv6} پیاده‌سازی کرده است.
این پژوهش از این پیاده‌سازی برای انتقال پروتکل \متن‌لاتین{CoAP} بر بستر \متن‌لاتین{UDP} و \متن‌لاتین{IPv6} استفاده کرده است.
هدف این پژوهش ارزیابی این الگوریتم بوده است.
این پژوهش بیان می‌کند منابع مصرفی در جهت استفاده از \متن‌لاتین{IPv6} نسبت به سود حاصل از آن بسیار کم است. در مقابل کارایی انرژی و داده در قطعه‌بندی کم است.
از سوی دیگر این پژوهش بیان می‌کند برای استفاده از قطعه‌بندی پیشنهادی \متن‌لاتین{IETF} نیاز است که ترتیب بسته‌ها حفظ شود.
از مزایای \متن‌لاتین{IPv6} می‌توان به زمانی اشاره کرد که یک گره بین دروازه‌ها جابجا می‌شود در صورت لزوم پروسه پیوستن به \متن‌لاتین{NS} را انجام می‌دهد که صورت استفاده از یک آدرس \متن‌لاتین{IPv6} ثابت اتصال تضمین خواهد شد.

\زیرقسمت{ارزیابی پروتکل \متن‌لاتین{MQTT}}

پژوهش \مرجع{Ferrari2018} قصد دارد پروتکل‌های پیام‌رسانی را از نظر تاخیر مقایسه کند. این پژوهش بیان می‌کند با وجود اهمیت تاخیر در راه‌کارهای صنعتی جهانی اما ارزیابی‌های صورت گرفته بر
پروتکل‌های پیام‌رسانی به خوبی تاخیر را مدنظر قرار نداده‌اند. این پژوهش یک روش ساخت‌یافته برای ارزیابی عملیاتی تاخیر انتها به انتها یک پروتکل پیام‌رسانی در حالت کلی ارائه می‌دهد که بتوان از آن
در شرایط خاص بهره برد.
در این پژوهش دو عدم قطعیت زمانی مطرح می‌شود، یک عدم قطعیت مربوط به زمان ثبت یک رویداد است که نسبت به زمان در آن لحظه دارای عدم قطعیت است و از سوی
دیگر زمان محلی در یک گره همانطور که پیشتر هم به آن پرداخته شد نسبت به زمان استاندارد دارای عدم قطعیت است.
پژوهشگران تمامی کیفیت سرویس‌های پروتکل \متن‌لاتین{MQTT} را در ارزیابی خود لحاظ کرده‌اند. در نهایت کیفیت سرویس پیشنهادی توسط
این پژوهش مقدار ۱ است. از سوی دیگر مستقل از کیفیت سرویس، فاصله تاثیر بیشتری بر تاخیر دارد.

در پژهش \مرجع{BertrandMartinez2020} پژوهشگران قصد استفاده از یک روش ساختارمند برای ارزیابی کارگذار‌های پیام \متن‌لاتین{MQTT} را دارند.
این پژوهش بیان می‌کند که در حوزه ارزیابی کارگذار‌های پیام \متن‌لاتین{MQTT} سه بحث کلی وجود دارد:

\شروع{فقرات}
\فقره ارزیابی کمی
\فقره ارزیابی کیفی
\فقره ساختار ارزیابی
\پایان{فقرات}

در پژوهش‌های پیشین به بحث‌های کمی بسیار پرداخته شده است اما بحث‌های کیفی مانند قابلیت اطمینان نیز اهمیت فراوانی دارند
که کارهای کمتری به آن‌ها پرداخته‌اند. هدف از این پژوهش ارائه یک ساختار برای ارزیابی است که در آینده بتوان بر پایه آن ارزیابی‌هایی
را با اهداف دلخواه ولی بدون از دست دادن پارامترهای مهم و تاثیرگذار صورت داد.

این پژوهش متد ذکر شده را برای ارزیابی ۱۲ بستر \متن‌لاتین{MQTT} متن‌باز استفاده می‌کند. برای این ارزیابی سه سطح کیفیت سرویس در پروتکل \متن‌لاتین{MQTT}
مدنظر است و از سوی دیگر نیازمندی‌های غیرکارکردی مانند کیفیت مستندات، گستره تنظیمات، کارکرد بر سیستم‌عامل‌های مختلف و \نقاط‌خ. در نهایت سه بستر انتخاب
با معیارهای کیفی برای مقایسه کارایی استفاده می‌شوند.

در پژوهش \مرجع{Palmese2021} پژوهشگران قصد مقایسه دو پروتکل انتشار و اشتراک \متن‌لاتین{CoAP} و \متن‌لاتین{MQTT-SN} را به صورت عملیاتی دارند.
پروتکل \متن‌لاتین{CoAP} به تازگی در پیش‌نویسی که توسط \متن‌لاتین{IETF} منتشر شده است از مدل انتشار و اشتراک پشتیبانی می‌کند و \متن‌لاتین{MQTT-SN}
مدل تغییر یافته پروتکل \متن‌لاتین{MQTT} برای شبکه‌های سنسوری است. هر دو این پروتکل‌ها بر پایه \متن‌لاتین{UDP} بوده و مقایسه آن‌ها منصفانه به نظر می‌رسد.
این پژوهش بیان می‌کند پروتکل \متن‌لاتین{CoAP} برای شبکه‌هایی با پویایی بالا انتخاب عاقلانه‌ای به نظر می‌رسد.

در نهایت جمع‌بندی این پژوهش به این شرح است. زمانی ماهیت شبکه تحت ثاثیر پارامترهایی مانند \متن‌لاتین{Duty Cycle} بوده
و اشیا نمی‌توانند یک ارتباط دائم داشته باشند استفاده از \متن‌لاتین{CoAP} گزینه‌ی بهتری است. از سوی دیگر پشتیبانی پروتکل
\متن‌لاتین{CoAP} از قطعه‌بندی آن را برای کاربردهایی با پیام‌های بزرگ مناسب می‌کند.

پژوهش \مرجع{sensors-19-00007} قصد ارزیابی بین پروتکل‌های \متن‌لاتین{MQTT} و \متن‌لاتین{CoAP} که به ترتیب بر بسترهای \متن‌لاتین{UDP} و \متن‌لاتین{TCP} فعالیت می‌کنند، را دارد.
ارزیابی بر شبکه زیرساخت \متن‌لاتین{NB-IoT} صورت می‌گیرد که با فعالیت روی باند دارای لایسنس و نبود \متن‌لاتین{duty-cycle}، امکان اجرای پروتکل \متن‌لاتین{TCP} را نیز فراهم می‌آورد.
این پژوهش در نهایت نشان می‌دهد پروتکل \متن‌لاتین{MQTT} نسبت به پروتکل \متن‌لاتین{CoAP} کارآیی کمتری در معیارهای تاخیر، پوشش و ظرفیت سیستم دارد.

در \مرجع{Mishra2021} پژوهشگران دست به ارزیابی کارایی کارگذار‌های پیام برای پروتکل \متن‌لاتین{MQTT} زده‌اند. در این پژوهش معیارهای نرخ پیام، مصرف \متن‌لاتین{CPU} و تاخیر مورد نظر بوده‌اند.
در این میان تاخیر مدت زمانی است که از ارسال پیام در \متن‌لاتین{Publisher} تا دریافت آن در \متن‌لاتین{Subscriber} طول می‌کشد.
در نهایت این پروژهش به این جمع‌بندی میرسد که کارگذار‌های پیام غیرگسترش‌پذیر مانند \متن‌لاتین{Mosquitto} که از تعداد مشخصی نخ استفاده می‌کنند برای محیط‌های محدود مناسب‌تر هستند.
از سوی دیگر از میان کارگذار‌های پیام گسترش‌پذیر \متن‌لاتین{ActiveMQ} کارآیی بالایی داشته و \متن‌لاتین{EMQ X}، \متن‌لاتین{VerneMQ} و \متن‌لاتین{HiveMQ} کارآیی مناسبی دارند.

پژوهش \مرجع{Govindan2015} بیان می‌کند که پارامترهای شبکه‌ای در راستان تضمین کیفیت \متن‌لاتین{MQTT-SN} نیاز به ارزیابی
انتها به انتها دارند که تا به حال به آن پرداخته نشده است.
هدف این پروژهش فراهم آوردن کیفیت سرویس بر پایه اولویت درخواست‌ها، تاخیر انتها به انتها پایین و نرخ دریافت بالا است.
این پژوهش برای نیازمندی‌های سرویس انتها به انتها و سیستمی مشخص می‌کند که چه تعداد نود را می‌توان پشتیبانی کرد
و از سوی دیگر قابلیت اطمینانی که می‌توان با تعداد مشخصی از نودها بدست آورد را مشخص می‌کند.
با توجه به این دروازه‌ها به محض دریافت اطلاعات آن‌ها را منتقل می‌کنند، صف‌ها روی سرور قرار خواهند داشت.
در این پژوهش فرض شده است برای هر داده یک درخواست ارسال می‌شود و دو حالت کلی در نظر گرفته شده است. در حالت اول
زمانی که درخواست برای داده می‌رسد داده از پیش در سرور وجود دارد (توسط شی ارسال شده است) و در حالت دوم نیاز است
که درخواست داده برای شی ارسال شود. البته این فرض در ساختار پروتکل \متن‌لاتین{MQTT} غیرضروری به نظر می‌رسد چرا که
بعد از درخواست اشتراک سرور به صورت خودکار داده‌ها ارسال می‌کند و نیازی به درخواست نیست.
در ادامه تاخیر بین اتصال و درخواست اشتراک، تاخیر \متن‌لاتین{TCP} بر پایه الگوریتم \متن‌لاتین{Jacobson}،
تاخیر ارتباط بی‌سیم \متن‌لاتین{UDP} و تاخیر صف مدل‌سازی می‌شوند.
برای تاخیر صف در سرور ارسال بسته‌ها توسط اشیا به صورت توزیع پوآسن مدل‌سازی می‌شود و سرور به صورت
یک سیستم \متن‌لاتین{Round-Robin} با قسمت‌های مساوی مدل‌سازی می‌شود.
مدل‌سازی برای سرور در واقعا یک روش \متن‌لاتین{Processor Sharing} است.
از آنجایی که فرض شد داده‌های ارسالی در یک بسته جای می‌گیرند، مدل صف سرور $M/D/1$ خواهد بود.

\زیرقسمت{ارزیابی شبکه دسترسی}

پژوهش \مرجع{Liang2020} با مقایسه فناوری‌های بی‌سیم \متن‌لاتین{WiFi}، \متن‌لاتین{IEEE 802.15.4}، \متن‌لاتین{Bluetooth}
و \متن‌لاتین{LoRa} که برای ساختمان‌های هوشمند مناسب به نظر می‌رسند، \متن‌لاتین{LoRa} را به عنوان بهترین گزینه انتخاب کرده
و بیان می‌کند با توجه به محیط پیچیده و از دست رفت زیاد سیگنال در ساختمان این فناوری ممکن است با مشکلاتی مواجه شود.
پژوهش حاضر مدعی است که بیشتر کارهای گذشته پارامترهایی مانند نرخ دریافت بسته را در شبکه‌های \متن‌لاتین{LoRa} ارزیابی کرده‌اند و این در حالی است
که تاخیر پارامتر مهمی برای کاربران انتهایی است که نیازمند بررسی دقیق است.

پژوهش \مرجع{Almojamed2021} به دنبال بررسی جابجایی در شبکه‌های \متن‌لاتین{LoRaWAN} است و از همین رو با استفاده از شبیه‌سازی با نرم‌افزار \متن‌لاتین{OMNET++}
دو مدل جابجایی شناخته شده را برای تاثیر جابجایی بر کارایی شبکه‌ی \متن‌لاتین{LoRaWAN} مورد تحقیق قرار می‌دهد. این پژوهش بیان می‌کند در نظر گرفتن جابجایی در شبکه‌های
\متن‌لاتین{LoRaWAN} ایده‌ی جدیدی نیست اما پژوهش‌های حاضر مدل‌های جابجایی کمی را در نظر گرفته‌اند و سرعت و تعداد اشیای آن‌ها محدود بوده است.
علاوه بر این، برخی از این پژوهش‌ها بر روی جابجایی از منظر فراگرد بین شبکه‌های مختلف تمرکز کرده‌اند یا جابجایی را تنها برای بخشی از اشیا در نظر گرفته‌اند.

مساله مهم در نتایج این پژوهش عملکرد بهتر شبکه \متن‌لاتین{LoRaWAN} در زمان جابجایی گرهها با سرعت زیاد برای تعداد بالای گره‌ها است.
این پژوهش استکارگذار می‌کند جابجایی گره‌ها باعث می‌شود که به دروازه‌ها نزدیک شوند و بتوانند از فاکتور گسترش
کمتر در جهت ارتباط استفاده کنند و از همین رو تداخل در شبکه کاهش پیدا می‌کند و نرخ دریافت بسته‌ها افزایش پیدا می‌کند.
این افزایش سرعت البته تاثیر زیادی بر مصرف توان نداشته و مصرف توان به صورت کلی با افزایش تعداد اشیا، افزایش پیدا می‌کند.
در نهایت این پژوهش نتیجه‌گیری می‌کند نتایج با وجود استفاده از مکانیزم استاندارد نرخ داده تطبیق‌پذیر خوب بوده است اما
بهتر است از مکانیزمی استفاده شود که برای گرههای متحرک بهینه شده باشد.

پژوهش \مرجع{Augustin2016} با هدف ارزیابی توانایی گیرنده \متن‌لاتین{LoRa} و \متن‌لاتین{LoRaWAN} آزمایش عملی را انجام داده است.
در این آزمایش دروازه در یک فضای بسته قرار گرفته است و فرستنده
به صورت متحرک در فضای شهری حرکت کرده است. آزمایش دوم در فضای باز صورت پذیرفته است و آزمایش نهایی در جهت ارزیابی \متن‌لاتین{LoRaWAN} بوده است.
در نتیجه این ارزیابی مشخص می‌شود که پروتکل \متن‌لاتین{LoRaWAN} نسبت به افزایش نرخ بسته‌ها حساس است و نمی‌تواند برای لود بالا کارایی داشته باشد.
این پژوهش پیشنهاد تغییر لایه دسترسی همزمان از \متن‌لاتین{ALOHA} را مطرح می‌کند و از سوی دیگر بیان می‌کند این پروتکل برای مصارف حساس به تاخیر طراحی نشده است.
یک راه حل پیشنهادی برای افزایش گذردهی ارسال یک بسته
بیش از یکبار است که البته باید خطر افزایش تصادم را نیز در نظر گرفت.

\زیرقسمت{ارزیابی سیستم اینترنت اشیا}

پژوهش \مرجع{Moura2021} استقرار یک پلتفرم اینتنرت اشیا و پیاده‌سازی سرویس‌ها را ارائه می‌دهد.
این پژوهش نیازمندی‌های پلتفرم اینترنت اشیا را چنین برمی‌شمارد: هزینه پیاده‌سازی پایین، نظارت و پشتیانی از کنترل سیستم‌های الکتریکی مختلف شامل نور و تهویه مطبوع
اتوماسیون کنترل سیستم‌های الکتریکی با دخالت انسانی پایین،
ارزیابی مصرف انرژی و
ادغام با زیرساخت فعلی در قالب ارتباطات و انرژی.

در پژوهش \مرجع{FerrndezPastor2018} تلاش برای طراحی معماری در کشاورزی دقیق بوده است. این پژوهش بیان می‌کند که کشاورزان مهارت زیادی در کشاورزی بدست آورده‌اند اما آشنایی کمی با سیستم‌های اینترنت اشیا دارند
بنابراین کاربران اینترنت اشیا می‌بایست در بهبود استفاده و ترکیب آن مشارکت کنند. طراحی با مرکزیت کاربر یا اختصارا \متن‌لاتین{UCD}، متد توسعه‌ای است که گارانتی می‌کند محصول، نرم‌افزار و یا وب‌سایت به سادگی قابل استفاده باشند.
در بحث کشاورزی دقیق، طراحی با محوریت کاربر پروسه طراحی را تعریف می‌کند که کشاورزان بر چگونگی شکل‌گیری طراحی تاثیر می‌گزارند.

این پژوهش با توجه به مواردی که ادامه می‌آید پروتکل \متن‌لاتین{MQTT} را برای لایه اپلیکشن انتخاب می‌کند.
\شروع{فقرات}
\فقره پروتکل \متن‌لاتین{MQTT} یک پروتکل اشتراک و انتشار است که برای دستگاه‌هایی با منابع محدود طراحی شده است. مدلی که توسط شرکت‌های بزرگ به صورت جهانی اجرا شده است
و می‌تواند با سیستم‌های قدیمی نیز کار کند.
\فقره تمام پیام‌های موضوعاتی که از کلماتی تشکیل شده است که با ``/'' از یکدیگر جدا شده‌اند. یک فرمت مرسوم \متن‌لاتین{/place/device-type/device-id/measurement-type/status}
است. مشترکین می‌توانند روی اندازه‌گیری‌هایی که از یک کلاس خاص از اشیا می‌آید، مشترک شوند.
\فقره پهنای باند لازم برای پروتکل \متن‌لاتین{MQTT} بسیار کم بوده و ماهیت آن به گونه‌ای است که از منظر مصرف انرژی بسیار کارا است.
\فقره رابط برنامه نویسی آن بسیار ساده است و در سمت کلاینت حافظه کمی مصرف می‌کند. این باعث می‌شود که برای سیستم‌های نهفته انتخاب مناسبی باشد.
\فقره سطوح مختلف کیفیت سرویس در این پروتکل می‌تواند عملیات‌های قابل اطمینان فراهم آورد.
\پایان{فقرات}

پژوهش \مرجع{Bharadwaj2016} در هندوستان انجام شده است و با توجه به حجم بالای زباله‌های شهری قصد دارد یک شیوه‌ی هوشمند برای مدیریت پسماند ارائه کند که در ابعاد شهری و عملیاتی قابل استفاده باشد.
سیستم پیشنهادی بر پایه شبکه ارتباطی \متن‌لاتین{LoRaWAN} و پروتکل \متن‌لاتین{MQTT} کار می‌کند.
پارامترهای اندازه‌گیری شده در سطل‌های زباله مشابه با سایر کارها در این حوزه متشکل شده از وزن سطل، میزان پر بودن سطل و گازهای سمی داخل است.

پژوهش \مرجع{Tseng2021} قصد دارد مهندسی عمران را با مهندسی برق ترکیب کند تا بتوان یک دانشگاه هوشمند و سبز بسازد. این پژوهش هدف اصلی خود را تاثیر پوشش سقف بتونی با صفحات خورشیدی
بر دمای محیط عنوان می‌کند و برای این ارزیابی از سنسورهای \متن‌لاتین{LoRa} با توجه به مصرف کم و برد بالا استفاده می‌کند.

پژوهش \مرجع{Jurva2020} به مساله پیچیدگی در مدیریت صحن هوشمند دانشگاه می‌پردازد. این پژوهش با معرفی مدل میکرو اپراتورها و ارزیابی آن تلاش می‌کند این مساله را حل کند.
میکرو اپراتور یک سرویس دهنده محلی است که زیرساخت دیجیتال و سرویس‌های دانشگاه را مدیریت می‌کند.
در چهارچوب پیشنهادی مدیریت زیرساخت و شبکه دانشگاه هوشمند توسط میکرو اپراتور صورت می‌پذیرد. این اپراتور می‌بایست با سیاست‌گذاران برای استفاده از پهنای باند در فضای دانشگاه
به توافق برسد، از زیرساخت‌های اپراتورهای فعلی برای سرویس خود بهره بگیرد، برای کسب مجوزهای لازم با مدیران دانشگاه در ارتباط باشد و \نقاط‌خ

در پژوهش \مرجع{Baldo2021} محققان بیان می‌کنند معماری‌های پیشنهادی تا به امروز بر سیستم‌های تک نقش متمرکز بودند و پژوهش حاضر قصد دارد سیستمی متشکل از سرویس‌هایی با توپولوژی‌ها متفاوت را
برای مدیریت هوشمند پسماند ارائه دهد. این معماری کلاس‌های مختلف \متن‌لاتین{LoRaWAN} را که پیشتر کمتر به آن‌ها پرداخته شده بود، در برگیرد.
این معماری با سطل‌های ساده هوشمند شروع شده، شامل دورریزهایی است که با کاربران تعامل می‌کنند و دوربین‌هایی که سطل‌ها را برای جلوگیری از حریق نظارت می‌کنند.
برای هر یک از این اجزا به ترتیب از کلاس‌های $A$ و $B$ استفاده می‌کنند.

پژوهشگران در \مرجع{sensors-20-02078} بیان می‌کنند که کارهای زیادی در کشاورزی دقیق با هدف مصرف توان پایین پیشنهاد شده‌اند اما تعداد کمی از آن‌ها در عمل تست شده‌اند.
این پژوهش به دنبال تست عملیاتی مصرف توان در کشاورزی و کاهش آن است.
در ابتدا از بین فناوری‌های ارتباطی موجود بیان می‌شود زمانی که داده‌ی زیادی برای ارسال موجود نیست مانند سنسورهای کشاورزی، استفاده از \متن‌لاتین{LoRa} گزینه خوبی است.
در ضمن پوشش \متن‌لاتین{LTE} در نواحی غیرشهری و کشاورزی کافی نیست بنابراین استفاده از \متن‌لاتین{NB-IoT} گزینه خوبی نیست.
این پژوهش حسگرهای صنعتی کشاورزی را بررسی کرده و برای ارسال داده‌ها نرخ منطقی ۳۰ دقیقه را محاسبه می‌کند.

هر دو پژوهش \مرجع{SanchezIborra2020} و \مرجع{Santa2020} توسط یک تیم انجام شده و تکمیل شده یکدیگر است. این پروژه‌ها وسایل حمل و نقل نوظهور مانند دوچرخه و اسکوترهای برقی را هدف قرار می‌دهند.
هدف طراحی یک سیستم \متن‌لاتین{OBU} یا \متن‌لاتین{on-board unit} است که بتوان از آن‌ها برای گردآوری داده از این وسایل استفاده کرد.
این پژوهش‌ها بیان می‌کنند از بین راه‌کارهای توان پایین با برد بالا بهتر از برای گردآوری داده‌های غیرحیاتی از \متن‌لاتین{LoRaWAN} یا \متن‌لاتین{NB-IoT} استفاده کرد
و فناوری‌های سلولی مانند \متن‌لاتین{GSM} یا \متن‌لاتین{LTE} برای ارتباط‌های حیاتی حفظ کرد.
این پژوهش‌ها یک دستگاه \متن‌لاتین{OBU} با هر دو فناوری \متن‌لاتین{LoRaWAN} و \متن‌لاتین{NB-IoT} پیاده‌سازی کرده و آن را در عمل ارزیابی می‌کنند.
برای ارزیابی \متن‌لاتین{NB-IoT} از زیرساخت عملیاتی \متن‌لاتین{Vodafone} استفاده شده است.

پژوهش \مرجع{Islam2021} به بحث استفاده از \متن‌لاتین{UAV}ها، ارتباطات \متن‌لاتین{LoRaWAN} و ارتباطات ماهواره‌ای در کشاوری پرداخته است.
این پژوهش بیان می‌کند که محدودیت‌های ارتباطی در این حوزه به قدر کافی مورد توجه قرار نگرفته است و تلاش دارد چالش‌های ارتباطی موجود در کشاورزی هوشمند را مرور کند.
این محدودیت‌های ارتباطی در کنار حسگرها، برای \متن‌لاتین{UAV}ها نیز مطرح هستند.
این پژوهش به عنوان یک راهکار در بحث مشکلات ارتباطی بحث \متن‌لاتین{Mesh LoRa} را مطرح می‌کند و مشکلات زیر را برای آن برمی‌شمارد.
\شروع{فقرات}
\فقره کارآیی و قابلیت بسیار پایین است.
\فقره نیاز به نگهداری از راه دور برای دروازه‌ها وجود دارد.
\فقره نیاز به پردازش لبه در دروازه‌ها وجود دارد.
\پایان{فقرات}

پژوهشگران در \مرجع{Cruz2021} استفاده آزمایشی از \متن‌لاتین{LoRa} و \متن‌لاتین{LoRaWAN} به عنوان زیرساخت مدیریت پسماند را گزارش می‌دهند.
در واقع مشارکت اصلی این پژوهش در ارزیابی واقعی سنسورها، شبکه، کارایی انرژی و فناوری‌های ارتباطی است که در قالب یک مدیریت هوشمند پسماند شهری رخ می‌دهد.
آزمایش‌های یک پژوهش در دو سطح رخ می‌دهند. در سطح اول هدف ارزیابی پوشش شبکه‌ای \متن‌لاتین{LoRa} برای مانیتورینگ مخازن پسماند سطحی و زیرزمینی است.
در سطح دوم هدف بررسی ظرفیت شبکه برای ارسال داده‌های مورد نیاز اپلیکشن‌ها است.
در نهایت این پژوهش بیان می‌کند که می‌توان از \متن‌لاتین{LoRa} برای زیرساخت مدیریت هوشمند پسماند شهری استفاده کرد.

پژوهشگران \مرجع{sensors-20-06721} تجربه بیش از دو سال نگهداری از شبکه‌ی سنسورهای فضای بسته دانشگاه \متن‌لاتین{oulu} کشور فلاند مبتنی بر \متن‌لاتین{LoRaWAN} در این پژوهش مرور می‌کنند.
در این پژوهش هیچ استفاده‌از \متن‌لاتین{ADR}، بسته‌های \متن‌لاتین{downlink} و \متن‌لاتین{ACK} نشده است.
این گره‌ها برای اتصال به شبکه از فعال‌سازی \متن‌لاتین{OTAA} استفاده می‌کنند.
یکی از موارد مهمی که در این پژوهش به آن اشاره می‌شود، از دست رفتن بسته‌ها به جز در شبکه‌ی دسترسی و در \متن‌لاتین{Backend} است.
منظور از شبکه \متن‌لاتین{Backend} زیرساخت ارتباط میان \متن‌لاتین{NS} و سرور پلتفرم است.
این بازه‌های از دست رفتن بیش از ۵۰ درصد بسته‌ها در \متن‌لاتین{Backend} به صورت دوره‌های ۱.۵ ماه رخ می‌دادند. این پژوهش به بررسی بیشتر این موضوع نپرداخته است و دلیلی ارائه نمی‌دهد.
