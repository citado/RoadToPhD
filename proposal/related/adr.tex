\قسمت{نرخ‌داده تطبیق پذیر}

یکی از بحث‌های در شکبه‌های \متن‌لاتین{LoRaWAN} قابلیت لایه‌ی لینک این شبکه برای تغییر پارامترهای ارتباطی در جهت بهبود کیفیت ارتباط است. در استاندارد \متن‌لاتین{LoRaWAN} یک روش پیشنهادی ساده مطرح شده است
اما پژوهش‌های زیادی دست به بهبود و ارزیابی آن در شرایط متفاوت زده‌اند. از سوی دیگر پارامتر \متن‌لاتین{SF} در این شبکه‌ها به صورت شبه عمود بوده و اجازه ارتباط همزمان را می‌دهد بنابراین پژوهش‌های زیادی دست به طرح مساله
برای تخصص این پارامتر زده‌اند.

\زیرقسمت{مرجع \مرجع{sensors-20-03061-v2}}

در پژوهش \مرجع{sensors-20-03061-v2} از شبه عمود بودن \متن‌لاتین{SF}ها در شبکه‌های \متن‌لاتین{LoRaWAN} برای انتقال همزمان پیام‌های دورسنجی و اخطار استفاده می‌شود.
این پژوهش دو استراتژی برای تخصیص \متن‌لاتین{SF}های متمایز برای داده‌های اخطاری و دورسنجی پیشنهاد می‌دهد.
آزمایش عملی این پژوهش به دلیل نیاز به تعداد بالای نود و نیاز به تغییر رویه تخصص \متن‌لاتین{SF}ها امکان‌پذیر نیست و بنابراین آزمایش در محیط شبیه‌سازی \متن‌لاتین{ns-3} صورت می‌گیرد.
برای شبیه‌سازی از سه محیط مختلف استفاده شده است، یک محیط بسته و دو محیط باز که به ترتیب یک و چهار \دروازه دارند.

در نهایت می‌توان نشان داد جدا کردن پیام‌های اخطار از دورسنجی می‌تواند به فراهم آوردن یک کران بالا برای تاخیر نیز کمک کند. پژوهشگران قصد دارند در پژوهش‌های آتی ارزیابی عملی از این الگوریتم‌ها داشته باشند
و مدل ریاضی برای موفقیت ارسال بسته‌ها بدست بیاورند. چهارچوب \متن‌لاتین{Network Calculus} می‌تواند به این پژوهش در بدست آوردن یک کران بالا برای تاخیر کمک کند.
