\قسمت{شبکه‌های \متن‌لاتین{Mesh}}

یکی از راه‌ها افزایش کارایی شبکه‌های \متن‌لاتین{LoRa} استفاده از لایه‌ی فیزیکی \متن‌لاتین{LoRa} و تشکیل یک شبکه \متن‌لاتین{Mesh} است. در این شبکه گرهها با کمک یکدیگر داده‌ها را ارسال کرده و به دست دروازه می‌رسانند.
پروتکل دسترسی همزمان در \متن‌لاتین{LoRaWAN} برای دسترسی تک گام در همبندی ستاره‌ای طراحی شده است و نمی‌تواند بسته را بین دستگاه‌ها جابجا کند.
البته در چنین شبکه‌هایی چالش‌های جدیدتری مانند چگونگی اضافه شدن یا حذف شدن یک گره و چگونگی ساخته شدن شبکه مطرح است.
دسته‌ای از پژوهش‌ها به پیاده‌سازی و ارزیابی چنین شبکه‌هایی پرداخته‌اند.

پژوهشگران در \مرجع{Lee2018} ادعا می‌کنند این اولین کاری است که تجربه واقعی در پیاده‌سازی \متن‌لاتین{Mesh LoRa} داشته است
و در آن بیان می‌شود با این روش نیاز به افرایش تعداد دروازه‌ها از بین می‌رود.
نویسندگان معتقد هستند که استفاده از \متن‌لاتین{ALOHA} توانایی \متن‌لاتین{LoRa} در پشتیبانی کردن تعداد زیادی از اشیا را از بین برده است.
این پژوهش در مورد مقدار بهینه پارامترهای \متن‌لاتین{RSSI}، \متن‌لاتین{PDR} و \نقاط‌خ صحبت می‌کند و همانطور که بیان شد، ادعا می‌کند که برای رسیدن به این مقدارهای بهینه تنها دو راه افزایش تعداد دروازه‌ها یا استفاده از شبکه‌ی \متن‌لاتین{Mesh} وجود دارد.


پژوهش \مرجع{Kim2020} از لایه‌ی فیزیکی \متن‌لاتین{LoRa} استفاده می‌کند و قصد دارد که با استفاده از آن یک شبکه \متن‌لاتین{Mesh} با هزینه‌ی پایین تشکیل دهد.
این پژوهش قصد دارد که گرهها از \متن‌لاتین{SF}های مختلف روی یک کانال فرکانسی استفاده کنند ولی بتوانند بدون نیاز به گرههای گران قیمت با یکدیگر ارتباط بگیرند.
برای این امر گرهها با دریافت پیش‌آیند سعی بر رمزگشایی آن با مقدارهای متفاوت \متن‌لاتین{SF} می‌کنند و با این روش می‌توانند \متن‌لاتین{SF} مورد نظر ارسال کننده را پیدا کنند.
