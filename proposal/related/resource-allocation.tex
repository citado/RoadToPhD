\قسمت{برنامه‌ریزی شبکه}

در شبکه‌های \متن‌لاتین{LoRaWAN} نیاز به استقرار دروازه‌ها و برنامه‌ریزی شبکه وجود دارد بنابراین بهینه‌سازی در محیط با توجه به هزینه‌های نگهداری و استقرار و نیازمندی‌های کیفیت سرویس
در این شبکه‌ها مطرح می‌شود. استقرار دروازه‌ها باعث هزینه‌ی خرید و نگهداری آن شده و می‌تواند هزینه نگهداری و راه‌اندازی سیستم را افزایش دهد،
بنابراین برنامه‌ریزی شبکه \متن‌لاتین{LoRaWAN} شامل جایگذاری دروازه‌ها، مساله حیاتی است که بر هزینه راه‌اندازی، نگهداری و کیفیت سرویس تاثیر می‌گذارد
\مرجع{Matni2020}.
از سوی دیگر تنظیمات بهینه اشیا در شبکه‌های \متن‌لاتین{LoRa} می‌تواند باعث بهبود سیستم از نظر گذردهی و مصرف بهینه شود
\مرجع{Ousat2019}.

تفاوت اصلی در مسائل جایگذاری دروازه در شبکه‌های \متن‌لاتین{LoRa} با آنچه پبیشتر در حوزه اینترنت اشیا صورت پذیرفته است،
از عدم شکل‌گیری وابستگی میان اشیا و دروازه منشا می‌گیرد. همانطور که پیشتر اشاره شد در شبکه‌های \متن‌لاتین{LoRa}
بسته توسط تمامی دروازه‌هایی که امکان دریافت آن بسته را داشته باشند، دریافت شده و به سرور شبکه ارسال می‌شوند.
بنابراین در این شبکه‌ها اشیا می‌توانند به صورت همزمان برای چند دروازه ارسال داشته باشند
\مرجع{Ousat2019}.

به صورت کلی هدف کارهای در این حوزه افزایش نرخ بسته‌هایی است که با موفقیت دریافت می‌شوند.
کارهای این حوزه را می‌توان به کارهای مبتنی بر مدل‌سازی ریاضی، مبتنی بر تداخل، مبتنی بر اتصال
و مبتنی بر سیستم تقسیم‌بندی کرد.
در کارهای مبتنی بر تداخل،
برای آزمایش‌هایی که بر روی تداخل و \متن‌لاتین{Capture Effect}
در یک کانال \متن‌لاتین{LoRa} صورت می‌پذیرند، یک مدل خطا در نظر گرفته می‌شود.
این مدل می‌تواند در ازمایش‌هایی که تداخل بین ارتباط‌هایی با فاکتورهای گسترش مختلف را نظارت می‌کنند،
اعمال شود.
در کارهای مبتنی بر اتصال و مبتنی بر سیستم ظرفیت \متن‌لاتین{LoRaWAN} به واسطه حداکثر تعداد
دستگاه‌های انتهایی که می‌توانند در سناریوهای تک و چند سلولی ارتباط برقرار کنند، آنالیز می‌گردد
\مرجع{Farhad2020}.

پژوهش \مرجع{Farhad2020} دو شِمای تخصیص فاکتور گسترش را برای بهبود نرخ بسته‌های موفق پیشنهاد می‌دهد
که تاثیر تداخل را کاهش می‌دهند.
این پژوهش بیان می‌کند در شِمای پیشنهادی \متن‌لاتین{LoRaWAN} که فاکتور گسترش با نرسیدن \متن‌لاتین{Ack}
افزایش پیدا می‌کند مشکل داشته و باعث می‌شود بیشتر دستگاه‌ها روی فاکتورهای گسترش بزرگ قرار بگیرند.
در شِمای پیشنهادی این پژوهش با از دست رفتن بسته‌ها فاکتور گسترش افزایش پیدا می‌کند اما با رسیدن صحیح
بسته‌ها و برآورده شدن کیفیتی مشخصی از شبکه فاکتور گسترش کاهش پیدا می‌کند.
این پژوهش برای تخصیص اولیه فاکتورهای گسترش از فاصله و کفیفت کانال استفاده می‌کند و فرض می‌کند
که مشخصات کانال با زمان تغییر نمی‌کند.

پژوهش \مرجع{Matni2020} الگوریتم \متن‌لاتین{DPLACE} را برای جایگذاری دروازه‌ها در کاربردهای پویای اینترنت اشیا پیشنهاد می‌دهد.
این الگوریتم دستگاه‌های اینترنت اشیا را به خوشه‌هایی با میزانی از شباهت تقسیم می‌کند تا اجازه بدهد که الگوریتم جایگذاری دروازه‌ها
به بهترین شیوه ممکن به این دستگاه‌ها سرویس ارائه کند.

پژوهش \مرجع{Ousat2019} به مساله‌های تخصیص فاکتور گسترش، تخصیص توان و جایگذاری دروازه‌ها به صورت توامان پرداخته است.
مساله طرح شده به صورت یک مساله بهینه‌سازی غیرخطی صحیح و غیرصحیح فرمول‌بندی شده است که تنها برای شبکه‌های کوچک قابل حل است.
پژوهشگران برای برنامه‌ریزی شبکه‌های بزرگ \متن‌لاتین{LoRa} یک الگوریتم تقریبی پیشنهاد می‌دهند.

پژوهش \مرجع{Karthikeya2016} بیان می‌کند در دنیای اینترنت اشیا شبکه‌ها به واسطه‌ی \متن‌لاتین{Coordinating Device} یا اختصارا \متن‌لاتین{CD} هماهنگ می‌شوند.
در شبکه‌های مختلف نام‌های متفاوتی به \متن‌لاتین{CD}ها داده می‌شود، مثلا در شبکه‌های \متن‌لاتین{WiFi} به آن \متن‌لاتین{Access Point} گفته می‌شود و یا در شبکه‌های حسگری
به آن \متن‌لاتین{Cluster Head} می‌گویند.
\متن‌لاتین{CD}ها نیاز دارند داده‌های خود را به اینترنت ارسال کنند. این ارسال اطلاعات به واسط دستگاه‌هایی به نام \متن‌لاتین{IoT Gateway} یا اختصارا \متن‌لاتین{IGW}ها
صورت می‌پذیرد.
از آنجایی که \متن‌لاتین{CD}ها به فناوری‌های مختلفی تعلق دارند، \متن‌لاتین{IGW}ها برای اهداف تبدیل پروتکل و داده گردانی اهمیت بیشتری پیدا می‌کنند.

یک \متن‌لاتین{IGW} دارای رابط‌های اختصاصی برای فناوری‌های پشتیبانی شده توسط شبکه‌های اینترنت اشیا است.
هر \متن‌لاتین{CD} باید به حداقل یک \متن‌لاتین{IGW} برای ارتباط متصل باشد، این محدودیت سختگیرانه باعث می‌شود، جایگذاری راهبردی \متن‌لاتین{IGW}ها یک قمست کلیدی
در طراحی شبکه باشد.
این پژوهش بیان می‌کند که برای اولین بار به مساله جایگذاری دروازه‌ها در شبکه‌های اینترنت اشیا پرداخته است.
این پژوهش مساله جایگذاری دروازه‌ها را در قالب یک مساله بهینه‌سازی با هدف کمینه‌سازی تعداد کل دروازه‌هایی که در یک شهر هوشمند
جایگذاری می‌شوند، فرمول‌بندی می‌کند.
در ادامه این پژوهش الگوریتم \متن‌لاتین{Network Intersection based Candidate Gateway Location}
یا به اختصار \متن‌لاتین{NewIoTGateway-Select} را برای مشخص کردن کمینه تعداد دروازه‌های لازم جهت جایگذاری ارائه می‌دهد.
الگوریتم \متن‌لاتین{NewIoTGateway-Select} به وسیله‌ی گام‌های زیر عمل می‌کند:
\شروع{فقرات}
\فقره محاسبه مکان‌های ممکن برای جایگذاری دروازه‌ها
\فقره انتخاب مکان‌های بهینه برای دروازه‌ها از میان مکان‌ها نامزد شده
\فقره پشتیبانی از خرابی دروازه و لینک در شبکه با به کار بستن الگوریتم‌های \متن‌لاتین{k-coverage} و \متن‌لاتین{k-connectivity}
\پایان{فقرات}

