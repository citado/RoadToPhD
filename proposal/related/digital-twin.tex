\قسمت{\متن‌لاتین{Digital Twin}}

سرویس‌های اینترنت اشیا نیازمند برقراری یک تطابق واقعی در محیط‌های پویا و بهینه‌سازی منابع محدود محاسباتی و ارتباطی هستند.
بنابراین برای فراهم آوردن سرویس‌های اینترنت اشیا نیاز به پیشنهاد سیستم‌های بی‌سیم خود نگهدارنده داریم که می‌توانند با کمترین
تعامل ممکن از کاربران انتهایی عمل کنند. چنین سیستم‌های خود نگهدارنده‌ای می‌توانند به صورت توامان کارکردهای شبکه‌ای سازگار و
بهینه‌سازی منابع را به وسیله تکنیک‌هایی در حوزه یادگیری ماشین، تئوری بهینه‌سازی، تئوری بازی‌ها و تئوری انطباق انجام دهند.
برای فعال‌سازی کارای این موارد، نیازمند استفاده از مفهوم \متن‌سیاه{\متن‌لاتین{Digital Twin}} یا \متن‌سیاه{همسان دیجیتال} هستیم
\مرجع{Khan2022}.

به صورت کلی می‌توان همسان‌های دیجیتال را به دسته‌های زیر تقسیم کرد \مرجع{Khan2022}:
\شروع{فقرات}
\فقره \متن‌سیاه{همسان دیجیتال نظارتی} که از نظارت بر وضعیت سیستم فیزیکی پشتیبانی می‌کند.
\فقره \متن‌سیاه{همسان دیجیتال شبیه‌سازی} با استفاده از ابزارهای شبیه‌سازی گوناگون و یادگیری ماشین دیدی در رابطه با وضعیت‌های آینده فراهم می‌آورد.
\فقره \متن‌سیاه{همسان دیجیتال عملیاتی} اجازه می‌دهد تکنیسین‌ها با سیستم سایبر فیزیکی تعامل کرده عملیات‌های مختلفی اضافه بر تحلیل و طراحی سیستم را صورت دهند.
\پایان{فقرات}

به صورت کلی همسان دیجیتال را می‌توان به عنوان یک نمونه عینی و مجازی از سیستم‌ها یا اشیا دنیای فیزیکی تعریف کرد.
یک همسان دیجیتال به صورت پیوسته از متناظر فیزیکی خود داده دریافت کرده تا بتواند یک مدل مجازی به روز ارائه کند و مدل مجازی
می‌تواند بازخوردهایی را به دنیای فیزیکی از طریق همان کانال ارتباطی ارائه بدهد
\مرجع{Angin2020}.

\زیرقسمت{مرجع \مرجع{Angin2020}}

پژوهش \مرجع{Angin2020} به بررسی همسان دیجیتالی برای یک مزرعه می‌پردازد، که به وسیله‌ی آن می‌توان به صورت همزمان وضعیت مزرعه را از طریق
همسان دیجیتال آن نظاره کرده و عمل‌های مناسبی که از پردازش هوشمند داده‌های جمع‌آوری شده پیشنهاد می‌شوند، انجام داد.
چهارچوب پیشنهادی بر پایه دو قسمت ساخته شده است. برای قسمت جمع‌آوری داده و ارتباطات، یک شبکه‌ی حسگر بی‌سیم بهینه \متن‌لاتین{LoRa} تشکیل شده است.
گرههای انتهایی شبکه‌ی حسگر بی‌سیم داده‌های محیطی از مزرعه را جمع‌آوری می‌کنند و این داده‌ها را به واسطه \دروازه برای سرورهای ابری ارسال می‌کنند.
قسمت کشاورزی هوشمند با استفاده از الگوریتم‌های یادگیری ماشین، برای کارهایی مانند تشخیص بیماری و یا تشخیص کمبود مواد مغذی بر روی برگ‌های گیاه،
به همراه ارزیابی ارتباط میان نتایج بدست آمده و داده‌های جمع‌آوری شده توسط شبکه‌ی حسگر بی‌سیم، در راستای فراهم آوردن یک نگاه کلی از مزرعه عمل می‌کند.

این پژوهش یک چهارچوب قابل گسترش همتای دیجیتال برای مزرعه دقیق مبتنی بر اینترنت اشیا را پیشنهاد می‌دهد که قابلیت نظارت همزمان بر شرایط مزرعه،
پردازش هوشمند و ابری عکس‌ها و داده‌های جمع‌آوری شده از مزرعه و پیشنهاد مراقب‌های زراعی برای حل مشکلات شناسایی شده با هزینه بهینه را به ارمغان
می‌آورد.

همسان دیجیتال ارائه‌‌ای به روز و دقیق از وضعیت مزرعه به صورت بصری با تصویرهای گرفته شده توسط پهباد و در قالب پارامترهای خاک شامل سطح‌های \متن‌لاتین{pH}،
نیتروژن، نمک، فسفر و پتاسیم، دما و ربوطب که به صورت دوره‌ای از حسگرها جمع‌آوری میشوند، ارائه می‌کند.
برتری اصلی یک چهارچوب همسان دیجیتال، فراهم آوردن دسترسی از دور به نمایش آخرین وضعیت مزرعه بر بستر وب است که به حد زیادی کارهای مزرعه‌دار را
به واسطه از بین بردن فرآیندهای نظارت دستی، به خصوص در مزارع بزرگ ساده‌تر می‌کند.
ساختار چهارچوب همسان دیجیتال پیشنهادی در شکل \رجوع{شکل: چهارچوب همسان دیجیتال مزرعه} آورده شده است.

\شروع{شکل}
\درج‌تصویر[width=\textwidth]{./img/farmfield-digital-twin-framework.png}
\تنظیم‌ازوسط
\شرح{چهارجوب همسان دیجیتال مزرعه \مرجع{Angin2020}}
\برچسب{شکل: چهارچوب همسان دیجیتال مزرعه}
\پایان{شکل}

همانطور که بیان شد این پژوهش نیاز دارد که \دروازهها به زیرساخت اینترنت متصل باشند تا بتواند اندازه‌گیری‌های صورت پذیرفته توسط حسگرهای \متن‌لاتین{LoRa} را
به سرورهای ابری ارسال کنند. با توجه به اینکه عموما مزارع در محل‌های غیرشهری قرار داشته و زیرساخت اینترنت ندارند، این مساله حیاتی خواهد بود.
برای حل این مساله، پژوهش حاضر از نسخه تغییریافته پروتکل مسیریابی \متن‌لاتین{AODV} استفاده کرده و به وسیله آن \دروازههایی که به اینترنت متصل نیستند بسته‌های خود
را به صورت گام به گام برای سایر \دروازهها ارسال می‌کنند تا به \دروازهای با ارتباط به اینترنت برسند.
این مسیریابی و ارتباط میان \دروازهها در بستر \متن‌لاتین{TCP/IP} است.

این پژوهش بیان می‌کند با وجود اینکه معماری پیشنهاد شده می‌تواند برای کارهای مختلفی در حوزه‌ی کشاورزی هوشمند استفاده شود، پژوهش حاضر تنها به تشخیص بیماری و علف‌های هرز
با استفاده از تصاویر گیاه می‌پردازد.
معماری پیشنهادی بر پایه استفاده از الگوریتم‌های یادگیری عمیق در سرورهای ابری برای تشخیص بیماری گیاهان مبتنی بر عکس‌های پهبادی است.
تشخیص دقیق بیماری به واسطه ساختن یک پایگاه داده‌ای کشاورزی هوشمند تکامی از عکس‌های گیاهان سالم و بیمار ممکن است.
این پایگاه داده‌ای مجموعه‌ی داده‌ آموزشی برای الگوریتم‌های یادگیری عمیق را تشکیل می‌دهد که برای تشخصی همزمان بیماری گیاهان از آن استفاده می‌شود.
در ادامه این پژوهش بیان می‌کند که به واسطه شبکه‌ی عصبی پیچشی یادگیری عمیق صورت می‌گیرد.

در نهایت این پژوهش به ارزیابی چهارچوب پیشنهادی پرداخته و برای این امر دو قسمت این چهارچوب به صورت جداگانه ارزیابی می‌شوند.
ارزیابی قسمت ارتباطی با استفاده از \متن‌لاتین{OmNet++} و \متن‌لاتین{FLoRa} صورت پذیرفته است و پارامترهای محدوده پوشش و توان مصرفی ارزیابی شده‌اند.
قسمت دوم مربوط به ارزیابی الگوریتم یادگیری عمیق بوده است که دقت این الگوریتم با استفاده از تقسیم یک مجموعه داده به مجموعه داده آموزش، تست و ارزیابی
صورت پذیرفته است.
