\قسمت{کنترل دسترسی همزمان}

مساله دسترسی همزمان سال‌ها است که در شبکه‌ها و به خصوص شبکه‌های بی‌سیم مطرح است. راه‌های زیادی برای تخصیص منابع و کنترل دسترسی همزمان وجود دارد
و هر یک از پروتکل‌های \متن‌لاتین{LPWAN} با توجه به ساختار خود، چالش‌های منحصر به فردی در این حوزه دارند.
همانطور که بیان شد، \متن‌لاتین{LoRaWAN} از پروتکل کنترل دسترسی همزمان \متن‌لاتین{ALOHA} استفاده می‌کند. این پروتکل سربار کمی دارد ولی در شبکه‌های شلوغ به خوبی عمل نمی‌کند.
از این رو پژوهش‌های زیادی الگوریتم‌های جدید پیشهاد کرده و آن‌ها را به صورت شبیه‌سازی یا واقعی ارزیابی می‌کنند.

در \متن‌لاتین{NB-IoT} دسترسی همزمان تنها در زمان درخواست منابع وجود دارد
اما پروسه تخصیص منابع خود مصرف توان بالایی دارد و از همین رو پژوهش‌هایی به آن پرداخته‌اند.

البته هدف اصلی این رساله شبکه‌های \متن‌لاتین{LoRa} هستند و بررسی پژوهش‌هایی در حوزه شبکه‌های دیگری چون \متن‌لاتین{NB-IoT} از بابت الگوریتم‌های پیشنهادی صورت پذیرفته است.

پژوهش \مرجع{Chen2019} قصد دارد تا به واسطه پروتکل \متن‌لاتین{LoRa} تصاویر کشاورزی را به صورت مطمئن منتقل کند
و روشی جایگزین روش \متن‌لاتین{Stop and Wait} در شبکه‌ی \متن‌لاتین{LoRa} برای انتقال مطمئن با نام \متن‌لاتین{Multi-Packet LoRa} یا اختصارا \متن‌لاتین{MPLR} پیشنهاد می‌دهد.
این روش بسته‌ها را به صورت دسته‌ای ارسال کرده و برای آن‌ها به صورت دسته‌ای \متن‌لاتین{Ack} ارسال می‌کند تا زمان انتظار برای \متن‌لاتین{Ack}ها و ترافیک مربوط به آن‌ها را کاهش دهد.
برای هر دسته‌ی ارسالی از بسته‌ها یک \متن‌لاتین{Ack} شامل وضعیت دریافت صحیح یا غیرصحیح بسته‌ها دریافت می‌شود. با توجه به اینکه پروتکل \متن‌لاتین{LoRa} به صورت \متن‌لاتین{Half-Duplex}
عمل می‌کند امکان ارسال همزمان بسته‌ها و \متن‌لاتین{Ack} وجود ندارد.

در \مرجع{Beltramelli2021} پژوهشگران یک لایه \متن‌لاتین{MAC} جدید برای شبکه‌های \متن‌لاتین{LoRaWAN} پیشنهاد داده‌اند.
این پژوهش خود قصد دارد از شیوه‌ی \متن‌لاتین{Slotted ALOHA} یا اختصارا \متن‌لاتین{S-ALOHA} با همگام‌سازی خارج از باند
و با استفاده از موج \متن‌لاتین{FM} استفاده کند.

پژوهش \مرجع{Lee2021} از جمله مقالاتی است که در بهبود لایه \متن‌لاتین{MAC} برای بهبود شبکه‌های \متن‌لاتین{LoRa} در مقیاس بزرگ کار کرده است.
این پژوهش ساختار جدیدی برای لایه \متن‌لاتین{MAC} ارائه داده و بر این ساختار بحث ارسال گروهی پیام‌های \متن‌لاتین{ACK} را مطرح می‌کند.
عملکرد این لایه پیشنهادی بر پایه ارسال متناوب \متن‌لاتین{Beacon}ها برای همگام‌سازی و مشخص شدن ساختار \متن‌لاتین{Frame}ها است.

پژوهش \مرجع{Polonelli2019} روش \متن‌لاتین{Slotted ALOHA} برای دسترسی همزمان در شبکه‌های \متن‌لاتین{LoRaWAN} به جای استفاده از \متن‌لاتین{Pure ALOHA} ارائه می‌دهد.
این روش بر روی روش \متن‌لاتین{Pure ALOHA} که روش استاندارد است ارائه شده و از این رو نیازی به تغییر در کتابخانه‌های فعلی نیست.

همانطور که اشاره شد پروتکل پیشنهادی در این پژوهش کاملا با پیاده‌سازی فعلی \متن‌لاتین{LoRaWAN} همگام است و از این رو این پژوهش ارزیابی عملیاتی از پروتکل پیشنهادی داشته است.
در کنار ارزیابی تئوری نتایج ارزیابی عملیاتی حتی بهتر نیز بوده‌اند. در نهایت این پروتکل پیشنهادی تنها با افزودن یک سربار برای ارسال زمان در پیام‌های \متن‌لاتین{ACK} گذردهی به مراتب بهتر از دو برابر
نسبت به نسخه‌ی اصلی بدست می‌آورد.
تنها مساله باقی‌مانده در همگام‌سازی شروع توسط گره‌ها است که در صورت عدم ارسال داده برای بازه‌ی طولانی و استفاده از کریستال‌های ارزان قیمت
می‌تواند اختلاف زمان زیادی بین گره‌ها به وجود بیاورد.

