\قسمت{کنترل دسترسی همزمان}

مساله دسترسی همزمان سال‌ها است که در شبکه‌ها و به خصوص شبکه‌های بی‌سیم مطرح است. راه‌های زیادی برای تخصیص منابع و کنترل دسترسی همزمان وجود دارد
و هر یک از پروتکل‌های \متن‌لاتین{LPWAN} با توجه به ساختار خود، چالش‌های منحصر به فردی در این حوزه دارند.
همانطور که بیان شد، \متن‌لاتین{LoRaWAN} از پروتکل کنترل دسترسی همزمان \متن‌لاتین{ALOHA} استفاده می‌کند. این پروتکل سربار کمی دارد ولی در شبکه‌های شلوغ به خوبی عمل نمی‌کند.
از این رو پژوهش‌های زیادی الگوریتم‌های جدید پیشهاد کرده و آن‌ها را به صورت شبیه‌سازی یا واقعی ارزیابی می‌کنند.

در \متن‌لاتین{NB-IoT} دسترسی همزمان تنها در زمان درخواست منابع وجود دارد
اما پروسه تخصیص منابع خود مصرف توان بالایی دارد و از همین رو پژوهش‌هایی به آن پرداخته‌اند.

البته هدف اصلی این رساله شبکه‌های \متن‌لاتین{LoRa} هستند و بررسی پژوهش‌هایی در حوزه شبکه‌های دیگری چون \متن‌لاتین{NB-IoT} از بابت الگوریتم‌های پیشنهادی صورت پذیرفته است.

\زیرقسمت{مرجع \مرجع{Chen20192}}

پژوهش \مرجع{Chen2019} قصد دارد تا به واسطه پروتکل \متن‌لاتین{LoRa} تصاویر کشاورزی را به صورت مطمئن منتقل کند.
این پژوهش بیان می‌کند در محیط‌های کشاورزی امکان استفاده از ارتباطات با سیم وجود نداشته و با توجه به پوشش‌دهی کم ارتباطات
\متن‌لاتین{WSN} توجهات را به سمت شبکه‌های \متن‌لاتین{LPWAN} جلب کرده است.
این پژوهش بیان می‌کند ارسال یک عکس ۱ مگابایتی با استفاده از بیشترین نرخ داده‌ی ممکن در لایه‌ی فیزیکی شبکه \متن‌لاتین{LoRa}
۴۹۸ ثانیه طول می‌کشد و شامل مصرف انژی بالا و از سوی دیگر باعث بار زیادی روی شبکه است
که با استفاده از پروتکل استاندارد \متن‌لاتین{LoRaWAN} و \متن‌لاتین{ALOHA} باعث نرخ زیادی از تصادم می‌شود.

این پژوهش بیان می‌کند حتی عکس‌هایی که بسیار فشرده شده هستند، نیز در یک بسته \متن‌لاتین{LoRa} با حداکثر اندازه ۲۵۵ بایت جا نمی‌گیرند.
بنابراین روشی جایگزین روش \متن‌لاتین{Stop and Wait} در شبکه‌ی \متن‌لاتین{LoRa} برای انتقال مطمئن با نام \متن‌لاتین{Multi-Packet LoRa} یا اختصارا \متن‌لاتین{MPLR} پیشنهاد می‌دهد.
این روش بسته‌ها را به صورت دسته‌ای ارسال کرده و برای آن‌ها به صورت دسته‌ای \متن‌لاتین{Ack} ارسال می‌کند تا زمان انتظار برای \متن‌لاتین{Ack}ها و ترافیک مربوط به آن‌ها را کاهش دهد.
برای هر دسته‌ی ارسالی از بسته‌ها یک \متن‌لاتین{Ack} شامل وضعیت دریافت صحیح یا غیرصحیح بسته‌ها دریافت می‌شود. با توجه به اینکه پروتکل \متن‌لاتین{LoRa} به صورت \متن‌لاتین{Half-Duplex}
عمل می‌کند امکان ارسال همزمان بسته‌ها و \متن‌لاتین{Ack} وجود ندارد.

در ادامه این پژوهش برای زمانی که چند گره در حال ارسال عکس به یک دروازه هستند، یک پروتکل تخصیص کردن کانال پیشنهاد می‌دهد.
رویه پیشنهادی برای دروازه‌هایی است که پشتیبانی همزمان از چند کانال را ندارند.
در این رویه یک کانال به کنترل و یک کانال به داده تخصیص داده می‌شود. اشیا برای ارسال می‌بایست درخواست تشکیل ارتباط را روی کانال کنترلی ارسال کرده
و در صورت تایید، ارتباط را روی کانال داده شکل بدهند.
به این ترتیب همواره یک ارتباط وجود دارد و تداخلی روی کانال داده رخ نمی‌دهد.
انتخاب کانال داده تصادفی نبوده و دروازه کانالی را انتخاب می‌کند که کمترین از دست رفت بسته را داشته است.
در واقع این پژوهش با استفاده از کانال‌های مختلف برای کنترل و داده و در عین حال برقراری تنها یک ارتباط روی یک کانال نیازی به استفاده از
الگوریتم‌های دسترسی همزمان ندارد.

در کنار راه‌کارهای ارتباطی پیشنهادی، این پژوهش عکس‌ها را نیز به وسیله‌ی کتابخانه \متن‌لاتین{Pillow} در زبان \متن‌لاتین{Python} فشرده‌سازی می‌کند.
میزان این فشرده‌سازی با توجه به کاربرد قابل تعیین کردن است.

\زیرقسمت{مرجع \مرجع{Beltramelli2021}}

در \مرجع{Beltramelli2021} پژوهشگران یک لایه \متن‌لاتین{MAC} جدید برای شبکه‌های \متن‌لاتین{LoRaWAN} پیشنهاد داده‌اند. در این پروژه در ابتدا کارهای پیشین در زمینه بهبود کنترل دسترسی همزمان
مورد بحث قرار گرفته است. این پژوهش خود قصد دارد از شیوه‌ی \متن‌لاتین{Slotted ALOHA} یا اختصارا \متن‌لاتین{S-ALOHA} با همگام‌سازی خارج از باند استفاده کند.
در نهایت در این پروژهش این شیوه در شبیه‌سازی و شرایط واقعی شهری آزمایش می‌شود. در نظر داشته باشید که روش پیشنهادی صرفا برای برنامه‌های نظارتی بوده که پیام‌های \متن‌لاتین{downlink} ندارند.

روش‌های همگام‌سازی خارج از باند این پژوهش عبارتند از:

\شروع{فقرات}
\فقره \متن‌لاتین{GNSS}: \متن‌لاتین{GPS} امروزه یکی از گسترش‌یافته‌ترین \متن‌لاتین{GNSS}ها بوده و می‌تواند با دقت خوبی همگام‌سازی زمان را انجام دهد اما هزینه و توان مصرفی آن برای بیشتر راهکارهای اینترنت اشیا زیاد است.
\فقره \متن‌لاتین{RCCs}:‌ در این سیستم از موج‌های بلند در بازه $40kHz$ تا $80kHz$ استفاده می‌شود. این امواج در مرکز اروپا در دسترس بوده و می‌توان از آن‌ها برای همگامی زمانی تا دقت $0.1$ ثانیه استفاده کرد.
\فقره \متن‌لاتین{FM-RDS}: در این سیستم داده‌های همگام‌سازی از موج \متن‌لاتین{FM} استخراج می‌شود اما لزوما همه‌ی پایگاه‌های \متن‌لاتین{FM} این داده را ارسال نمی‌کنند و لزوما نیز همه پایگاه‌ها خود زمان صحیح ندارند.
\پایان{فقرات}

از بین این روش‌ها، پژوهش حاضر از \متن‌لاتین{FM-RDS} استفاده می‌کند. ارزیابی واقعی به وسیله دو دستگاه صورت گرفته است و برای ارزیابی با تعداد اشیا بالا از شبیه‌سازی \متن‌لاتین{LoRaEnergySim} به زبان \متن‌لاتین{Python} و به صورت تغییریافته استفاده شده است.

\زیرقسمت{مرجع \مرجع{Lee2021}}

پژوهش \مرجع{Lee2021} از جمله مقالاتی است که در بهبود لایه \متن‌لاتین{MAC} برای بهبود شبکه‌های \متن‌لاتین{LoRa} در مقیاس بزرگ کار کرده است.
این پژوهش ساختار جدیدی برای لایه \متن‌لاتین{MAC} ارائه داده و بر این ساختار بحث ارسال گروهی پیام‌های \متن‌لاتین{ACK} را مطرح می‌کند.
عملکرد این لایه پیشنهادی بر پایه ارسال متناوب \متن‌لاتین{Beacon}ها برای همگام‌سازی و مشخص شدن ساختار \متن‌لاتین{Frame}ها است.

هر \متن‌لاتین{Frame} خود از \متن‌لاتین{Subframe}هایی تشکیل شده است که هر یک در \متن‌لاتین{slot}های مشخصی به گرهها
اجازه می‌دهد تا \متن‌لاتین{uplink} و \متن‌لاتین{downlink} داشته باشند.

در این ساختار، پژوهش قصد دارد به صورت گروهی \متن‌لاتین{Ack} ارسال کند. برای ارسال گروهی این \متن‌لاتین{Ack}ها یک پیام مشخص ارسال می‌شود.
این پیام مشخص می‌تواند با توجه به اندازه‌ای که دارد شامل تعداد مختلفی از \متن‌لاتین{Ack}ها باشد. همانطور که پیشتر اشاره شد این اندازه بسته خود وابسته به
نرخ ارسال داده و \متن‌لاتین{SF}ها است. از آنجایی که \متن‌لاتین{SF}ها به صورت شبه عمود هستند، دروازه‌ها می‌توانند از \متن‌لاتین{SF}ها به صورت همزمان
استفاده کنند اما با توجه به تغییر نرخ ارسال، آن‌ها به تعداد \متن‌لاتین{slot}ها متفاوتی نیاز خواهند داشت و بنابراین مساله تخصیص بهینه \متن‌لاتین{SF}ها وجود خواهد داشت که این پژوهش به
آن نیز پرداخته است.

\زیرقسمت{مرجع \مرجع{Polonelli2019}}

پژوهش \مرجع{Polonelli2019} روش \متن‌لاتین{Slotted ALOHA} برای دسترسی همزمان در شبکه‌های \متن‌لاتین{LoRaWAN} به جای استفاده از \متن‌لاتین{Pure ALOHA} ارائه می‌دهد.
این روش بر روی روش \متن‌لاتین{Pure ALOHA} که روش استاندارد است ارائه شده و از این رو نیازی به تغییر در کتابخانه‌های فعلی نیست.
این پژوهش بیان می‌کند که این روش برای گرههای کلاس $A$ ارائه می‌شود چرا که گرههای کلاس‌های $B$ و $C$ محدودیت مصرفی کمی دارند.

روش \متن‌لاتین{Slotted ALOHA} یا اختصارا \متن‌لاتین{S-ALOHA} از دهه هفتاد در شبکه‌های محلی بی‌سیم شناخته شده است. در این روش زمان به قسمت‌های مشخصی تقسیم می‌شود
و هر دستگاه تنها در ابتدای هر زمان می‌تواند شروع به ارسال کند و در صورت تداخل ارسال را متوقف می‌کند. بهره‌وری کانال در این شیوه به صورت تئوری ۳۷ درصد است.

در ادامه این پژوهش برای ارزیابی کارایی روش پیشنهادی شروع به محاسبه پارامترهای لازم می‌کند. اولین پارامتر $T$ مدت زمان لازم برای ارسال موفقیت آمیز یک بسته است. در \متن‌لاتین{LoRaWAN}
برای ارسال یک بسته و سپس دریافت \متن‌لاتین{ACK} آن نیاز به صبر کردن تا رسیدن برای اولین بازه دریافت است، این باز تاخیر \متن‌لاتین{RX1 Delay} نامیده می‌شود و با توجه به زمان بالای ارسال
به خصوص در \متن‌لاتین{SF}های بالا عملا این بازه نباید برای ارسال سایر گرهها استفاده شود. با توجه به این موضوع خواهیم داشت:

\[
  T_{LoRaWAN} = T_{ack} + T + RX1\_Delay \approx 2.22T
\]

از همین زمان مشخص می‌شود که در صورت استفاده از ارتباط نیمه دو طرفه (درخواست \متن‌لاتین{ACK} برای پیام‌های ارسالی) در \متن‌لاتین{LoRaWAN} در صورت استفاده از \متن‌لاتین{Pure ALOHA}
یا انحصارا \متن‌لاتین{P-ALOHA} گذردهی به شدت افت می‌کند. پژوهش حاضر این را با استفاده از شبیه‌سازی در \متن‌لاتین{MATLAB} نیز به نمایش گذاشته است.(شکل \رجوع{شکل: گذردهی شبکه LoRaWAN})

\شروع{شکل}
\درج‌تصویر[width=\textwidth]{./img/lorawan-throughput.png}
\تنظیم‌ازوسط
\شرح{شبیه‌سازی گذردهی \متن‌لاتین{LoRaWAN} در زمان استفاده از بسته‌های با تایید تحت فاکتورهای گسترش مختلف \مرجع{Polonelli2019}}
\برچسب{شکل: گذردهی شبکه LoRaWAN}
\پایان{شکل}

در ادامه با در نظر گرفتن همین زمان ارسال و استفاده از \متن‌لاتین{S-ALOHA} گذردهی تا دو برابر افزایش پیدا می‌کند. یکی از مسائل مهم در استفاده از نسخه \متن‌لاتین{S-ALOHA} مساله همگام‌سازی گرهها است
در این پژوهش برای این امر از بسته‌های \متن‌لاتین{ACK} استفاده می‌شود و به این ترتیب گرهها با در نظر گرفتن زمان رسیدن \متن‌لاتین{ACK} زمان خود را با دروازه همگام می‌کنند.

در این پژوهش فرض شده است که از کانال ۶ در پروتکل \متن‌لاتین{LoRa} استفاده می‌شود و برای جلوگیری از پیچیدگی محاسباتی تغییر کانال توسط گرهها لحاظ نشده است. بازه‌های زمانی در \متن‌لاتین{S-ALOHA}
پیشنهادی به جز زمان $T_{LoRaWAN}$ محاسبه شده یک بازه اطمینان برای طولانی شدن پروسه همگام‌سازی نیز دارند. در نظر داشته باشید که برای اجرای رویه همگام‌سازی نیاز است که در ابتدا از سوی شی ارسال
صورت بپذیرد.

همانطور که اشاره شد پروتکل پیشنهادی در این پژوهش کاملا با پیاده‌سازی فعلی \متن‌لاتین{LoRaWAN} همگام است و از این رو این پژوهش ارزیابی عملیاتی از پروتکل پیشنهادی داشته است.
در کنار ارزیابی تئوری نتایج ارزیابی عملیاتی حتی بهتر نیز بوده‌اند. در نهایت این پروتکل پیشنهادی تنها با افزودن یک سربار برای ارسال زمان در پیام‌های \متن‌لاتین{ACK} گذردهی به مراتب بهتر از دو برابر
نسبت به نسخه‌ی اصلی بدست می‌آورد. تنها مساله باقی‌مانده در همگام‌سازی شروع توسط گرهها است که در صورت عدم ارسال داده برای بازه‌ی طولانی و استفاده از کریستال‌های ارزان قیمت می‌تواند
اختلاف زمان زیادی بین گرهها به وجود بیاورد.

\زیرقسمت{مرجع \مرجع{Lee2017}}

در پژوهش \مرجع{Lee2017} هدف ارائه یک روش تخصیص منابع مبتنی بر پیش‌بینی برای شبکه \متن‌لاتین{NB-IoT} است.
این پژوهش بیان می‌کند کارهای پیشین در این حوزه به بحث پیش‌بینی ترافیک اینترنت اشیا و ارتباط میان ترافیک‌های \متن‌لاتین{uplink} و \متن‌لاتین{downlink} نپرداخته‌اند.

حتی بعد از زمانی که ارتباط حامل رادیویی برقرار شده است و \متن‌لاتین{RRC} در وضعیت اتصال است، در \متن‌لاتین{NB-IoT} نمی‌توان بسته‌های \متن‌لاتین{uplink} را بدون
رویه درخواست زمان‌بندی ارسال کرد. مساله این پژوهش مشکل مصرف توان در این پروسه درخواست زمان‌بندی است.

مشابه با شبکه‌های \متن‌لاتین{LTE}، در \متن‌لاتین{NB-IoT} منابع رادیویی برای ارسال بسته‌ها بین دستگاه‌های کاربران (\متن‌لاتین{UE}ها) مشترک است و
زمان‌بند \متن‌لاتین{eNodeB} به صورت پویا این منابع را بر پایه سیاست زمان‌بندی به هر دستگاه کاربر تخصیص می‌دهد.
برخلاف \متن‌لاتین{LTE}، \متن‌لاتین{NB-IoT} منابع رادیویی اختصاصی برای درخواست‌های زمان‌بندی ندارد و بنابراین این درخواست‌ها تنها ممکن است از طریق روش‌های دسترسی تصادفی ارسال شوند.
از آنجایی که این رویه با سایر دستگاه‌ها رقابت می‌کند، ممکن تصادم به وجود بیاید و در این شرایط دستگاه کاربر می‌بایست بعد از یک زمان مشخص دوباره دسترسی تصادفی را تکرار کند.
در پاسخ به این درخواست \متن‌لاتین{eNodeB} پاسخی را ارسال می‌کند که شامل اطلاعات پیشگیری از تصادم و دستور زمان‌بندی است.
از آنجایی که زمان‌بند \متن‌لاتین{eNodeB} هیچ اطلاعی از حجم اطلاعات در بافر دستگاه کاربر ندارد، یک منبع کوچک رادیویی برای دریافت وضعیت بافر دستگاه کاربر در نظر می‌گیرد.
دستگاه کاربر پس از دریافت دستور زمان‌بندی \متن‌لاتین{uplink}، گزارش وضعیت بافر خود را ارسال کرده و \متن‌لاتین{eNodeB} بعد از دریافت آن به دستگاه کاربر اطلاع داده و به صورت پیوسته
منابع رادیویی \متن‌لاتین{uplink} را تخصیص می‌دهد تا همه‌ی اطلاعات بافر به صورت کامل منتقل شوند.
این مراحل در شکل \رجوع{شکل: زمان‌بندی در NB-IoT} نمایش داده شده است.

\شروع{شکل}
\تنظیم‌ازوسط
\درج‌تصویر[height=\textwidth]{img/nbiot-scheduling.png}
\شرح{رویه درخواست زمان‌بندی به صورت دسترسی تصادفی \مرجع{Lee2017}}
\برچسب{شکل: زمان‌بندی در NB-IoT}
\پایان{شکل}

این پژوهش بیان می‌کند در کنار دست‌دادهایی که در \متن‌لاتین{NB-IoT} صورت می‌گیرد، مساله مصرف توان برای دست‌دادهایی که پروتکل‌های لایه انتقال
استفاده می‌شوند مانند \متن‌لاتین{CoAP}، \متن‌لاتین{DTLS} و \نقاط‌خ هم وجود دارد.
در واقع اگر مساله انتقال یک بسته \متن‌لاتین{uplink} باشد سربار پروسه درخواست زمان‌بندی زیاد نخواهد بود اما اگر یک دست‌داد میان شبکه و دستگاه
کاربر لازم باشد، سربار این رویه بیشتر خواهد بود و مصرف توان آن نیز افزایش پیدا می‌کند.. شکل \رجوع{شکل: دست‌داد میان شبکه و دستگاه کاربر در NB-IoT} این مورد را بهتر نمایش می‌دهد.

\شروع{شکل}
\تنظیم‌ازوسط
\درج‌تصویر[height=.8\textwidth]{img/nbiot-handshake.png}
\شرح{دست‌داد میان شبکه و دستگاه کاربر در \متن‌لاتین{NB-IoT} \مرجع{Lee2017}}
\برچسب{شکل: دست‌داد میان شبکه و دستگاه کاربر در NB-IoT}
\پایان{شکل}

روش پیشنهادی این پژوهش که بر پایه پیش‌بینی عمل می‌کند، \متن‌لاتین{Prediction-Based Energy Saving Mechanism} یا مختصرا \متن‌لاتین{PBESM} نام دارد.
این روش تخصیص منابع را مبتنی بر پیش‌بینی صورت می‌دهد و باعث می‌شود مصرف انرژی به خاطر رویه‌های درخواست زمان‌بندی کاهش پیدا کند.
در این روش با استفاده از رخ داد \متن‌لاتین{uplink}ها و تاخیر پردازش که از بررسی بسته‌ها بدست می‌آید، \متن‌لاتین{eNodeB}
منابع رادیویی را برای انتقال بسته \متن‌لاتین{uplink} از پیش تخصیص می‌دهد و به این ترتیب اجازه می‌دهد تا بسته‌های \متن‌لاتین{uplink}
بدون رویه درخواست زمان‌بندی امکان ارسال داشته باشند.

همانطور که بیان شد این الگوریتم بسته‌ها را پردازش می‌کند و از این رو در صورتی که بخواهیم زمان پاسخ برنامه را در این الگوریتم محاسبه کنیم این الگوریتم نیاز دارد تا پروتکل لایه کاربرد
موردنظر پشتیبانی کند. در پیاده‌سازی پژوهشگران از این الگوریتم پروتکل‌های \متن‌لاتین{RLC}، \متن‌لاتین{RRC}، \متن‌لاتین{NAS} و \نقاط‌خ صورت پذیرفته است.
برای پیش‌بینی زمان پردازش مقدار ابتدایی با توجه به شبیه‌سازی‌ها پر شده و الگوریتم از میانگین متحرک و صحت پیش‌بینی صورت گرفته، استفاده می‌کند.
در سمت دستگاه کاربر پیام‌های از پیش تخصیص یافتن منابع ذخیره می‌شود و در صورتی که کاربر داده‌ای برای ارسال داشته باشد تا فرارسیدن این زمان بازه از پیش تخصیص یافته
صبر می‌شود اما اگر پیامی مبنی بر از پیش تخصیص یافتن منابع وجود نداشته باشد رویه درخواست سنتی زمان‌بندی صورت می‌گیرد. اگر منابعی از پیش تخصیص یافته باشد اما دستگاه
در آن داده‌ای برای ارسال نداشته باشد، هیچ پیامی ارسال نمی‌کند و این به منزله پیش‌بینی غلط است.

در نهایت این پژوهش بیان می‌کند با استفاده از \متن‌لاتین{PBESM} فاز دسترسی تصادفی حذف می‌شود و از همین رو مصرف توان کاهش پیدا می‌کند.
مشکل اصلی در روش \متن‌لاتین{PBESM} تاخیری است که داده ممکن است تا رسیدن به شروع بازه از پیش تخصیص یافته تحمل کند.
این تاخیر قابل کنترل است که با افزایش آن خطاهای پیش‌بینی کاهش پیدا می‌کند و مصرف توان بهتر خواهد شد ولی تاخیر روی داده‌ها بیشتر خواهد بود.
این پارامتر مصالحه‌ای میان تاخیر و توان مصرفی است. برای ارزیابی این روش از شبیه‌سازی در سطح سیستم استفاده شده است.
در این شبیه‌سازی از ۵ وضعیت متفاوت استفاده شده است:

\شروع{شمارش}
\فقره \متن‌سیاه{سناریو گزارش‌دهی دستگاه کاربر}، مانند سرویس اندازه‌گیری دوره‌ای، که در آن دستگاه اینترنت اشیا مقدار حسگر را
به شبکه با بسته‌های غیر \متن‌لاتین{IP} ارسال کرده و با دریافت تاییدیه از شبکه نشست کامل می‌شود.
\فقره در این سناریو برنامه کاربردی وضعیت شی و مقدار حسگر آن را بررسی می‌کند. مانند حالت اول، از بسته‌های غیر \متن‌لاتین{IP} استفاده می‌شود اما
نشست از سمت شبکه آغاز شده و با پاسخ دستگاه کاربر و تاییدیه‌اش خاتمه می‌یابد.
\فقره این سناریو مشابه با وضعیت ۲ است با این تفاوت که از بسته‌های \متن‌لاتین{IP} در آن استفاده می‌شود. پروتکل مورد استفاده \متن‌لاتین{UDP/DTLS/CoAP} بوده و فرض می‌شود نشست \متن‌لاتین{DTLS} می‌بایست ایجاد شود.
\فقره این سناریو مشابه با وضعیت ۲ است با این تفاوت که از بسته‌های \متن‌لاتین{IP} در آن استفاده می‌شود. پروتکل مورد استفاده \متن‌لاتین{UDP/DTLS/CoAP} بوده و فرض می‌شود نشست \متن‌لاتین{DTLS} وجود داشته و در صورت نیاز تمدید می‌شود.
\فقره این سناریو مشابه با وضعیت ۲ است با این تفاوت که از بسته‌های \متن‌لاتین{IP} در آن استفاده می‌شود. پروتکل مورد استفاده \متن‌لاتین{TCP} بوده که سرآیند به واسطه یک الگوریتم کاهش اندازه سرآیند قدرتمند کاهش پیدا کرده است.
\پایان{شمارش}

در نهایت برای شبیه‌سازی زمان پاسخ از توزیع \متن‌لاتین{Pareto} استفاده شده است. در انتها این پژوهش کاهش توان مصرفی را به صورت ریاضی و بر پایه احتمال موفیت آمیز بودن پیش‌بینی فرمول‌بندی می‌کند.
