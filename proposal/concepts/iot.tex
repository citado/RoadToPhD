\قسمت{زیست‌بوم اینترنت اشیا}

اینترنت اشیا اولین بار توسط \متن‌لاتین{Kevin Ashton} در سال ۱۹۹۹ پیشنهاد شد، او اینترنت اشیا را به عنوان
شبکه‌ای از اشیا هم‌کنشپذیر و قابل شناسایی به وسیله‌ی فرکانس رادیویی (\متن‌لاتین{RFID}) تعریف می‌کند.
کلمه‌های ``اینترنت'' و ``اشیا'' به معنی شبکه‌ای متصل و جهانی بر پایه حسگرها، ارتباطات، شبکه‌سازی و فناوری‌های پردازش داده است
که نسخه‌ی جدیدی از فناوری اطلاعات یا \متن‌لاتین{ICT} را عرضه می‌کند.
پیشرفت فناوری‌های حسگر بی‌سیم کمک شایانی به قابلیت‌های دستگاه‌های حسگر و بنابراین مفهوم پایه اینترنت اشیا کرده است.
اهمیت اینترنت اشیا برای کشورهای در حال توسعه و توسعه یافته بسیار واضح است و این کشورها شروع به سرمایه‌گذاری در این حوزه نموده‌اند
\مرجع{Li2014}.
در سال‌های اخیر با کاهش قیمت حسگرهای و عملگرها، تعداد دستگاه‌های اینترنت اشیا به سرعت در حال گسترش است و به سرعت در حال تبدیل کردن خود به یکی از اجزا زندگی ما هستند.
\مرجع{Mishra2021}.

تولد حقیقی اینترنت اشیا با توجه به تخمین \متن‌لاتین{Cisco} به بازه‌ای بین سال‌های ۲۰۰۸ تا ۲۰۰۹ بازمی‌گردد که برای اولین بار تعداد اشیا
متصل از جمعیت جهان در آن سال‌ها بیشتر شد. در سال ۲۰۱۰ تعداد اشیا متصل تقریبا دو برابر جمعیت جهان در آن سال شد و تقریبا به عدد ۱۲/۵ بیلیون رسید.
از آن سال‌ها به لطف پیشرفت‌های فناوری و سرمایه‌گذاری‌های قابل توجه شرکت‌ها، اینترنت اشیا در حال گسترش در زندگی روزمره است
\مرجع{Lombardi2021}.
با توجه به آغاز اینترنت اشیا از \متن‌لاتین{RFID}ها بسیاری از پژوهش‌ها رشد این فناوری در بستر \متن‌لاتین{RFID} را مورد بحث قرار داده‌اند
\مرجع{Ray2018}.

با آمدن اینترنت اشیا و ارتباطات ماشین به ماشین انتظار می‌رود به زودی افزایش زیادی در تعداد گرهها دیده شود. پیش‌بینی می‌شود تا سال ۲۰۲۵ بیش از ۷۵ بیلیون گره اینترنت اشیا داشته باشیم.
این گرهها شامل ماشین‌ها، حسگرها، شمارشگرها، دستگاه‌های فروش و \نقاط‌خ است \مرجع{Chaudhari2020}.
تحقیقات و پژوهش‌های متفاوت اینترنت اشیا را به شکل‌های مختلفی تعریف کرده‌اند که از جمله‌ی آن‌ها می‌توان به
\شروع{نقل}
یک زیرساخت شبکه‌ای پویا و جهانی با قابلیت‌های خودپیکربندی بر پایه پروتکل‌های ارتباطی استاندارد و هم‌کنشپذیر که در آن اشیا فیزیکی و مجازی دارای شناسه، صفات فیزیکی
و شخصیت مجازی بوده، از رابط‌های هوشمند استفاده می‌کنند و به صورت یکپارچه در شکبه اطلاعاتی مجمتع می‌شوند.
\پایان{نقل}

اشاره کرد
\مرجع{Ray2018} \مرجع{Razzaque2016}.

یکی از چالش‌ها در اینترنت اشیا نبود استانداردها و تاثیر آن بر گسترش در این حوزه است. در این سال‌ها فعالیت‌های زیادی در حوزه طراحی استانداردها
صورت گرفته است. مهمترین این استانداردها برای میان‌افزارها و رابط‌ها بوده‌اند. این استانداردها که هر یک توسط گروه‌ها و ارگان‌های خاصی تهیه می‌شوند که
خود می‌بایست هماهنگی و ارتباط داشته باشند
\مرجع{Li2014}.

پیش از آنکه بخواهیم به معماری‌های اینترنت اشیا بپردازیم، قصد داریم ویژگی‌های آن را مرور کنیم
\مرجع{Lombardi2021}:
\شروع{فقرات}
\فقره \متن‌سیاه{گسترش‌پذیر}، برای مدیریت تعداد رو به افزایش اشیا و سرویس‌ها بدون کاهش کارایی
\فقره \متن‌سیاه{همکنش‌پذیر}، تا دستگاه‌هایی از سازندگان مختلف بتوانند برای یک هدف مشترک همکاری کنند.
\فقره \متن‌سیاه{توزیع‌پذیر}، تا اجازه دهد یک محیط توزیع‌شده ساخته شود که در آن داده بعد از جمع‌آوری شدن از منابع مختلف، به وسیله موجودیت‌های مختلف و به شکل توزیع شده پردازش شود.
\فقره \متن‌سیاه{فعالیت با منابع کم}، از آنجایی که اشیا توان پردازشی پایینی دارند.
\فقره \متن‌سیاه{امن} تا از دسترسی‌های غیرمجاز جلوگیری کند.
\پایان{فقرات}

معماری اینترنت اشیا را می‌توان در چهار لایه با کارکردهای متفاوت دسته‌بندی کرد:

\شروع{فقرات}
\فقره \متن‌سیاه{لایه حسگر} با اشیا حاضر در هم آمیخته می‌شود تا بتواند وضعیت اشیا را حس کند. در واقع در این لایه سنسورها می‌توانند به صورت خودکار وضعیت محیط را ارزیابی کرده و با سایر اشیا تبادل اطلاعات داشته باشند.
در این لایه هزینه، اندازه، منابع و مصرف انرژی باید بهینه باشد. در این لایه شبکه‌بندی اشیا، ارتباطات و انتخاب پروتکل ارتباطی مناسب، تنوع و نحوه استقرار از سایر دغدغه‌ها است.
\فقره \متن‌سیاه{لایه شبکه} زیرساختی است که ارتباط بی‌سیم یا دارای سیم میان اشیا را پشتیبانی می‌کند. در این لایه اشیا به یکدیگر متصل شده و می‌توانند از اطراف باخبر شوند. این لایه می‌تواند داده‌ها را با زیرساخت فناوری اطلاعات فعلی
ترکیب کرده یا آن‌ها را برای واحد تصمیم‌گیری شامل سرویس‌های سطح بالا و پیچیده ارسال کند. در لایه شبکه دغدغه‌هایی مانند مدیریت شبکه، مصرف انرژی شبکه، نیازمندی‌های کیفیت سرویس، پیدا کردن اشیا و امنیت مطرح است.
\فقره \متن‌سیاه{لایه سرویس}، سرویس‌های لازم برای کاربران یا برنامه‌های کاربردی را مدیریت کرده یا می‌سازد. این لایه از فناوری میان‌افزار استفاده می‌کند، که یک فناوری کلیدی در سرویس‌ها و برنامه‌های کاربردی اینترنت اشیا است.
سرویس‌ها را می‌توان به عنوان یک فعل، شامل جمع‌آوری، جابجایی و ذخیره‌سازی داده یا ترکیب این‌ها برای رسیدن به یک هدف مشخص، در نظر گرفت.
این لایه خود از اجزا زیر تشکیل شده است.
\شروع{فقرات}
\فقره \متن‌سیاه{سرویس اکتشاف} این سرویس اشیایی که اطلاعات یا سرویس مشخصی را فراهم می‌آورند، پیدا می‌کند.
\فقره \متن‌سیاه{ترکیب سرویس‌ها} ارتباط میان اشیا متصل را ممکن می‌سازد. سرویس اکتشاف اشیا را برای یافتن سرویس موردنظر جستجو می‌کند و ترکیب سرویس‌ها با زمان‌بندی و باز ساخت سرویس‌های بیشتر مطمئن‌ترین سرویس را ارائه میدهد.
در واقع در اینترنت اشیا برخی از نیازمندی‌ها می‌توانند با یک سرویس برآورده شوند و برخی برای برآورده شدن به ترکیب سرویس‌ها نیاز دارند.
\فقره \متن‌سیاه{مدیریت ارزشمندی} درک می‌کند که داده‌های هر سرویس می‌بایست چگونه پردازش شوند.
\فقره \متن‌سیاه{رابط‌های کاربری سرویس‌ها} چگونگی ارتباط سرویس‌ها با کاربران و اشیا را مشخص می‌کنند.
\پایان{فقرات}
\فقره \متن‌سیاه{لایه رابط‌ها} شامل روش‌هایی برای ارتباط با کاربران و برنامه‌ها است. همانطور که بیان شد اینترنت اشیا از تعداد زیادی دستگاه متنوع تشکیل شده است و یک رابط کاربری کارا برای ساده‌سازی مدیریت و ارتباط اشیا لازم است.
\پایان{فقرات}
