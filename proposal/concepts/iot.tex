\قسمت{اینترنت اشیا}

اینترنت اشیا از اجزای مختلفی تشکیل شده است که کنار یکدیگر سیستم اینترنت اشیا را تشکیل می‌دهند. در این قسمت قصد داریم مروری بر این اجزا داشته باشیم.

در شکل \رجوع{شکل: اجزای کارکردی یک دستگاه اینترنت اشیا} اجزای کارکردی یک دستگاه اینترنت اشیا آورده شده است.
اینترنت اشیا بر پایه همین دستگاه‌ها که دارای حسگر و عملگر بوده و می‌توانند عملیات مانیتورینگ و کنترل را انجام دهند، بنا شده است.
این دستگاه‌ها می‌توانند با سایر دستگاه‌ها و برنامه‌های کاربردی متصل تبادل داده داشته باشند یا می‌توانند داده‌ها را از سایر دستگاه‌ها جمع‌آوری کرده
و آن‌ها را پردازش کنند یا در جهت پردازش آن‌ها را به سرورهای ابری ارسال کنند. این تقسیم پردازشی بین پردازش محلی و ابری می‌تواند بر پایه محدودیت‌های پردازشی و حافظه‌ای
صورت بپذیرد
\مرجع{Ray2018}.

یک دستگاه اینترنت اشیا می‌توانند چند رابط جهت ارتباط با سایر اشیا به صورت بی‌سیم یا دارای سیم داشته باشد.
رابط‌های یک دستگاه اینترنت اشیا را می‌توان به اقسام زیر تقسیم کرد:

\شروع{فقرات}
\فقره رابط‌های ورودی و خروجی برای ارتباط با حسگرها و عملگرها
\فقره رابط‌های شبکه
\فقره رابط‌های ذخیره‌سازی و حافظه
\فقره رابط‌های صوتی و تصویری
\پایان{فقرات}

دستگاه‌های اینترنت اشیا می‌توانند در اشکال متفاوتی ظاهر شوند که از جمله‌ی آن می‌توان به شکل‌های، حسگرهای پوشیدنی، ساعت‌های هوشمند،
روشنایی‌های \متن‌لاتین{LED}، خودرو و ماشین‌های صعنعی اشاره کرد. اشیا می‌توانند داده‌هایی در فرمت‌های گوناگون تولید کنند که بعد از پردازش توسط سیستم‌های
پردازش داده باعث اطلاعات کاربردی می‌شوند که می‌توان از آن‌ها برای واکنش‌های مستقیم یا غیرمستقیم استفاده نمود
\مرجع{Ray2018}.

\شروع{شکل}
\درج‌تصویر[width=\textwidth]{./img/iot-device-components.png}
\تنظیم‌ازوسط
\شرح{اجزای کارکردی یک دستگاه اینترنت اشیا \مرجع{Ray2018}}
\برچسب{شکل: اجزای کارکردی یک دستگاه اینترنت اشیا}
\پایان{شکل}

قسمت بعدی در اینترنت اشیا ارتباطات است، این قسمت وظیفه برقراری ارتباط میان دستگاه‌ها و سرورها را برعهده دارد.
سرویس‌ها قسمت سوم در اینترنت اشیا هستند. سرویس‌ها می‌توانند کارکردهای مختلفی برای مدل کردن دستگاه‌ها، کنترل دستگاه،
انتشار داده‌ها، آنالیز داده‌ها و اکتشاف دستگاه‌ها را فراهم آورند
\مرجع{Ray2018}.

بلوک مدیریت در اینترنت اشیا زیرساخت قانون‌گذاری در اینترنت اشیا را با استفاده از کارکردهای مدیریتی فراهم می‌آورد.
بلوک امنیت کارکردهایی چون احراز هویت، سطوح دسترسی، حریم خصوصی، یکپارچی پیام‌ها، یکپارچی محتوا و امنیت داده‌ها را فراهم می‌آورد.
در نهایت بلوک برنامه‌های کاربردی قرار دارد که رابط کاربری مشتریان بوده و ماژول‌های لازم در جهت کنترل و نظارت بر جنبه‌های مختلف سیستم اینترنت اشیا
را فراهم می‌آورد. برنامه‌های کاربردی به کاربران اجازه می‌دهند داده‌های خود را نمایش دهند، سیستم را ارزیابی کنند و گاها جنبه‌های مختلف آن را پیش‌بینی کنند
\مرجع{Ray2018}.

سیستم‌های اینترنت اشیا می‌بایست پویا بوده و خود تطبیق‌پذیر باشند. مثلا دوربین‌های نظارتی باید بتوانند به صورت خودکار دید در شب را فعال کنند یا در صورت تشخیص حرکت
کیفیت ضبط خود و سایر دوربین‌ها را افزایش دهند. سیستم‌های اینترنت اشیا می‌توانند قابلیت خودپیکربندی داشته باشند. آن‌ها ممکن است بتوانند بروزرسانی‌های سیستم خود یا
تنظیمات رابط شبکه‌ای را با کمترین دخالت کاربران انجام دهند. دستگاه‌های اینترنت اشیا از پروتکل‌های هم‌کنشپذیر برای ارتباط با زیرساخت و سایر اشیا استفاده می‌کنند.
این دستگاه‌ها در سیستم اینترنت اشیا می‌بایست یک شناسه یکتا مانند آدرس \متن‌لاتین{IP} یا \متن‌لاتین{URI} داشته باشند. دستگاه‌های اینترنت اشیا عموما با شبکه‌های اطلاعاتی
یکپارچه‌سازی می‌گردند. در این یکپارچه‌سازی اشیا می‌توانند به صورت پویا در شبکه کشف شوند و بتوانند ویژگی‌ها خود برای سایر اشیا و برنامه‌های کاربردی ارائه دهند. این یکپارچی
سیستم اینترنت اشیا را هوشمندتر می‌کند چرا که اشیا و زیرساخت در کنار یکدیگر می‌توانند هوشمندی بالاتری داشته باشند.
در نهایت سیستم‌های اینترنت اشیا می‌توانند با استفاده از اشیای متفاوت تصمیمات هوشمندانه‌تری بگیرند
\مرجع{Ray2018}.

در نهایت می‌توان ویژگی‌های اینترنت اشیا به شرح زیر خلاصه کرد
\مرجع{Angel2021}:
\شروع{فقرات}
\فقره \متن‌سیاه{اتصال متقابل}: اینترنت اشیا می‌تواند به زیرساخت ارتباطات جهانی متصل شود.
\فقره \متن‌سیاه{سرویس‌های مرتبط با اشیا}: اینتنرت اشیا در ارائه اشیا فیزیکی و مجازی، حریم خصوصی و سرویس‌های سازگار معنایی در محدوده اشیا مهارت دارد.
\فقره \متن‌سیاه{تنوع}: دستگاه‌های اینترنت اشیا وابسته به زیرساخت‌های سخت‌افزاری و شبکه‌های متنوع هستند.
\فقره \متن‌سیاه{منابع محدود}: دستگاه‌های اینترنت اشیا محدودیت‌های پردازشی و توان مصرفی دارند.
\فقره \متن‌سیاه{تغییرات پویا}: وضعیت دستگاه‌ها و محیط در اینترنت اشیا ممکن است به صورت پویا تغییر کند.
\فقره \متن‌سیاه{محیط‌های کنترل نشده}: دستگاه‌های اینترنت اشیا در محیط‌هایی با تنظیمات کنترل‌نشده مستقر می‌شوند.
\فقره \متن‌سیاه{ابعاد بسیار بزرگ}: دستگاه‌هایی که در اینترنت اشیا می‌بایست نظارت شوند یا آن‌هایی که با یکدیگر ارتباط دارند بسیار زیاد هستند و به صورت نمایی به رشد خود در آینده ادامه می‌دهند.
\پایان{فقرات}

از سوی دیگر ویژگی‌های زیرساخت اینترنت اشیا از نظر \مرجع{Razzaque2016} به شرح زیر است:

\شروع{فقرات}
\فقره \متن‌سیاه{دستگاه‌های ناهمگون}: عمدتا در اینترنت اشیا از حسگرها و عملگرهایی با زیرساخت‌های محاسباتی کم هزینه استفاده می‌شود اما
اینترنت اشیا تنها از حسگرها و دستگاه‌های نهفته تشکیل نشده است و به سیستم‌های قدرتمند برای انجام وظایف سنگینی چون مسیریابی، پردازش داده
و \نقاط‌خ نیز نیاز دارد.
\فقره \متن‌سیاه{منابع محدود}: حسگرها و سیستم‌های محاسباتی نهفته نیاز به یک ساختار کوچک دارند و همین توان پردازشی، حافظه و محاسبات آن‌ها را محدود می‌کند.
\فقره \متن‌سیاه{تعاملات لحظه‌ای}: در کاربردهای اینترنت اشیا تعاملات لحظه‌ای می‌تواند با حرکت اشیا و قرارگیری آن‌ها در پوشش ارتباطی یکدیگر صورت بپذیرد که باعث
تولید رویدادهای لحظه‌ای می‌گردد.
\فقره \متن‌سیاه{شبکه‌های فوق بزرگ و تعداد زیاد رویدادها}: در محیط‌های اینترنت اشیا ممکن است هزاران دستگاه یا شی حتی در یک محیط محلی با یکدیگر تعامل داشته باشند
که بسیار از شبکه‌های تداول بزرگتر است.
\فقره \متن‌سیاه{شبکه‌ی پویا و بدون زیرساخت}: نودها متحرک در شبکه می‌توانند در هر زمان ملحق شده یا از شبکه جدا شوند. از سوی دیگر نودها ممکن است به خاطر ارتباط ضعیف بی‌سیم
یا کمبود باتری قطع شوند. این فاکتورها شبکه‌ها را در اینترنت اشیا بسیار پویا می‌نمایند.
\فقره \متن‌سیاه{آگاه به متن}: تعداد زیادی از حسگرها حجم بالایی از اطلاعات را تولید می‌کنند که این اطلاعات بدون پردازش، تفسیر و فهم هیچ ارزشی ندارند.
محاسبات آگاه از متن با ذخیره‌سازی اطلاعات مرتبط با داده‌های سنسور تفسیر را آسان‌تر می‌کند.
آگاهی از متن (به خصوص در زمینه‌های زمانی و مکانی) نقش حیاتی در رفتارهای تطبیقی و خودمختار اشیا در اینترنت اشیا دارد.
\فقره \متن‌سیاه{هوشمندی}: در افق اینترنت اشیا شرکت اینتل، دستگاه‌ها یا اشیا هوشمند و سیستم‌های هوشمند دو عنصر کلیدی اینترنت اشیا هستند.
\فقره \متن‌سیاه{آگاه از موقعیت}: اطلاعات مکانی و موقعیتی درباره اشیا یا حسگرها در اینترنت اشیا حیاتی است چرا که
موقعیت در محاسبات آگاه از متن نقش حیاتی دارد.
\فقره \متن‌سیاه{توزیع‌شده}: اینتنرت سنتی خود یک شبکه توزیع شده است و بنابراین اینترنت اشیا نیز یک شبکه توزیع شده خواهد بود.
بعد مکانی قوی در اینترنت اشیا باعث می‌شود شبکه‌ی اینترنت اشیا هم به صورت جهانی و هم به صورت محلی توزیع شده باشد.
\پایان{فقرات}

در ادامه \مرجع{Razzaque2016} ویژگی‌های برنامه‌های کاربردی اینترنت اشیا را به شرح زیر برمی‌شمارد:

\شروع{فقرات}
\فقره \متن‌سیاه{کاربردهای متنوع}: اینترنت اشیا می‌تواند سرویس‌های خود را به تعداد زیادی از کاربردها در حوزه‌ها و محیط‌های مختلف ارائه کند.
\فقره \متن‌سیاه{هم‌زمانی}: برنامه‌های کاربردی که از اینترنت اشیا استفاده می‌کنند به صورت کلی می‌توانند به دو دسته همزمان و غیرهمزمان تقسیم شوند.
\پایان{فقرات}

\زیرقسمت{کاربردهای اینترنت اشیا}

\زیرزیرقسمت{کشاورزی هوشمند}

در کشاورزی هوشمند هدف افزایش کیفیت و حجم محصولات با استفاده از مدیریت هوشمند آبیاری، نظارت هوشمند و \نقاط‌خ است.
در ساده‌ترین سطح کشاورزی هوشمند، قرار دادن سنسورهایی برای رطوبت خاک، رسانایی و \نقاط‌خ در زمین کشاورزی است. در سطوح بالاتر در کنار حسگرها، عملگرهایی برای آبیاری، سم‌پاشی و \نقاط‌خ تعریف می‌شوند.
امروزه \متن‌لاتین{UAV}ها در کشورهای پیشرفته جزئی از صنعت کشاورزی شده‌اند و می‌توانند به کشاورزان برای آبیاری، سم‌پاشی و نظارت بر محصولاتشان کمک کنند. مدیریت و برنامه‌ریزی این \متن‌لاتین{UAV}ها خود می‌تواند
در کشاورزی هوشمند صورت بپذیرد. در نهایت بحث کشاورزی دقیق نیز مطرح است که می‌توان از عکس‌های ماهواره‌ای و سنجش از دور برای تخمین محصول برداشتی، شناسایی آفات و \نقاط‌خ بهره جست.

با توجه به قرارگیری زمین‌های کشاورزی در مناطق غیرشهری و دورافتاده بحث نحوه ارتباط از چالش‌های مهم این حوزه است، چرا که ارتباط‌های سلولی لزوما در این مناطق فعال نیستند.
از سوی دیگر ارتباط با \متن‌لاتین{UAV}ها با توجه به ماهیت متحرکی که دارند از دیگر چالش‌های این حوزه است. در نهایت می‌توان به توان پردازشی مورد نیاز برای اجرای الگوریتم‌های پیش‌بینی
و \نقاط‌خ روی داده‌های جمع‌آوری شده اشاره کرد.

\زیرزیرقسمت{صحن هوشمند دانشگاه}

امروز با هوشمند شدن شهرها و کشورها، دانشگاه‌ها نیز می‌بایست پروژه‌های دیجیتال‌سازی را برای شرکت در این رقابت رو به گسترش و کمک به آموزش آغاز کنند.
امروز دانشگاه‌های مجازی مزایای زیادی را به وجود آورده‌اند و صحن هوشمند دانشگاه می‌تواند برگ برنده‌ای برای رقابت دانشگاه‌های فیزیکی باشد. از سوی دیگر دانشجویان تنها به شرکت در کلاس و راه‌های سنتی
برای یادگیری قانع نخواهند بود و هوشمندسازی می‌تواند راهکارهای بیشتری برای آموزش ارائه دهد.
یکی از چالش‌های پیشرو در هوشمندسازی دانشگاه‌ها عدم توانایی مدیران و زیرساخت \متن‌لاتین{IT} دانشگاه در پذیرش این حجم از پردازش و اشیا است
\مرجع{Jurva2020}.

شهرهای هوشمند ارتباط نزدیکی با دانشگاه‌های هوشمند دارند. شهرها می‌توانند از محیط دانشگاه هم برای پایلوت کردن مدل شهر هوشمند و هم برای تحقیق و توسعه‌ی آن استفاده کنند.
امروزه مشخص شده است برای شهر هوشمند تنها نیاز به زیرساخت نیست بلکه نیاز به دانش هم وجود دارد و دانشگاه نقش منحصر به فردی در این موضوع دارد
\مرجع{Jurva2020}.

\زیرزیرقسمت{شبکه انرژی هوشمند}

شبکه‌های انرژی هوشمند در حال توسعه برای جایگزینی شبکه سنتی انرژی هستند تا بتوانند یک سرویس مطمئن و کارا انرژی را به کاربران ارائه دهند.
در این شبکه‌های هوشند، ارتباط دو طرفه و اندازه‌گیری هوشمند باعث ایجاد تعامل میان مصرف‌کننده و تامین‌کننده شده است. ماشین‌های برقی
نیز جزئی از این شبکه بوده و ذخیره‌سازی انرژی را بهبود داده‌اند
و باعث شده‌اند گاز دی اکسید کربن کمتری منتشر شود
\مرجع{Lin2017}.

این شبکه‌ها می‌توانند میزان مصرف، تولید و ذخیره‌سازی مصرف‌کنندگان را اندازه‌گیری کنند و می‌توانند با تامین‌کنندگان در جهت اعلام اطلاعات مصرف انرژی مشترک و دریافت قبض‌های لحظه‌ای تعامل کنند.
با این اطلاعات مصرف‌کنندگان می‌توانند بهینه‌تر مصرف کنند و تامین‌کنندگان می‌توانند میزان درستی از انرژی را وارد شبکه کنند.
در نهایت با توجه به اهمیت این اطلاعات، نیاز است که امنیت آن‌ها در جهت حفظ حریم خصوصی و صحت این اطلاعات، فراهم گردد
\مرجع{Lin2017}.

\زیرزیرقسمت{شهر هوشمند}

\متن‌سیاه{حمل و نقل هوشمند} که از آن به عنوان سامانه حمل و نقل هوشمند یا اختصارا \متن‌لاتین{ITS} یاد می‌شود، با استفاده از مدیریت هوشمند حمل و نقل، سیستم کنترل، شبکه‌های ارتباطی و تکنیک‌های محاسباتی
سیستم حمل و نقل را قابل اطمینان، امن و کارا می‌نماید.
در یک سیستم حمل و نقل هوشمند، تعداد زیادی خودرو هوشمند وجود دارند که به وسیله شبکه‌های بی‌سیم به یکدیگر متصل شده‌اند. خودروهای هوشمند می‌توانند به صورت کارا اطلاعات ترافیکی را جمع کرده
و به اشتراک بگذارند. از سوی دیگر این خودروها می‌توانند سفر‌های رانندگانشان را با کارایی زیاد، قابلیت اطمینان و ایمن برنامه‌ریزی کنند. در این سال‌ها خودروهای خودران نیز پا به عرصه ظهور گذاشته‌اند
\مرجع{Lin2017}.

ماشین‌های هوشمند در سیستم‌های حمل و نقل هوشمند داری تعدادی واحد کنترلی الکترونیکی یا اختصارا \متن‌لاتین{ECU} هستند. این واحدها زیر سیستم‌های خود را نظارت کرده و آن‌ها را کنترل می‌کنند.
این \متن‌لاتین{ECU}ها در یک شبکه‌ی داخلی قرار گرفته‌اند تا اطلاعات را با خودرو به اشتراک بگذارند. همانطور که پیشتر هم اشاره شد، خودروهای هوشمند دارای رابط‌های ارتباطی هستند که به آن‌ها اجازه می‌دهد
در قالب خودرو به خودرو (\متن‌لاتین{V2V}) یا خودرو به زیرساخت (\متن‌لاتین{V2I}) ارتباط داشته باشند
\مرجع{Lin2017}.

\متن‌سیاه{چراغ‌های ترافیکی} هوشمند به عنوان نودهای مِه می‌توانند آژیرهای یک آمبولانس را تشخیص داده و سیگنال‌های ترافیکی را در جهت باز کردن یک راه برای عبور آن تغییر دهند.
این دستگاه‌ها می‌توانند خودروهایی که به چراغ نزدیک می‌شوند را تشخیص داده و برای نمایش پیام‌های اخطار با آن‌ها همکاری کنند.
حتی این چراغ‌ها می‌توانند در حضور ترافیک فعال بوده و با گذر آن به صورت خودکار غیرفعال شوند.
داده‌ی زیادی که در سیستم حمل و نقل هوشمند تولید می‌شود اگر در یک ابر متمرکز پردازش شود باعث تاخیر زیادی می‌شود.
بنابراین نودهای مِه در چهارراه‌ها می‌تواند در جهت پردازش محلی داده‌ها استفاده شود و مردم را از وضعیت کلی آگاه کند و از سوی دیگر تاخیر را نیز تا حدی کاهش دهد
\مرجع{Angel2021}.

با گسترش مصرف و بزرگتر شدن شهرها یکی از موارد مهم در شهرهای هوشمند مدیریت پسماند است. \متن‌سیاه{مدیریت هوشمند پسماند} در ساده‌ترین سطح شامل نظارت بر میزان پر بودن سطل‌های آشغال
و کمک به مسیریابی بهینه ماشین جمع‌آوری زباله است. بدین ترتیب جلوی خالی کردن زود هنگام سطل‌ها و از سوی دیگر دیرکرد در خالی کردن سطل‌هایی که بیش از اندازه پر شده‌اند گرفته خواهد شد.

در کنار میزان پر بودن می‌توان دما سطل‌ها یا میزان گاز $CO_{2}$ آن‌ها را نیز اندازه‌گیری کرد. از سوی دیگر با استفاده از الگوریتم‌های هوش‌مصنوعی می‌توان فرآیند جمع‌اوری زباله‌ها را بهینه‌سازی نمود.
البته گاز کربن دی اکسید تنها گازی نیست که از سطل‌های عمومی زباله منتشر می‌گردد بلکه گازهای دیگری نیز وجود دارند که می‌توانند باعث بوی بد در سطح شهر شوند.

یکی از مسائل مهم در این حوزه تکنولوژی ارتباطی است، برای ارتباط می‌توان از تکنولوژی‌های سلولی چون \متن‌لاتین{GSM} یا \متن‌لاتین{GPRS} استفاده کرد که البته هزینه‌ی زیادی دارد.
از سوی دیگر گزینه استفاده از \متن‌لاتین{WiFi} با هزینه پایین وجود دارد که البته برد آن نیز بسیار کم خواهد بود.

سطل‌های زباله همواره در معرض آسیب‌های عمدی و غیرعمدی قرار دارند و در صورتی که بتوان با هزینه کمی نسبت به هزینه ساخت آن‌ها از آن‌ها محافظت کرد، این هزینه حتما ارزش خواهد شد.
این امر یکی از محرک‌های بحث سطل‌های هوشمند در مدیریت هوشمند پسماند است.

هوشمند‌سازی شهرهای گردش‌گری تحت عنوان \متن‌سیاه{گردش‌گری هوشمند} می‌تواند به گسترش این صنعت کمک کند. مثلا گردشگران می‌توانند نظر سایرین را در زمان بازدید از یک بنای تاریخی مطالعه کنند و نظر خودشان را به آن مجموعه اضافه کنند
یا حتی می‌توانند ویدیوهای سایرین یا ویدیوهای آموزشی در رابطه با آن بنا را در زمان بازدید مشاهده کنند
\مرجع{Taleb2017}.

\زیرزیرقسمت{ساختمان‌های هوشمند}

به طول میانگین ۸۰ درصد وقت افراد در ساختمان‌های سپری می‌شود. بنابراین همواره راه‌حل‌های جدید برای افزایش
امنیت، کارایی و راحتی ساختمان‌ها مورد نیاز است.
تکنولوژی‌های ارتباط بی‌سیم، اینترنت اشیا و حسگرها قالبا توسط ساختمان‌های هوشمند، برای
\شروع{فقرات}
\فقره ارسال و پردازش داده
\فقره کنترل و بهینه‌سازی بیشتر سامانه‌های مدیریت ساختمان
\پایان{فقرات}
مورد استفاده قرار می‌گیرند.
با این روش مدیران، صاحبان و مستاجران می‌توانند راحتی، امنیت و کارایی بیشتری را با هزینه کمتر بدست آورند.
استفاده از اینترنت اشیا به طرز چشم‌گیری کارایی کلی خانه‌های هوشمند را افزایش می‌دهد و هوشمندی و یکپارچگی
گسترده‌ای به ارمغان می‌آورد
\مرجع{Liang2020}.

جمع‌آوری داده و ارتباطات پایه‌های ساختمان‌های هوشمند هستند که می‌توان آن‌ها را به ارتباطات بی‌سیم و با سیم تقسیم کرد.
از ارتباطات‌های باسیم می‌توان به شبکه‌های \متن‌لاتین{Ethernet}، شبکه‌های \متن‌لاتین{BACnet}، شبکه‌های \متن‌لاتین{RS485}
و شبکه‌های مبتنی بر حامل برق اشاره کرد. با این وجود هزینه زیاد نیرو و ساخت پیچیده شبکه‌های باسیم توسعه ساختمان‌های هوشمند را
کند کرد. در سال‌های اخیر با توسعه سریع تکنولوژی‌های ارتباطی بی‌سیم در اینترنت اشیا مانند \متن‌لاتین{Zigbee}، \متن‌لاتین{WiFi}،
\متن‌لاتین{Bluetooth}، \متن‌لاتین{LoRa} و \نقاط‌خ، این تکنولوژی‌ها به شکل گسترده در ساختمان‌های هوشمند مورد استفاده قرار گرفته‌اند.
استفاده از شبکه‌های بی‌سیم به سادگی قابل گسترش است، که پیچیدگی سیم‌کشی را کاهش داده و باعث انعطاف‌پذیری بیشتر سیستم می‌شود
\مرجع{Liang2020}.

ساختمان‌های هوشمند نقش مهمی در کنترل و نگهداری خودکار دما، رطوبت، تهویه، امنیت، نور و سایر پارامترها در پروسه‌های ساختمانی
دارند
\مرجع{Liang2020}.
از سوی دیگر ساختمان‌ها نقش کلیدی در گرمایش جهانی به عنوان بزرگترین تهدید نسل بشر دارند.
ساختمان‌ها ۳۰ درصد از تولید گازهای گلخانه‌ای، نزدیک به ۴۰ درصد از منابع طبیعی و یک سوم از مصرف جهانی
انرژی را به خود اختصاص داده‌اند
\مرجع{Aliero2022}.

سامانه مدیریت انرژی ساختمان هوشمند یا اختصارا \متن‌لاتین{SBEMS} یک پیشرفت تکنولوژی است
که به وسیله‌ی تکنولوژی‌هایی مانند \متن‌لاتین{Zigbee} اجازه می‌دهد محصولات مختلف اینترنت اشیا
بر روی یک شبکه که به وسیله‌ی یک سیستم‌های خودکار رسیدگی می‌شود ارتباط برقرار کرده و داده به اشتراک بگذارند
\مرجع{Aliero2022}.

یک سامانه \متن‌لاتین{SBEMS} از سیستم هوشمند خنک کننده و تهویه، سیستم هوشمند روشنایی، سیستم دو شاخه هوشمند،
سیستم پنجره‌های هوشمند، بهینه‌سازی هوشمند انرژی با توجه داده‌های آب و هوایی و رفتار ساکنین، و داشبردهایی برای مدیریت تشکیل شده است.
از سوی دیگر در این سیستم‌ها می‌توان از منابع تجدیدپذیر انرژی مانند توربین‌های باد، پنل‌های خورشیدی و \نقاط‌خ نیز بهره برد
\مرجع{Aliero2022}.

بیشتر پژوهش‌ها در حوزه ساختمان‌های هوشمند بر روی کنترل‌کننده‌هایی تمرکز کرده‌اند که نیاز به داده‌ی مشخص از سوی کاربر یا اپراتور ساختمان دارد.
نیاز است که در این حوزه روی الگوریتم‌های یادگیری ماشین وقت بیشتری گذاشته شود.
بیشتر استراتژی‌های مصرف در این حوزه برای ساختمان‌های تجاری ارائه شده‌اند که در آن‌ها زمان‌بندی فعالیت‌ها ثابت و مشخص است که این استراتژی‌ها
در ساختمان‌های مسکونی قابل استفاده نیستند.
به طور مثال سیستم‌های خنک کننده و تهویه می‌بایست تعداد افراد حاضر را تشخیص داده تا بتوانند دمای مطبوعی را بدون اتلاف انرژی فراهم آورند،
از سوی دیگر این سیستم‌ها باید به عوامل محیطی چون آب و هوا، سطح دی‌اکسید کربن و \نقاط‌خ نیز توجه داشته باشند
\مرجع{Aliero2022}.

\زیرقسمت{سیستم‌های سایبر فیزیکال}

اگر بخواهیم خیلی کلی صحبت کنیم، سیستم‌های سایبر فیزیکال یا اختصارا \متن‌لاتین{CPS}، سیستم‌هایی هستند که به صورت بهینه قسمت‌های فیزیکی و سایبری را به وسیله ادغام تکنولوژی‌های ارتباطی و محاسباتی مدرن،
ادغام می‌کنند و هدفشان تغییر روش تعامل انسان، سایبر و محیط فیزیکی است.
\متن‌لاتین{CPS} ترکیب اجزای فیزیکی، سنسورها، عملگرها، شبکه‌های ارتباطی و مراکز کنترل است. سنسورها برای نظارت و اندازه‌گیری وضعیت اجزای فیزیکی مستقر می‌شوند.
عملگرها برای اطمینان از عملیات‌های مورد نظر بر اجزای فیزیکی مستقر می‌شوند. شبکه‌های ارتباطی برای رساندن داده‌های اندازه‌گیری شده و دستورات بازخوردی در میان حسگرها، عملگرها و مراکز کنترلی استفاده می‌شود.
مراکز کنترلی برای آنالیز داده‌های اندازه‌گیری شده و ارسال دستورات بازخوردی به عملگرها و اطمینان از عملکرد کلی سیستم در وضعیت موردنظر استفاده می‌شوند
\مرجع{Lin2017}.

با توجه به تعریفی که از \متن‌لاتین{CPS} ارائه شد، می‌دانیم که سیستم‌های سایبر فیزیکال و اینترنت اشیا هر دو قصد ایجاد تعامل میان دنیای سایبری و فیزیکی را دارند.
با توجه به شباهت‌های میان سیستم‌های سایبر فیزیکال و اینترنت اشیا، نیاز فوری به تفیکیک میان این دو موضوع وجود دارد.
سیستم‌های سایبر فیزیکال به ماهیت سیستم تاکید دارد، در \متن‌لاتین{CPS} لایه حسگر و عملگر در جهت جمع‌اوری داده همزمان و اجرای دستورات،
لایه ارتباط برای رساندن داده‌ها به لایه بالاتر (لایه کاربرد) و رساندن دستورات به لایه پایین‌تر (لایه حسگر و عملگر) و لایه کاربر یا کنترل در جهت پردازش داده‌ها
و تصمیم‌گیری وجود دارند. بنابراین می‌توان گفت معماری \متن‌لاتین{CPS} یک معماری عمودی است.
اما اینترنت اشیا یک معماری مبتنی بر شبکه‌سازی است که قصد دارد تعداد زیادی از دستگاه‌ها را برای نظارت و کنترل به وسیله‌ی تکنولوژی‌های مدرن در دنیای سایبری، به یکدیگر متصل کند.
بنابراین هدف اینترنت اشیا اتصال شبکه‌های مختلف است که به وسیله‌ی آن جمع‌اوری داده، اشتراک منابع، پردازش و مدیریت بتواند در میان شبکه‌های مختلف رخ دهد.
در نهایت می‌توان نتیجه گرفت که اینترنت اشیا یک معماری افقی است که قصد دارد لایه‌های ارتباط از سیستم‌های سایبر فیزیکال گوناگون را برای بدست آوردن ارتباط به یکدیگر متصل کند
\مرجع{Lin2017}.

اگر بخواهیم مثالی از ارتباط میان سیستم‌های سایبر فیزیکال و اینترنت اشیا بزنیم، شهرهای هوشمند مثال خوبی خواهد بود. شهرهای هوشمند از سیستم‌های سایبر فیزیکال متنوعی چون
شبکه برق هوشمند، حمل و نقل هوشمند، سلامت هوشمند و \نقاط‌خ تشکیل شده‌اند. لایه ارتباطی همه این سیستم‌ها به یکدیگر متصل شده‌اند و یک سرویس یکپارچه برای شهر هوشمند را
تشکیل داده‌اند
\مرجع{Lin2017}.

با توجه به آنچه بیان شد، لایه کنترل می‌بایست به گونه‌ای طراحی شود که بتواند از شبکه‌های گوناگون و اشتراکی برای داده‌ها استفاده کند.
در واقع، لایه کنترل در اینترنت اشیا بسیار پیچیده‌تر از اینترنت است و تا به حال به خوبی به آن پرداخته نشده است
\مرجع{Lin2017}.

\زیرقسمت{معماری‌های اینترنت اشیا}

پیش از آنکه بخواهیم به معماری‌های اینترنت اشیا بپردازیم، قصد داریم ویژگی‌های آن را مرور کنیم
\مرجع{Lombardi2021}:
\شروع{فقرات}
\فقره \متن‌سیاه{گسترش‌پذیر}، برای مدیریت تعداد رو به افزایش اشیا و سرویس‌ها بدون کاهش کارایی
\فقره \متن‌سیاه{همکنش‌پذیر}، تا دستگاه‌هایی از سازندگان مختلف بتوانند برای یک هدف مشترک همکاری کنند.
\فقره \متن‌سیاه{توزیع‌پذیر}، تا اجازه دهد یک محیط توزیع‌شده ساخته شود که در آن داده بعد از جمع‌آوری شدن از منابع مختلف، به وسیله موجودیت‌های مختلف و به شکل توزیع شده پردازش شود.
\فقره \متن‌سیاه{فعالیت با منابع کم}، از آنجایی که اشیا توان پردازشی پایینی دارند.
\فقره \متن‌سیاه{امن} تا از دسترسی‌های غیرمجاز جلوگیری کند.
\پایان{فقرات}

در این سال‌های معماری‌های زیادی برای اینترنت اشیا پیشنهاد شده است که بسیاری از آن‌ها در حوزه‌های مشخصی بوده‌اند.
یکی از نگاه‌ها در معماری اینترنت اشیا نگاه \متن‌لاتین{Service Oriented Architecture} یا مختصرا \متن‌لاتین{SOA} است.
در این نگاه از سرویس‌ها استفاده می‌شود و آنچه از \متن‌لاتین{SOA} امروزه در اینترنت اشیا استفاده می‌شود شامل میان‌افزار است، میان‌افزار نرم‌افزاری است که
میان برنامه‌های کاربردی و اشیا قرار گرفته و پیچیدگی‌ها را از دید کاربر پنهان می‌کند. این لایه زمان توسعه را کاهش داده و کمک می‌کند محصول برای بازار
زودتر حاضر شود. معماری‌هایی بر همین اساس برای شبکه‌های سنسور بی‌سیم، کشاورزی هوشمند، ارزیابی کیفیت آب هوشمند، \نقاط‌خ مطرح شده است که در پژوهش \مرجع{Ray2018} به طور خلاصه آمده‌اند
\مرجع{Ray2018}.

در این سال‌های سیستم‌های ابری زیادی برای اینترنت اشیا شکل گرفته‌اند که کارکردهایی مانند مدیریت اشیا، مدیریت سیستم، مدیریت تنوع، مدیریت داده‌ها، پردازش و
نظارت را فراهم می‌آورند. این سرویس‌های ابری عموما دارای \متن‌لاتین{Gateway} برای ارسال داده‌ها به ابر هستند. دو نمونه از چالش‌های مهم در این حوزه ارتباطات یکسان و
شناسه‌ی یکسان برای اشیا در این سیستم‌های ابری است. سازمان \متن‌لاتین{ETSI} در حال تلاش برای استانداردسازی این موارد است
\مرجع{Ray2018}.

یکی از مسائل مطرح در اینرتنت اشیا حجم داده‌ها است و معماری‌هایی برای پردازش و نگهداری این ابرداده‌ها پیشنهاد شده‌اند. این معماری‌ها از سیستم‌هایی چون \متن‌لاتین{Spark} یا \متن‌لاتین{Flink} استفاده می‌کنند.
از مسائل دیگر در معماری اینترنت اشیا می‌توان به \متن‌لاتین{Fog Computing} یا پردازش مِه اشاره کرد
\مرجع{Ray2018}.

اگر بخواهیم به معماری مبتنی بر سرویس مطابق با آنچه در شکل \رجوع{شکل: معماری مبتنی بر سرویس اینترنت اشیا} دقیق‌تر نگاه کنیم، آن را می‌توان در چهار لایه با کارکردهای متفاوت دسته‌بندی کرد:

\شروع{فقرات}
\فقره \متن‌سیاه{لایه حسگر} با اشیا حاضر در هم آمیخته می‌شود تا بتواند وضعیت اشیا را حس کند. در واقع در این لایه سنسورها می‌توانند به صورت خودکار وضعیت محیط را ارزیابی کرده و با سایر اشیا تبادل اطلاعات داشته باشند.
در این لایه هزینه، اندازه، منابع و مصرف انرژی باید بهینه باشد. در این لایه شبکه‌بندی اشیا، ارتباطات و انتخاب پروتکل ارتباطی مناسب، تنوع و نحوه استقرار از سایر دغدغه‌ها است.
\فقره \متن‌سیاه{لایه شبکه} زیرساختی است که ارتباط بی‌سیم یا دارای سیم میان اشیا را پشتیبانی می‌کند. در این لایه اشیا به یکدیگر متصل شده و می‌توانند از اطراف باخبر شوند. این لایه می‌تواند داده‌ها را با زیرساخت فناوری اطلاعات فعلی
ترکیب کرده یا آن‌ها را برای واحد تصمیم‌گیری شامل سرویس‌های سطح بالا و پیچیده ارسال کند. در لایه شبکه دغدغه‌هایی مانند مدیریت شبکه، مصرف انرژی شبکه، نیازمندی‌های کیفیت سرویس، پیدا کردن اشیا و امنیت مطرح است.
\فقره \متن‌سیاه{لایه سرویس}، سرویس‌های لازم برای کاربران یا برنامه‌های کاربردی را مدیریت کرده یا می‌سازد. این لایه از تکنولوژی میان‌افزار استفاده می‌کند، که یک تکنولوژی کلیدی در سرویس‌ها و برنامه‌های کاربردی اینترنت اشیا است.
سرویس‌ها را می‌توان به عنوان یک فعل، شامل جمع‌آوری، جابجایی و ذخیره‌سازی داده یا ترکیب این‌ها برای رسیدن به یک هدف مشخص، در نظر گرفت.
این لایه خود از اجزا زیر تشکیل شده است.
\شروع{فقرات}
\فقره \متن‌سیاه{سرویس اکتشاف} این سرویس اشیایی که اطلاعات یا سرویس مشخصی را فراهم می‌آورند، پیدا می‌کند.
\فقره \متن‌سیاه{ترکیب سرویس‌ها} ارتباط میان اشیا متصل را ممکن می‌سازد. سرویس اکتشاف اشیا را برای یافتن سرویس موردنظر جستجو می‌کند و ترکیب سرویس‌ها با زمان‌بندی و باز ساخت سرویس‌های بیشتر مطمئن‌ترین سرویس را ارائه میدهد.
در واقع در اینترنت اشیا برخی از نیازمندی‌ها می‌توانند با یک سرویس برآورده شوند و برخی برای برآورده شدن به ترکیب سرویس‌ها نیاز دارند.
\فقره \متن‌سیاه{مدیریت ارزشمندی} درک می‌کند که داده‌های هر سرویس می‌بایست چگونه پردازش شوند.
\فقره \متن‌سیاه{رابط‌های کاربری سرویس‌ها} چگونگی ارتباط سرویس‌ها با کاربران و اشیا را مشخص می‌کنند.
\پایان{فقرات}
\فقره \متن‌سیاه{لایه رابط‌ها} شامل روش‌هایی برای ارتباط با کاربران و برنامه‌ها است. همانطور که بیان شد اینترنت اشیا از تعداد زیادی دستگاه متنوع تشکیل شده است و یک رابط کاربری کارا برای ساده‌سازی مدیریت و ارتباط اشیا لازم است.
\پایان{فقرات}

معماری مبتنی بر سرویس یک سیستم پیچیده را در قالب یک مجموعه خوش تعریف از اشیا یا زیر سیستم‌های ساده می‌بیند.
این اجزای به صورت مستقل قابلیت استفاده و تغییر دارند بنابراین اجزای نرم‌افزاری و سخت‌افزاری در اینترنت اشیا می‌توانند به صورت کارایی،
باز استفاده یا به روزرسانی شوند
\مرجع{Li2014}.

میان‌افزار در لایه سرویس، در واقع یک نرم‌افزار است که انتزاعی را میان تکنولوژی‌های اینترنت اشیا و برنامه‌های کاربردی دخیل می‌کند.
در میان‌افزار جزئیات تکنولوژی پنهان می‌شود و رابط‌های استانداری تعریف می‌شود که اجازه می‌دهد برنامه‌نویسان بدون دغدغه هماهنگی میان برنامه‌ها و زیرساخت، بر توسعه برنامه‌ها تمرکز کنند.
بنابراین با استفاده از میان‌افزارها برنامه‌ها و دستگاه‌هایی با رابط‌های متفاوت می‌توانند با یکدیگر تبادل اطلاعات داشته و منابع را به اشتراک بگذارند
\مرجع{Lin2017}.

برای میان‌افزار می‌توان سودهای زیر را در نظر گرفت
\مرجع{Lin2017}:
\شروع{فقرات}
\فقره میان‌افزارها می‌توانند از برنامه‌های کاربردی متنوع پشتیبانی کنند.
\فقره میان افزار می‌تواند روی زیرساخت‌ها و سیستم‌عامل‌های مختلفی اجرا شود.
\فقره میان‌افزار می‌تواند از محاسبات توزیع شده و تعامل میان سرویس‌ها در بین شبکه‌ها، اشیا و برنامه‌های کاربردی گوناگون پشتیبانی کند.
\فقره میان‌افزار می‌تواند از پروتکل‌های استاندارد پشتیبانی کند.
\فقره میان‌افزار می‌تواند رابط‌های استاندارد فراهم آورد، قابلیت انتقال فراهم آورد و می‌تواند پروتکل‌های استاندارد، در جهت همکنش‌پذیری، فراهم آورد که باعث می‌شود میان‌افزار نقش مهمی در استانداردسازی داشته باشد.
\فقره میان‌افزار می‌تواند یک رابط سطح بالا برای برنامه‌های کاربردی فراهم آورد.
\پایان{فقرات}

پژوهش‌های میان‌افزارها را به ۵ دسته کلی تقسیم کرده‌اند
\مرجع{Lin2017}:
\شروع{فقرات}
\فقره \متن‌سیاه{میان‌افزارهای پیام محور}: این میان‌افزارها می‌توانند جابجایی مطمئن اطلاعات میان پلتفرم‌ها و پروتکل‌های ارتباطی را فراهم آورند.
\فقره \متن‌سیاه{میان‌افزارهای مبتنی بر وب معنایی}: این میان‌افزارها می‌توانند تعامل و همکنش‌پذیری را میان شبکه‌های حسگر مختلف فراهم کنند.
\فقره \متن‌سیاه{میان‌افزارهای سرویس موقعیت و نظارتی}: این میان‌افزارها موقعیت اشیا را با سایر اطلاعات ترکیب کرده و از آن برای فراهم آوردن سرویس با ارزش یکپارچه استفاده می‌کنند.
\فقره \متن‌سیاه{میان‌افزارهای ارتباطی}: این میان‌افزارها ارتباطات قابل اطمینان میان اشیا و برنامه‌های کاربردی متنوع فراهم می‌آورند.
\فقره \متن‌سیاه{میان‌افرارهای فراگیر}: این میان‌افزارها برای محیط‌های محاسباتی فراگیر طراحی شده‌اند و می‌توانند روی چندین زیرساخت گوناگون اجرا شوند.
\پایان{فقرات}

\شروع{شکل}
\درج‌تصویر[width=\textwidth]{./img/iot-soa-architecture.png}
\تنظیم‌ازوسط
\شرح{معماری مبتنی بر سرویس اینترنت اشا \مرجع{Li2014}}
\برچسب{شکل: معماری مبتنی بر سرویس اینترنت اشیا}
\پایان{شکل}

یک سرویس می‌تواند به صورت مجموعه‌ای از داده‌ها و رفتارهای وابسته یا قسمتی از یک دستگاه یا ویژگی از یک دستگاه دیده شود که یک کارکرد خاص را صورت می‌دهد.
برای معرفی سرویس‌ها نیاز به یک استاندارد است که در این حوزه استاندادرهای متنوعی پیشنهاد شده است
مثل استفاده از \متن‌لاتین{XML} در استاندارد پیشنهادی از \متن‌لاتین{Special Interest Group} مربوط به \متن‌لاتین{Bluetooth}
\مرجع{Li2014}.

یکی دیگر از معماری‌های مرسوم در اینترنت اشیا، معماری‌های لایه‌ای هستند. ساده‌ترین این معماری‌ها سه لایه \متن‌لاتین{Perception}، \متن‌لاتین{Network} و \متن‌لاتین{Application}
است. از سایر معماری‌های لایه‌ای می‌توان به معماری ۴ لایه‌ای \متن‌لاتین{Things}، \متن‌لاتین{Edge}، \متن‌لاتین{Network} و \متن‌لاتین{Application} و معماری ۵ لایه‌ای
\متن‌لاتین{Business}، \متن‌لاتین{Application}، \متن‌لاتین{Service}، \متن‌لاتین{Object Abstraction} و \متن‌لاتین{Objects} اشاره کرد. در ادامه به توضیح بیشتر این معماری‌ها می‌پردازیم
\مرجع{FerrndezPastor2018}.

همانطور که بیان شد، معماری سه‌لایه‌ای یک معماری ساده و پایه‌ای مشتمل بر لایه‌های \متن‌لاتین{Perception} یا ادراک، \متن‌لاتین{Network} یا شبکه و \متن‌لاتین{Application} یا کاربرد است.
این لایه‌های در ادامه بیشتر بحث خواهند شد.
\شروع{فقرات}
\فقره \متن‌سیاه{لایه ادراک} که با نام لایه‌ی حسگر نیز شناخته می‌شود در پایین این معماری اینترنت اشیا پیاده‌سازی می‌شود. این لایه با اجزا و دستگاه‌های فیزیکی به واسطه دستگاه‌های هوشمند ارتباط برقرار می‌کند.
هدف اصلی این لایه اتصال اشیا به شبکه اینترنت اشیا است تا بتوان اطلاعات وضعیتی این اشیا را به واسطه دستگاه‌های هوشمند مستقر شده، اندازه‌گیری، جمع‌اوری و پردازش کرد و در نهایت داده‌های پردازش شده به
وسیله‌ی رابط لایه، به لایه‌ی بالاتر ارسال شود.
\فقره \متن‌سیاه{لایه شبکه} که با نام لایه انتقال نیز شناخته می‌شود به عنوان یک لایه‌ی میانی در این معماری اینترنت اشیا پیاده‌سازی می‌شود. این لایه داده‌های پردازش شده توسط لایه ادراک را دریافت کرده و
مسیرهایی را برای انتقال این داده‌ها و اطلاعات به دستگاه‌ها، هاب‌ها و برنامه‌های کاربردی توسط شبکه مجتمع مشخص می‌کند. لایه شبکه در اینترنت اشیا لایه‌ی مهمی است چرا که دستگاه‌های زیادی (مانند
هاب‌ها، سوئیچ‌ها، \متن‌لاتین{Gateway}ها، زیرساخت محاسبات ابری و \نقاط‌خ) و تکنولوژی‌های ارتباطی زیادی (مانند \متن‌لاتین{Bluetooth}، \متن‌لاتین{WiFi} و \نقاط‌خ) در این لایه جمع شده‌اند.
\فقره \متن‌سیاه{لایه کاربرد} که با نام لایه کسب و کار هم شناخته می‌شود در بالای این معماری اینترنت اشیا پیاده‌سازی شده است. این لایه با دریافت داده‌های ارسالی از لایه‌ی شبکه از آن‌ها برای فراهم آوردن
سرویس‌ها و عملیات‌های لازم استفاده می‌کند.
\پایان{فقرات}

با وجود سادگی معماری سه لایه، کارکردهای لایه‌ی شبکه و کاربرد بسیار متنوع و گسترده است. به صورت مثال لایه شبکه در کنار مسیریابی می‌تواند سرویس‌هایی برای تجمبع‌داده‌ها ارائه دهد یا لایه کاربرد
می‌تواند سرویس‌هایی برای آنالیز داده‌ها ارائه کند. بنابراین لایه سرویس میان لایه‌ی کاربرد و شبکه ایجاد می‌شود که باعث معماری مبتنی بر سرویس خواهد که پیشتر به آن پرداختیم
\مرجع{Lin2017}.

\زیرقسمت{اشیا هوشمند}

اشیا هوشمند پایه‌هایی هستند که اینترنت اشیا بر اساس آن‌ها بنا شده است. این اشیا می‌توانند، اشیایی با استفاده روزمره (مانند یخچال، تلویزیون، خودرو و \نقاط‌خ) یا دستگاه‌های ساده‌ای مجهز به
سنسورها و قابلیت‌های پردازشی باشند. به صورت کلی اشیا هوشمند ویژگی‌های اساسی زیر را دارند
\مرجع{Lombardi2021}:
\شروع{فقرات}
\فقره \متن‌سیاه{ارتباطات}: اشیا می‌توانند به یکدیگر و به منابعی در اینترنت در جهت استفاده از داده‌ها و سرویس‌ها، به روزرسانی وضعیتشان و همکاری برای رسیدن به یک هدف مشترک متصل شوند.
\فقره \متن‌سیاه{شناسایی}: اشیا می‌بایست به صورت یکتا قابل شناسایی باشند.
\پایان{فقرات}
با توجه به یک کاربرد خاص، یک یا چند مورد از صفاتی که در ادامه می‌آیند می‌توانند اضافه شوند:
\شروع{فقرات}
\فقره \متن‌سیاه{آدرس‌پذیری}: اشیا می‌توانند به صورت مستقیم قابل دسترس یا آدرس‌دهی باشند تا بتوانند از دور تنظیم یا تحقیق شوند.
\فقره \متن‌سیاه{حسگری و عملگری}: اشیا می‌توانند به واسطه حسگرها و عملگرها از محیط اطراف خود اطلاعات جمع‌آوری کرده و آن را تغییر دهند.
\فقره \متن‌سیاه{پردازش نهفته اطلاعات}: اشیا هوشمند می‌توانند توانایی‌های محاسباتی داشته باشند تا بتوانند نتایج حاصل از سنسورها را پردازش کرده یا عملگرها را کنترل نمایند.
\فقره \متن‌سیاه{مکان‌یابی}: اشیا می‌توانند از محل فیزیکی خود باخبر باشند یا مکان‌یابی شوند.
\فقره \متن‌سیاه{رابط کاربری}: اشیا می‌توانند برای ارتباط مناسب با کاربران از صفحه‌های نمایش یا سایر رابط‌ها استفاده کنند.
\پایان{فقرات}

پلتفرم‌های سخت‌افزاری زیادی در بازار با این ویژگی‌ها موجود هستند که از میان آن‌ها می‌توان به \متن‌لاتین{Raspberry Pi}، \متن‌لاتین{Arduino}، \متن‌لاتین{Beaglebone Black} و \نقاط‌خ اشاره کرد
\مرجع{Lombardi2021}.
