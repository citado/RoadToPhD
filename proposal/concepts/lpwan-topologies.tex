\قسمت{همبندی‌های شبکه‌های \متن‌لاتین{LPWAN}}

به صورت کلی دو همبندی در این شبکه‌ها مطرح است. همبندی \متن‌لاتین{Mesh} یا توری
و همبندی \متن‌لاتین{Start} یا ستاره.
در همبندی \متن‌لاتین{Mesh} همه نودها به یکدیگر متصل هستند از این رو زمانی که هدف
افزایش پوشش‌دهی و کاهش توان مصرفی است این همبندی ترجیح داده نمی‌شود. همبندی
\متن‌لاتین{Star} همبندی انتخابی در شبکه‌های \متن‌لاتین{LPWAN} است.
در این همبندی با استفاده از یک نود می‌توان به تعداد زیادی نود دسترسی داشت که
خود باعث کاهش هزینه است
\مرجع{Chaudhari2020}.

در شبکه‌های \متن‌لاتین{Star} نود به صورت مستقیم با یک یا چند \متن‌لاتین{Gateway} در ارتباط هستند
و با سایر نودها ارتباطی ندارند. در این شبکه‌ها اطلاعات توسط \متن‌لاتین{Gateway} برای لایه‌های بالاتر ارسال می‌شود.
خود \متن‌لاتین{Gateway}ها ارتباطی با یکدیگر ندارند و از این رو این شبکه‌ها بسیار از شبکه‌های \متن‌لاتین{Mesh} ساده‌تر هستند.
در این شبکه‌ها نودها دارای ایراد به سادگی شناسایی می‌شوند ولی در صورت ایراد \متن‌لاتین{Gateway} تمام نودهای
متصل به آن دست می‌روند
\مرجع{Chaudhari2020}.

در شبکه‌های \متن‌لاتین{Mesh} کامل همه نودها به یکدیگر متصل هستند ولی در شبکه‌های \متن‌لاتین{Mesh} جزئی یا \متن‌لاتین{Partial Mesh} نودها
همه با یکدیگر ارتباط ندارند و با نودهایی که بیشترین پیام را رد و بدل کرده‌اند در ارتباط هستند.
مزیت این شبکه‌ها در وجود چندین راه ارتباطی بین نودها، امکان ارسال و دریافت همزمان داده‌ها از مسیرهای متفاوت،
گسترش‌پذیری آسان و قابلیت بهبود خودکار است. از سوی دیگر معایب آن شامل افزایش تاخیر به خاطر مسیرهایی با چند گام و
افزایش هزینه و پیچیدگی است
\مرجع{Chaudhari2020}.
