\قسمت{ارتباطات و شبکه‌ها}

ارتباطات و شبکه بخش مهمی از بلاک‌های کارکردی اینترنت اشیا را تشکیل می‌دهند.
این پروتکل‌ها با توجه به گستردگی اشیا و تنوع قابلیت‌های آن‌ها، بسیار متنوع شده‌اند و البته نباید نیازمندی‌های کیفیت سرویس متنوع برای اشیا را نیز از یاد برد.
قطعا پروتکل \متن‌لاتین{IPv6} نقش بسیار مهمی در اینتنرت اشیا خواهد شد و بسیاری از پروتکل‌ها تلاش برای استفاده از آن در محیط‌هایی با محدودیت‌های گوناگون دارند.
این فناوری‌های ارتباطی می‌توانند در سه گروه کلی طبقه‌بندی شوند که در ادامه به همراه مثال‌هایی آورده شده‌اند:

\شروع{فقرات}
\فقره \متن‌سیاه{\متن‌لاتین{Session/Application}}: \متن‌لاتین{MQTT}، \متن‌لاتین{CoAP}، \متن‌لاتین{AMQP}، \متن‌لاتین{HTTP}، \متن‌لاتین{SOAP}
\فقره \متن‌سیاه{\متن‌لاتین{Network}}: \متن‌لاتین{6LowPAN}، \متن‌لاتین{RPL}، \متن‌لاتین{IPsec}، \متن‌لاتین{TCP/UDP}، \متن‌لاتین{DTLS}، \متن‌لاتین{CORPL}
\فقره \متن‌سیاه{\متن‌لاتین{Perception/Things}}: \متن‌لاتین{WiFi}، \متن‌لاتین{Bluetooth Low Energy}، \متن‌لاتین{Z-Wave}، \متن‌لاتین{ZigBee}، \متن‌لاتین{LoRaWAN}، \متن‌لاتین{LTE}
\پایان{فقرات}

به صورت کلی می‌توان فناوری‌های ارتباطی لایه فیزیکی را در دسته‌های برد کوتاه، متوسط و بلند دسته‌بندی کرد.
از دیگر پارامترهای مهم برای این دسته از پروتکل‌ها نرخ انتقال داده‌ و توان مصرفی است.
به صورت کلی اگر شبکه‌هایی با توان مصرفی پایین را
در نظر بگیریم، می‌توانیم آن را به دو دسته تقسیم کنیم \مرجع{Augustin2016}\مرجع{Li2014}:

\شروع{فقرات}
\فقره شبکه‌های محلی توان پایین که عموما بردشان زیر یک کیلومتر است. از جمله این شبکه‌ها می‌توان به \متن‌لاتین{IEEE 802.15.4}، \متن‌لاتین{IEEE 802.11ah}، \متن‌لاتین{Bluetooth} و \متن‌لاتین{BLE} اشاره کرد.
در شبکه‌های محلی توان پایین، شبکه‌هایی مانند \متن‌لاتین{Zigbee} وجود دارند که بر پایه \متن‌لاتین{IEEE 802.15.4} بوده اما لایه شبکه را نیز افزون بر لایه‌های پیوند داده و فیزیکی دارا هستند. شبکه‌های \متن‌لاتین{Zigbee}
از همبندی‌های متنوعی پشتیبانی می‌کنند و حتی الگوریتم‌های مسیریابی نیز برای آن‌ها وجود دارد.
این شبکه‌ها عموما از همبندی \متن‌لاتین{Mesh} استفاده می‌کنند و با این شیوه می‌توان پوشش آن‌ها را گسترش داد.

\فقره شبکه‌های گسترده توان پایین که عموما بردشان بالای یک کیلومتر است و تحت قالب \متن‌لاتین{LPWAN} به آن‌ها می‌پردازیم.
فناوری‌های بسیاری در حوزه \متن‌لاتین{LPWAN} به بازار عرضه شده‌اند که از جمله‌ی آن‌ها می‌توان به \متن‌لاتین{SigFox}، \متن‌لاتین{NB-IoT}، \متن‌لاتین{LTE Cat-M} و \متن‌لاتین{LoRaWAN}
اشاره کرد.
\پایان{فقرات}

\شروع{شکل}
\تنظیم‌ازوسط
\درج‌تصویر[height=.5\textwidth]{img/wireless-tech-cov-thr.png}
\شرح{مقایسه فناوری‌های ارتباط بی‌سیم از نظر گذردهی و برد \مرجع{Lee2017}}
\پایان{شکل}
