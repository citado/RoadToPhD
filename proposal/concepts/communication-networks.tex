\قسمت{ارتباطات و شبکه‌ها}

ارتباطات و شبکه بخش مهمی از بلاک‌های کارکردی اینترنت اشیا را تشکیل می‌دهند.
این پروتکل‌ها با توجه به گستردگی اشیا و تنوع قابلیت‌های آن‌ها، بسیار متنوع شده‌اند و البته نباید نیازمندی‌های کیفیت سرویس متنوع برای اشیا را نیز از یاد برد.
قطعا پروتکل \متن‌لاتین{IPv6} نقش بسیار مهمی در اینتنرت اشیا خواهد شد و بسیاری از پروتکل‌ها تلاش برای استفاده از آن در محیط‌هایی با محدودیت‌های گوناگون دارند.
این فناوری‌های ارتباطی می‌توانند در سه گروه کلی طبقه‌بندی شوند که در ادامه به همراه مثال‌هایی آورده شده‌اند:

\شروع{فقرات}
\فقره \متن‌سیاه{\متن‌لاتین{Session/Application}}: \متن‌لاتین{MQTT}، \متن‌لاتین{CoAP}، \متن‌لاتین{AMQP}، \متن‌لاتین{HTTP}، \متن‌لاتین{SOAP} و \نقاط‌خ
\فقره \متن‌سیاه{\متن‌لاتین{Network}}: \متن‌لاتین{6LowPAN}، \متن‌لاتین{RPL}، \متن‌لاتین{IPsec}، \متن‌لاتین{TCP/UDP}، \متن‌لاتین{DTLS}، \متن‌لاتین{CORPL} و \نقاط‌خ
\فقره \متن‌سیاه{\متن‌لاتین{Perception/Things}}: \متن‌لاتین{WiFi}، \متن‌لاتین{Bluetooth Low Energy}، \متن‌لاتین{Z-Wave}، \متن‌لاتین{ZigBee}، \متن‌لاتین{LoRaWAN}، \متن‌لاتین{LTE} و \نقاط‌خ
\پایان{فقرات}

به صورت کلی می‌توان این فناوری‌های ارتباطی لایه فیزیکی را در دسته‌های برد کوتاه، متوسط و بلند دسته‌بندی کرد.
از سوی دیگر یکی از پارامترهای مهم برای این دسته از پروتکل‌ها نرخ داده‌ای است. یکی دیگر از پارامترهای مهم در شبکه‌های اینترنت اشیا توان مصرفی شبکه است، به صورت کلی اگر شبکه‌های توان پایین را
در نظر بگیریم، می‌توانیم آن را به دو دسته تقسیم کنیم \مرجع{Augustin2016}\مرجع{Li2014}:

\شروع{فقرات}
\فقره شبکه‌های محلی توان پایین که عموما بردشان زیر یک کیلومتر است. این دسته از شبکه‌ها شامل \متن‌لاتین{IEEE 802.15.4}، \متن‌لاتین{IEEE 802.11ah}، \متن‌لاتین{Bluetooth}، \متن‌لاتین{BLE} و \نقاط‌خ است.
این شبکه‌ها عموما از همبندی \متن‌لاتین{Mesh} استفاده می‌کنند و با این شیوه می‌توان پوشش آن‌ها را گسترش داد.

\فقره شبکه‌های گسترده توان پایین که عموما بردشان بالای یک کیلومتر است و تحت قالب \متن‌لاتین{LPWAN} به آن‌ها پرداختیم.
\پایان{فقرات}

\شروع{شکل}
\تنظیم‌ازوسط
\درج‌تصویر[height=.5\textwidth]{img/wireless-tech-cov-thr.png}
\شرح{مقایسه فناوری‌های ارتباط بی‌سیم از نظر گذردهی و برد \مرجع{Lee2017}}
\پایان{شکل}

پروتکل‌هایی مانند \متن‌لاتین{ModBus} برای اتوماسیون صنعتی، \متن‌لاتین{KNX} برای هوشمند‌سازی ساختمان‌ها و \متن‌لاتین{Wireless M-Bus} برای اندازه‌گیری آب و گاز مصرفی از سال‌های پیش وجود داشته‌اند
و برای یک کاربردهای خاص تعریف شده‌اند.
در شبکه‌های محلی توان پایین، شبکه‌هایی مانند \متن‌لاتین{Zigbee} وجود دارند که بر پایه \متن‌لاتین{IEEE 802.15.4} بوده اما لایه شبکه را نیز افزون بر لایه‌های پیوند داده و فیزیکی دارا هستند. شبکه‌های \متن‌لاتین{Zigbee}
از همبندی‌های متنوعی پشتیبانی می‌کنند و حتی الگوریتم‌های مسیریابی نیز برای آن‌ها وجود دارد.
پیشتر به معرفی \متن‌لاتین{LPWAN} پرداختیم، فناوری‌های بسیاری در حوزه \متن‌لاتین{LPWAN} به بازار عرضه شده‌اند که از جمله‌ی آن‌ها می‌توان به \متن‌لاتین{SigFox}، \متن‌لاتین{NB-IoT}، \متن‌لاتین{LTE Cat-M} و \متن‌لاتین{LoRaWAN}
اشاره کرد.

\متن‌لاتین{SigFox} قصد دارد یک پوشش جهانی را در قالب یک اپراتور شبکه که در کشورهای مختلف با استفاده از شرکت‌های تابعه اجرا می‌شود، فراهم آورد.
این شبکه به صورت کامل از سخت‌افزار تا لایه شبکه در انحصار همین شرکت است و همکاری با آن تنها راه برای عملیاتی کردن این شبکه است.
\متن‌لاتین{NB-IoT} توسط شرکت‌های مخابراطی به عنوان یک جایگزین در ارتباطات اینترنت اشیا، نسبت به فناوری‌های زیرگیگاهرتز \متن‌لاتین{LPWAN} ارائه می‌شود.
از آنجایی \متن‌لاتین{NB-IoT} در طیف فرکانسی دارای لایسنس فعالیت می‌کند، می‌تواند قابلیت اطمینان بیشتری در ترافیک نسبت به سایر فناوری‌های زیرگیگاهرتز ارائه دهد.
برخلاف \متن‌لاتین{SigFox} و \متن‌لاتین{NB-IoT}، \متن‌لاتین{LoRaWAN} قابلیت ارائه به صورت شبکه‌های خصوصی و ادغام آسان با پلتفرم‌های شبکه‌ای جهانی مانند \متن‌لاتین{The Things Network} را فراهم می‌آورد.
به همین دلیل و از سوی دیگر باز بودن استاندارد، \متن‌لاتین{LoRaWAN} توجه جامعه محققان را از اولین نمود خود در بازار جلب کرده است
\مرجع{sensors-18-03995}
\مرجع{Mekki2019}.

در جدول \رجوع{جدول: مقایسه فناوری‌های LPWAN} فناوری‌های مطرح \متن‌لاتین{LPWAN} در معیارهای مختلف مقایسه شده‌اند. این مقایسه نشان می‌دهد فناوری‌های \متن‌لاتین{LoRaWAN}
و \متن‌لاتین{Sigfox} نسبت به سایر فناوری‌ها در طول عمر دستگاه، ظرفیت شبکه، نرخ داده تطبیق‌پذیر و هزینه برتری دارند \مرجع{Almuhaya2022}.

\begin{table}
\caption{مقایسه فناوری‌های \متن‌لاتین{LPWAN} \مرجع{SanchezIborra2016} \مرجع{Mekki2019} \مرجع{Naik2018} \مرجع{Almuhaya2022}}
\label{جدول: مقایسه فناوری‌های LPWAN}
\begin{latin}\begin{tabularx}
  {\textwidth}
  {|*{6}{X|}}
  \toprule

  &
  LoRaWAN &
  Sigfox &
  NB-IoT &
  Ingenu &
  Telensa \\

  \midrule

  Band &
  Sub-GHz ISM &
  Sub-GHz ISM &
  Licensed &
  2.4 GHz ISM &
  Sub-GHz ISM \\

  \midrule

  Data Rate (uplink) &
  50 (FSK) 0.3--37.5 (LoRa) kbps &
  100 bps &
  64 kbps &
  624 kbps &
  62.5 bps \\

  \midrule

  Range &
  5 km &
  10 km &
  35 km &
  15 km &
  1 km \\

  \midrule

  Number of Channels &
  8 &
  360 &
  --- &
  40 &
  --- \\

  \midrule

  MAC &
  ALOHA &
  none &
  Non-Access Stratum &
  --- &
  --- \\

  \midrule

  Topology &
  Star-of-Stars &
  Star &
  Star &
  Star / Tree &
  Star / Tree \\

  \midrule

  Adaptive Data Rate &
  Yes &
  No &
  No &
  Yes &
  No \\

  \midrule

  Payload Length &
  256 B &
  12 B &
  1600 B &
  10 kB &
  65 kB \\

  \midrule

  Handover &
  No &
  No &
  Yes &
  Yes &
  --- \\

  \midrule

  Authentication / Encryption &
  AES 128 &
  No &
  LTE Encryption &
  --- &
  --- \\

  \midrule

  Over the air update &
  --- &
  --- &
  --- &
  --- &
  --- \\

  \midrule

  Battery life &
  10Y+ &
  10Y+ &
  --- &
  --- &
  10Y+ \\

  \midrule

  Bi-Directional &
  Yes &
  Yes &
  Yes &
  Yes &
  Yes \\

  \bottomrule
\end{tabularx}\end{latin}
\end{table}

شبکه‌های \متن‌لاتین{Sigfox} پهنای باند بسیار کمی داشته و محدودیت‌های زیادی برای اندازه بسته و تعداد بسته‌ها در نظر گرفته است.
برای \متن‌لاتین{uplink} در این شبکه‌ها در روز ۱۴۰ بسته و برای \متن‌لاتین{downlink} در روز تنها ۴ بسته ظرفیت وجود دارد.
با توجه به مدل تجاری خاص آن که پیشتر
به آن پرداخته شد، توجه‌ها بیشتر به سوی \متن‌لاتین{LoRaWAN} معطوف شده است
\مرجع{Adelantado2017}.
نرخ ارسال داده در این شبکه‌ها ۱۰۰ بیت بر ثانیه بوده و بسته‌ها می‌توانند حداکثر ۱۲ بایتی باشند
که البته این محدودیت‌ها به طول عمر بالای باتری و پوشش‌دهی گسترده منجر شده است
\مرجع{SanchezIborra2016}.

\متن‌لاتین{Dash7} یک استاندارد باز ارائه شده توسط \متن‌لاتین{Dash7 Alliance} است.
برخلاف فناوری‌های مطرح دیگر در این حوزه، \متن‌لاتین{Dash7} از همبندی درختی دو گامی استفاده می‌کند
که به معماری شبکه‌های سنتی \متن‌لاتین{WSN} شباهت دارد.
برتری اصلی این پروتکل پوشش‌دهی بیشتر در مقایسه با راهکارهای \متن‌لاتین{WSN} به خاطر استفاده از باندهای
زیرگیگاهرتزی و امکان ارتباط مسقتیم میان اشیا است
\مرجع{SanchezIborra2016}.

مستقل از ارتباط رادیویی که برای شکل دادن شبکه‌ی \متن‌لاتین{M2M} از آن استفاده شده است، دستگاه انتهایی یا ماشین می‌بایست داده خود را از طریق اینرتنت قابل دسترسی کنند.
دستگاه اینترنت اشیا عموما منابع محدودی دارند و این به آن معناست که باید با حافظه، توان پردازشی، توان شبکه‌ای و باتری محدودی فعالیت کنند.
بنابراین کارایی ارتباط ماشین به ماشین وابستگی زیادی به پروتکل زیرین مورد استفاده در اپلیکشن اینرتنت اشیا دارد
\مرجع{Mishra2021}.

پروتکل‌های ارتباطی زیادی در لایه شبکه و کاربرد اینترنت اشیا مطرح است که می‌توان از بین آن‌ها به \متن‌لاتین{MQTT}، \متن‌لاتین{CoAP}، \متن‌لاتین{AMQP} و \متن‌لاتین{HTTP} اشاره کرد
\مرجع{Mishra2021}. در ادامه این رساله به این پروتکل‌ها نیز پرداخته خواهد شد.

\زیرقسمت{\متن‌لاتین{NB-IoT}}

همانطور که اشاره شد شبکه \متن‌لاتین{NB-IoT} از باند دارای لایسنس استفاده می‌کند و همین موضوع باعث می‌شود که مساله ازدحام حل شود و سرویس‌های
قابل اطمینان بیشتری برای برنامه‌های حساس به وجود بیاید.
فرکانس رادیویی این شبکه از \متن‌لاتین{Narrowband}های ۱۸۰ کیلوهرتزی تشکیل شده است که برابر با یک بلاک ریسورس فیزیکی یا \متن‌لاتین{PRB} در شبکه \متن‌لاتین{LTE} است.
سه مد عملیاتی برای شبکه‌های \متن‌لاتین{NB-IoT} وجود دارد.
در مد اول یا \متن‌لاتین{Stand-Alone} از طیف فرکانسی اختصاصی برای \متن‌لاتین{NB-IoT} استفاده می‌شود، که این طیف می‌تواند از باندهای فرکانسی \متن‌لاتین{GSM} باشد.
در مد دوم یا \متن‌لاتین{Guard-Band} از منابع استفاده نشده در باند محافظ \متن‌لاتین{LTE} استفاده می‌شود.
در مد سوم یا \متن‌لاتین{In-Band} از ۲۰۰ کیلوهرتز پهنای باند مربوط به \متن‌لاتین{LTE} استفاده می‌شود. در این شیوه نیاز به طیف بیشتری نیاز نیست اما نگاشت منابع می‌بایست به گونه‌ای باشد
که عمود بودن با شبکه فعلی را تضمین کند و ممکن است پهنای باند \متن‌لاتین{LTE} کاهش پیدا کند
\مرجع{Mekki2019}
\مرجع{Lee2017}.

لایه‌ی کنترلی شبکه‌های سلولی حاضر برای سرویس‌هایی با منابع بالا مانند سرویس دسترسی به اینترنت، چندرسانه‌ای و صوتی
طراحی شده‌اند.
لایه‌ی کنترلی شبکه‌های \متن‌لاتین{LTE} پیش از ارسال پیام نیاز به جابجایی ۱۱ پیام دارد، این امر در شبکه‌های \متن‌لاتین{NB-IoT}
بهینه شده است و لایه کنترلی این شبکه‌ها تنها نیاز به ۴ پیام دارد. این امر در شکل \رجوع{شکل: مقایسه شمای ارسال در LTE و NB-IoT}
نمایش داده شده است
\مرجع{Lee2017}.

\شروع{شکل}
\تنظیم‌ازوسط
\درج‌تصویر[width=\textwidth]{img/nbiot-deployment-scenarios.png}
\شرح{سناریوهای استقرار شبکه \متن‌لاتین{NB-IoT} \مرجع{Lee2017}}
\پایان{شکل}

\شروع{شکل}
\تنظیم‌ازوسط
\درج‌تصویر[width=\textwidth]{img/cp-nb-iot-vs-lte.png}
\شرح{مقایسه شِمای ارسال در \متن‌لاتین{LTE} \متن‌لاتین{(a)} و \متن‌لاتین{NB-IoT} \متن‌لاتین{(b)} \مرجع{Lee2017}}
\برچسب{شکل: مقایسه شمای ارسال در LTE و NB-IoT}
\پایان{شکل}

شبکه \متن‌لاتین{NB-IoT} در قیاس با سایر شبکه‌های \متن‌لاتین{3GPP} بیشینه‌ی نرخ داده‌ی کمتری داشته،
پوشش آن بیشتر بوده و پیچیدگی سخت‌افزاری آن کاهش پیدا کرده است.
بنابراین \متن‌لاتین{NB-IoT} می‌تواند هزینه و انرژی مصرفی را کاهش دهد که دو مساله اصلی در استفاده از فناوری‌های
شبکه سلولی در دستگاه‌های اینترنت اشیا است
\مرجع{Lee2017}.

گرهها در شبکه‌ی \متن‌لاتین{NB-IoT} می‌توانند دو حالت \متن‌لاتین{eDRX} و \متن‌لاتین{PSM} را برای صرفه‌جویی انتخاب کنند.
در مد \متن‌لاتین{eDRX} دستگاه برای مدت تا ۱۷۵ دقیقه مودم خود را خاموش می‌کند.
در \متن‌لاتین{DRX} که پیشتر هم در شبکه‌های سلولی وجود داشته است همین رویه برای بازه‌ی کوتاه $2.56$ ثانیه خاموش می‌شده است
و تفاوت \متن‌لاتین{eDRX} در همین مدت زمان است.
در نظر داشته باشید که مساله زمان از این جهت مطرح است که در صورت خاموش بودن مودم پاسخ \متن‌لاتین{downlink} با تاخیر مواجه می‌شود.
در روش \متن‌لاتین{PSM} مودم برای مدتهای طولانی مانند چندین ماه خاموش می‌شود.
این حالت برای سنسورهایی که صرفا در شرایط مشخصی \متن‌لاتین{uplink} دارند، کاربرد دارد
\مرجع{Lee2017}.

\زیرقسمت{\متن‌لاتین{LoRa}}

لایه‌ی فیزیکی \متن‌لاتین{LoRa} که در \متن‌لاتین{LoRaWAN} استفاده می‌شود، در سال ۲۰۱۴ توسط \متن‌لاتین{Semtech} به ثبت رسید
و بنابراین برای بررسی‌ها کاملا باز نیست. مطالبی که در ادامه می‌آید بخشی بر اساس قسمت‌های باز استاندارد و بخشی بر اساس آزمایش‌های
تجربی بدست آمده‌اند.
از ویژگی‌های \متن‌لاتین{LoRa} می‌توان به توان عملیاتی پایین، نرخ پایین داده و برد ارتباطی بالا اشاره کرد
\مرجع{sensors-18-03995}
\مرجع{Adelantado2017}.

از سال ۲۰۱۵ جامعه تحقیقاتی شروع به مطالعه در رابطه با کارآیی و ویژگی‌های مختلف فناوری‌های \متن‌لاتین{LoRa} و \متن‌لاتین{LoRaWAN} کرد.
از آن تاریخ مقلات متعددی در ژورنال‌ها و کنفرانس‌های عملی در سراسر دنیا چاپ و ارائه شده‌اند
\مرجع{sensors-18-03995}.

ماژولیشن \متن‌لاتین{LoRa} بر پایه \متن‌لاتین{Chirp Spread Spectrum} بوده و به صورت دوره‌ای سیگنال‌های \متن‌لاتین{chirp}ای تولید می‌کنند که همه آن‌ها بازه زمانی یکسانی دارند.
\متن‌لاتین{chirp} یک سیگنال سینوسی است که فرکانس آن با زمان به صورت خطی افزایش یا کاهش پیدا می‌کند.
یک \متن‌لاتین{chirp} به وسیله‌ی پروفایل زمانی فرکانس لحظه‌ی آن که در بازه‌ی زمانی \متن‌لاتین{T} از فرکانس $f_0$ به فرکانس $f_1$
تغییر می‌کند، شناخته می‌شود.
در \متن‌لاتین{LoRa} دو نوع \متن‌لاتین{chirp} تعریف شده است. \متن‌لاتین{chirp} بالا که فرکانس پروفایل زمانی آن با فرکانس مینیمال
\(f_{\min} = -\frac{BW}{2}\)
شروع شده و با فرکانس ماکسیمال
\(f_{\max} = \frac{BW}{2}\)
خاتمه می‌یابد و \متن‌لاتین{chirp} پایین که فرکانس آن پروفایل زمانی آن با فرکانس ماکسیمال
شروع شده و با فرکانس مینیمال خاتمه می‌یابد.
برای ورودی‌های دیجیتال مختلف، یک ماژولاتور \متن‌لاتین{chirp}های مختلفی تولید می‌کند که نسبت به \متن‌لاتین{chirp} پایه شیف فرکانسی خورده‌اند
\مرجع{sensors-18-03995} \مرجع{Kufakunesu2020}.
برای هر فاکتور گسترش \متن‌لاتین{SF} به اندازه‌ی $2^{SF}$، \متن‌لاتین{chirp} وجود دارد که با فرکانس‌های مختلفی آغاز می‌شوند.
در ماژولیشن \متن‌لاتین{LoRa}، نرخ \متن‌لاتین{chirp} بر ثانیه برابر با پهنای باند است.

\شروع{شکل}
\درج‌تصویر[width=\textwidth]{./img/lora-mod.png}
\تنظیم‌ازوسط
\شرح{ماژولیشن \متن‌لاتین{LoRa}}
\پایان{شکل}

\شروع{شکل}
\درج‌تصویر[width=\textwidth]{./img/lora-chirp-sf.png}
\تنظیم‌ازوسط
\شرح{\متن‌لاتین{chirp}های پایه}
\پایان{شکل}

\متن‌لاتین{LoRa} از باند فرکانسی بدون مجوز استفاده می‌کند بنابراین برای راه‌اندازی شبکه‌ی آن نیاز به تهیه هیچ مجوزی نیست. البته باید در نظر داشته که نرخ پیام در این باندهای بدون مجوز توسط قانون‌گذاران محدود شده است.
یکی از محدودیت‌های مهم در شبکه‌های \متن‌لاتین{LoRa} محدودیت \متن‌لاتین{Duty Cycle} است که استفاده از کانال را محدود می‌کند. این محدودیت بیان می‌کند برای استفاده از کانال به مدت $T_{a}$ شی می‌بایست
حداقل به اندازه $T_{s}$ که از رابطه \رجوع{معادله: چرخه وظیفه} بدست می‌آید، ارسالی نداشته باشد
\مرجع{Cruz2021}
\مرجع{Adelantado2017}.

\begin{align}
  \label{معادله: چرخه وظیفه}
  T_{s} = T_{a}\left( \frac{1}{d} - 1 \right)
\end{align}

لایه فیزیکی \متن‌لاتین{LoRa} با توجه به ویژگی‌های گسترده‌ای که دارد در راهکارهای دیگری به جز \متن‌لاتین{LoRaWAN} نیز استفاده شده است که از جمله‌ی آن می‌توان به \متن‌لاتین{Meshed LoRa} اشاره کرد
\مرجع{Beltramelli2021}.

پارامترهای فاکتور گسترش یا به اختصار \متن‌لاتین{SF}، پهنای باند و نرخ‌کدگذاری قابل تنظیم هستند و می‌توانند روی زمان ارسال بسته، نرخ ارسال، مصرف انرژی و برد ارتباطی تاثیر داشته باشند.
در ادامه به مرور این پارامترها و تاثیرشان می‌پردازیم.

به صورت غیر رسمی فاکتور گسترش لگاریتم مبنای ۲ تعداد \متن‌لاتین{chirp}ها در هر علامت است. مقدار فاکتور گسترش بین ۷ تا ۱۲ است.
با افزایش فاکتور گسترش پوشش‌دهی بیشتر می‌شود اما بهای آن کاهش نرخ بیت و افزایش زمان ارسال\پانویس{Time on Air} است (معادله \رجوع{معادله: زمان علامت در LoRa})
\مرجع{Augustin2016}.

در بسته‌های \متن‌لاتین{LoRa} از تصحیح خطا جلورونده یا مختصرا \متن‌لاتین{FEC} استفاده می‌شود.
در این فرآیند بیت‌های تصحیح خطا به داده‌های ارسال اضافه می‌شوند.
این بیت‌های اضافه شده کمک می‌کنند تا داده‌های از دست رفته به خاطر تداخل بازگردانی شوند.
بیت‌های بیشتر این پروسه بازگردانی را ساده‌تر می‌کنند اما باعث هدر رفت پهنای باند و عمر باتری می‌شوند.
در \متن‌لاتین{LoRa} ما نرخ‌های کدگذاری $4/5$، $4/6$، $4/7$ و $4/8$ را داریم.

\begin{table}
\caption{توانایی \متن‌لاتین{LoRa} در تشخیص و تصحیح خطا \مرجع{Pham2020}}
\begin{latin}\begin{tabularx}
  {\textwidth}
  {|*{3}{X|}}
  \toprule
  Coding rates &
  Error detection (bits) &
  Error correction (bits) \\
  \midrule
  $4/5$ &
  0 &
  0 \\
  \midrule
  $4/6$ &
  1 &
  0 \\
  \midrule
  $4/7$ &
  2 &
  1 \\
  \midrule
  $4/8$ &
  3 &
  1 \\
  \bottomrule
\end{tabularx}\end{latin}
\end{table}

پهنای باند در \متن‌لاتین{LoRa} می‌تواند بین ۱۲۵، ۲۵۰ و ۵۰۰ کیلوهرتز باشد و با توجه به استفاده از باند بدون لایسنس این پهنای باند وابسته به پارامتر‌های منطقه‌ای و فاکتور گسترش است.
به طور مثال در باند فرکانسی ۸۶۸ مگاهرتز ۸ کانال متفاوت وجود دارد که ۷ کانال ابتدایی تنها با پهنای باند ۱۲۵ کیلوهرتز کار می‌کنند و کانال آخر می‌تواند با پهنای باند‌های
۱۲۵، ۲۵۰ و ۵۰۰ کیلوهرتز کار کند. از این بین ۳ کانال ۱۲۵ کیلوهرتزی اجباری بوده و می‌توان در صورت لزوم کانال‌های بیشتری را نیز فعال کرد.

ارسال‌هایی که روی یک کانال صورت می‌پذیرند اما فاکتورهای گسترش متفاوتی دارند می‌توانند تقریبا بدون تداخل ارسال شوند پس زوج کانال و فاکتور گسترش در \متن‌لاتین{LoRa} می‌تواند به عنوان
یک کانال مجازی شناخته شود. از آنجایی که این کانال‌های مجازی فاکتورهای گسترش متفاوتی دارند، نرخ بیت ارسالی روی همه آن‌ها یکسان نخواهد بود
\مرجع{Augustin2016}.

\شروع{شکل}
\درج‌تصویر[width=\textwidth]{./img/lora-868-channels.jpg}
\تنظیم‌ازوسط
\شرح{کانال‌های \متن‌لاتین{LoRa} در باند فرکانسی ۸۶۸ مگاهرتز}
\پایان{شکل}

در \متن‌لاتین{LoRa} نرخ باد یا نرخ علائم از رابطه‌ی \رجوع{معادله: نرخ باد یا علائم در LoRa}
و زمان یک علامت از رابطه‌ی \رجوع{معادله: زمان علامت در LoRa}
محاسبه می‌گردد.
که در آن‌ها \متن‌لاتین{BW} پهنای باند و \متن‌لاتین{SF} فاکتور گسترش است
\مرجع{Augustin2016}.

\begin{align}
  \label{معادله: نرخ باد یا علائم در LoRa}
  R_{s} = \frac{BW}{2^{SF}}
\end{align}

\begin{align}
  \label{معادله: زمان علامت در LoRa}
  T_{s} = \frac{2^{SF}}{BW}
\end{align}

در ادامه نرخ داده‌ی ارسالی را می‌توان با استفاده از رابطه \رجوع{معادله: نرخ داده در LoRa} محاسبه کرد.

\begin{align}
  \label{معادله: نرخ داده در LoRa}
  R_{b} = SF \times \frac{BW}{2^{SF}} \times CR
\end{align}

در رابطه \رجوع{معادله: نرخ داده در LoRa} \متن‌لاتین{CR} نرخ کدگذاری، \متن‌لاتین{SF} فاکتور گسترش و \متن‌لاتین{BW} پهنای باند است
\مرجع{Augustin2016}.

\شروع{شکل}
\درج‌تصویر[width=.5\textwidth]{./img/lora-packet.png}
\تنظیم‌ازوسط
\شرح{ساختار بسته \متن‌لاتین{LoRa} \مرجع{Augustin2016}}
\برچسب{شکل: بسته LoRa}
\پایان{شکل}

رابطه \رجوع{معادله: تعداد علائم مورد نیاز در LoRa} مشخص می‌کند برای ارسال یک داده به چه تعداد علامت نیاز داریم. این پارامتر با $n_{s}$ نمایش داده می‌شود.

\begin{align}
  \label{معادله: تعداد علائم مورد نیاز در LoRa}
  n_{s} = 8 + \max\left( \left\lceil \frac{8PL - 4SF + 8 + CRC + H}{4 \times (SF - DE)} \right\rceil \times \frac{4}{CR}, 0 \right)
\end{align}

در رابطه \رجوع{معادله: تعداد علائم مورد نیاز در LoRa} در صورت فعال بودن \متن‌لاتین{CRC} مقدار آن برابر ۱۶ و در غیر این صورت برابر صفر است.
\متن‌لاتین{CR} نرخ کدگذاری،
\متن‌لاتین{PL} اندازه داده،
\متن‌لاتین{SF} فاکتور گسترش است.
در این رابطه \متن‌لاتین{H} اندازه سرآیند بوده که در صورت فعال بودن برابر ۲۰ و در غیر این صورت صفر است.
در این رابطه \متن‌لاتین{DE} در صورت فعال بودن حالت نرخ داده پایین یا \متن‌لاتین{low data rate} برابر ۲ و در غیر این صورت برابر صفر است
\مرجع{Augustin2016}
\مرجع{Pham2020}.

همانطور که محاسبات دیده می‌شود، استفاده از مقدارهای بالاتر برای فاکتور گسترش زمان ارسال را بیشتر کرده و تاثیر \متن‌لاتین{Duty Cycle} را بیشتر می‌کند.
با استفاده از فاکتورهای گشترش بالاتر نرخ ارسال کاهش پیدا کرده اما قابلیت اطمینان در مسافت‌های بالاتر بیشتر می‌شود.
در فاکتورهای گسترش پایین مسافت‌های کمتری قابل دسترس بوده اما زمان ارسال پایین‌تر می‌آید. در این فاکتورهای گسترش
امکان دست‌یابی به نرخ داده بالاتر وجود دارد اما دروازه می‌بایست حساسیت بالاتری داشته باشد.
پژوهش \مرجع{Adelantado2017} بیان می‌کند احتمال استفاده از فاکتورهای گسترش بالاتر بیشتر است.

ساختار فریم \متن‌لاتین{LoRa} در شکل \رجوع{شکل: بسته LoRa} قابل مشاهده است. استفاده از سرآیند اختیاری بوده است و در صورتی که مواردی مانند اندازه بسته،
نرخ کدگذاری و وجود \متن‌لاتین{CRC} از پیش هماهنگ شده باشند نیازی به استفاده از آن نیست. سرآیند از نرخ کدگذاری $4/8$ استفاده کرده و دارای یک \متن‌لاتین{CRC}
برای خود است.

یکی دیگر از پارامترهای قابل تنظیم در \متن‌لاتین{LoRa}، توان ارسال است. توان ارسال می‌تواند در بازه $-4dBm$ تا $20dBm$
باشد اما این مقادیر با محدودیت‌های سخت‌افزاری و قانونی محدود می‌شوند
\مرجع{Kufakunesu2020}.

\شروع{لوح}
\تنظیم‌ازوسط
\شروع{لاتین}\شروع{جدول}{ccccc}
\toprule
Data Rate & Spreading Factor & Bandwidth[KHz] & Bit rate[kbps] & Sensivity [dBm] \\
\midrule
DR0 & 12 & 125 & 0.293 & -137 \\
DR1 & 11 & 125 & 0.537 & -134.5 \\
DR2 & 10 & 125 & 0.976 & -132 \\
DR3 & 9  & 125 & 1.757 & -129 \\
DR4 & 8  & 125 & 3.125 & -126 \\
DR5 & 7  & 125 & 5.4680 & -123 \\
DR6 & 7  & 250 & 10.936 & -122 \\
\bottomrule
\پایان{جدول}\پایان{لاتین}
\شرح{نرخ داده و حساسیت \متن‌لاتین{LoRa} بر پایه پارامترهای متفاوت برای باند فرکانسی ۸۶۸ مگاهرتز \مرجع{ElChall2019}}
\پایان{لوح}

\زیرقسمت{\متن‌لاتین{LR-FHSS}}

یکی از تکنیک‌ها در شبکه‌های بی‌سیم استفاده از \متن‌لاتین{Frequency Hopping} است. در این تکنیک با هماهنگی در میان ارسال کننده و گیرنده فرکانس‌های ارسال در زمان
تغییر می‌کند. پیاده‌سازی این شیوه در شبکه‌های \متن‌لاتین{LoRaWAN} در قالب \متن‌لاتین{LR-FHSS} یا
\متن‌لاتین{Frequency Hopping Spread Spectrum}
صورت می‌پذیرد. در این روش هر کانال به تعدادی زیرکانال شکسته شده و سرآیند بسته روی همه این زیرکانال‌ها ارسال می‌شود.
خود داده اما قطعه قطعه شده و هر قطعه به وسیله‌ی یک زیرکانال ارسال می‌گردد.
از آنجایی که دروازه روی همه‌ی این کانال‌ها گوش می‌دهد می‌تواند بسته را دوباره بازسازی کند بنابراین
در دروازه نیازی به از پیش دانستن ترتیب این پرش‌ها یا فرکانس و پهنای باند دقیق این کانال‌ها ندارد
و این اطلاعات از طریق سرآیندهای بسته که به صورت تکراری ارسال می‌گردند قابل بازیابی است.
ذکر این نکته نیز خالی از لطف نیست که این ماژولیشن سربار بیشتری برای تشخیص سیگنال در گیرنده نسبت به \متن‌لاتین{LoRa} دارد
\مرجع{Boquet2021}.

\متن‌لاتین{LR-FHSS} در سال ۲۰۲۰ توسط \متن‌لاتین{Semtech} برای پشتیبانی از شبکه‌های وسیع پیشنهاد شده است.
ای ماژولیشن با \متن‌لاتین{LoRa} سازگار بوده و تنها برای \متن‌لاتین{uplink} پیشنهاد شده است و برای \متن‌لاتین{downlink} از همان
\متن‌لاتین{LoRa} استفاده خواهد شد
\مرجع{Boquet2021}.

پژوهش \مرجع{Boquet2021} با استفاده از شبیه‌سازی دست به ارزیابی این لایه فیزیکی زده است و یادآور می‌شود که هدف از ارائه این لایه فیزیکی
افزایش تعداد بسته‌های ارسالی توسط یک شی نیست بلکه افزایش کلی ظرفیت شبکه است. در شکل \رجوع{شکل: مقایسه LoRa و LR-FHSS} این افزایش ظرفیت شبکه بر پایه همین
شبیه‌سازی کاملا مشهود است. در شکل \رجوع{شکل: مقایسه LoRa و LR-FHSS} همچنین مشخص است که با افزایش تعداد اشیا در شبکه کارایی \متن‌لاتین{LR-FHSS} بیشتر از \متن‌لاتین{LoRa}
می‌گردد. از سوی دیگر این پژوهش بیان می‌کند در صورت افزایش حجم داده از ۱۰ بایت به ۵۰ بایت این بهبود بسیار زودتر به وقوع می‌پیوندد
\مرجع{Boquet2021}.

\شروع{شکل}
\درج‌تصویر[width=\textwidth]{./img/lr-fhss-limits.png}
\تنظیم‌ازوسط
\شرح{مقایسه دریافت صحیح اطلاعات با افزایش تعداد اشیا برای بسته‌های ۱۰ بیتی در \متن‌لاتین{LoRa} و \متن‌لاتین{LR-FHSS} \مرجع{Boquet2021}}
\برچسب{شکل: مقایسه LoRa و LR-FHSS}
\پایان{شکل}

از گپ‌های تحقیقاتی این حوزه می‌توان به مشخص کردن دنباله‌ی پرش‌های فرکانسی در \متن‌لاتین{LR-FHSS} و همزیستی \متن‌لاتین{LoRa} و \متن‌لاتین{LR-FHSS} اشاره کرد.

\زیرقسمت{\متن‌لاتین{LoRaWAN}}

\متن‌لاتین{LoRaWAN} پروتکل لایه لینک و شبکه بوده که شامل پروتکل کنترل دسترسی چندگانه\پانویس{MAC} نیز است.
این پروتکل اجازه می‌دهد تا دستگاه‌هایی با لایه فیزیکی \متن‌لاتین{LoRa} با برنامه‌های کاربردی ارتباط برقرار کنند.
این پروتکل توسط \متن‌لاتین{LoRa Alliance}، گروهی متشکل از \متن‌لاتین{IBM}، \متن‌لاتین{Semtech}، \متن‌لاتین{Actility} و \نقاط‌خ،
توسعه پیدا کرده و برای همگان قابل استفاده است.
این پروتکل برای ارتباط دستگاه به دستگاه ایجاد نشده است و تنها هدف آن ارتباط اشیا با دروازه و \متن‌لاتین{Network Server} است.
در صورت نیاز به ارتباط بین دستگاه‌ها می‌بایست از دروازه و \متن‌لاتین{Network Server} استفاده کرد یا اینکه
تنها لایه‌ی فیزیکی \متن‌لاتین{LoRa} را مورد استفاده قرار داد.
از سوی دیگر \متن‌لاتین{LoRaWAN} از لایه‌های فیزیکی \متن‌لاتین{LoRa}، \متن‌لاتین{FSK} و \متن‌لاتین{LR-FHSS} پشتیبانی می‌کند
\مرجع{Cruz2021}
\مرجع{Augustin2016}.

یک شبکه‌ی \متن‌لاتین{LoRaWAN} در ساده‌ترین شکل از اجزای زیر تشکیل شده است \مرجع{Fujdiak2022}:

\شروع{شمارش}
\فقره یک دستگاه حسگر یا عملگر که توان مصرفی و محاسبات محدودی دارد.
\فقره یک دروازه که عنصر شبکه‌ای برای دریافت و ارسال اطلاعات از و به دستگاه‌ها است. این عنصر شبکه برای ارتباط سرور شبکه
از زیرساخت \متن‌لاتین{IP} و لایه فیزیکی با گذردهی بالا مانند \متن‌لاتین{Ethernet} استفاده می‌کند.
\فقره سرور شبکه که پیام‌های دریافت شده از یک مجموعه دروازه‌ها را به برنامه‌های کاربردی می‌رساند و برعکس.
سرور شبکه وظیفه حذف بسته‌های تکراری، ارسال \متن‌لاتین{Ack}
و رمزگشایی آن‌ها برعهده دارد. از سوی دیگر بسته‌های ارسالی به دستگاه‌ها در سرور شبکه ساخته و صف می‌شوند.
\فقره سرور اتصال که پروسه‌های فعال‌سازی بر پایه \متن‌لاتین{OTAA} و \متن‌لاتین{ABP} را برای دستگاه‌های انتهایی مدیریت می‌کند.
\فقره سرور برنامه کاربردی که می‌تواند در بستر اینترنت قرار داشته باشد و داده‌ها را از طریق سرور شبکه برای اشیا ارسال و دریافت کند.
این سرور برنامه کاربردی کلید رمزنگاری متقارنی با اشیا داشته و می‌تواند داده‌ها را رمزگذاری و رمزگشایی کند. استاندارد در رابطه با چگونگی ادغام
کاربران انتهایی با سرور برنامه کاربردی صبحتی نکرده اما عموما در پیاده‌سازی‌ها از پروتکل \متن‌لاتین{MQTT} یا یک ارتباط ساده \متن‌لاتین{TCP/IP} استفاده شده است \مرجع{Carvalho2019}.
\پایان{شمارش}

\شروع{شکل}
\تنظیم‌ازوسط
\درج‌تصویر[width=\textwidth]{./img/nrm-home.png}
\شرح{مدل مرجع شبکه \متن‌لاتین{LoRaWAN} --- شبکه‌ی خانگی}
\برچسب{شکل: مدل مرجع شبکه LoRaWAN شبکه‌ی خانگی}
\پایان{شکل}

\شروع{شکل}
\تنظیم‌ازوسط
\درج‌تصویر[width=\textwidth]{./img/nrm-roaming.png}
\شرح{مدل مرجع شبکه \متن‌لاتین{LoRaWAN} --- شبکه‌ی فراگرد}
\برچسب{شکل: مدل مرجع شبکه LoRaWAN شبکه‌ی فراگرد}
\پایان{شکل}

\شروع{شکل}
\درج‌تصویر[width=\textwidth]{./img/lora-architecture-osi.png}
\تنظیم‌ازوسط
\شرح{معماری شبکه \متن‌لاتین{LoRaWAN} از نگاه مدل لایه‌ای \متن‌لاتین{OSI} \مرجع{Ertrk2019}}
\پایان{شکل}

برخلاف شبکه‌های سلولی سنتی، در \متن‌لاتین{LoRaWAN} ارتباطی میان دروازه و دستگاه انتهایی شکل نمی‌گیرد.
دروازه‌ها در واقع نقش رله‌ای در لایه لینک را ایفا می‌کنند و بعد از افزودن اطلاعات مبنی بر کیفیت پیام دریافتی آن
را به سرور شبکه ارسال می‌کنند. بنابراین دستگاه‌های انتهایی با سرور شبکه ارتباط دارند که وظیفه آن رمزگشایی بسته‌ها، حذف بسته‌های تکراری و
انتخاب دروازه مناسب جهت ارسال بسته به دستگاه انتهایی است.
بنابراین می‌توان گفت که در شبکه‌ی \متن‌لاتین{LoRaWAN} عملا دروازه از دید دستگاه‌های انتهایی پنهان است
\مرجع{Augustin2016}.

در حوزه امنیت \متن‌لاتین{LoRaWAN}، دولایه از امنیت را تعریف می‌کند. لایه اول امنیت میان شی و شبکه است در حالی که لایه دوم میان شی و برنامه کاربردی است.
به این صورت می‌توان مطمئن شد که تنها برنامه کاربردی است که می‌تواند داده‌های ارسالی توسط دستگاه را رمزگشایی کند
\مرجع{Cruz2021}. شبکه‌ی \متن‌لاتین{LoRaWAN} از معدود سیستم‌های اینترنت اشیا است که رمزنگاری انتها به انتها را پیاده‌سازی کرده است
\مرجع{Kufakunesu2020}.

در ضمن \متن‌لاتین{LoRaWAN} ویژگی‌های دیگری مانند نرخ داده تطبیقی\پانویس{ADR} را اضافه می‌کند. در نرخ داده تطبیقی شبکه با دستگاه در رابطه با پارامترهای لایه‌ی فیزیکی \متن‌لاتین{LoRa} مذاکره می‌کند
که در نتجیه آن کارآیی مصرف بهینه می‌شود. شکل \رجوع{شکل: لایه‌های لورا} مدل لایه‌ای \متن‌لاتین{LoRa} و \متن‌لاتین{LoRaWAN} را نمایش می‌دهد
\مرجع{Cruz2021}.

\شروع{شکل}
\درج‌تصویر[width=\textwidth]{./img/lora-layers.png}
\تنظیم‌ازوسط
\شرح{مدل لایه‌ای \متن‌لاتین{LoRa} و \متن‌لاتین{LoRaWAN} \مرجع{Cruz2021}}
\برچسب{شکل: لایه‌های لورا}
\پایان{شکل}

در شبکه‌های \متن‌لاتین{LoRaWAN} سه کلاس کاری می‌توان برای اشیا در نظر گرفت.

\شروع{فقرات}
\فقره در کلاس ($A$ یا \متن‌لاتین{All}) شی هر زمان که به خواهد شروع به ارسال داده کرده و دو پریود متوالی آینده را برای دریافت \متن‌لاتین{Downlink} خواهد داشت. این کلاس پایین‌ترین مصرف انرژی را دارد چرا که شی تنها در زمان‌هایی که لازم است
روشن می‌شود و می‌تواند دوباره خاموش شود. با توجه به ساختار دریافت \متن‌لاتین{Downlink} در این کلاس می‌توان به سادگی برای پیام‌ها \متن‌لاتین{Ack} دریافت کرد اما برای سایر پیام‌های \متن‌لاتین{Downlink} می‌بایست تا تصمیم شی برای ارسال داده صبر کرد.
\فقره در کلاس ($B$ یا \متن‌لاتین{Beacon}) گره به صورت همگام می‌تواند \متن‌لاتین{Downlink} دریافت کند برای اینکار گره به جز دو بازه دریافت که در کلاس $A$ تعریف شده بود یک بازه دریافت قابل پیش‌بینی نیز دارد.
این کلاس مصرف بالاتری دارد چرا که نیاز است یک بازه دوره‌ای برای دریافت \متن‌لاتین{Downlink} روشن شود. مزیت این کلاس قابلیت دریافت \متن‌لاتین{Downlink} حتی در زمان‌هایی که ارسالی ندارد، است. برای همگام‌سازی اشیا کلاس $B$ از \متن‌لاتین{Beacon}های دوره‌ای استفاده می‌شود.
\فقره در کلاس ($C$ یا \متن‌لاتین{Continues}) بیشترین مصرف توان را داشته و شی در هر زمان می‌تواند داده دریافت کند. این کلاس از اشیا از قابلیت‌های کلاس $A$ پشتیبانی کرده اما قابلیت‌های کلاس $B$ را ندارند.
\پایان{فقرات}

پیشتر به ساختار بسته‌ها در لایه‌ی فیزیکی \متن‌لاتین{LoRa} پرداختیم. در \متن‌لاتین{LoRaWAN} برای پیام‌های \متن‌لاتین{uplink} استفاده از سرآیند و \متن‌لاتین{CRC} اجباری است
و بنابراین نمی‌توان در \متن‌لاتین{LoRaWAN} از فاکتور گسترش ۶ استفاده کرد. از سوی دیگر تنها استفاده از سرآیند در بسته‌های \متن‌لاتین{downlink} اجباری است.
نرخ کدگذاری در \متن‌لاتین{LoRaWAN} مشخص نشده و قابل تغییر است
\مرجع{Augustin2016}.

\شروع{شکل}
\درج‌تصویر[width=\textwidth]{./img/lorawan-packet.png}
\تنظیم‌ازوسط
\شرح{بسته‌های پروتکل \متن‌لاتین{LoRaWAN} (سایز فیلدها به بیت آمده است) \مرجع{Augustin2016}}
\برچسب{شکل: بسته‌های LoRaWAN}
\پایان{شکل}

ساختار پیام‌های \متن‌لاتین{LoRaWAN} در شکل \رجوع{شکل: بسته‌های LoRaWAN} آورده شده است. \متن‌لاتین{DevAddr} آدرس دستگاه است.
\متن‌لاتین{FPort} پورت برای مالتی‌پلکس است و در صورتی که مقدار آن صفر باشد نشان‌دهنده‌ی این موضوع است که بسته تنها شامل
دستورات لایه‌ی \متن‌لاتین{MAC} است. بسته‌های \متن‌لاتین{uplink} ادرس مقصد و بسته‌های \متن‌لاتین{downlink} آدرس مبدا ندارند
چرا که هر شی تنها با یک سرور شبکه در ارتباط است
\مرجع{Augustin2016}.

در \متن‌لاتین{LoRaWAN} از الگوریتم کنترل دسترسی همزمان \متن‌لاتین{ALOHA} استفاده میشود. در این الگوریتم بسته‌ها می‌توانند اندازه‌های متغیر داشته باشند، گرهها هر زمان که قصد داشته باشند داده ارسال کنند و نیازی به همگام‌سازی زمانی ندارد.
مشکل اصلی این روش در تعداد برخوردهای بالا در زمانی است که شبکه گرههای زیادی دارد، از این رو روش‌های ``گوش‌دادن پیش از حرف‌زدن'' مانند \متن‌لاتین{CSMA/CA} کارایی بهتری دارند.
چنین الگوریتم‌هایی در مقابل نیاز به همگام‌سازی دارند به این معنی که می‌بایست اشیا یک \متن‌لاتین{clock} محلی مشترک داشته باشند
\مرجع{Beltramelli2021}.

به صورت کلی می‌توان الگوریتم‌های دسترسی همزمان را در سه گروه اصلی قرار داد:

\شروع{فقرات}
\فقره الگوریتم‌های قطعه‌بندی کانال
\فقره الگوریتم‌های دسترسی تصادفی
\فقره الگوریتم‌های نوبت‌دهی
\پایان{فقرات}

که الگوریتم \متن‌لاتین{ALOHA} یا \متن‌لاتین{CSMA} در دسته الگوریتم‌های دسترسی تصادفی است. روش \متن‌لاتین{ALOHA} خود به دو گونه \متن‌لاتین{Pure ALOHA} و \متن‌لاتین{Slotted ALOHA}
قابل پیاده‌سازی است. در روش \متن‌لاتین{Pure ALOHA} که در \متن‌لاتین{LoRaWAN} نیز استفاده می‌شود، گرهها هر زمان که داده‌ای داشته باشند می‌توانند آن را ارسال کنند و در صورت رخ دادن تصادم
با احتمال $p$ داده را باز ارسال می‌کنند. در روش \متن‌لاتین{Slotted ALOHA} نیاز است که گرهها با یکدیگری همگام بوده و در ابتدای بازه‌های مشخصی شروع به ارسال کنند و در صورت
وقوع تصادم بعد از صبر کردن در ابتدای بازه‌ی بعدی با احتمال $p$ باز ارسال را شروع می‌کنند. کارایی روش \متن‌لاتین{Slotted ALOHA} از روش \متن‌لاتین{ALOHA} بیشتر بوده است ولی نیاز به همگام‌سازی گرهها دارد
که خود موضوعی هزینه‌بر است. پژوهش‌های متعددی روش‌های همگام‌سازی برای \متن‌لاتین{LoRaWAN} در جهت استفاده از \متن‌لاتین{Slotted ALOHA} پیشنهاد داده‌اند.
می‌توان نشان داد کارایی پروتکل \متن‌لاتین{ALOHA} برابر $\frac{1}{2e}$ است و از سوی دیگر کارایی پروتکل \متن‌لاتین{Slotted ALOHA} تقریبا دو برابر بوده و برابر $\frac{1}{e}$ است.

ذکر این نکته خالی از لطف نیست که تفاوت آنچه در \متن‌لاتین{LoRaWAN} به عنوان \متن‌لاتین{ALOHA} پیاده‌سازی شده است و آنچه به عنوان \متن‌لاتین{ALOHA} شناخته می‌شود در اندازه بسته‌ها است،
در \متن‌لاتین{LoRaWAN} اندازه بسته‌ها متغیر است ولی در \متن‌لاتین{ALOHA} اندازه‌ی بسته‌ها ثابت در نظر گرفته می‌شود
\مرجع{Augustin2016}.

ارسال اطلاعات همگام‌سازی در صورتی که از طریق همان بستر بی‌سیم رخ بدهد به آن همگام‌سازی داخل باند گفته می‌شود. در این روش نیاز است که به محدودیت‌های بستر رادیویی احترام گذاشت.
در این شیوه ممکن است کیفیت ارتباط رادیویی به علت ارسال همین اطلاعات به میزان غیرقابل قبولی افت کند. برای حل این مشکل می‌توان از راه‌حل‌های خارج از باند استفاده کرد
\مرجع{Beltramelli2021}.

پروتکل \متن‌لاتین{LoRaWAN} یک پروتکل تک گام است و از مسیریابی یا ارسال چند گامی پیام پشتیبانی نمی‌کند. از مهم‌ترین ویژگی‌های این پروتکل، مکانیزم نرخ داده تطبیق‌پذیر و تشخیص
موقعیت مکانی بدون نیاز به \متن‌لاتین{GPS} و تنها با ۳ دروازه است
\مرجع{Ertrk2019}.

اولین نسخه از استاندارد \متن‌لاتین{LoRaWAN} در سال ۲۰۱۵ منتشر شد. در طی این سال‌ها بهبودهایی در آن حاصل شد که مهمترین نسخه‌های آن ۱.۰.۳ و ۱.۱ است
\مرجع{Ertrk2019}.

\شروع{شکل}
\درج‌تصویر[width=\textwidth]{./img/lorawan-gps.png}
\تنظیم‌ازوسط
\شرح{تشخیص موقعیت مکانی در \متن‌لاتین{LoRaWAN} \مرجع{Ertrk2019}}
\پایان{شکل}

برای پیوستن یک شی به شبکه \متن‌لاتین{LoRaWAN} نیاز است که آن شی فعال‌سازی شود. دو راه برای فعال‌سازی اشیا وجود دارد: \متن‌لاتین{Over-The-Air Activation} یا مختصرا \متن‌لاتین{OTAA}
و \متن‌لاتین{Activation By Personalization} یا مختصرا \متن‌لاتین{ABP}
\مرجع{Augustin2016}.

پروسه فعال‌سازی می‌بایست اطلاعات زیر را در اختیار اشیا قرار بدهد:
\شروع{فقرات}
\فقره آدرس دستگاه انتهایی (\متن‌لاتین{DevAddr}): یک شناسه‌ی ۳۲ بیتی که ۷ بیت آن نماینده شبکه و ۲۵ بیت آن آدرس دستگاه انتهایی در شبکه است.
\فقره شناسه برنامه کاربردی (\متن‌لاتین{AppEUI}): شناسه عمومی برنامه کاربردی که از فضای آدرس \متن‌لاتین{IEEE EUI64} انتخاب شده و به صورت یکتا صاحب این دستگاه انتهایی را مشخص می‌کند.
\فقره کلید نشست شبکه (\متن‌لاتین{NetSKey}): کلیدی که میان دستگاه و سرور شبکه برای محاسبه و اطمینان از یکپارچگی اطلاعات استفاده می‌شود.
\فقره کلید نشست برنامه کاربردی (\متن‌لاتین{AppSKey}): کلیدی که میان دستگاه و برنامه کاربردی برای رمزگذاری و رمزگشایی اطلاعات استفاده می‌شود.
\پایان{فقرات}

در \متن‌لاتین{OTAA} فرآیند عضویت شامل یک درخواست عضویت و پاسخی مبتنی بر پذیرفته شدن عضویت است که برای هر نشست جدید استفاده می‌شود.
بنابر پاسخی که مبتنی بر پذیرفته شدن عضویت می‌آید، دستگاه انتهایی کلیدهای جدید نشست شبکه و برنامه کاربردی را دریافت می‌کند.
در فرایند \متن‌لاتین{ABP} این کلیدها به صورت مستقیم روی دستگاه‌های انتهایی ذخیره شده‌اند
\مرجع{Augustin2016}.

در \متن‌لاتین{LoRaWAN} دستورات \متن‌لاتین{MAC}ای تعریف شده است که اجازه می‌دهد پارامترهای دستگاه‌های انتهایی سفارشی شود.
از بین این دستورات تنها \متن‌لاتین{LinkCheckReq} از سمت شی ارسال شده و هدف آن بررسی ارتباط است.
سایر این دستورات از سوی سرور شبکه ارسال می‌شوند و می‌توانند پارامترهایی مانند نرخ داده تنظیم پذیر را تغییر دهند
\مرجع{Augustin2016}.

آنجه در \متن‌لاتین{LoRaWAN} بیان می‌شود در رابطه با دستگاه‌ّها است و در رابطه با سرور شبکه حرفی به میان نیامده است اما سرور شبکه نقشه حیاتی در کارکرد شبکه ایفا می‌کند.
اگر بخواهیم در شبکه میلیون دستگاه را هندل کنیم در این صورت بهینه‌سازی این دستگاه‌ّها بر عهده سرور شبکه خواهد بود. در صورتی که سرور شبکه دستورات \متن‌لاتین{MAC}
درستی را ارسال نکند یا پارامترها را به شکل اشتباهی تغییر دهد می‌تواند کاملا کارکرد شبکه را مختل کند
\مرجع{Augustin2016}.

از سوی دیگر در رابطه با نقش \دروازهها در \متن‌لاتین{LoRaWAN} تنها به عنوان یک رله اکتفا می‌شود و مسئولیت انتخاب بهترین \دروازه برای ارسال
برعهده سرور شبکه خواهد بود. تنها مسئولیت \دروازه زمان‌بندی صحیح در جهت ارسال است و این زمان‌بندی می‌بایست به گونه‌ای باشد که از تصادم در هنگام ارسال
\متن‌لاتین{Downlink} جلوگیری کند. البته استاندارد \متن‌لاتین{LoRaWAN} در رابطه با این زمان‌بندی صحیتی به میان نمیاورد
\مرجع{Augustin2016}.

با وجود اینکه \متن‌لاتین{LoRaWAN} عموما برای سناریوهای ارتباطات ثابت استفاده می‌شود، کاربردهایی از آن در سناریوهای متحرک نیز وجود دارد.
\متن‌لاتین{LoRaWAN} می‌تواند برای سناریوهای متحرکی چون نظارت بر ناوگان، اجاره دوچرخه، آتش‌سوزی جنگل‌ها و دنبال کردن حیوانات
استفاده شود \مرجع{Almojamed2021}.

پیام‌هایی که از سمت گره به \دروازه ارسال می‌شوند، \متن‌لاتین{Uplink} یا اختصارا \متن‌لاتین{UL} نامیده می‌شوند و
پیام‌هایی که از سمت \دروازه به گره ارسال می‌شوند، \متن‌لاتین{Donwlink} یا اختصارا \متن‌لاتین{DL} نامیده می‌شوند.
به صورت کلی دو نوع پیام در شبکه‌های \متن‌لاتین{LoRaWAN} وجود دارد، پیام‌های بدون تاییدیه که در آن‌ها گره نیازی به دریافت پاسخ
از \متن‌لاتین{NS} ندارد و پیام‌های با تاییدیه که نیاز به دریافت پاسخ دارند
\مرجع{Kufakunesu2020}.

یکی از مسائلی که \متن‌لاتین{LoRaWAN} به آن پرداخته است، مساله فراگردی است. در \متن‌لاتین{LoRaWAN} این امکان وجود دارد
که یک شی بتواند بین شبکه‌هایی با مدیریت‌های مختلف جابجا شود. در این شرایط نیاز است که لایه پیوند داده شی توسط یک
سرور شبکه کنترل شود و اطلاعات شی روی یک سرور شبکه‌ای دیگر قرار داشته باشد. مدل مرجع شبکه‌ی \متن‌لاتین{LoRaWAN}
به ترتیب در زمان فراگرد و در زمان استفاده از شبکه‌ی خانگی در شکل‌های \رجوع{شکل: مدل مرجع شبکه LoRaWAN شبکه‌ی فراگرد}
و \رجوع{شکل: مدل مرجع شبکه LoRaWAN شبکه‌ی خانگی} آمده است.
با توجه به تعریف فراگَردی در پروتکل \متن‌لاتین{LoRaWAN} این پروتکل به یک گزینه خوب در کنترل و مدیریت ناوگان بدل شده است.

از سال ۲۰۱۵ تا به امروز نسخه‌های مختلفی از استاندارد \متن‌لاتین{LoRaWAN} منتشر شده است.
به ترتیب نسخه $1.0$ در سال ۲۰۱۵، نسخه $1.0.1$ در سال ۲۰۱۶، نسخه $1.0.2$ در سال ۲۰۱۶،
نسخه $1.1$ در سال ۲۰۱۷، نسخه $1.0.3$ در سال ۲۰۱۸، نسخه $1.0.4$ در سال ۲۰۲۰ منتشر شده‌اند.
نسخه $1.1$ ویژگی‌های فراگردی و بهبودهای امنیتی را به \متن‌لاتین{LoRaWAN} اضافه کرد اما با توجه به عدم مهاجرت
صنعت استانداردهای $1.0$ ادامه پیدا کردند
\مرجع{Fujdiak2022}

\زیرزیرقسمت{مکانیزم نرخ داده تطبیق‌پذیر}

\متن‌لاتین{LoRaWAN} از مکانیزم نرخ داده تطبیق‌پذیر برای بهینه‌سازی نرخ‌داده، زمان ارسال و انرژی مصرفی به صورت پویا استفاده می‌کند.
در این مکانیزم پارامترهای ارسال که عبارتند از پهنای باند (\متن‌لاتین{BW})، فاکتور گسترش (\متن‌لاتین{SF})،
توان ارسال (\متن‌لاتین{TP}) و نرخ کدگذاری (\متن‌لاتین{CR})، کنترل می‌شوند.
بهینه‌سازی مکانیزم نرخ داده تطبیق‌پذیر ظرفیت شبکه را افزایش می‌دهد چرا که بسته‌هایی که با فاکتورهای گسترش مختلف ارسال می‌شوند،
بر یکدیگر عمود بوده و بنابراین می‌توانند به صورت همزمان دریافت شوند و زمان ارسال کاهش پیدا کند.
اگر بخواهیم بهتر بیان کنیم، مکانیزم نرخ داده تطبیق‌پذیر پارامترهای ارسال \متن‌لاتین{uplink} دستگاه‌های \متن‌لاتین{LoRa}
را با توجه به بودجه لینک کنترل می‌کند. برای استفاده از مکانیزم نرخ داده تطبیق‌پذیر، این مکانیزم می‌بایست در دستگاه انتهایی فعال باشد
\مرجع{Kufakunesu2020}.

الگوریتم نرخ داده تطبیق‌پذیر عملا در سمت گره و در سمت سرور شبکه پیاده‌سازی می‌شود. استاندارد \متن‌لاتین{LoRaWAN}
در رابطه با پیاده‌سازی آن در سمت سرور شبکه صحبتی نکرده است و در نهایت این دو الگوریتم می‌توانند به صورت غیرهمگام با یکدیگر
اجرا شوند. در شکل \رجوع{شکل: فلوچارت الگوریتم نرخ داده تطبیق‌پذیر در سمت گره} فلوچارت الگوریتم نرخ‌داده تطبیق‌پذیر
در سمت گره آورده شده است
\مرجع{Kufakunesu2020}.
این مکانیزم به گونه‌ای است که در سمت گره می‌توان نرخ داده را کاهش داده و در سمت سرور شبکه این نرخ را افزایش داد
\مرجع{Potsch2017}.

\شروع{شکل}
\تنظیم‌ازوسط
\درج‌تصویر[height=.5\textwidth]{img/lorawan-node-adr-flowchart.png}
\شرح{فلوچارت الگوریتم نرخ داده تطبیق‌پذیر در سمت گره \متن‌لاتین{LoRaWAN} \مرجع{Kufakunesu2020}}
\برچسب{شکل: فلوچارت الگوریتم نرخ داده تطبیق‌پذیر در سمت گره}
\پایان{شکل}

الگوریتم‌های زیادی برای نرخ داده تطبیق‌پذیر پیشنهاد شده‌اند که هر یک برای برآورده شدن هدف و شاخص کارایی متفاوتی در شبکه‌ی \متن‌لاتین{LoRaWAN}
تلاش می‌کنند.
در بستر متن‌باز \متن‌لاتین{The Things Network}
الگوریتمی بر پایه الگوریتم توصیه‌شده \متن‌لاتین{Semtech} برای مکانیزم نرخ داده تطبیق‌پذیر پیاده‌سازی شده است
\مرجع{Kufakunesu2020}. در ادامه به مرور تعدادی از این الگوریتم‌ها می‌پردازیم.

الگوریتمی با هدف تشخیص سطح تراکم در شبکه با استفاده از یادگیری ماشین و در ادامه اعمال نتیجه برای کنترل نرخ داده پیشنهاد شده است.
این الگوریتم از نرخ داده، قدرت سیگنال دریافتی و تعداد ارتباطات در \دروازه به عنوان شاخص‌های یادگیری استفاده می‌کند.
در زمان تشخیص ازدحام، الگوریتم پیشنهادی به جای کاهش دادن نرخ داده، زمان عقب‌کشیدن نمایی را افزایش می‌دهد.
نتایج حاکی از افزایش کارایی شبکه و دقت در کنترل نرخ داده است. یادگیری در سرور شبکه صورت می‌گیرد و نتایج را به دستگاه‌هایی انتهایی می‌دهد.
محاسبات یادگیری ماشین روی یک سیستم متمرکز صورت می‌پذیرد. نقطه قوت این الگوریتم در نظر گرفتن سطح تراکم در شبکه و نقطه ضعف آن استفاده
از پیام‌های \متن‌لاتین{Downlink} است
\مرجع{Kufakunesu2020}.

الگوریتمی با هدف تخصیص بهینه فاکتور گسترش در راستای بهینه‌سازی تصادم‌ها و میرایی پیشنهاد شده است.
مدل پیشنهادی قصد دارد کیفیت گرههای انتهایی توزیع شده در شبکه را با انتصاب فاکتورهای گسترشی که کیفیت ارسال را بیشینه می‌کنند، بهینه کند
\مرجع{Kufakunesu2020}.

یکی از الگوریتم‌های پیشنهادی هدف کاهش زمان همگرایی در مکانیزم نرخ داده تطبیق‌پذیر برای گره و \دروازه را دنبال می‌کند.
در ادامه این پژوهش با استفاده از شبیه‌سازی (به وسیله‌ی \متن‌لاتین{ns3}) نشان می‌دهد که با استفاده از رویش پیشنهادی این زمان همگرایی برای گره کاهش به سزایی دارد
\مرجع{Kufakunesu2020}.

الگوریتم پیشنهادی با ارائه روش‌های ابتکاری تلاش می‌کند تا فاکتورهای گسترش را میان گرههای در پوشش یک \دروازه به گونه‌ای تقسیم کند
که با توجه به شبه‌عمود بودن این فاکتورهای گسترش بتوان بیشترین بهره از کانال رادیویی برد. نقطه ضعف این روش پشتیبانی آن از تنها یک \دروازه
است
\مرجع{Kufakunesu2020}.

\زیرقسمت{چالش‌های \متن‌لاتین{LoRa} و \متن‌لاتین{LoRaWAN}}

با وجود گسترش و استفاده روزافزون شبکه‌های \متن‌لاتین{LoRaWAN} هنوز چالش‌هایی در این شبکه وجود دارد. اولین چالش مربوط به تاثیر
\متن‌لاتین{Duty Cycle} بر اندازه شبکه است. با افزایش تعداد گرهها در شبکه نرخ پیام‌هایی که به درستی دریافت می‌شوند توسط تصادم و
\متن‌لاتین{Duty Cycle} محدود می‌شوند. همانطور که پیشتر بیان شد \متن‌لاتین{LoRaWAN} از \متن‌لاتین{ALOHA} استفاده می‌کند
که خود یکی از دلایل اصلی در ایجاد تصادم با افزایش تعداد گرهها در شبکه است. آنچه پژوهش \مرجع{Adelantado2017} در این حوزه
با شبیه‌سازی بر پایه سه کانال با بسته‌های ۱۰ بایتی محاسبه کرده است در شکل \رجوع{شکل: محدودیت‌های LoRaWAN} قابل مشاهده است
\مرجع{Adelantado2017}.

\شروع{شکل}
\درج‌تصویر[width=.5\textwidth]{./img/lora-limits-1}
\تنظیم‌ازوسط
\شرح{تاثیر افزایش اشیا و نرخ ارسال با در نظر گرفتن سه کانال بر بسته‌های دریافتی در پروتکل \متن‌لاتین{LoRaWAN} \مرجع{Adelantado2017}}
\برچسب{شکل: محدودیت‌های LoRaWAN}
\پایان{شکل}

دومین چالش شبکه‌های \متن‌لاتین{LoRaWAN} بحث قابلیت اطمینان است. قابلیت اطمینان در این شبکه‌ها توسط \متن‌لاتین{Acknowledgement} تامین می‌شود.
اما باز به دلیل وجود \متن‌لاتین{Duty Cycle} و تاثیر آن بر \دروازه طراحی شبکه و برنامه‌های می‌بایست به گونه‌ای باشد که تعداد \متن‌لاتین{Acknowledgement}ها کمینه شوند.
همین مشکل استفاده از \متن‌لاتین{LoRaWAN} برای کاربردهایی که قابلیت اطمینان بسیار بالا می‌خواهند را زیر سوال می‌برد
\مرجع{Adelantado2017}.

اگر کاربردهایی مانند کشاورزی هوشمند یا کنترل محیط را در نظر بگیریم که داده‌های تولیدی به صورت دوره‌ای یا غیردوره‌ای بوده و محدودیت‌های سخت‌گیرانه‌ای از نگاه تاخیر ندارند و تنها نیاز به پوشش بالا وجود دارد،
استفاده از \متن‌لاتین{LoRaWAN} گزینه‌ی مناسبی است. در رابطه با کاربردهایی مانند اتوماسیون صنعتی که نیاز به تاخیر مشخص و محدود شده وجود دارد، استفاده از \متن‌لاتین{LoRaWAN} نیاز به در نظر گرفتن تمهیدات
خاصی است از جمله، کوچک نگاه داشتن فاکتور گسترش یا به عبارت دیگر نزدیکی اشیا به \دروازه که باعث کاهش زمان ارسال و تاثیر \متن‌لاتین{Duty Cycle} می‌گردد و از سوی دیگر تعداد کانال‌ها می‌بایست
با دقت انتخاب شود تا با شکست ارسال در یک کانال بتوان بدون مشکل \متن‌لاتین{Duty Cycle} از کانال دیگری برای ارسال بهره برد
\مرجع{Adelantado2017}.

\گرنادرست
\زیرقسمت{\متن‌لاتین{WiFi 7}}

اندکی پس از انتشار \متن‌لاتین{WiFi 6} کارگروه \متن‌لاتین{IEEE 802.11} به همراه \متن‌لاتین{WiFi Alliance} شروع به طراحی نسل بعدی آن در شبکه‌های بی‌سیم محلی با نام \متن‌لاتین{WiFi 7} کردند.
یکی از اجزای \متن‌لاتین{WiFi 7}، \متن‌لاتین{IEEE 802.11be} است. قرار است در این نسل از \متن‌لاتین{Time-Sensitive Networking} یا \متن‌لاتین{TSN} برای ارتباط‌هایی با تاخیر کم و قابلیت
اطمینان بالا پشتیبانی شود
\مرجع{Adame2021}.

\متن‌لاتین{TSN} در ابتدا برای شبکه‌ها اترنت (\متن‌لاتین{IEEE 802.3}) طراحی شده بود اما به آرامی راه خود را به شبکه‌های بی‌سیم باز می‌کند. در \متن‌لاتین{TSN} سعی می‌شود
هیچ بسته‌ای به خاطر ازدحام بافرها از دست نرود، بسته‌های کمی در خرابی تجهیزات از دست بروند و تاخیر انتها به انتها گارانتی شده باشد.
کارگروه \متن‌لاتین{IEEE 802.11be} برای طراحی لایه \متن‌لاتین{MAC} و \متن‌لاتین{PHY} در می ۲۰۱۹ شکل گرفت. یکی از اهداف \متن‌لاتین{WiFi 7} کاهش بدترین حالت تاخیر و \متن‌لاتین{Jitter} است
که برای آن، کارگروه در حال بررسی استانداردهای \متن‌لاتین{TSN} است
\مرجع{Adame2021}.

با وجود اینکه هرگز \متن‌لاتین{WiFi} نخواهد توانست تاخیر محدودی را با توجه به ماهیت خود در استفاده از باندهای فرکانسی بدون مجوز، ارائه دهد اما استفاده از مفاهیم \متن‌لاتین{TSN}
می‌تواند آن را در زمره فناوری‌های پیشرو در \متن‌لاتین{6G} قرار دهد
\مرجع{Adame2021}.

به صورت سنتی \متن‌لاتین{WiFi} برای مدیریت دسترسی همزمان از \متن‌لاتین{Distributed Coordination Function} یا مختصرا \متن‌لاتین{DCF} استفاده می‌کند.
این شیوه بر پایه حس حامل و عقب‌نشینی نمایی عمل می‌کند. از مشکلات اصلی آن می‌توان به عدم قابلیت برای اولویت‌دهی ترافیک و از سوی دیگر غیرقابل پیش‌بینی بودن
آن اشاره کرد. در واقع در \متن‌لاتین{DCF} چند ایستگاه می‌توانند باعث اشباع شدن کانل شده و بنابراین نمی‌توان گارانتی از نظر زمانی برای داده‌ها ارائه داد
\مرجع{Adame2021}.

برای حل این مشکل روش \متن‌لاتین{EDCF} یا \متن‌لاتین{Enhanced DCF} در \متن‌لاتین{IEEE 802.11e} پیشنهاد شد. در این روش امکان اولویت‌دهی بر پایه
کاتالوگ‌های دسترسی اضافه شد. در ادامه این شیوده در \متن‌لاتین{IEEE 802.11aa} برای ارتباطات صدا و تصویر بهبود بیشتری یافت.
با این حال هیچ یک از این استانداردها کیفیت سرویس را در شرایطی که \متن‌لاتین{WiFi} دارای بار اضافه است، گارانتی نمی‌کنند
\مرجع{Adame2021}.

در لایه انتقال وجود بافر در پروتکل \متن‌لاتین{TCP} باعث تاخیرهای زیادی می‌شود و این امر کار برای انتقال جریان‌های ترافیکی \متن‌لاتین{TCP}
با استانداردهای \متن‌لاتین{TSN} سخت می‌کند. از سوی دیگر تکنیک‌های شبکه‌های سیمی مانند روش‌های نوین مدیریت صف و \نقاط‌خ در اینجا
کارایی زیادی ندارد
\مرجع{Adame2021}.

در استاندارد \متن‌لاتین{IEEE 802.11be} حالت عملیاتی چند کاناله وجود دارد. با استفاده از این حالت امکان افزایش بهره‌وری با ارسال همزمان
روی چند کانال به وجود می‌آید و از سوی دیگر می‌توان یک بسته یکسان را در چند کانال ارسال کرده تا از رسیدن آن مطمئن شد. در نهایت ارسال‌کننده
می‌تواند کانال با تاخیر کمتر را انتخاب کرده و تاخیر را کاهش دهد. این حالت عملیاتی خود می‌تواند در دو حالت همزمان و غیرهمزمان استفاده شود.
در حالت همزمان بعد از ارسال از کانال اصلی یک مدتی صبر شده و بعد می‌توان از کانال ثانویه استفاده کرد این در حالتی است که در حالت غیرهمزمان
هر دو کانال می‌توانند همزمان استفاده شوند ولی امکان تداخل میان آن‌ها وجود دارد
\مرجع{Adame2021}.
\رگ
