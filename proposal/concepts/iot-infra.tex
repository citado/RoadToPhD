\قسمت{زیرساخت‌های اینترنت اشیا}

حجم داده و پردازش مورد نیاز در اینترنت اشیا بسیار زیاد است و برای همین نیاز به زیرساختی قابل گسترش وجود دارد.
برای پاسخ به این نیاز پردازش ابری با گسترش‌پذیری و ظرفیت بالا، بهترین گزینه است.
در پردازش ابری عموما برنامه‌ها به صورت مجازی‌سازی شده اجرا می‌شوند که این امر کارایی، امنیت، گسترش‌پذیری و کاهش هزینه
را نسبت به اجرای روی \متن‌لاتین{bare-metal} به ارمغان می‌آورد
\مرجع{Botez2021}.

استفاده از کانتینرها برای اجرای برنامه‌ها بر بسترهای ابری به جای استفاده از مجازی‌سازی می‌تواند گسترش‌پذیری
بیشتری را به همراه داشته باشد
\مرجع{Botez2021}.

در شبکه‌های \متن‌لاتین{5G} بحث استفاده از کارکردهای مجازی شبکه مطرح شده است و هدف حذف کارکردهای فیزیکی
و استفاده از پیاده‌سازی‌های نرم‌افزاری آن‌ها است. اما هنوز مشکلاتی وجود دارد،
این پیاده‌سازی‌ها نرم‌افزاری در قالب ماشین‌های مجازی سربار زیادی دارند
و به سادگی نمی‌توانند آن‌ها را گسترش داد، مدیریت کرد یا هماهنگ نمود.
از این روی \متن‌لاتین{Cloud-Native Network Functions}ها مطرح می‌شوند که پیاده‌سازی کارکردها به صورت ابرزی بوده
و می‌توان برای مدیریت آن‌ها به زیرساخت‌هایی چون \متن‌لاتین{Kubernetes} استفاده کرد که خود بسترهایی برای
گسترش خودکار، خطاپذیری و \نقاط‌خ را فراهم می‌آورد
\مرجع{Botez2021}.
