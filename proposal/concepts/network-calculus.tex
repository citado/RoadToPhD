\قسمت{\متن‌لاتین{Network Calculus}}

یکی از مسائل مطرح در اینترنت اشیا بدست آوردن کران‌های بالا در جهت تضمین پارامترهای کیفیت سرویس است. به طور مثال
قصد داریم بیشترین تاخیری که ممکن است برای یک بسته رخ دهد را بدست آوریم.
ابزار \متن‌لاتین{Network Calculus} در چنین مسائلی می‌تواند کمک کننده باشد و از سوی دیگر مسائل شبکه‌های اینترنت اشیا
بعد از لایه‌ی پیوند داده را می‌توان به عنوان مسائل جریان مدل‌سازی کرد.

\متن‌لاتین{Network Calculus} مجموعه‌ای از پیشرفت‌های اخیر است که دید عمیقی در مساله‌های جریان در شبکه‌ها ایجاد می‌کند.
پایه \متن‌لاتین{Network Calculus} در تئوری ریاضی \متن‌لاتین{Dioid}ها و مشخصا \متن‌لاتین{Min-Plus dioid} نهفته است.
در ادامه به مرور مفاهیم اصلی این حوزه می‌پردازیم.

\زیرقسمت{منحنی ورودی}

جریان با تابع تجمعی $R(t)$، دارای $\alpha$ به عنوان جریان ورودی (بیشین) است اگر:

\[
  R(t) - R(s) \le \alpha(t - s) \forall t,s \ge 0
\]

که در آن $\alpha$ یک تابع صعودی است. به عنوان مثال اگر فرض کنیم جریان ورودی با الگوریتم \متن‌لاتین{Leaky Bucket} با پارامترهای $r$ و $b$، محدود شده است داریم:

\[
  \alpha(t) = rt + b
\]

جریان‌های ورودی را می‌توان با یکدیگر جمع کرد.

\زیرقسمت{پیچش \متن‌لاتین{Min-Plus}}

پیچش دو جریان $f_{1}$ و $f_{2}$ در جبر \متن‌لاتین{Min-Plus} به شکل زیر تعریف می‌شوند:

\[
  f(t) = \inf_{s \ge 0}(f_{1}(s) + f_{2}(t-s))
\]
\[
  f = f_{1} \otimes f_{2}
\]

این پیچش، ویژگی‌ها خوب پیچیش معمول را دارد:

\[
  (f_{1} \otimes f_{2}) \otimes f_{3} = f_{1} \otimes (f_{2} \otimes f_{3})
\]
\[
  f_{1} \otimes f_{2} = f_{2} \otimes f_{1}
\]

با توجه به این تعریف می‌توان گفت $\alpha$ یک منحنی ورودی برای $R$ خواهد بود اگر و تنها اگر

\[
  R \le R \otimes \alpha
\]

