\قسمت{مقدمه}

اینترنت اشیا اولین بار توسط \متن‌لاتین{Kevin Ashton} در سال ۱۹۹۹ پیشنهاد شد، او اینترنت اشیا را به عنوان
شبکه‌ای از اشیا هم‌کنشپذیر و قابل شناسایی به وسیله‌ی فرکانس رادیویی (\متن‌لاتین{RFID}) تعریف می‌کند.
کلمه‌های ``اینترنت'' و ``اشیا'' به معنی شبکه‌ای متصل و جهانی بر پایه حسگرها، ارتباطات، شبکه‌سازی و فناوری‌های پردازش داده است
که نسخه‌ی جدیدی از فناوری اطلاعات یا \متن‌لاتین{ICT} را عرضه می‌کند.
پیشرفت فناوری‌های حسگر بی‌سیم کمک شایانی به قابلیت‌های دستگاه‌های حسگر و بنابراین مفهوم پایه اینترنت اشیا کرده است.
اهمیت اینترنت اشیا برای کشورهای در حال توسعه و توسعه یافته بسیار واضح است و این کشورها شروع به سرمایه‌گذاری در این حوزه نموده‌اند
\مرجع{Li2014}.

تولد حقیقی اینترنت اشیا با توجه به تخمین \متن‌لاتین{Cisco} به بازه‌ای بین سال‌های ۲۰۰۸ تا ۲۰۰۹ بازمی‌گردد که برای اولین بار تعداد اشیا
متصل از جمعیت جهان در آن سال‌ها بیشتر شد. در سال ۲۰۱۰ تعداد اشیا متصل تقریبا دو برابر جمعیت جهان در آن سال شد و تقریبا به عدد ۱۲/۵ بیلیون رسید.
از آن سال‌ها به لطف پیشرفت‌های فناوری و سرمایه‌گذاری‌های قابل توجه شرکت‌ها، اینترنت اشیا در حال گسترش در زندگی روزمره است
\مرجع{Lombardi2021}.

با آمدن اینترنت اشیا و ارتباطات ماشین به ماشین انتظار می‌رود به زودی افزایش زیادی در تعداد گرهها دیده شود. پیش‌بینی می‌شود تا سال ۲۰۲۵ بیش از ۷۵ بیلیون گره اینترنت اشیا داشته باشیم.
این گرهها شامل ماشین‌ها، حسگرها، شمارشگرها، دستگاه‌های فروش و \نقاط‌خ است \مرجع{Chaudhari2020}.

یکی از بحث‌ها در اینترنت اشیا نبود استانداردها و تاثیر آن بر گسترش در این حوزه است. در این سال‌ها فعالیت‌های زیادی در حوزه طراحی استانداردها
صورت گرفته است. مهمترین این استانداردها برای میان‌افزارها و رابط‌ها بوده‌اند. این استانداردها که هر یک توسط گروه‌ها و ارگان‌های خاصی تهیه می‌شوند که
خود می‌بایست هماهنگی و ارتباط داشته باشند
\مرجع{Li2014}.

تحقیقات و پژوهش‌های متفاوت اینترنت اشیا را به شکل‌های مختلفی تعریف کرده‌اند که از جمله‌ی آن‌ها می‌توان به
\شروع{نقل}
یک زیرساخت شبکه‌ای پویا و جهانی با قابلیت‌های خودپیکربندی بر پایه پروتکل‌های ارتباطی استاندارد و هم‌کنشپذیر که در آن اشیا فیزیکی و مجازی دارای شناسه، صفات فیزیکی
و شخصیت مجازی بوده، از رابط‌های هوشمند استفاده می‌کنند و به صورت یکپارچه در شکبه اطلاعاتی مجمتع می‌شوند.
\پایان{نقل}

\شروع{نقل}
یک زیرساخت جهانی برای اجماع اطلاعاتی که سرویس‌های پیشرفته‌ای را با برقراری ارتباط میان اشیا فیزیکی و مجازی بر پایه فناوری‌های حاضر، هم‌کنشپذیر و رو به گسترش اطلاعاتی
و ارتباطی فراهم می‌آورد.
\پایان{نقل}

\شروع{نقل}
اینترنت اشیا به افراد و اشیا اجازه می‌دهد در هر زمان و مکانی و با استفاده از هر فرد یا وسیله‌ای به طریق هر شبکه و مسیری و با هر سرویسی به یکدیگر متصل شوند.
\پایان{نقل}

اشاره کرد
\مرجع{Ray2018} \مرجع{Razzaque2016}.
با توجه به آغاز اینترنت اشیا از \متن‌لاتین{RFID}ها بسیاری از پژوهش‌ها رشد این فناوری در بستر \متن‌لاتین{RFID} را مورد بحث قرار داده‌اند
\مرجع{Ray2018}.

در سال‌های اخیر با کاهش قسمت حسگرهای و عملگرها، تعداد دستگاه‌های اینترنت اشیا به سرعت در حال گسترش است و به سرعت در حال تبدیل کردن خود به یکی از اجزا زندگی ما هستند.
در نتیجه رد پای فاحش اینترنت اشیا امروزه در همه جا قابل مشاهده است
\مرجع{Mishra2021}.
