\قسمت{معماری شبکه‌های \متن‌لاتین{LPWAN}}

معماری معمول شبکه‌های \متن‌لاتین{LPWAN} در شکل \رجوع{شکل: معماری معمول شبکه‌های LPWAN} آمده است.
کارکرد پایه یک دستگاه \متن‌لاتین{LPWAN} جمع‌اوری داده و پاسخ به ورودی‌های دریافتی از شبکه است.
داده‌های جمع‌اوری شده در یک لینک رادیویی مشخص برای ایستگاه دسترسی بی‌سیم ارسال می‌گردد.
ایستگاه بی‌سیم لینک رادیویی، برای مدیریت دستگاه و تبادل اطلاعات فراهم می‌آورد.
این ایستگاه در ارتباط با \دروازه یا \متن‌لاتین{Concentrator} قرار دارد که در برخی از موارد هسته نامیده می‌شود.
هسته وظیفه پشتبانی از لایه‌های کاربر و کنترل را دارد و همچنین وظیفه تبدیل پروتکل‌های قابل فهم برای شبکه و برنامه‌های کاربردی نیز از وظایف هسته است
\مرجع{Chaudhari2020}.

به خاطر نزدیکی \دروازه به اشیا در برخی از موارد از آن برای پردازش در لبه در کاربردهای همزمان استفاده می‌گردد.
از سوی دیگر پردازش و ذخیره‌سازی در لبه می‌توان بار پردازش ابری را کاهش دهد. در برخی از تکنولوژی‌های
از \دروازه برای کنترل درخواست و اولویت‌دهی نیز استفاده می‌گردد
\مرجع{Chaudhari2020}.

\شروع{شکل}
\درج‌تصویر[width=\textwidth]{./img/lpwan-arch.png}
\تنظیم‌ازوسط
\شرح{معماری معمول شبکه‌های \متن‌لاتین{LPWAN} \مرجع{Chaudhari2020}}
\برچسب{شکل: معماری معمول شبکه‌های LPWAN}
\پایان{شکل}

در معماری مرسوم، دستگاه به طور مستقیم با شبکه‌ی \متن‌لاتین{LPWAN} در ارتباط است.
تنظیمات دسترسی دیگری را نیز می‌توان در نظر گرفت. دو نمونه از معروف‌ترین آن‌ها
در ادامه آمده است.
معماری اول حالتی است که ارتباط اشیا از طریق تکنولوژی‌هایی مانند \متن‌لاتین{Zigbee}،
\متن‌لاتین{WiFi} و \نقاط‌خ فراهم می‌آید. \دروازه متناظر این اشیا از طریق
شبکه \متن‌لاتین{LPWAN} متصل می‌شود.
معماری دوم حالتی است که اشیا از چند شبکه \متن‌لاتین{LPWAN} به صورت همزمان استفاده می‌کنند
\مرجع{Chaudhari2020}.
