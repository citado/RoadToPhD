\فصل{شبیه‌سازی}

در راستای ارزیابی انتها به انتها یک سیستم \متن‌لاتین{LoRa} یکی از راه‌های مورد استفاده،
بهره‌گیری از شبیه‌سازی است. شبیه‌سازی مدل‌سازی کامپیوتری یک راه‌حل مناسب برای بهبود تجربه
کاوش در کارایی سیستم و ارزیابی تکنیک‌های عملکردی آن در روش‌های مورد پیش‌بینی یا تصور است
\مرجع{Almuhaya2022}.

در این فصل قصد داریم مروری به روش‌های شبیه‌سازی برای یک سیستم انتها به انتها
\متن‌لاتین{LoRa} داشته باشیم.

\قسمت{ابزارهای شبیه‌سازی \متن‌لاتین{LoRaWAN}}

در حوزه \متن‌لاتین{LoRa} شبیه‌سازهای بسیار تخصصی و رایگان وجود دارند.
این شبیه‌سازهای \متن‌لاتین{LoRa} در این حوزه تحقیقاتی برای ارزیابی سناریوهای مختلف \متن‌لاتین{LoRa}
توسعه یافته و مورد استفاده قرار گرفته‌اند
\مرجع{Almuhaya2022}.

\زیرقسمت{\متن‌لاتین{LoRaSIM}}

\متن‌لاتین{LoRaSIM} بر پایه \متن‌لاتین{SimPy} به عنوان یک شبیه‌ساز رویداد گسسته به وسیله‌ی پایتون برای
شبیه‌سازی، تحقیق و آنالیز گسترش‌پذیری و کارکرد تصادم در شبکه‌های \متن‌لاتین{LoRaWAN} توسعه پیدا کرده است.
این شبیه‌ساز رابط گرافیکی نداشته اما می‌تواند نمودارها و اشکال را برای نمایش با سایر ابزارها خروجی دهد.
مدل انتشار رادیویی بر پایه مدل‌های شناخته‌شده \متن‌لاتین{path-loss} در مسیرهای طولانی در \متن‌لاتین{LoRaSIM}
پیاده‌سازی شده است
\مرجع{Almuhaya2022}.

\زیرقسمت{ns-3}

\متن‌لاتین{ns-3} یک شبیه‌ساز شبکه رویداد گسسته است که در اواسط سال ۲۰۰۶
به زبان‌های \متن‌لاتین{C++} و \متن‌لاتین{python} ایجاد شده اما هنوز هم به شدت در حال توسعه است.
