\فصل{مسائل پیشنهادی}

\قسمت{مقدمه}

همانطور که بیان شد مسائل مختلفی در حوزه شبکه‌های توان پایین با رنج بالا مطرح می‌باشد و ما در این رساله قصد داریم روی مساله ارزیابی کارایی تمرکز کنیم.
حوزه اصلی که انتخاب کردیم، حوزه‌های:

\شروع{فقرات}
\فقره صحن دانشگاه هوشمند یا \متن‌لاتین{Smart Campus}
\فقره کشاورزی هوشمند یا \متن‌لاتین{Smart Farming}
\پایان{فقرات}

است.

در جدول \رجوع{جدول: شاخصه‌ها و کاربردها} ارتباط میان شاخصه‌ها و کاربردهای اینترنت اشیا آورده شده است.
در شهر هوشمند ما نیاز به پوشش‌دهی بالا، تعداد حسگر و عملگرهای بالا، هزینه‌ی کم و مصرف توان معقول داریم چرا
که شهر هوشمند از نظر گسترگی نیاز به تعداد زیادی شی و پوشش بالا دارد ولی به علت وجود زیرساخت توزیع انرژی مناسب
اشیا می‌توانند منبع انرژی مستقل داشته باشند و یا تعویض باتری‌های آن‌ها در بازه‌های زمانی کوتاه صورت بپذیرد.
از سوی دیگر در کشاورزی هوشمند می‌توان میزان بیشتری برای اشیا هزینه کرد اما با توجه به نبود زیرساخت مناسب
اشیا می‌بایست برای مدت‌های طولانی به باتری وابسته باشند.

\begin{table}
\label{جدول: شاخصه‌ها و کاربردها}
\caption{ارتباط شاخصه‌ها با حوزه‌ها به ترتیب $H$ ارتباط زیاد، $M$ ارتباط متوسط و $L$ ارتباط کم \مرجع{Chaudhari2020}}
\begin{latin}\begin{tabularx}
  {\textwidth}
  {|*{6}{X|}}
  \toprule
  Applications &
  Coverage &
  Capacity &
  Cost &
  Low Power &
  Enhanced Characteristics \\
  \midrule
  Smart Cities &
  H &
  H &
  H &
  M &
  H \\
  \midrule
  Smart Environment &
  M &
  H &
  H &
  H &
  M \\
  \midrule
  Smart Water &
  H &
  M &
  M &
  M &
  L \\
  \midrule
  Automotives and Logistics &
  H &
  H &
  M &
  L &
  H \\
  \midrule
  Smart Agriculture and Farming &
  H &
  H &
  M &
  H &
  L \\
  \bottomrule
\end{tabularx}\end{latin}
\end{table}

\قسمت{مسائل پیشنهادی}

\زیرقسمت{ارزیابی کارآیی \متن‌لاتین{Backbone} مبتنی بر \متن‌لاتین{IP} شبکه‌های \متن‌لاتین{LoRaWAN}}

همانطور که بیان شد در شبکه‌های \متن‌لاتین{LoRaWAN} یک ساختار مشخص به عنوان زیرساخت \متن‌لاتین{IP} وجود دارد.
این زیرساخت تاثیر زیادی بر نرخ داده، حجم داده، تعداد حسگرها و \نقاط‌خ دارد. پیش‌بینی منابع مورد نیاز و معماری آن از مراحل مهم طراحی یک شبکه \متن‌لاتین{LoRaWAN} می‌باشد.
در پژوهش \مرجع{sensors-20-06721} به مشکل از دست رفتن پیام‌ها در لایه \متن‌لاتین{IP} اشاره شده ولی به ان پرداخته نشده بود که خود نمایانگر اهمیت این موضوع می‌باشد.

یک ارتباط مهم در این معماری، ارتباط \متن‌لاتین{AS-hNS} می‌باشد که سامانه متن‌باز \متن‌لاتین{Chirpstack} و سایر سامانه‌ها با پروتکل \متن‌لاتین{MQTT} پیاده‌سازی شده است.
می‌توان با جایگزین کردن این قسمت با پروتکل \متن‌لاتین{NATS Jetstream} کارایی بهتری بدست آورد که البته خود نیاز به ارزیابی دقیق دارد.

در ادامه می‌توان به تاثیر رویه ارسال و نرخ ارسال داده‌ها برای کارآیی کلی شبکه هسته اشاره کرد.
اشیا می‌توانند به سه روش کلی زیر ارسال داده داشته باشند که هر یک نیازمندی‌های کیفیت سرویس خاص خود را ایجاد می‌کند.

\شروع{فقرات}
\فقره \متن‌لاتین{Periodic}
\فقره \متن‌لاتین{Self Trigger}
\فقره \متن‌لاتین{Trigger by Event}
\پایان{فقرات}

شبکه‌ی هسته می‌بایست بین بسته‌های دریافتی بسته‌های با الویت بیشتر را پردازش کند و الگوریتم‌زمان‌بندی داشته باشد و یکی از نیازمندی‌های در اعلان‌های مهم داشتن یک کران بالا برای تاخیر است.
چرا که در شبکه‌های اینترنت اشیا کلاس‌های مختلفی از داده می‌توانند حضور داشته باشند و به جز شبکه‌ی زیرساخت، شبکه‌ی هسته نیز در برآورده شدن کیفیت سرویس مورد نظر آن‌ها موثر است.

در نهایت می‌توان با استفاده از چهارچوب‌های ریاضی تئوری صف و \متن‌لاتین{Network Calculus} تاخیر میانگین و کران بالای تاخیر را بدست آورد.

\زیرقسمت{استفاده از زیرساخت \متن‌لاتین{IPv6} در شبکه‌های \متن‌لاتین{LoRaWAN}}

کارگروه \متن‌لاتین{lpwan} در بدنه استاندارسازی \متن‌لاتین{IETF} برای استفاده از \متن‌لاتین{IPv6} در شبکه‌های \متن‌لاتین{LoRaWAN} استانداردهای زیادی را منتشر کرده است.
این استانداردها می‌بایست ارزیابی شوند که پژوهش‌هایی به این امر پرداخته‌اند باید در نظر داشت موارد همچون تاثیر حرکت اشیا در کارایی این پروتکلها و استفاده از لایه‌های کاربرد متنوع همچون \متن‌لاتین{QUIC}، \متن‌لاتین{NATS Jetstream} و \نقاط‌خ هنوز نیاز به ارزیابی دارد.

\زیرقسمت{پیاده‌سازی یک پلتفرم با قابلیت شناسایی خودکار داده‌ها و اشیا}

یک لایه مهم در اینترنت اشیا، لایه اپلیکشن و پلتفرم می‌باشد. پس جمع‌اوری داده‌ها در شبکه‌های \متن‌لاتین{LoRa} نیاز به پردازش، دسته‌بندی، صحت‌سنجی و \نقاط‌خ
آن‌ها می‌رسد. دو مساله کلی در اینجا مطرح است، مدل داده‌ای سنسور و کدگذاری استفاده شده در داده‌های دریافتی. تلاش بر این است که این موضوع به صورت
یکسان و استاندارد شده صورت بگیرند و از این رو استانداردهای \متن‌لاتین{Semantic Definition Format} و \متن‌لاتین{Sensor Measurement List} به ترتیب برای مدل داده‌ای و کدگذاری مطرح شده‌اند.

مساله‌ای که به آن پرداخته نشده است، بحث تاثیر این دو استاندارد بر کارایی لایه‌های مختلف شبکه توان پایین با برد بالا می‌باشد که رساله حاضر قصد بررسی آن را دارد.

\زیرقسمت{پیاده‌سازی شبکه‌ \متن‌لاتین{Meshed Lora} بر پایه \متن‌لاتین{Nvidia Jetson} و استفاده از الگوریتم‌های یادگیری ماشین برای تشکیل گراف بهینه‌}

همانطور که بیان شد، یک مساله مورد بحث در شبکه‌های \متن‌لاتین{LoRa} استفاده از لایه فیزیکی به تنهایی و شکل دادن یک شبکه‌ی \متن‌لاتین{Mesh} می‌باشد.
یکی از بحث‌های مهم در این شبکه‌ها چگونگی تشکیل گراف و انتخاب پدر برای گره‌ها می‌باشد. برای این امر می‌توان از الگوریتم‌های یادگیری تقویتی عمیق استفاده کرد
و در جهت کاهش تغییر با استفاده از سخت‌افزار \متن‌لاتین{Nvidia Jetson} آن را در \متن‌لاتین{Gateway} و نزدیکی شبکه اجرا نمود.

\زیرقسمت{‌پیاده‌سازی الگوریتم‌های هوش‌مصنوعی و یادگیری عمیق در لایه کاربرد}

یکی از مسائل حوزه اینترنت اشیا که به آن توجه کمی شده است استفاده عملی از داده‌های جمع‌آوری شده در اینترنت اشیا است. می‌توان این داده‌ها را با استفاده از الگوریتم‌های یادگیری عمیق و هوش مصنوعی به داده‌های کاربردی تبدیل کرده
و چالش‌های موجود در حوزه‌های اینترنت اشیا را پاسخ داد.

\زیرقسمت{ارزیابی شبکه \متن‌لاتین{IEEE 802.11be}}

\متن‌لاتین{WiFi 7} و \متن‌لاتین{IEEE 802.11be} تکنولوژی‌های جدیدی هستند که به دنبال پشتیبانی از کاربردهای حساس به زمان می‌باشند. هنوز این تکنولوژی‌ها در عمل مورد سنجش قرار نگرفته‌اند.
سنجش آن‌ها می‌تواند نقاط ضعف و قوت آن‌ها را بهتر مشخص کند. استاندارد نهایی برای سال ۲۰۲۴ خواهد بود ولی اولین نسخه‌ی آن در سال ۲۰۲۲ منتشر خواهد شد. برای شبیه‌سازی و ارزیابی در این
مساله می‌بایست از نرم‌افزارهای شبیه‌ساز استفاده کرد.

\قسمت{پروژه‌های کارشناسی}

از آنجایی که در مساله پیشنهادی نیاز به پیاده‌سازی استانداردها و پروتکل‌هایی وجود دارد، این موارد می‌توانند در قالب پروژه کارشناسی تعریف شوند.

\شروع{فقرات}
\فقره پیاده‌سازی کتابخانه‌ی \متن‌لاتین{SenML} با زبان \متن‌لاتین{C} یا \متن‌لاتین{C++} بر پایه پروتکل \متن‌لاتین{CBOR}
\پایان{فقرات}
