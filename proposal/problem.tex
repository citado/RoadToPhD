\فصل{بیان مسائل و رویکردهای حل مساله}
\برچسب{فصل: بیان مسائل و رویکردهای حل مساله}

\قسمت{بیان مساله}

در این مساله قصد داریم $S$ فاکتورها گسترش جهت ارسال اطلاعات توسط $n$ حسگر را در $T$ بازه‌ی زمانی مشخص کنیم.
حسگرها در شروع هر بازه‌ی زمانی در صورتی که فاکتور گسترش به آن‌ها تخصیص پیدا کرده باشد داده‌ی جدیدی را ارسال می‌کنند.
در صورتی که شی در یک بازه‌ی زمانی ارسالی نداشته باشد عمر اطلاعات آن یک واحد افزایش می‌یابد.
هدف کمینه کردن عمر اطلاعات همه حسگر‌ها است.

\شروع{لوح}

\تنظیم‌ازوسط

\برچسب{جدول: معرفی متغیرهای تصمیم‌گیری}
\شرح{معرفی متغیرهای تصمیم‌گیری}

\شروع{جدول}{وپ{.5\پهنای‌کاغذ}}

متغیر تصمیم‌گیری & معرفی \\

\خط‌پر

$x_{it}$ & متغیر صحیح و نامنفی که عمر اطلاعات حسگر $i$ام در بازه‌ی زمانی $t$ را مشخص می‌کند. \\

$p_{i,s,t}$ & متغیر دودویی که مشخص می‌کند فاکتور گسترش $s$ در بازه‌ی زمانی $t$ به حسگر $i$ام تخصیص پیدا کرده است. \\

\پایان{جدول}

\پایان{لوح}


\شروع{لوح}

\تنظیم‌ازوسط

\برچسب{جدول: معرفی پارامترها}
\شرح{معرفی پارامترها}

\شروع{جدول}{وپ{.5\پهنای‌کاغذ}}

پارامتر & معرفی \\

\خط‌پر

$N$ & تعداد حسگرها \\

$T$ & تعداد بازه‌های زمانی \\

$S$ & تعداد فاکتورهای گسترش \\

\پایان{جدول}

\پایان{لوح}

در ادامه مدل‌سازی ریاضی این مساله بیان می‌شود:

\begin{align}
  \sum_{i = 0}^{N} p_{i, s, t} &= 1 \quad \forall c \in \{1, \cdots, S\}, \forall t \in \{1, \cdots, T\}
\end{align}
