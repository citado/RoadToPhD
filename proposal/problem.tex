\فصل{بیان مسائل و رویکردهای حل مساله}
\برچسب{فصل: بیان مسائل و رویکردهای حل مساله}

\قسمت{بیان مساله}

در این مساله قصد داریم $S$ فاکتور گسترش جهت ارسال اطلاعات توسط $n$ حسگر را در $T$ بازه‌ی زمانی تخصیص دهیم.
حسگرها در شروع هر بازه‌ی زمانی در صورتی که فاکتور گسترش به آن‌ها تخصیص پیدا کرده باشد داده‌ی جدیدی را ارسال می‌کنند.
در صورتی که شی در یک بازه‌ی زمانی ارسالی نداشته باشد عمر اطلاعات آن یک واحد افزایش می‌یابد.
هدف کمینه کردن میانگین عمر اطلاعات همه حسگر‌ها است.

\شروع{لوح}

\تنظیم‌ازوسط

\برچسب{جدول: معرفی متغیرهای تصمیم‌گیری}
\شرح{معرفی متغیرهای تصمیم‌گیری}

\شروع{جدول}{وپ{.5\پهنای‌کاغذ}}

متغیر تصمیم‌گیری & معرفی \\

\خط‌پر

$x_{it}$ & متغیر صحیح و نامنفی که عمر اطلاعات حسگر $i$ام در بازه‌ی زمانی $t$ را مشخص می‌کند. \\

$p_{i,s,t}$ & متغیر دودویی که مشخص می‌کند فاکتور گسترش $s$ در بازه‌ی زمانی $t$ به حسگر $i$ام تخصیص پیدا کرده است. \\

$\mu_{i,t}$ & متغیر پیوسته و نامنفی که توان ارسالی شی $i$ام در بازه‌ی زمانی $t$ را نشان می‌دهد. \\

\پایان{جدول}

\پایان{لوح}


\شروع{لوح}

\تنظیم‌ازوسط

\برچسب{جدول: معرفی پارامترها}
\شرح{معرفی پارامترها}

\شروع{جدول}{وپ{.5\پهنای‌کاغذ}}

پارامتر & معرفی \\

\خط‌پر

$N$ & تعداد حسگرها \\

$T$ & تعداد بازه‌های زمانی \\

$S$ & تعداد فاکتورهای گسترش \\

$P_{max}$ & حداکثر توان ارسالی حسگرها \\

$P_{min, s}$ & حداقل توان ارسالی حسگرها در صورت استفاده از فاکتور گسترش $s$ \\

\پایان{جدول}

\پایان{لوح}

در ادامه مدل‌سازی ریاضی این مساله بیان می‌شود:

\begin{align}
  \sum_{i = 1}^{N} p_{i, s, t} &\le 1 \quad \forall s \in \{1, \cdots, S\}, \forall t \in \{1, \cdots, T\} \label{eq:constr_subchannel_limit} \\
  \sum_{s = 1}^{S} p_{i, s, t} &\le 1 \quad \forall i \in \{1, \cdots, N\}, \forall t \in \{1, \cdots, T\} \label{eq:constr_thing_limit} \\
  -T * \sum_{s = 1}^{S} p_{i, s, t} + x_{i, t} + 1 &\le x_{i, t + 1} \quad \forall i \in \{1, \cdots, N\}, \forall t \in \{1, \cdots, T - 1\} \label{eq:constr_aoi_limit} \\
  \mu_{i,t} &\ge P_{min, s} * p_{i, s, t} \quad \forall i \in \{1, \cdots, N\}, \forall t \in \{1, \cdots, T\}, \forall s \in \{1, \cdots, S\} \label{eq:power_range_1} \\
  \mu_{i, t} &\le P_{max} * \sum_{s = 1}^{S} p{i, s, t} \quad \forall i \in \{1, \cdots, N\}, \forall t \in \{1, \cdots, T\} \label{eq:power_range_2}
\end{align}

\begin{align}
  \min \sum_{i = 1}^{N} \sum_{t = 1}^{T} x_{i, t} / T
\end{align}


محدودیت \رجوع{eq:constr_subchannel_limit} بیان می‌کند هر فاکتور گسترش در هر بازه‌ی زمانی تنها به یک حسگر می‌تواند تخصیص پیدا کند.
محدودیت \رجوع{eq:constr_thing_limit} بیان می‌کند در هر بازه‌ی زمانی حداکثر یک فاکتور گسترش می‌تواند به یک شی تخصیص پیدا کند.
محدودیت \رجوع{eq:constr_aoi_limit} بیان می‌کند مقدار عمر اطلاعات در صورت تخصیص نیافتن فاکتور گسترش به یک حسگر می‌بایست در بازه‌ی زمانی بعدی یک واحد افزایش پیدا کند.
محدودیت‌های \رجوع{eq:power_range_1} و \رجوع{eq:power_range_2} در صورت تخصیص یافتن فاکتور گسترش حداقل توان ارسالی را بر پایه فاکتور گسترش تخصیص یافته مشخص می‌کنند.
