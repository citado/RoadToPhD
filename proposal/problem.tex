\فصل{مسائل ییشنهادی}

\قسمت{مقدمه}

همانطور که بیان شد مسائل مختلفی در حوزه شبکه‌های توان پایین با رنج بالا مطرح می‌باشد و ما در این رساله قصد داریم روی مساله ارزیابی کارایی تمرکز کنیم.
حوزه اصلی که انتخاب کردیم، حوزه‌های:

\شروع{فقرات}
\فقره محوطه دانشگاه هوشمند - \متن‌لاتین{Smart Campus}
\فقره کشاورزی هوشمند - \متن‌لاتین{Smart Farming}
\پایان{فقرات}

است.

\قسمت{مسائل پیشنهادی}

\زیرقسمت{ارزیابی کارآیی \متن‌لاتین{Backbone} مبتنی بر \متن‌لاتین{IP} شبکه‌های \متن‌لاتین{LoRaWAN}}

همانطور که بیان شد در شبکه‌های \متن‌لاتین{LoRaWAN} یک ساختار مشخص به عنوان زیرساخت \متن‌لاتین{IP} وجود دارد.
این زیرساخت تاثیر زیادی بر نرخ داده، حجم داده، تعداد حسگرها و \نقاط‌خ دارد. پیش‌بینی منابع مورد نیاز و معماری آن از مراحل مهم طراحی یک شبکه \متن‌لاتین{LoRaWAN} می‌باشد.
در پژوهش \مرجع{sensors-20-06721} به مشکل از دست رفتن پیام‌ها در لایه \متن‌لاتین{IP} اشاره شده ولی به ان پرداخته نشده بود که خود نمایانگر اهمیت این موضوع می‌باشد.

یک ارتباط مهم در این معماری، ارتباط \متن‌لاتین{AS-hNS} می‌باشد که سامانه متن‌باز \متن‌لاتین{Chirpstack} و سایر سامانه‌ها با پروتکل \متن‌لاتین{MQTT} پیاده‌سازی شده است.
می‌توان با جایگزین کردن این قسمت با پروتکل \متن‌لاتین{NATS Jetstream} کارایی بهتری بدست آورد که البته خود نیاز به ارزیابی دقیق دارد.

در ادامه می‌توان به تاثیر رویه ارسال و نرخ ارسال داده‌ها برای کارآیی کلی شبکه هسته اشاره کرد.
اشیا می‌توانند به سه روش کلی زیر ارسال داده داشته باشند که هر یک نیازمندی‌های کیفیت سرویس خاص خود را ایجاد می‌کند.

\شروع{فقرات}
\فقره \متن‌لاتین{Periodic}
\فقره \متن‌لاتین{Self Trigger}
\فقره \متن‌لاتین{Trigger by Event}
\پایان{فقرات}

شبکه‌ی هسته می‌بایست بین بسته‌های دریافتی بسته‌های با الویت بیشتر را پردازش کند و الگوریتم‌زمان‌بندی داشته باشد و یکی از نیازمندی‌های در اعلان‌های مهم داشتن یک کران بالا برای تاخیر است.

\زیرقسمت{استفاده از زیرساخت \متن‌لاتین{IPv6} در شبکه‌های \متن‌لاتین{LoRaWAN}}

کارگروه \متن‌لاتین{lpwan} در بدنه استاندارسازی \متن‌لاتین{IETF} برای استفاده از \متن‌لاتین{IPv6} در شبکه‌های \متن‌لاتین{LoRaWAN} استانداردهای زیادی را منتشر کرده است.
این استانداردها می‌بایست ارزیابی شوند که پژوهش‌هایی به این امر پرداخته‌اند باید در نظر داشت موارد همچون تاثیر حرکت اشیا در کارایی این پروتکلها و استفاده از لایه‌های کاربرد متنوع همچون \متن‌لاتین{QUIC}، \متن‌لاتین{NATS Jetstream} و \نقاط‌خ هنوز نیاز به ارزیابی دارد.

\زیرقسمت{پیاده‌سازی یک پلتفرم با قابلیت شناسایی خودکار داده‌ها و اشیا}

یک لایه مهم در اینترنت اشیا، لایه اپلیکشن و پلتفرم می‌باشد. پس جمع‌اوری داده‌ها در شبکه‌های \متن‌لاتین{LoRa} نیاز به پردازش، دسته‌بندی، صحت‌سنجی و \نقاط‌خ
آن‌ها می‌رسد. دو مساله کلی در اینجا مطرح است، مدل داده‌ای سنسور و کدگذاری استفاده شده در داده‌های دریافتی. تلاش بر این است که این موضوع به صورت
یکسان و استاندارد شده صورت بگیرند و از این رو استانداردهای \متن‌لاتین{Semantic Definition Format} و \متن‌لاتین{Sensor Measurement List} به ترتیب برای مدل داده‌ای و کدگذاری مطرح شده‌اند.

مساله‌ای که به آن پرداخته نشده است، بحث تاثیر این دو استاندارد بر کارایی لایه‌های مختلف شبکه توان پایین با برد بالا می‌باشد که رساله حاضر قصد بررسی آن را دارد.

\زیرقسمت{پیاده‌سازی شبکه‌ \متن‌لاتین{Meshed Lora} بر پایه \متن‌لاتین{Nvidia Jetson} و استفاده از الگوریتم‌های یادگیری ماشین برای تشکیل گراف بهینه‌}

همانطور که بیان شد، یک مساله مورد بحث در شبکه‌های \متن‌لاتین{LoRa} استفاده از لایه فیزیکی به تنهایی و شکل دادن یک شبکه‌ی \متن‌لاتین{Mesh} می‌باشد.
یکی از بحث‌های مهم در این شبکه‌ها چگونگی تشکیل گراف و انتخاب پدر برای گره‌ها می‌باشد. برای این امر می‌توان از الگوریتم‌های یادگیری تقویتی عمیق استفاده کرد
و در جهت کاهش تغییر با استفاده از سخت‌افزار \متن‌لاتین{Nvidia Jetson} آن را در \متن‌لاتین{Gateway} و نزدیکی شبکه اجرا نمود.


\قسمت{پروژه‌های کارشناسی}

از آنجایی که در مساله پیشنهادی نیاز به پیاده‌سازی استانداردها و پروتکل‌هایی وجود دارد، آن‌ها را در قالب پروژه کارشناسی تعریف می‌کنیم.

\شروع{فقرات}
\فقره پیاده‌سازی کتابخانه‌ی \متن‌لاتین{SenML} با زبان \متن‌لاتین{C} یا \متن‌لاتین{C++} و \متن‌لاتین{CBOR}
\پایان{فقرات}
