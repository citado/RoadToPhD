\فصل{بیان مسائل و رویکردهای حل مساله}
\برچسب{فصل: بیان مسائل و رویکردهای حل مساله}

\قسمت{بیان مساله}

در این مساله قصد داریم $S$ فاکتور گسترش جهت ارسال اطلاعات توسط $n$ حسگر را در $T$ بازه‌ی زمانی تخصیص دهیم.
حسگرها در شروع هر بازه‌ی زمانی در صورتی که فاکتور گسترش به آن‌ها تخصیص پیدا کرده باشد داده‌ی جدیدی را ارسال می‌کنند.
در صورتی که شی در یک بازه‌ی زمانی ارسالی نداشته باشد عمر اطلاعات آن یک واحد افزایش می‌یابد.
این اشیا جابجا می‌شوند و مکان آن‌ها در هر بازه‌ی زمانی متفاوت است اما این مکان از پیش مشخص شده است.
با توجه به مشخص بود این مکان می‌توان حداقل فاکتورگسترش برای اشیا را مشخص کرد و در صورتی که بخواهیم آن شی ارسال داشته باشد
باید فاکتور گسترش بزرگ یا مساوی آن حداقل به شی تخصیص پیدا کند.
هدف کمینه کردن میانگین عمر اطلاعات همه حسگر‌ها است.

\شروع{لوح}

\تنظیم‌ازوسط

\برچسب{جدول: معرفی متغیرهای تصمیم‌گیری}
\شرح{معرفی متغیرهای تصمیم‌گیری}

\شروع{جدول}{وپ{.5\پهنای‌کاغذ}}

متغیر تصمیم‌گیری & معرفی \\

\خط‌پر

$x_{it}$ & متغیر صحیح و نامنفی که عمر اطلاعات حسگر $i$ام در بازه‌ی زمانی $t$ را مشخص می‌کند. \\

$p_{i,s,t}$ & متغیر دودویی که مشخص می‌کند فاکتور گسترش $s$ در بازه‌ی زمانی $t$ به حسگر $i$ام تخصیص پیدا کرده است. \\

$\mu{i,t}$ & توان مصرفی حسگر $i$ام در بازه‌ی زمانی $t$ \\

$SF_{i,t}$ & فاکتور گسترش انتخاب شده برای حسگر $i$ام در بازه‌ی زمانی $t$ \\

$N^{payload}_{i,t}$ & اندازه‌ی بسته حسگر $i$ام در بازه‌ی زمانی $t$ \\

\پایان{جدول}

\پایان{لوح}


\شروع{لوح}

\تنظیم‌ازوسط

\برچسب{جدول: معرفی پارامترها}
\شرح{معرفی پارامترها}

\شروع{جدول}{وپ{.5\پهنای‌کاغذ}}

پارامتر & معرفی \\

\خط‌پر

$N$ & تعداد حسگرها \\

$T$ & تعداد بازه‌های زمانی \\

$S$ & تعداد فاکتورهای گسترش \\

$P_{i, s, t}$ & توان مصرفی حسگر $i$ام در بازه‌ی زمانی $t$ در صورتی که از فاکتور گسترش $s$ استفاده کند \\

$PL_{i}$ & اندازه بسته حسگر $i$ام \\

$CR$ & نرخ کدگذاری \\

$BW$ & پهنای باند \\

$\tau$ & اندازه‌ی یک بازه‌ی زمانی \\

\پایان{جدول}

\پایان{لوح}

در ادامه مدل‌سازی ریاضی این مساله بیان می‌شود:

\begin{align}
  &\min \sum_{i = 1}^{N} \sum_{t = 1}^{T} x_{i, t} / T \\
  &s.t. \notag \\
  \sum_{i = 1}^{N} p_{i, s, t} &\le 1 \quad \forall s \in \{1, \cdots, S\}, \forall t \in \{1, \cdots, T\} \label{eq:constr_subchannel_limit} \\
  \sum_{s = 1}^{S} p_{i, s, t} &\le 1 \quad \forall i \in \{1, \cdots, N\}, \forall t \in \{1, \cdots, T\} \label{eq:constr_thing_limit} \\
  -T * \sum_{s = 1}^{S} p_{i, s, t} + x_{i, t} + 1 &\le x_{i, t + 1} \quad \forall i \in \{1, \cdots, N\}, \forall t \in \{1, \cdots, T - 1\} \label{eq:constr_aoi_limit} \\
  \mu_{i, t} &= \sum_{s = 1}^{S} p_{i, s, t} P_{i, s, t}  \quad \forall i \in \{i, \cdots, N\}, \forall t \in \{1, \cdots, T\} \label{eq:constr_power_range} \\
  SF_{i, t} &= \sum_{s = 1}^{S} s p_{i, s, t} \quad \forall i \in \{i, \cdots, N\}, \forall t \in \{1, \cdots, T\} \label{eq:constr_set_sp} \\
  N^{payload}_{i, t} &= 8 + \max\left[ ceil \left( \frac{8 PL_{i} 28 + 16}{4 SF_{i, t}} \right) \times (CR + 4), 0 \right]
                       \quad \forall i \in \{i, \cdots, N\}, \forall t \in \{1, \cdots, T\}
                       \label{eq:constr_payload} \\
  N^{payload}_{i, t} \times \frac{2^{SP_{i, t}}}{BW} &\le \tau \quad \forall i \in \{i, \cdots, N\}, \forall t \in \{1, \cdots, T\}
                                                       \label{eq:constr_time_slot_threshold} \\
\end{align}

محدودیت \رجوع{eq:constr_subchannel_limit} بیان می‌کند هر فاکتور گسترش در هر بازه‌ی زمانی تنها به یک حسگر می‌تواند تخصیص پیدا کند.
محدودیت \رجوع{eq:constr_thing_limit} بیان می‌کند در هر بازه‌ی زمانی حداکثر یک فاکتور گسترش می‌تواند به یک شی تخصیص پیدا کند.
محدودیت \رجوع{eq:constr_aoi_limit} بیان می‌کند مقدار عمر اطلاعات در صورت تخصیص نیافتن فاکتور گسترش به یک حسگر می‌بایست در بازه‌ی زمانی بعدی یک واحد افزایش پیدا کند.
اشیا می‌توانند از فاکتورهای گسترش متفاوتی استفاده کنند اما همانطور که بیان شد در فاکتورهای گسترش بالاتر نرخ ارسال کاهش پیدا کرده اما توان مصرفی کاهش پیدا می‌کند این در حالی است که
در فاکتورهای گسترش پایین نرخ ارسال بیشتری وجود داشته و توان مصرفی بالاتر خواهد بود. محدودیت \رجوع{eq:constr_power_range} ارتباط میان توان مصرفی و فاکتور گسترش را در فاصله‌ی
معین شده در آن بازه‌ی زمانی برای حسگر، مشخص می‌کند.
اندازه‌ی هر بازه‌ی زمانی ثابت و برابر با $\tau$ است و برای اینکه تداخلی میان اشیا به وجود نیاید، ارسال داده هر شی باید در این بازه زمانی به اتمام برسد.
محدودیت \رجوع{eq:constr_time_slot_threshold} این مورد را تضمین می‌کند.
