\فصل{بیان مسائل}

همانطور که بیان شد مسائل مختلفی در حوزه شبکه‌های توان پایین با رنج بالا مطرح می‌باشد و ما در این رساله قصد داریم روی مساله ارزیابی کارایی انتها به انتها تمرکز کنیم.
از آنجایی که اینترنت اشیا در حوزه‌های گوناگونی مطرح می‌شود که هر یک از این حوزه‌ها نیازمندی‌های خاص خود را دارند، پیش از بیان مسائل حوزه‌هایی که قصد داریم
مسائل را در آن‌ها بررسی کنیم بیان می‌کنیم. در ادامه شاخصه‌های مهم در حوزه‌های اینترنت اشیا را مرور می‌کنیم و در نهایت به بیان مساله و اجزای آن می‌پردازیم.

حوزه‌های اصلی که انتخاب کردیم، حوزه‌های:

\شروع{فقرات}
\فقره صحن دانشگاه هوشمند یا \متن‌لاتین{Smart Campus}
\فقره کشاورزی هوشمند یا \متن‌لاتین{Smart Farming}
\پایان{فقرات}

است. همانطور که در پروژهش \مرجع{Mekki2019} و سایر پژوهش‌ها آمده است در کشاورزی هوشمند با توجه به قرارگیری در محل‌های غیرشهری امکان استفاده
از \متن‌لاتین{NB-IoT} وجود نداشته و \متن‌لاتین{LoRa} یکی از بهترین گزینه‌ها است.
در رابطه با دانشگاه‌ها و شهرهای هوشمند، نیز یکی از شبکه‌های پر استفاده با توجه به راه‌اندازی آسان و هزینه پایین آن شبکه‌های \متن‌لاتین{LoRa} هستند
که می‌توانند در کنار سایر شبکه‌ها نیز مورد استفاده قرار بگیرند.

در جدول \رجوع{جدول: شاخصه‌ها و کاربردها} ارتباط میان شاخصه‌ها و کاربردهای اینترنت اشیا آورده شده است.
در شهر هوشمند ما نیاز به پوشش‌دهی بالا، تعداد حسگر و عملگرهای بالا، هزینه‌ی کم و مصرف توان معقول داریم چرا
که شهر هوشمند از نظر گسترگی نیاز به تعداد زیادی شی و پوشش بالا دارد ولی به علت وجود زیرساخت توزیع انرژی مناسب
اشیا می‌توانند منبع انرژی مستقل داشته باشند و یا تعویض باتری‌های آن‌ها در بازه‌های زمانی کوتاه صورت بپذیرد.
از سوی دیگر در کشاورزی هوشمند می‌توان میزان بیشتری برای اشیا هزینه کرد اما با توجه به نبود زیرساخت مناسب
اشیا می‌بایست برای مدت‌های طولانی به باتری وابسته باشند.

\begin{table}
\label{جدول: شاخصه‌ها و کاربردها}
\caption{ارتباط شاخصه‌ها با حوزه‌ها به ترتیب $H$ ارتباط زیاد، $M$ ارتباط متوسط و $L$ ارتباط کم \مرجع{Chaudhari2020}}
\begin{latin}\begin{tabularx}
  {\textwidth}
  {|*{6}{X|}}
  \toprule
  Applications &
  Coverage &
  Capacity &
  Cost &
  Low Power &
  Enhanced Characteristics \\
  \midrule
  Smart Cities &
  H &
  H &
  H &
  M &
  H \\
  \midrule
  Smart Environment &
  M &
  H &
  H &
  H &
  M \\
  \midrule
  Smart Water &
  H &
  M &
  M &
  M &
  L \\
  \midrule
  Automotives and Logistics &
  H &
  H &
  M &
  L &
  H \\
  \midrule
  Smart Agriculture and Farming &
  H &
  H &
  M &
  H &
  L \\
  \bottomrule
\end{tabularx}\end{latin}
\end{table}

\قسمت{بیان مسائل}

مساله اصلی ما ارزیابی کارایی انتها به انتها یک شبکه \متن‌لاتین{LoRa} است، این شبکه اجزای مختلفی دارد که هر یک از این اجزا در کارایی کلی سیستم تاثیر دارند.
معماری کلی سیستمی که قصد ارزیابی آن را داریم در شکل \رجوع{شکل: معماری انتها به انتها سیستم مورد ارزیابی} آورده شده است.
از پایین‌ترین سطح شبکه‌ی دسترسی مبتنی بر تکنولوژی \متن‌لاتین{LoRa} وجود دارد، در لایه فیزیکی این شبکه می‌توان بین \متن‌لاتین{LoRa}، \متن‌لاتین{FSK}
و \متن‌لاتین{LR-FHSS} انتخاب کرد. در لایه پیوند داده در کنار \متن‌لاتین{LoRaWAN} می‌توان از نسخه‌های تغییریافته آن یا حتی شبکه \متن‌لاتین{Meshed LoRa} نیز استفاده کرد.
قسمت بعدی در این معماری \متن‌لاتین{Gateway}ها هستند که می‌توانند بخشی از پردازش لبه را نیز انجام بدهند. در صورتی که در لایه‌ی شبکه بخواهیم از پروتکل \متن‌لاتین{IPv6} استفاده کنیم
\متن‌لاتین{Gateway} می‌تواند یکی از انتخاب‌ها برای تبدیل پروتکل به نسخه اصلی \متن‌لاتین{IPv6} باشد.
در ادامه \متن‌لاتین{Network Server} یا سرور شبکه قرار دارد. در این مرحله انتخاب پیاده‌سازی خود سرور شبکه و از سوی دیگر لینک ارتباطی آن با
\متن‌لاتین{Gateway}ها انتخاب‌های زیادی را ایجاد می‌کند. از سوی دیگر سرور شبکه خود الگوریتم‌های مهمی از جمله نرخ داده تطبیق‌پذیر، احراز هویت، شناسایی و حذف بسته‌های تکراری را شامل می‌شود
که پیاده‌سازی و انتخاب تاثیر زیادی در کارایی شبکه خواهد داشت. انتخاب مرسوم‌تر برای تبدیل پروتکل \متن‌لاتین{IPv6} سرور شبکه است چرا که در این مرحله داده‌های تکراری دریافت شده از
\متن‌لاتین{Gateway}ها حذف شده است.
لایه بعدی سرور اپلیکشن است، برای ارتباط سرور اپلیکشن با سرور شبکه انتخاب‌های زیادی دارد که البته به پیاده‌سازی انتخابی برای این سرور شبکه و اپلیکیشن نیز وابسته است
و ممکن است قابل تغییر نباشد. کارایی سرور اپلیکیشن در رمزگشایی و از سوی دیگر تبدیل‌های پروتکلی لازم می‌تواند تاثیر به سزایی در کارایی سیستم داشته باشد.
سرور اپلکیشن در نهایت داده را در اختیار پلتفرم قرار می‌دهد. پلتفرم‌های زیادی موجود هستند که هر یک ویژگی‌های متفاوتی دارند.
این پلتفرم‌ها می‌توانند به صورت متن‌باز یا تجاری باشند و برخی نسخه‌ی ابری نیز دارند.
یکی از قسمت‌های مهم
در پلتفرم چگونگی پردازش داده‌های دریافتی و مدل داده‌ای اشیا است. در این زمینه تلاش‌های زیادی برای استاندارد سازی صورت پذیرفته است که می‌توان از
جمله‌ی آن‌ها به \متن‌لاتین{OneDM} از \متن‌لاتین{ETSI} و \متن‌لاتین{SenML} از \متن‌لاتین{IETF} اشاره کرد. استفاده از این استانداردها یکپارچگی
بیشتری فراهم می‌آورد و البته می‌تواند سربار هم داشته باشد و این مساله را می‌توان در این ارزیابی انتها به انتها مورد توجه قرار داد.
از سوی دیگر پلتفرم‌ها عموما سرویس‌هایی مانند اکتشاف، عیب‌یابی و \نقاط‌خ را نیز ارائه می‌دهند که یک مدل اطلاعاتی مشترک می‌تواند باعث شود تا آن‌ها
بتوانند اشیای متنوع‌تری را پشتیبانی کنند.
ذکر این نکته خالی از لطف نیست که مساله مدل اطلاعاتی تنها در لایه پلتفرم مطرح نیست، در صورتی که از پردازش لبه یا پردازش مِه استفاده شود نودهای پردازشی این لایه
هم برای همکنش‌پذیری به یک مدل داده‌ای و اطلاعاتی یکسان، نیازمند هستند.
در نهایت داده‌های جمع‌اوری و پردازش شده در پلتفرم در کنار سایر سرویس‌ها مانند مدیریت اشیا در اختیار کاربران و سایر برنامه‌های کاربردی قرار می‌گیرد.
برنامه‌های کاربردی می‌توانند نیازمندی‌های متفاوتی از جمله نیازمندی‌های کیفیت سرویس داشته باشند و در این ارزیابی انتها به انتها می‌توانیم این نیازمندی‌ها را
نیز مورد توجه قرار دهیم.

این رساله قصد ارزیابی کلی و جزئی همه این قسمت‌ها را دارد. در هر قسمت انتخاب‌های مختلفی وجود دارد که هر یک تاثیرات منحصر به فرد خود را دارند،
این رساله برای هر یک از این انتخاب‌ها در صورت وجود به پژوهش‌های پیشین ارجاع داده و در صورت لزوم و عدم پوشش کافی توسط کارهای پیشین، ارزیابی‌های لازم را صورت می‌دهد.
این ارزیابی‌ها در قالب شبیه‌سازی و پیاده‌سازی واقعی صورت خواهند پذیرفت و نتایج بدست آمده با آنچه که می‌توان از طریق تئوری مانند تئوری صف یا \متن‌لاتین{Network Calculus}
بدست آورد مقایسه می‌شوند. تئوری صف به ما برای محاسبه حالت میانگین کمک خواهد کرد و \متن‌لاتین{Network Calculus} از سوی دیگر برای یافتن کران‌های بالا کاربردی است.
اگر بخواهیم در رابطه با ابزارهای تئوری دقیقتر صحبت کنیم، ارسال داده توسط اشیا به صورت تصادفی است و ترافیک ماهیت تصادفی دارد بنابراین تئوری صف ابزار مناسبی برای تحلیل ترافیک به نظر می‌رسد،
از سوی دیگر همانطور که بیان شد سیستم از اجزای مختلفی تشکیل شده است که هر یک پردازش منحصر به فرد خود را دارند، \متن‌لاتین{Network Calculus} بیان می‌کند در صورت مدل سازی هر جز
می‌توانیم با استفاده از پیچش \متن‌لاتین{min-plus} مدل نهایی سیستم را بدست آوریم.

\شروع{sidewaysfigure}
\درج‌تصویر[width=\textwidth]{./img/e2e-iot-lorawan.png}
\تنظیم‌ازوسط
\شرح{معماری انتها به انتها سیستم مورد ارزیابی و معرفی کوتاه هر قسمت}
\برچسب{شکل: معماری انتها به انتها سیستم مورد ارزیابی}
\پایان{sidewaysfigure}

پارامترهای مورد ارزیابی به قرار زیر خواهند بود:

\شروع{فقرات}
\فقره تاخیر
\فقره نرخ دریافت بسته‌ها
\فقره گذردهی
\فقره نرخ تصادم
\فقره توان مصرفی
\پایان{فقرات}

از سوی دیگر پارامترهای موثر به شرح زیر خواهند بود:

\شروع{فقرات}
\فقره نرخ ارسال اشیا
\فقره تعداد اشیا
\فقره جابجایی اشیا و مدل جابجایی
\فقره قرارگیری اشیا در محیط بسته یا باز
\فقره قرارگیری اشیا در محیط شهری، نیمه‌شهری و غیرشهری
\فقره نیازمندی‌های کیفیت سرویس اشیا
\فقره تنوع اشیا، ساختار داده‌ها و مدل اطلاعاتی
\فقره پروتکل‌های ارتباطی
\فقره الگوریتم‌های رمزگذاری، نرخ‌داده تنظیم‌پذیر و \نقاط‌خ
\فقره شیوه استقرار سرویس‌های نرم‌افزاری (استفاده از ماشین‌های فیزیکی، مجازی یا سرورهای ابری)
\فقره زبان برنامه‌نویسی
\فقره معماری نرم‌افزاری
\فقره خرابی برخی از نودهای سیستم در معماری توزیع‌شده
\پایان{فقرات}

نگاه دیگر به این سیستم نگاه لایه‌ای خواهد بود. در نگاه لایه‌ای بسته در شبکه‌ی دسترسی با لایه‌ی فیزیکی \متن‌لاتین{LoRa} بوده و با ورود به لایه هسته،
لایه فیزیکی آن تغییر کرده و می‌تواند لایه‌های بالاتر آن نیز تغییر کرده یا بدون تغییر باقی بماند. برای این لایه‌های بالاتر انتخاب‌های مختلفی وجود دارد.
خلاصه‌ای از این انتخاب‌ها در شکل \رجوع{شکل: مدل انتها به انتها لایه‌ای داده‌ّها} دیده می‌شود.
تبدیل پروتکل در لایه هسته می‌تواند بیش از یکبار هم صورت بپذیرد که در این صورت مدل‌های لایه‌ای بیشتری خواهیم داشت. هر یک از این تبدیل‌ها
سربار دارد و انتخاب مناسب پروتکل‌ها و تبدیل‌ها می‌تواند تاثیر زیادی بر کارایی سیستم داشته باشد.

یکی از انتخاب‌های مطرح، استفاده از \متن‌لاتین{IPv6} و \متن‌لاتین{UDP} در لایه دسترسی و تبدیل آن به پروتکل اصلی در لایه هسته است. با این روش
اشیا عملا به صورت مستقبم به شبکه اینترنت متصل شده و با آدرس \متن‌لاتین{IP} مورد دسترسی قرار می‌گیرند.

\شروع{شکل}
\درج‌تصویر[width=\textwidth]{./img/e2e-iot-layered.png}
\تنظیم‌ازوسط
\شرح{مدل انتها به انتها لایه‌ای داده‌ها}
\برچسب{شکل: مدل انتها به انتها لایه‌ای داده‌ّها}
\پایان{شکل}

\زیرقسمت{ارزیابی کارایی سرورهای \متن‌لاتین{MQTT} و \متن‌لاتین{NATS}}

با توجه به گسترش روزافزون اینترنت اشیا و تعداد اشیا متصل، نیاز به پلتفرم‌های گسترش پذیر برای مدیریت ارتباط این اشیا و جمع‌اوری پیام‌های آن‌ها
از مسائل مهم در این حوزه است. یکی از اجزا مهم در همه پلتفرم‌های اینترنت اشیا و معماری مورد ارزیابی این رساله دلال‌های پیام هستند. یکی از
معروفترین پروتکل‌های انتشار و اشتراک پروتکل \متن‌لاتین{MQTT} است ولی پروتکل‌های دیگری با این معماری مانند \متن‌لاتین{NATS} یا
\متن‌لاتین{CoAP} هم در این حوزه وجود دارند. حتی در رابطه با معماری مورد بحث این رساله سایر ساختارهای ارتباطی مانند درخواست و پاسخ
هم می‌توانند مورد استفاده قرار بگیرند و برای رسیدن به کارایی بهینه می‌بایست مورد ارزیابی قرار بگیرند.

همانطور که پیشتر ارائه شد، پژوهش‌های زیادی به ارزیابی کارایی سرورهای \متن‌لاتین{MQTT} پرداخته‌اند اما پارامترهای زیادی مانند تعداد کانکشن‌های باز،
منابع مصرفی، ابرزی بودن و چگونگی گسترش‌پذیری در این سرورها مسائلی هستند که هنوز نیاز به ارزیابی دارد.
از سوی دیگر این سرورها با توجه به تنوع داده‌های ورودی نیاز به پشتیبانی
از اولویت‌ها دارند و بحث در رابطه با کران بالای تاخیر در آن‌ها بر پایه \متن‌لاتین{Network Calculus}
و استفاده از تئوری صف برای یافتن حالت میانه مطالبی است که پیشتر به آن‌ها پرداخته نشده است و می‌تواند پشتیبانی
از کیفیت سرویس‌های مختلف را به ارمغان بیاورد.

سرور و پروتکل \متن‌لاتین{NATS} که به صورت متن باز منتشر شده است نسبت به \متن‌لاتین{MQTT} ساده‌تر بوده و سربار کمتری دارد. ارزیابی آن
برای پیاده‌سازی در اشیا و مقایسه آن با \متن‌لاتین{MQTT} می‌تواند راهکار ساده‌تر و با سربار کمتری برای ارتباط در اینتنرت اشیا ارائه کند.
البته پروتکل \متن‌لاتین{NATS} تنها کیفیت سرویس ``صفر'' از پروتکل \متن‌لاتین{MQTT} پشتیبانی می‌کند و برای کیفیت سرویس بالاتر نیاز به
پروتکل \متن‌لاتین{NATS Jetstream} است که پیچیدگی بیشتری داشته اما کیفیت سرویس ``یک'' از \متن‌لاتین{MQTT} فراهم می‌آورد.
مقایسه این دو پروتکل در کیفیت سرویس‌های مختلف می‌تواند باعث انتخاب درستی شود که در نهایت کارایی انتها به انتها را بهبود ببخشد.

از سوی دیگر یک ارتباط مهم در معماری زیرساخت \متن‌لاتین{IP} در شبکه‌های \متن‌لاتین{LoRaWAN}،
ارتباط \متن‌لاتین{AS-hNS} است (ارتباط میان \متن‌لاتین{Network Server} و \متن‌لاتین{Application Server})
که در سامانه متن‌باز \متن‌لاتین{Chirpstack} و بیشتر سامانه‌ها با پروتکل \متن‌لاتین{MQTT} پیاده‌سازی شده است.
این ارزیابی به انتخاب پروتکل صحیح برای این ارتباط نیز کمک مهمی خواهد کرد.

\زیرقسمت{ارزیابی کارآیی \متن‌لاتین{Backbone} مبتنی بر \متن‌لاتین{IP} شبکه‌های \متن‌لاتین{LoRaWAN}}

همانطور که بیان شد در شبکه‌های \متن‌لاتین{LoRaWAN} یک ساختار مشخص به عنوان زیرساخت \متن‌لاتین{IP} وجود دارد.
این زیرساخت تاثیر زیادی بر نرخ داده، حجم داده، تعداد حسگرها و \نقاط‌خ دارد. پیش‌بینی منابع مورد نیاز و معماری آن از مراحل مهم طراحی یک شبکه \متن‌لاتین{LoRaWAN} می‌باشد.
در پژوهش \مرجع{sensors-20-06721} به مشکل از دست رفتن پیام‌ها در لایه \متن‌لاتین{IP} اشاره شد، که البته به آن پرداخته نشده بود، که خود نمایانگر اهمیت این موضوع می‌باشد.

در ادامه می‌توان به تاثیر رویه ارسال و نرخ ارسال داده‌ها برای کارآیی کلی شبکه هسته اشاره کرد.
اشیا می‌توانند به سه روش کلی زیر ارسال داده داشته باشند که هر یک نیازمندی‌های کیفیت سرویس خاص خود را ایجاد می‌کند.

\شروع{فقرات}
\فقره \متن‌لاتین{Periodic}: در این روش اشیا به صورت دوره‌ای داده را ارسال می‌کنند.
\فقره \متن‌لاتین{Self Triggered}: در این روش اشیا در فواصل زمانی گوناگون داده‌ای را ارسال می‌کنند که اهمیت بالایی داشته و می‌بایست در بازه زمانی مشخصی دریافت شود.
\فقره \متن‌لاتین{Trigger by Event}: در این روش اشیا به یک رویداد خارجی داده را ارسال می‌کنند.
\پایان{فقرات}

شبکه‌ی هسته می‌بایست بین بسته‌های دریافتی بسته‌های با اولویت بیشتر را پردازش کند و الگوریتم‌زمان‌بندی داشته باشد و یکی از نیازمندی‌ها، در اعلان‌های مهم، داشتن یک کران بالا برای تاخیر است.
چرا که در شبکه‌های اینترنت اشیا کلاس‌های مختلفی از داده می‌توانند حضور داشته باشند و به جز شبکه‌ی دسترسی، شبکه‌ی هسته نیز در برآورده شدن کیفیت سرویس مورد نظر آن‌ها موثر است.

در نهایت می‌توان با استفاده از چهارچوب‌های ریاضی تئوری صف و \متن‌لاتین{Network Calculus} که پیشتر به آن اشاره شد،
تاخیر میانگین و کران بالای تاخیر را برای اجزای مختلف این سیستم از جمله
سرور \متن‌لاتین{MQTT} بدست آورد و آن را با مقادیر بدست آمده در آزمایش‌های عملی
مقایسه کرد.

\زیرقسمت{استفاده از زیرساخت \متن‌لاتین{IPv6} در شبکه‌های \متن‌لاتین{LoRaWAN}}

کارگروه \متن‌لاتین{lpwan} در بدنه استاندارسازی \متن‌لاتین{IETF} برای استفاده از \متن‌لاتین{IPv6} در شبکه‌های \متن‌لاتین{LoRaWAN} استانداردهای زیادی را منتشر کرده است.
این استانداردها می‌بایست ارزیابی شوند که البته پژوهش‌هایی به این امر پرداخته‌اند. باید در نظر داشت موارد همچون تاثیر حرکت اشیا در کارایی این پروتکل‌ها و استفاده از لایه‌های کاربرد متنوع همچون \متن‌لاتین{QUIC}، \متن‌لاتین{NATS Jetstream} و \نقاط‌خ هنوز نیاز به ارزیابی دارد.
همانطور که اشاره شد استفاده از پروتکل‌های استاندارد نیاز به تبدیل پروتکل در قسمت‌های مختلف سیستم را کاهش می‌کند و می‌تواند به کارایی بیشتر سیستم کمک کند البته سربار حاصل از استفاده از این پروتکل‌ها نیازمند ارزیابی دقیق است.

\زیرقسمت{پیاده‌سازی یک پلتفرم با قابلیت شناسایی خودکار داده‌ها و اشیا}

یک لایه مهم در اینترنت اشیا، لایه اپلیکشن و پلتفرم می‌باشد. پس جمع‌اوری داده‌ها در شبکه‌های \متن‌لاتین{LoRa} نوبت به پردازش، دسته‌بندی، صحت‌سنجی و \نقاط‌خ
آن‌ها می‌رسد. دو مساله کلی در اینجا مطرح است، مدل داده‌ای سنسور و کدگذاری استفاده شده در داده‌های دریافتی. تلاش بر این است که این موضوع‌ها به صورت
یکسان و استاندارد شده صورت بگیرند و از این رو استانداردهای \متن‌لاتین{Semantic Definition Format} و \متن‌لاتین{Sensor Measurement List} به ترتیب برای مدل داده‌ای و کدگذاری مطرح شده‌اند.

مساله‌ای که به آن پرداخته نشده است، بحث تاثیر این دو استاندارد بر کارایی لایه‌های مختلف شبکه توان پایین با برد بالا است که رساله حاضر قصد بررسی آن را در شبکه‌ی \متن‌لاتین{LoRa} را دارد.
می‌توان با ارزیابی این موارد کارایی این استانداردها را مشخص کرده و از سوی دیگر راه را برای رسیدن به یک زیرساخت مشترک مدیریت یکسان اشیا هموار کرد.

از سوی دیگر همانطور که در مرور کارها به آن پرداخته شد، یکی از مسائلی که کارهای متنوعی روی آن انجام شده است پشتیبانی اشیا از چند شبکه ارتباطی و انتخاب شبکه‌ای با بهترین
پارامترها است. در این مساله مهم است که اشیا یک تعریف استاندارد داشته باشند تا بتوان در لایه پلتفرم آن‌ها را به صورت خودکار در سرورهای شبکه‌ای ثبت کرد.
و همانطور که در پژوهش \مرجع{Chen2019} آمده بود، یکی از چالش‌های اصلی برای این شبکه‌های ترکیبی چگونگی پیاده‌سازی آن‌ها بر پایه بسترهای موجود است.

\زیرقسمت{پردازش در لبه}

پردازش در لبه از مواردی است که امروزه در شبکه‌های اینترنت اشیا به صورت گسترده مورد استفاده قرار می‌گیرد.
همانطور که اشاره شد با توجه به معماری مورد استفاده در شبکه \متن‌لاتین{LPWAN} جایگاه این لبه می‌تواند متفاوت باشد.

ارائه یک معماری برای مدیریت ساده و از دور، برای لبه به منظور انجام پردازش‌های متنوع و از سوی دیگر ارزیابی کارایی این روش می‌تواند یک چالش تحقیقاتی در این حوزه باشد.
این معماری برای پشتیبانی از اشیا متحرک نیاز به جابجایی محل پردازش برای حفظ کیفیت سرویس دارد.

\زیرقسمت{پیاده‌سازی شبکه‌ \متن‌لاتین{Meshed Lora} بر پایه \متن‌لاتین{Nvidia Jetson} و استفاده از الگوریتم‌های یادگیری ماشین برای تشکیل گراف بهینه‌}

همانطور که بیان شد، یک مساله مورد بحث در شبکه‌های \متن‌لاتین{LoRa} استفاده از لایه فیزیکی به تنهایی و شکل دادن یک شبکه‌ی \متن‌لاتین{Mesh} می‌باشد.
یکی از بحث‌های مهم در این شبکه‌ها چگونگی تشکیل گراف و انتخاب پدر برای گره‌ها می‌باشد. برای این امر می‌توان از الگوریتم‌های یادگیری تقویتی عمیق استفاده کرد
و در جهت کاهش تغییر در ساختار فعلی شبکه، با استفاده از سخت‌افزار \متن‌لاتین{Nvidia Jetson} آن را در \متن‌لاتین{Gateway} و نزدیکی شبکه اجرا نمود.
در نهایت با توجه با محوریت این رساله تاثیر این انتخاب‌ها بر کارایی سیستم انتها به انتها مدنظر است.

\زیرقسمت{پیاده‌سازی الگوریتم‌های هوش‌مصنوعی و یادگیری عمیق در لایه کاربرد}

یکی از مسائل حوزه اینترنت اشیا که به آن توجه کمی شده است استفاده عملی از داده‌های جمع‌آوری شده در اینترنت اشیا است. می‌توان این داده‌ها را با استفاده از الگوریتم‌های یادگیری عمیق و هوش مصنوعی به داده‌های کاربردی تبدیل کرده
و چالش‌های موجود در حوزه‌های اینترنت اشیا را پاسخ داد. به طور مثال در کاربردهای پیشنهاد شده می‌توان از داده‌های جمع‌آوری شده از سنسورهای آلودگی هوا در صحن دانشگاه برای پیش‌بینی روزهای آلوده سال و از سنسورهای کشاورزی
برای پیش‌بینی وضعیت محصول استفاده کرد.
این اطلاعات در واقع محصول نهایی سیستم هستند و در صورتی که بخواهیم، سیستم انتها به انتها را مدنظر قرار دهیم کیفیت داده‌ی خروجی، حجم داده‌ی جمع‌آوری شده و \نقاط‌خ وابسته به انتخاب‌هایی است که در طراحی سیستم صورت پذیرفته است.

\گرنادرست
\زیرقسمت{ارزیابی شبکه \متن‌لاتین{IEEE 802.11be}}

\متن‌لاتین{WiFi 7} و \متن‌لاتین{IEEE 802.11be} تکنولوژی‌های جدیدی هستند که به دنبال پشتیبانی از کاربردهای حساس به زمان می‌باشند. هنوز این تکنولوژی‌ها در عمل مورد سنجش قرار نگرفته‌اند.
سنجش آن‌ها می‌تواند نقاط ضعف و قوت آن‌ها را بهتر مشخص کند. استاندارد نهایی برای سال ۲۰۲۴ خواهد بود ولی اولین نسخه‌ی آن در سال ۲۰۲۲ منتشر خواهد شد. برای شبیه‌سازی و ارزیابی در این
مساله می‌بایست از نرم‌افزارهای شبیه‌ساز استفاده کرد.

با توجه به ماهیت این شبکه بحث ارزیابی کران بالای تاخیر در آن‌ها بسیار ارزشمند بوده و می‌توان از تکنیک‌های \متن‌لاتین{Network Calculus} برای این امر استفاده کرد.

\زیرقسمت{ارزیابی شبکه \متن‌لاتین{NB-IoT}}

شبکه‌های \متن‌لاتین{NB-IoT} با زیرساخت فعلی شبکه‌های سلولی فعالیت می‌کنند، از این رو برای فراهم آوردن کیفیت سرویس مناسب نیاز دارند تا بین ترافیک کاربران عادی
و ترافیک اینترنت اشیا تفاوت قائل شوند. در اینجا بحث تقسیم‌بندی شبکه و استفاده از کارکردهای مجازی شده پیش می‌آید که در در این حوزه می‌توان ارزیابی صورت داد.
از سوی دیگر مفاهیمی همچون کارکردهای شبکه‌ای ابرزی در اینجا می‌تواند نقش پررنگی در شبکه \متن‌لاتین{Backhaul} داشته باشد.

\قسمت{پروژه‌های کارشناسی}

از آنجایی که در مساله پیشنهادی نیاز به پیاده‌سازی استانداردها و پروتکل‌هایی وجود دارد، این موارد می‌توانند در قالب پروژه کارشناسی تعریف شوند.

\شروع{فقرات}
\فقره پیاده‌سازی کتابخانه‌ی \متن‌لاتین{SenML} با زبان \متن‌لاتین{C} یا \متن‌لاتین{C++} بر پایه پروتکل \متن‌لاتین{CBOR}
\فقره پیاده‌سازی یک زیرساخت لبه با قابلیت مدیریت چندین نود لبه از دور
\پایان{فقرات}

\رگ
