\newpage\clearpage

\pagestyle{abstract}

\vspace*{-1cm}
\section*{\centering \abstractname}
%\addcontentsline{toc}{chapter}{چکیده}
\vspace*{.5cm}
% abstract in persian

اینترنت اشیا از آغاز خود در سال ۱۹۹۹ تا به امروز گسترش زیادی پیدا کرده است.
نیاز به کاهش هزینه در گرهها و گستشر پوشش شبکه باعث شده است که الگوی \متن‌لاتین{LPWAN}\پانویس{Low Power Wide Area Network} مطرح شود.
فناوری‌های زیادی در حوزه \متن‌لاتین{LPWAN}ها مطرح شدند که معروف‌ترین آن‌ها \متن‌لاتین{LoRaWAN} است. مدل تجاری و متن باز بودن
لایه دسترسی آن دلیل شهرت \متن‌لاتین{LoRaWAN} است.
با توجه به باز بودن استاندارد در تعریف شبکه‌ی \متن‌لاتین{Backbone} و انتقال بخش عمده‌ای از پیچیدگی، در راستای کاهش توان مصرفی
شبکه دسترسی، به شبکه \متن‌لاتین{Backbone} انتخاب‌های زیادی در این شبکه وجود داشته و پردازش زیادی در صورت می‌گیرد.
در این رساله قصد داریم ارزیابی کارایی انتها به انتهایی در این شبکه صورت داده و ارتباط پارامترهایی مانند نرخ ارسال داده، تعداد اشیا
به پارامترهای کیفیت سرویسی چون تاخیر انتها به انتها را آشکار کنیم.
از سوی دیگر ساختار داده در لایه کاربرد می‌تواند روی این پارامترهای کیفیت سرویس تاثیر داشته باشد، از این مساله‌ی دیگری که رساله حاضر به آن می‌پردازد
ساختار داده در لایه کاربرد است و تاثیر آن را بر روی پارامترهای کیفیت سرویس بررسی می‌کند.
برای این کار رساله حاضر شبکه را به بخش‌هایی تقسیم و سعی در مدل‌سازی آن‌ها بر پایه تئوری صف و \متن‌لاتین{Network Calculus}
می‌نماید.


\vspace*{2cm}

{%
  \noindent\large\textbf{واژه‌های کلیدی:}
}\par
\vspace*{.5cm}
% keywords in persian
