\newpage\clearpage

\pagestyle{abstract}

\vspace*{-1cm}
\section*{\centering \abstractname}
%\addcontentsline{toc}{chapter}{چکیده}
\vspace*{.5cm}
% abstract in persian

اینترنت اشیا از آغاز خود در سال ۱۹۹۹ تا به امروز گسترش زیادی پیدا کرده است.
نیاز به کاهش هزینه در گره‌ها و گسترش پوشش شبکه باعث شده است که الگوی
شبکه‌های توان پایین با برد بلند یا
\متن‌لاتین{LPWAN}\پانویس{Low Power Wide Area Network} مطرح شود.
فناوری‌های زیادی در حوزه \متن‌لاتین{LPWAN}ها مطرح شدند که از جمله‌ی آن‌ها می‌توان به \متن‌لاتین{LoRaWAN}، \متن‌لاتین{Sigfox} و \متن‌لاتین{NB-IoT}
اشاره کرد. شبکه \متن‌لاتین{LoRaWAN} بر پایه لایه فیزیکی \متن‌لاتین{LoRa} فعالیت می‌کند و
معروف‌ترین شبکه \متن‌لاتین{LPWAN} به شمار می‌رود. مدل تجاری و باز بودن
لایه دسترسی آن، دلیل شهرت \متن‌لاتین{LoRaWAN} است.
در بیشتر شبکه‌های \متن‌لاتین{LPWAN} در جهت کاهش پیچیدگی و هزینه در گره‌ها پیچیدگی به شبکه‌ی هسته منتقل می‌شود و \متن‌لاتین{LoRaWAN}
نیز از این قاعده مستثنا نیست.
با توجه به باز بودن استاندارد ارائه شده از سوی \متن‌لاتین{LoRaWAN} در تعریف شبکه‌ی \متن‌لاتین{Backbone} و
انتقال بخش عمده‌ای از پیچیدگی، در راستای کاهش توان مصرفی
شبکه دسترسی، به شبکه \متن‌لاتین{Backbone} انتخاب‌های زیادی در این شبکه وجود داشته و پردازش زیادی در آن صورت می‌گیرد.
این پیشنهاد رساله قصد دارد ارزیابی کارایی انتها به انتهایی در شبکه دسترسی \متن‌لاتین{LoRaWAN}
از دروازه تا سرور شبکه و سرور اپلیکیشن،
صورت داده و ارتباط پارامترهایی از جمله نرخ ارسال داده، تعداد اشیا
به پارامترهای کیفیت سرویسی چون تاخیر انتها به انتها و نرخ از دست رفت بسته را آشکار کنیم.
از سوی دیگر ساختار داده و پروتکل مورد استفاده در لایه کاربرد می‌تواند روی این پارامترهای کیفیت سرویس تاثیر داشته باشد،
از این رو مساله‌ی دیگری که پیشنهاد رساله حاضر به آن می‌پردازد
ارزیابی لایه کاربرد شبکه \متن‌لاتین{LoRaWAN} از گره تا اپلیکیشن
و تاثیر آن را بر روی پارامترهای کیفیت سرویس است.
برای حل این دو مساله پیشنهاد رساله حاضر شبکه \متن‌لاتین{LoRaWAN} را
مطابق استانداردهای ارائه شده
به بخش‌هایی تقسیم و سعی در مدل‌سازی آن‌ها بر پایه تئوری صف و \متن‌لاتین{Network Calculus}
می‌نماید.

\vspace*{2cm}

{%
  \noindent\large\textbf{واژه‌های کلیدی:}
}\par
\vspace*{.5cm}
اینترنت اشیا، شبکه‌های توان پایین با برد بالا، شبکه \متن‌لاتین{LoRaWAN}، ارزیابی کارایی، تاخیر، نرخ از دست رفت بسته
% keywords in persian
