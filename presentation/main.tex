\documentclass[dvipsnames]{beamer}
\usetheme{metropolis}

\usepackage{multicol}

\usepackage{xecolor}
\usepackage{amsmath}
% \usefonttheme[onlymath]{serif} % change the math font

\usepackage{tabularx}
\usepackage{booktabs}
\usepackage[style=numeric,sorting=ynt]{biblatex}
\addbibresource{references.bib}
\usepackage[localise]{xepersian}

\settextfont[
	Path = fonts/,
	UprightFont = *-Regular,
	BoldFont = *-Bold,
	ItalicFont = *-Variable
]{Vazir}
\setlatintextfont[
	Path = fonts/,
	UprightFont = *-Regular,
	BoldFont = *-Bold,
	ItalicFont = *-Italic
]{Neuton}

% ---------------------------------------------------------------------------------
% Colors
% ---------------------------------------------------------------------------------
\definecolor{نارنجی}{rgb}{1.0, 0.31, 0.0}

% To force beamer use numbering in captions
\setbeamertemplate{caption}[numbered]{}% Number float-like environments

\setbeamertemplate{footline}[frame number]
\setbeamertemplate{section in toc}[circle]
\setbeamertemplate{blocks}[rounded][shadow=true]
\setbeamercolor{block title}{bg=orange}
\setbeamercolor{block body}{bg=lightgray}
\setbeamercolor{headline}{bg=orange}
\setbeamersize{text margin left=1cm,text margin right=1cm}
\setbeamertemplate{frametitle continuation}{\insertcontinuationcount}

\setbeamertemplate{headline}
{
	\begin{beamercolorbox}{section in head/foot}
		\vspace{2pt}\insertnavigation{\paperwidth}\vspace{2pt}
	\end{beamercolorbox}
}

\setbeamertemplate{footline}
{%
	\leavevmode%
	\hbox{%
		\begin{beamercolorbox}[wd=.333333\paperwidth,ht=2.25ex,dp=1ex,center]{author in head/foot}%
			\usebeamerfont{author in head/foot}\insertshortauthor%
		\end{beamercolorbox}%
		\begin{beamercolorbox}[wd=.333333\paperwidth,ht=2.25ex,dp=1ex,center]{title in head/foot}%
			\usebeamerfont{title in head/foot}\insertshorttitle%
		\end{beamercolorbox}%
		\begin{beamercolorbox}[wd=.333333\paperwidth,ht=2.25ex,dp=1ex,right]{date in head/foot}%
			\usebeamerfont{date in head/foot}\insertsection\hspace*{2em}
			\insertframenumber~ از \inserttotalframenumber{} \hspace*{2ex}%
		\end{beamercolorbox}
	}%
}

\eqcommand{موسسه}{institute}

% ---------------------------------------------------------------------------------
\عنوان{جایگذاری دروازه‌ها در شبکه‌های توری \متن‌لاتین{LoRa} با استفاده از یادگیری نمایش گرافی}
\نویسنده{پرهام الوانی}
\موسسه{%
	دانشکده مهندسی کامپیوتر\\
	دکتر بهادر بخشی و دکتر مهدی راستی
}
\date{\today}
\titlegraphic{\vspace{4.5cm}\flushleft\includegraphics[height=50pt]{images/logo}}

\setbeamertemplate{title}{%
	\linespread{1.0}%
	\inserttitle%
	\par%
	\vspace*{0.5em}
}
\setbeamertemplate{subtitle}{%
	\insertsubtitle%
	\par%
	\vspace*{0.5em}
}

\AtBeginSection[]
{%
	\begin{frame}{فهرست}
		\tableofcontents[currentsection]
	\end{frame}
	\begin{frame}
		\begin{center}
			\insertsectionnumber. \insertsection%
		\end{center}
		\usebeamertemplate*{title separator}
	\end{frame}
}

\begin{document}

\begin{persian}

	% ------------------------------------------
	% Title frame (0)
	% ------------------------------------------
	{%
		\setbeamertemplate{footline}{}
		\begin{frame}
			\titlepage%
		\end{frame}
	}

	% -------------------------------------------------------------------------------
	\begin{frame}{فهرست}
		\tableofcontents[pausesections]
	\end{frame}

	% -------------------------------------------------------------------------------
	\قسمت{مقدمه}

	\begin{frame}{اینرتنت اشیا}
		\تنظیم‌ازوسط
		\درج‌تصویر[width=\textwidth]{./images/iot-growth.jpg}
	\end{frame}

	\begin{frame}{شبکه‌های توان پایین با برد بلند}
		\شروع{فقرات}
		\فقره عدم همخوانی تکنولوژی‌های حاضر شبکه‌های حسگر بی‌سیم با گسترش روزافزون اینترنت اشیا
		\شروع{فقرات}
		\فقره کاهش هزینه‌های هر واحد
		\فقره گسترده‌تر کردن پوشش شبکه
		\فقره کاهش توان مصرفی نودهای در لبه
		\فقره شبکه‌های گسترش‌پذیر
		\پایان{فقرات}
		\pause
		\فقره \متن‌لاتین{LPWAN} یا \متن‌لاتین{Low-Power Wide Area Network}
		\پایان{فقرات}
	\end{frame}

	\begin{frame}{شبکه‌های توان پایین با برد بلند}
		\شروع{فقرات}
		\فقره مهم‌ترین شبکه‌های توان پایین با برد بلند:
		\شروع{فقرات}
		\فقره \متن‌لاتین{LoRa}
		\فقره \متن‌لاتین{Sigfox}
		\فقره \متن‌لاتین{NB-IoT}
		\پایان{فقرات}
		\pause
		\فقره شاخصه‌های شبکه‌های \متن‌لاتین{LoRa}
		\شروع{فقرات}
		\فقره استفاده از باند فرکانسی بدون مجوز
		\فقره امکان راه‌اندازی شبکه توسط اشخاص ثالث
		\پایان{فقرات}
		\pause
		\فقره پرداختن پژوهش‌های زیادی در این سال‌ها به شبکه‌های \متن‌لاتین{LoRa}
		\فقره \متن‌لاتین{LoRa} بیشترین تکنولوژی استفاده شده
		\پایان{فقرات}
	\end{frame}

	\begin{frame}{شبکه‌های \متن‌لاتین{LoRaWAN}}
		\شروع{فقرات}
		\فقره لایه پیوند داده بر پایه لایه فیزیکی \متن‌لاتین{LoRa}
		\فقره استاندارد رایگان بر خلاف \متن‌لاتین{LoRa}
		\فقره عدم وجود ارتباط میان دروازه‌ها و گره‌ها وجود ندارد
		\فقره داده‌ی ارسالی توسط هر گره می‌تواند به وسیله‌ی یک یا چند دروازه دریافت شود.
		\فقره دروازه‌ها وظیفه ارسال ترافیک گره‌های \متن‌لاتین{LoRa} به سرور شبکه و برعکس را دارا هستند.
		\فقره عملیات‌هایی از جمله احراز هویت، شناسایی و حذف بسته‌های تکراری توسط سرور شبکه صورت می‌پذیرد.
		\پایان{فقرات}
	\end{frame}

	\begin{frame}
		\شروع{شکل}
		\تنظیم‌ازوسط
		\درج‌تصویر[width=\textwidth]{./images/nrm-home.png}
		\شرح{مدل مرجع شبکه \متن‌لاتین{LoRaWAN} - شبکه‌ی خانگی}
		\پایان{شکل}
	\end{frame}

	\قسمت{کارهای مرتبط}

	\قسمت{تعریف مساله}

	\begin{frame}{جایگذاری دروازه‌ها در شبکه توری \متن‌لاتین{LoRa}}
	\شروع{فقرات}
	\فقره مدل انتشار سیگنال
	\فقره در نظر گرفتن تداخل و استفاده از فاکتورهای گسترش متفاوت
	\فقره مشخص بودن محل گره‌ها از پیش
	\فقره بین این گره‌ها در صورتی وجود ارتباط (بدون در نظر گرفتن تداخل و تنها با استفاده از مدل انتشار سیگنال) یال وجود دارد.
	\فقره با توجه به فاکتورهای گسترش مختلف ۶ گراف گوناگون تشکیل می‌شود.
	\فقره افزایش قابلیت اطمینان با استفاده همزمان از چند دروازه برای یک گره
	\پایان{فقرات}
	\end{frame}

	\begin{frame}{جایگذاری دروازه‌ها در شبکه توری \متن‌لاتین{LoRa}}
	\شروع{فقرات}
	\فقره کشاورزی هوشمند
	\فقره پیش‌بینی تقاضای ترافیک
	\فقره جابجایی گره‌ها (در نظر گرفتن همبندی یا محیط پویا به جای همبندی ساکن)
	\پایان{فقرات}
	\end{frame}

	% -------------------------------------------------------------------------------
	\begin{frame}[allowframebreaks]{مراجع}
		\begin{latin}
			\printbibliography[title=مراجع]
		\end{latin}
	\end{frame}

\end{persian}
\end{document}
