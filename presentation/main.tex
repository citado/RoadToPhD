\documentclass[dvipsnames]{beamer}
\usetheme{metropolis}

\usepackage{multicol}

\usepackage{xecolor}
\usepackage{amsmath}
% \usefonttheme[onlymath]{serif} % change the math font

\usepackage{tabularx}
\usepackage{booktabs}
\usepackage[style=numeric,sorting=ynt]{biblatex}
\addbibresource{references.bib}
\usepackage[localise]{xepersian}

\settextfont[
	Path = fonts/,
	UprightFont = *-Regular,
	BoldFont = *-Bold,
	ItalicFont = *-Variable
]{Vazir}
\setlatintextfont[
	Path = fonts/,
	UprightFont = *-Regular,
	BoldFont = *-Bold,
	ItalicFont = *-Italic
]{Neuton}

% ---------------------------------------------------------------------------------
% Colors
% ---------------------------------------------------------------------------------
\definecolor{نارنجی}{rgb}{1.0, 0.31, 0.0}

% To force beamer use numbering in captions
\setbeamertemplate{caption}[numbered]{}% Number float-like environments

\setbeamertemplate{footline}[frame number]
\setbeamertemplate{section in toc}[circle]
\setbeamertemplate{blocks}[rounded][shadow=true]
\setbeamercolor{block title}{bg=orange}
\setbeamercolor{block body}{bg=lightgray}
\setbeamercolor{headline}{bg=orange}
\setbeamersize{text margin left=1cm,text margin right=1cm}
\setbeamertemplate{frametitle continuation}{\insertcontinuationcount}

\setbeamertemplate{headline}
{
	\begin{beamercolorbox}{section in head/foot}
		\vspace{2pt}\insertnavigation{\paperwidth}\vspace{2pt}
	\end{beamercolorbox}
}

\setbeamertemplate{footline}
{%
	\leavevmode%
	\hbox{%
		\begin{beamercolorbox}[wd=.333333\paperwidth,ht=2.25ex,dp=1ex,center]{author in head/foot}%
			\usebeamerfont{author in head/foot}\insertshortauthor%
		\end{beamercolorbox}%
		\begin{beamercolorbox}[wd=.333333\paperwidth,ht=2.25ex,dp=1ex,center]{title in head/foot}%
			\usebeamerfont{title in head/foot}\insertshorttitle%
		\end{beamercolorbox}%
		\begin{beamercolorbox}[wd=.333333\paperwidth,ht=2.25ex,dp=1ex,right]{date in head/foot}%
			\usebeamerfont{date in head/foot}\insertsection\hspace*{2em}
			\insertframenumber~ از \inserttotalframenumber{} \hspace*{2ex}%
		\end{beamercolorbox}
	}%
}

\eqcommand{موسسه}{institute}

% ---------------------------------------------------------------------------------
\عنوان{تازگی اطلاعات در شبکه‌های \متن‌لاتین{LoRa}}
\نویسنده{پرهام الوانی}
\موسسه{%
	دانشکده مهندسی کامپیوتر\\
	دکتر بهادر بخشی و دکتر مهدی راستی
}
\date{\today}
\titlegraphic{\vspace{4.5cm}\flushleft\includegraphics[height=50pt]{images/logo}}

\setbeamertemplate{title}{%
	\linespread{1.0}%
	\inserttitle%
	\par%
	\vspace*{0.5em}
}
\setbeamertemplate{subtitle}{%
	\insertsubtitle%
	\par%
	\vspace*{0.5em}
}

\AtBeginSection[]
{%
	\begin{frame}{فهرست}
		\tableofcontents[currentsection]
	\end{frame}
	\begin{frame}
		\begin{center}
			\insertsectionnumber. \insertsection%
		\end{center}
		\usebeamertemplate*{title separator}
	\end{frame}
}

\begin{document}

\begin{persian}

	% ------------------------------------------
	% Title frame (0)
	% ------------------------------------------
	{%
		\setbeamertemplate{footline}{}
		\begin{frame}
			\titlepage%
		\end{frame}
	}

	% -------------------------------------------------------------------------------
	\begin{frame}{فهرست}
		\tableofcontents[pausesections]
	\end{frame}

	% -------------------------------------------------------------------------------
	\قسمت{مقدمه}

	\begin{frame}{اینرتنت اشیا}
		\تنظیم‌ازوسط
		\درج‌تصویر[width=\textwidth]{./images/iot-growth.jpg}
	\end{frame}

	\begin{frame}{شبکه‌های توان پایین با برد بلند}
		\شروع{فقرات}
		\فقره عدم همخوانی تکنولوژی‌های حاضر شبکه‌های حسگر بی‌سیم با گسترش روزافزون اینترنت اشیا
		\شروع{فقرات}
		\فقره کاهش هزینه‌های هر واحد
		\فقره گسترده‌تر کردن پوشش شبکه
		\فقره کاهش توان مصرفی نودهای در لبه
		\فقره شبکه‌های گسترش‌پذیر
		\پایان{فقرات}
		\pause
		\فقره \متن‌لاتین{LPWAN} یا \متن‌لاتین{Low-Power Wide Area Network}
		\پایان{فقرات}
	\end{frame}

	\begin{frame}{شبکه‌های توان پایین با برد بلند}
		\شروع{فقرات}
		\فقره مهم‌ترین شبکه‌های توان پایین با برد بلند:
		\شروع{فقرات}
		\فقره \متن‌لاتین{LoRa}
		\فقره \متن‌لاتین{Sigfox}
		\فقره \متن‌لاتین{NB-IoT}
		\پایان{فقرات}
		\pause
		\فقره شاخصه‌های شبکه‌های \متن‌لاتین{LoRa}
		\شروع{فقرات}
		\فقره استفاده از باند فرکانسی بدون مجوز
		\فقره امکان راه‌اندازی شبکه توسط اشخاص ثالث
		\پایان{فقرات}
		\pause
		\فقره پرداختن پژوهش‌های زیادی در این سال‌ها به شبکه‌های \متن‌لاتین{LoRa}
		\فقره \متن‌لاتین{LoRa} بیشترین تکنولوژی استفاده شده
		\پایان{فقرات}
	\end{frame}

	\begin{frame}{شبکه‌های \متن‌لاتین{LoRaWAN}}
		\شروع{فقرات}
		\فقره لایه پیوند داده بر پایه لایه فیزیکی \متن‌لاتین{LoRa}
		\فقره استاندارد رایگان بر خلاف \متن‌لاتین{LoRa}
		\فقره عدم وجود ارتباط میان دروازه‌ها و گره‌ها وجود ندارد
		\فقره داده‌ی ارسالی توسط هر گره می‌تواند به وسیله‌ی یک یا چند دروازه دریافت شود.
		\فقره دروازه‌ها وظیفه ارسال ترافیک گره‌های \متن‌لاتین{LoRa} به سرور شبکه و برعکس را دارا هستند.
		\فقره عملیات‌هایی از جمله احراز هویت، شناسایی و حذف بسته‌های تکراری توسط سرور شبکه صورت می‌پذیرد.
		\پایان{فقرات}
	\end{frame}

	\begin{frame}
		\شروع{شکل}
		\تنظیم‌ازوسط
		\درج‌تصویر[width=\textwidth]{./images/nrm-home.png}
		\شرح{مدل مرجع شبکه \متن‌لاتین{LoRaWAN} - شبکه‌ی خانگی}
		\پایان{شکل}
	\end{frame}

	\قسمت{کارهای مرتبط}

	\قسمت{تعریف مساله}

	\begin{frame}{تخصیص منابع با هدف کمینه‌سازی میانگین عمر اطلاعات در شبکه‌ی \متن‌لاتین{LoRa}}
	\شروع{فقرات}
	\فقره هدف برنامه‌ریزی برای $n$ بازه‌ی زمانی است و فرض می‌شود اشیا از پیش با یکدیگر همگام شده‌اند.
	\فقره $k$ حسگر
	\فقره $t$ بازه‌ی زمانی که اشیا در ابتدای آن بر اساس منابع از پیش تخصیص‌یافته، ارسال یا عدم ارسال اطلاعات را صورت می‌دهند.
	\فقره تخصیص $s$ فاکتورهای گسترش و $n$ زیرکانال‌های عمود برهم به اشیا برای جلوگیری از تصادم
	\شروع{فقرات}
	\فقره $p_{k,n,s}(t)$ در صورت تخصیص زیرکانال $n$ و فاکتورگسترش $s$ به شی $k$ در بازه‌ی زمانی $t$ برابر یک خواهد بود.
	\فقره هر دو تایی زیرکانال و فاکتور گسترش در یک بازه‌ی زمانی حداکثر به یک شی تخصیص می‌یابد.
	\پایان{فقرات}
	\فقره در ابتدای هر بازه زمانی تخصیص زیرکانال‌ها و فاکتورهای گسترش برای برخی از اشیا صورت گرفته و آن اشیا در صورت داشتن داده آن را ارسال خواهند کرد.
	\پایان{فقرات}
	\end{frame}

	\begin{frame}{تخصیص منابع با هدف کمینه‌سازی میانگین عمر اطلاعات در شبکه‌ی \متن‌لاتین{LoRa} (ادامه)}
	\شروع{فقرات}
	\فقره تنها یک دوازه در نظر گرفته می‌شود و فرض شده است اشیا در پوشش این دروازه قرار دارند.
	\فقره اشیایی که تخصیص منابع به آن‌ها صورت پذیرفته است در انتهای بازه عمر اطلاعاتشان به روزرسانی می‌گردد.
	\فقره این مساله مشکل همگام‌سازی را مدنظر قرار نداده و از همین رو ممکن است، عملیاتی \متن‌سیاه{نباشد}.
	\فقره در نظر گرفتن عدم قطعیت برای وجود داده در اشیا در هر بازه زمانی، به عبارت دیگر یک شی ممکن است در یک بازه زمانی داده‌ای نداشته باشد.
	\فقره در نظر گرفتن عدم قطعیت برای پارامتر از دست رفت مسیر اشیا
	\پایان{فقرات}
	\end{frame}

	\begin{frame}{تخصیص منابع با هدف کمینه‌سازی میانگین عمر اطلاعات در شبکه‌ی \متن‌لاتین{LoRa} (ادامه)}
	\شروع{فقرات}
	\فقره عمر اطلاعات هر شی در هر بازه زمانی که ارسال نداشته باشد یک واحد افزایش پیدا می‌کند.
	\پایان{فقرات}
	\end{frame}

	\زیرقسمت{عمر اطلاعات}

	\begin{frame}{عمر اطلاعات در شبکه‌های بی‌سیم}
	\شروع{فقرات}
	\فقره موقعیت مکانی فرستنده/گیرنده‌ها که روی از دست رفت مسیر آن‌ها تاثیر دارد.
	\فقره وضعیت ارسال کننده‌های هم‌کانال\پانویس{co-channel} که روی تداخل تاثیر دارد.
	\فقره تاثیر گره‌های همسایه بر پویایی صف ارسال کننده‌ها که از آن با نام \متن‌لاتین{spatially interacting queues} یاد می‌شود.
	\فقره بسته‌های اطلاعاتی با استفاده از پردازه برنولی و مستقل تولید می‌شوند. این بسته‌ها در یک بافر نامتناهی قرار می‌گیرند.
	\فقره استفاده از لایه دسترسی همزمان \متن‌لاتین{ALOHA}
	\فقره استفاده از نسبت سیگنال به نویز برای تشخیص دریافت صحیح بسته
	\پایان{فقرات}
	\end{frame}

	% -------------------------------------------------------------------------------
	\begin{frame}[allowframebreaks]{مراجع}
		\begin{latin}
			\printbibliography[title=مراجع]
		\end{latin}
	\end{frame}

\end{persian}
\end{document}
