% !TeX TS-program = xelatex

\documentclass[dvipsnames]{beamer}
\usetheme{metropolis}

\usepackage{multicol}

\usepackage{ragged2e} % who justifies the text
\usepackage{xecolor}
\usepackage{amsmath}
% \usefonttheme[onlymath]{serif} % change the math font

\usepackage{tabularx}
\usepackage{booktabs}
\usepackage[style=numeric,sorting=ynt]{biblatex}
\addbibresource{references.bib}
\usepackage[localise]{xepersian}

\settextfont{Vazir}
\setlatintextfont{Roboto}

% ---------------------------------------------------------------------------------
% Colors
% ---------------------------------------------------------------------------------
\definecolor{نارنجی}{rgb}{1.0, 0.31, 0.0}

% ---------------------------------------------------------------------------------
% Settings to force Beamer works with Xepersian and RTL typesetting
% ---------------------------------------------------------------------------------
% \raggedleft

% For right to left lists (itemize and enumerate)
\makeatletter
\newcommand{\لیست‌فارسی}{\raggedleft\rightskip\@totalleftmargin}
\makeatother
% Correct the bullet for RTL texts
\setbeamertemplate{itemize item}{\scriptsize\raise1.25pt%
	\hbox{\donotcoloroutermaths$\blacktriangleleft$}}

% To force beamer use numbering in captions
\setbeamertemplate{caption}[numbered]{}% Number float-like environments

\setbeamertemplate{footline}[frame number]
\setbeamertemplate{section in toc}[circle]
\setbeamertemplate{blocks}[rounded][shadow=true]
\setbeamercolor{block body}{bg=lightgray}
\setbeamercolor{headline}{bg=orange}
\setbeamersize{text margin left=1cm,text margin right=1cm}

\setbeamertemplate{headline}
{
	\begin{beamercolorbox}{section in head/foot}
		\vspace{2pt}\insertnavigation{\paperwidth}\vspace{2pt}
	\end{beamercolorbox}
}

% ---------------------------------------------------------------------------------
% To force beamer use numbering in captions
\setbeamertemplate{caption}[numbered]{}% Number float-like environments

\setbeamertemplate{footline}
{%
	\leavevmode%
	\hbox{%
		\begin{beamercolorbox}[wd=.333333\paperwidth,ht=2.25ex,dp=1ex,center]{author in head/foot}%
			\usebeamerfont{author in head/foot}\insertshortauthor%
		\end{beamercolorbox}%
		\begin{beamercolorbox}[wd=.333333\paperwidth,ht=2.25ex,dp=1ex,center]{title in head/foot}%
			\usebeamerfont{title in head/foot}\insertshorttitle%
		\end{beamercolorbox}%
		\begin{beamercolorbox}[wd=.333333\paperwidth,ht=2.25ex,dp=1ex,right]{date in head/foot}%
			\usebeamerfont{date in head/foot}\insertsection\hspace*{2em}
			\insertframenumber~ از \inserttotalframenumber{} \hspace*{2ex}%
		\end{beamercolorbox}
	}%
}
\setbeamertemplate{section in toc}[circle]
\setbeamertemplate{blocks}[rounded][shadow=true]
\setbeamercolor{block title}{bg=orange}
\setbeamercolor{block body}{bg=lightgray}
\setbeamersize{text margin left=1cm,text margin right=1cm}
\setbeamertemplate{frametitle continuation}{\insertcontinuationcount}

% ---------------------------------------------------------------------------------
\title{ارزیابی کارآیی شبکه‌های توان پایین}
\subtitle{}
\author{پرهام الوانی}
\institute{%
	دانشکده مهندسی کامپیوتر\\
	دکتر مهدی راستی
}
\date{\today}
\titlegraphic{\vspace{4.5cm}\flushleft\includegraphics[height=50pt]{images/logo}}

\begin{document}

\makeatletter

\setbeamertemplate{title}{%
	\linespread{1.0}%
	\inserttitle%
	\par%
	\vspace*{0.5em}
}
\setbeamertemplate{subtitle}{%
	\insertsubtitle%
	\par%
	\vspace*{0.5em}
}

\AtBeginSection[]
{%
	\begin{frame}{فهرست}
		\tableofcontents[currentsection]
	\end{frame}
	\begin{frame}
		\begin{center}
			\insertsectionnumber. \insertsection%
		\end{center}
		\usebeamertemplate*{title separator}
	\end{frame}
}

\makeatother


\begin{persian}
	% ------------------------------------------
	% Title frame (0)
	% ------------------------------------------
	{%
		\setbeamertemplate{footline}{}
		\begin{frame}
			\titlepage%
		\end{frame}
	}

	% -------------------------------------------------------------------------------
	\begin{frame}{فهرست}
		\tableofcontents[pausesections]
	\end{frame}

	% -------------------------------------------------------------------------------
	\section{شبکه‌های \متن‌لاتین{LoRa}}

	\section{کارهای مرتبط}
	\begin{frame}{}
		\begin{table}[h]
			\caption{مرور پژوهش‌های حوزه ارزیابی کارایی شبکه‌های توان پایین در لایه دسترسی}
			\vspace{0.3cm}
			\tiny{}
			\begin{tabularx}{\textwidth}{XXXXXXXXXXp{1cm}}
				\toprule
				مرجع &
							 تحرک &
											\متن‌لاتین{coverage} &
																						شبکه &
																									 لایه انتقال &
																																 لایه اپلیکشن &
																																								اندازه بسته &
																																															\متن‌لاتین{Coding Rate} &
																																																											 تاخیر &
																																																															 محیط &
																																																																			شبیه‌سازی \\
				\midrule
				\cite{sensors-18-00772-v3} &
																		 سیار &
																						گزارش شده &
																												\متن‌لاتین{LoRaWAN} &
																																						 ندارد &
																																										 ندارد &
																																														 گزارش شده &
																																																				 گزارش شده &
																																																										 گزارش نشده &
																																																																	باز &
																																																																				واقعی / \متن‌لاتین{cloudRF} \\
				\midrule
				\cite{sensors-19-00007} &
																	ثابت &
																				 گزارش شده &
																										 \متن‌لاتین{LoRaWAN} &
																																					\متن‌لاتین{TCP / UDP} &
																																																 \متن‌لاتین{MQTT / CoAP} &
																																																													گزارش شده &
																																																																			گزارش نشده &
																																																																									 گزارش شده &
																																																																															 باز &
																																																																																		 \متن‌لاتین{Ericsson internal simulator} \\
				\midrule
				\cite{sensors-20-03061-v2} &
																		 ثابت &
																						گزارش شده &
																												\متن‌لاتین{LoRaWAN} &
																																						 ندارد &
																																										 ندارد &
																																														 گزارش شده &
																																																				 گزارش شده &
																																																										 گزارش شده &
																																																																 باز / بسته &
																																																																							\متن‌لاتین{ns-3} \\
				\midrule
				\cite{sensors-20-00280-v2} &
																		 ثابت &
																						گزارش شده &
																												\متن‌لاتین{LoRaWAN} &
																																						 \متن‌لاتین{UDP over IPv6} &
																																																				\متن‌لاتین{CoAP}
																																																				گزارش شده &
																																																										گزارش شده &
																																																																گزارش شده &
																																																																						باز / بسته &
																																																																												 واقعی \\
				\midrule
				\cite{sensors-20-00280-v2} &
																		 ثابت &
																						گزارش شده &
																												\متن‌لاتین{LoRaWAN} &
																																						 ندارد &
																																										 ندارد &
																																														 گزارش شده &
																																																				 گزارش شده &
																																																										 گزارش شده &
																																																																 بسته &
																																																																				واقعی \\
				\bottomrule
			\end{tabularx}
		\end{table}
	\end{frame}

	\begin{frame}{\cite{sensors-18-03995}}
		\شروع{فقرات}
		\لیست‌فارسی
		\فقره مرور جامع مسائل و چالش‌های شبکه‌های \متن‌لاتین{LoRaWAN}
		\پایان{فقرات}
	\end{frame}

	\begin{frame}{\cite{sensors-18-00772-v3}}
		\شروع{فقرات}
		\لیست‌فارسی
		\فقره یک نود متحرک که پارامترهای ارسالش با زمان تغییر می‌کند.
		\فقره ناحیه‌های شهری، نیمه‌شهری و غیرشهری
		\فقره ارزیابی رادیویی بر پایه مدل انتشار \متن‌لاتین{Okumura-Hata} پیش از انجام ارزیابی عملیاتی
		\فقره نود متحرک برای بازه‌های زمانی ثابت می‌ماند تا تاثیر حرکت در پارامترها دیده شود.
		\فقره در انتها این پژوهش بیان می‌کند برای اجرای یک زیرساخت \متن‌لاتین{LoRaWAN} باید مصالحه‌ای بین کیفیت لینک، نرخ داده‌ی انتقالی و سیار بودن نود برقرار شود.
		\پایان{فقرات}
	\end{frame}

	\begin{frame}{\cite{sensors-19-00007}}
		\شروع{فقرات}
		\لیست‌فارسی
		\فقره ارزیابی بین \متن‌لاتین{MQTT} و \متن‌لاتین{CoAP} که به ترتیب بر بسترهای \متن‌لاتین{UDP} و \متن‌لاتین{TCP} فعالیت می‌کنند.
		\فقره شبکه زیرساخت \متن‌لاتین{NB-IoT} می‌باشد که با فعالیت روی باند دارای لایسنس و نبود \متن‌لاتین{duty-cycle} امکان اجرای \متن‌لاتین{tcp} را نیز فراهم می‌آورد.
		\فقره نشان می‌دهد \متن‌لاتین{MQTT} نسبت به \متن‌لاتین{CoAP} کارآیی کمتری در معیارهای تاخیر، پوشش و ظرفیت سیستم دارد.
		\پایان{فقرات}
	\end{frame}

	\begin{frame}{\cite{sensors-20-03061-v2}}
		\شروع{فقرات}
		\لیست‌فارسی
		\فقره \رنگ‌متن{نارنجی}{جداسازی پیام‌های دورسنجی از اخطارها} به وسیله‌ی شبه عمود بودن \متن‌لاتین{SF}ها در شبکه‌های \متن‌لاتین{LoRaWAN}.
		\فقره استفاده از دو استراتژی برای تخصیص \متن‌لاتین{SF}های متمایز برای داده‌های اخطاری و داده‌های دورسنجی
		\فقره آزمایش عملی این پژوهش به دلیل نیاز به تعداد بالای نود و نیاز به تغییر رویه تخصص \متن‌لاتین{SF}ها امکان‌پذیر نیست.
		\پایان{فقرات}
	\end{frame}

	\begin{frame}{\cite{sensors-20-00280-v2}}
		\شروع{فقرات}
		\لیست‌فارسی
		\فقره پیاده‌سازی الگوریتم \متن‌لاتین{Static Context Header Compression (SCHC)} برای متن‌لاتین{IPv6} و استفاده از آن برای انتقال \متن‌لاتین{CoAP} بر پایه \متن‌لاتین{UDP}
		\فقره منابع مصرفی در جهت استفاده از \متن‌لاتین{IPv6} نسبت به سود حاصل از آن بسیار کم است. در مقابل کارایی انرژی و داده در \متن‌لاتین{fragmentation} کم است.
		\فقره زمانی یک نود بین \متن‌لاتین{gateway}ها جابجا می‌شود در صورت لزوم پروسه پیوستن به \متن‌لاتین{NS} را انجام می‌دهد. در صورت استفاده از یک آدرس \متن‌لاتین{IPv6} ثابت اتصال تضمین می‌شود.
		\فقره برای استفاده از قطعه‌بندی \متن‌لاتین{IETF} نیاز است که ترتیب بسته‌ها حفظ شود.
		\فقره در نظر گرفتن نودهای متحرک و سناریوهای انتها به انتها چیزی که در این پژوهش به آن پرداخته نشده است.
		\پایان{فقرات}
	\end{frame}

	\begin{frame}{\cite{sensors-20-06721.pdf}}
		\شروع{فقرات}
		\لیست‌فارسی
		\فقره تجربه بیش از دو سال مدیریت و نگهداری شبکه‌های سنسورهای داخلی مبتنی بر \متن‌لاتین{LoRa}
		\فقره از دست رفتن بسته‌ها به جز در انتقال در \متن‌لاتین{Backend} هم رخ می‌دهد.
		\فقره مقالات بسیاری را بر پایه این تجربه به چاپ رسانده‌اند.
		\پایان{فقرات}
	\end{frame}


	\begin{frame}{\cite{sensors-20-02078}}
		\شروع{فقرات}
		\لیست‌فارسی
		\فقره کارهای زیادی برای کشاورزی دقیق و با مصرف باطری کم پیشنهاد شده‌اند اما تعداد کمی از آن‌ها در عمل تست شده‌اند.
		\فقره این پژوهش قصد دارد مصرف انرژی قسمت‌های مختلف را در شرایط واقعی محاسبه و کاهش دهد.
		\فقره زمانی که داده‌ی زیادی برای ارسال موجود نیست مانند سنسورهای کشاورزی، استفاده از \متن‌لاتین{LoRa} گزینه خوبی است.
		\فقره پوشش \متن‌لاتین{LTE} در نواحی غیرشهری و کشاورزی کافی نیست بنابراین استفاده از \متن‌لاتین{NB-IoT} گزینه خوبی نیست.
		\پایان{فقرات}
	\end{frame}

	\begin{frame}{\cite{sensors-21-01924-v2}}
		\شروع{فقرات}
		\لیست‌فارسی
		\فقره ارائه یک روش کاهش حجم داده مبتنی بر حذف داده‌های تکراری
		\فقره این روش به صورت عام بوده و می‌توان آن را با روش \متن‌لاتین{SCHC} ترکیب کرد.
		\پایان{فقرات}
	\end{frame}

	\begin{frame}{\cite{sustainability-12-08443}}
	  \شروع{فقرات}
	  \لیست‌فارسی
	  \فقره پیاده‌سازی یک \متن‌لاتین{OBU} برای اسوکترها و وسایل حمل و نقل نوظهور (مانند دوچرخه یا اسکوترهای برقی)
	  \فقره استفاده از \متن‌لاتین{NB-IoT} و \متن‌لاتین{LoRaWAN} برای ارتباط
	  \پایان{فقرات}
	\end{frame}

	\begin{frame}{نوآوری‌ها}
		\شروع{فقرات}
		\لیست‌فارسی
		\فقره در نظر گرفتن شی متحرک با جابجایی میان \متن‌لاتین{Gateway}ها و انجام رویه‌ی \متن‌لاتین{Roaming}
		\فقره در نظر گرفتن کلاس‌های مختلف اشیا
		\فقره در نظر گرفتن استفاده مشترک در شبکه‌های \متن‌لاتین{WiFi}
		\فقره در نظر گرفتن سیستم انتها به انتها مبتنی بر \متن‌لاتین{IPv6}
		\فقره در نظر گرفتن نودهای عملگر
		\فقره در نظر گرفتن الویت و کیفیت سرویس
		\پایان{فقرات}
	\end{frame}

	% -------------------------------------------------------------------------------
	\begin{frame}[allowframebreaks]{مراجع}
		\begin{latin}
			\printbibliography[title=مراجع]
		\end{latin}
	\end{frame}

\end{persian}
\end{document}
