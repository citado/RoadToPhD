% !TeX TS-program = xelatex

\documentclass[dvipsnames]{beamer}
\usetheme{metropolis}

\usepackage{multicol}

\usepackage{ragged2e} % who justifies the text
\usepackage{xecolor}
\usepackage{amsmath}
% \usefonttheme[onlymath]{serif} % change the math font

\usepackage{tabularx}
\usepackage{booktabs}
\usepackage[style=numeric,sorting=ynt]{biblatex}
\addbibresource{references.bib}
\usepackage[localise]{xepersian}

\settextfont[
	Path = fonts/,
	UprightFont = *-Regular,
	BoldFont = *-Bold,
	ItalicFont = *-Variable
]{Vazir}
\setlatintextfont[
	Path = fonts/,
	UprightFont = *-Regular,
	BoldFont = *-Bold,
	ItalicFont = *-Italic
]{Neuton}

% ---------------------------------------------------------------------------------
% Colors
% ---------------------------------------------------------------------------------
\definecolor{نارنجی}{rgb}{1.0, 0.31, 0.0}

% ---------------------------------------------------------------------------------
% Settings to force Beamer works with Xepersian and RTL typesetting
% ---------------------------------------------------------------------------------
% \raggedleft

\eqcommand{مرجع‌پرانتزی}{parencite}

% For right to left lists (itemize and enumerate)
\makeatletter
\newcommand{\لیست‌فارسی}{\raggedleft\rightskip\@totalleftmargin}
\makeatother
% Correct the bullet for RTL texts
\setbeamertemplate{itemize item}{\scriptsize\raise1.25pt%
	\hbox{\donotcoloroutermaths$\blacktriangleleft$}}

% To force beamer use numbering in captions
\setbeamertemplate{caption}[numbered]{}% Number float-like environments

\setbeamertemplate{footline}[frame number]
\setbeamertemplate{section in toc}[circle]
\setbeamertemplate{blocks}[rounded][shadow=true]
\setbeamercolor{block title}{bg=orange}
\setbeamercolor{block body}{bg=lightgray}
\setbeamercolor{headline}{bg=orange}
\setbeamersize{text margin left=1cm,text margin right=1cm}
\setbeamertemplate{frametitle continuation}{\insertcontinuationcount}

\setbeamertemplate{headline}
{
	\begin{beamercolorbox}{section in head/foot}
		\vspace{2pt}\insertnavigation{\paperwidth}\vspace{2pt}
	\end{beamercolorbox}
}

\setbeamertemplate{footline}
{%
	\leavevmode%
	\hbox{%
		\begin{beamercolorbox}[wd=.333333\paperwidth,ht=2.25ex,dp=1ex,center]{author in head/foot}%
			\usebeamerfont{author in head/foot}\insertshortauthor%
		\end{beamercolorbox}%
		\begin{beamercolorbox}[wd=.333333\paperwidth,ht=2.25ex,dp=1ex,center]{title in head/foot}%
			\usebeamerfont{title in head/foot}\insertshorttitle%
		\end{beamercolorbox}%
		\begin{beamercolorbox}[wd=.333333\paperwidth,ht=2.25ex,dp=1ex,right]{date in head/foot}%
			\usebeamerfont{date in head/foot}\insertsection\hspace*{2em}
			\insertframenumber~ از \inserttotalframenumber{} \hspace*{2ex}%
		\end{beamercolorbox}
	}%
}

% ---------------------------------------------------------------------------------
\title{ارزیابی انتها به انتها شبکه \متن‌لاتین{LoRaWAN}}
\subtitle{}
\author{پرهام الوانی}
\institute{%
	دانشکده مهندسی کامپیوتر\\
	دکتر بهادر بخشی و دکتر مهدی راستی
}
\date{\today}
\titlegraphic{\vspace{4.5cm}\flushleft\includegraphics[height=50pt]{images/logo}}

\setbeamertemplate{title}{%
	\linespread{1.0}%
	\inserttitle%
	\par%
	\vspace*{0.5em}
}
\setbeamertemplate{subtitle}{%
	\insertsubtitle%
	\par%
	\vspace*{0.5em}
}

\AtBeginSection[]
{%
	\begin{frame}{فهرست}
	  \RaggedLeft
	  \tableofcontents[currentsection]
	\end{frame}
	\begin{frame}
		\begin{center}
			\insertsectionnumber. \insertsection%
		\end{center}
		\usebeamertemplate*{title separator}
	\end{frame}
}

\begin{document}

\begin{persian}

	% ------------------------------------------
	% Title frame (0)
	% ------------------------------------------
	{%
		\setbeamertemplate{footline}{}
		\begin{frame}
		  \titlepage%
		\end{frame}
	}

	% -------------------------------------------------------------------------------
	\begin{frame}{فهرست}
	  \RaggedLeft
	  \tableofcontents[pausesections]
	\end{frame}

	% -------------------------------------------------------------------------------
	\قسمت{مقدمه}

	\begin{frame}{اینرتنت اشیا}
		\شروع{فقرات}
		\فقره اینترنت اشیا اولین بار توسط \متن‌لاتین{Kevin Ashton} در سال ۱۹۹۹ پیشنهاد شد
		\فقره با توجه به تخمین \متن‌لاتین{Cisco} در بازه‌ای بین سال‌های ۲۰۰۸ تا ۲۰۰۹ که برای اولین بار تعداد اشیا
		متصل از جمعیت جهان بیشتر شد، اینترنت اشیا متولد شد.
		\فقره در سال ۲۰۱۰ تعداد اشیا متصل تقریبا دو برابر جمعیت جهان در آن سال شد و تقریبا به عدد ۱۲/۵ بیلیون رسید.
		\فقره با پیشرفت‌های تکنولوژی و سرمایه‌گذاری‌های قابل توجه شرکت‌ها، اینترنت اشیا در حال گسترش در زندگی روزمره است
		\پایان{فقرات}
	\end{frame}

	\begin{frame}{شبکه‌های توان پایین با برد بلند}
		\شروع{فقرات}
		\فقره در حال حاضر بیشتر شبکه‌های اینترنت اشیا از بسترهای \متن‌لاتین{WSN} مانند \متن‌لاتین{Zigbee}، \متن‌لاتین{Bluetooth} یا \متن‌لاتین{WiFi} ساخته شده‌اند.
		\فقره این تکنولوژی‌ها با گسترش روزافزون اینترنت اشیا همخوانی ندارند، در آینده ما نیاز داریم تا هزینه هر واحد را کاهش دهیم، پوشش شبکه را گسترده‌تر کنیم، توان مصرفی نودهای در لبه را کاهش دهیم و شبکه‌های گسترش‌پذیر داشته باشیم.
		\فقره الگو جدیدی تحت عنوان \متن‌لاتین{LPWAN} یا \متن‌لاتین{Low-Power Wide Area Network} برای پوشش دادن این شکاف به وجود آمده است.
		\پایان{فقرات}
	\end{frame}

	\begin{frame}{شبکه‌های توان پایین با برد بلند}
		\شروع{فقرات}
		\لیست‌فارسی
		\فقره در این سال‌های شبکه‌های مختلفی تحت عنوان \متن‌لاتین{LPWAN} پیشنهاد شده‌اند که از جمله‌ی آن‌ها می‌توان به \متن‌لاتین{LoRa}، \متن‌لاتین{Sigfox} و \متن‌لاتین{NB-IoT} اشاره کرد.
		\فقره با توجه به استفاده از باند فرکانسی بدون مجوز در \متن‌لاتین{LoRa} و امکان راه‌اندازی شبکه‌های \متن‌لاتین{LoRa} توسط اشخاص ثالث پژوهش‌های زیادی در این سال‌ها به این نوع شبکه‌ها پرداخته‌اند.
		\فقره \متن‌لاتین{LoRa} بیشترین تکنولوژی استفاده شده در کاربردهای اینترنت اشیا با برد بالا، نرخ داده کم و توان مصرفی پایین است.
		\پایان{فقرات}
	\end{frame}

	\begin{frame}{شبکه‌های \متن‌لاتین{LoRaWAN}}
		\شروع{فقرات}
		\فقره \متن‌لاتین{LoRa} توسط \متن‌لاتین{Semtech} ثبت شده است و برای همگان انتشار نیافته است.
		\فقره \متن‌لاتین{LoRaWAN} یک لایه پیوند داده بر پایه لایه فیزیکی \متن‌لاتین{LoRa} است.
		\فقره \متن‌لاتین{LoRaWAN} یک استاندارد رایگان (منتشر شده از سوی \متن‌لاتین{LoRa Alliance} در سال ۲۰۱۵) بوده و در اختیار همگان است.
		\فقره در \متن‌لاتین{LoRaWAN} ارتباطی میان دروازه‌ها و گره‌ها وجود ندارد و داده‌ی ارسالی توسط هر گره می‌تواند به وسیله‌ی یک یا چند دروازه دریافت شود.
		\فقره دروازه‌ها وظیفه ارسال ترافیک گره‌های \متن‌لاتین{LoRa} به سرور شبکه و برعکس را دارا هستند.
		\فقره عملیات احراز هویت، شناسایی و حذف بسته‌های تکراری توسط سرور شبکه صورت می‌پذیرد.
		\فقره سرور شبکه تنظیمات شبکه را به وسیله‌ی مکانیزم نرخ داده تطبیق‌پذیر را نیز انجام می‌دهد.
		\پایان{فقرات}
	\end{frame}

	\begin{frame}
		\begin{figure}
		\includegraphics[width=\textwidth]{./images/nrm-home.png}
		\centering
		\caption{مدل مرجع شبکه \متن‌لاتین{LoRaWAN} - شبکه‌ی خانگی}
		\end{figure}
	\end{frame}

	\section{کارهای مرتبط}

	\begin{frame}{ارزیابی شبکه دسترسی}
		\fontsize{5pt}{6pt}\selectfont
		\begin{tabularx}{\textwidth}{|*{13}{X|}}
			\toprule
			مرجع &
			جابجایی &
			پوشش‌دهی &
			شبکه &
			لایه انتقال &
			لایه اپلیکشن &
			اندازه بسته &
			نرخ کدگذاری &
			تاخیر &
			محیط &
			\متن‌لاتین{LoRa Mesh} &
			توان مصرفی &
			شبیه‌سازی \\
			\midrule
			\cite{sensors-18-00772-v3} &
			سیار &
			گزارش شده &
			\متن‌لاتین{LoRa} &
			ندارد &
			ندارد &
			گزارش شده &
			گزارش شده &
			گزارش نشده &
			باز &
			ندارد &
			گزارش نشده &
			واقعی / \متن‌لاتین{cloudRF} \\
			\midrule
			\cite{sensors-19-00007} &
			ثابت &
			گزارش شده &
			\متن‌لاتین{NB-IoT} &
			\متن‌لاتین{TCP / UDP} &
			\متن‌لاتین{MQTT / CoAP} &
			گزارش شده &
			گزارش نشده &
			گزارش شده &
			باز &
			ندارد &
			گزارش نشده &
			\متن‌لاتین{Ericsson inter. sim.} \\
			\midrule
			\cite{sensors-20-03061-v2} &
			ثابت &
			گزارش شده &
			\متن‌لاتین{LoRa} &
			ندارد &
			ندارد &
			گزارش شده &
			گزارش شده &
			گزارش شده &
			باز / بسته &
			ندارد &
			گزارش نشده &
			\متن‌لاتین{ns-3} \\
			\midrule
			\cite{sensors-20-00280-v2} &
			ثابت &
			گزارش شده &
			\متن‌لاتین{LoRa} &
			\متن‌لاتین{UDP ov IPv6} &
			\متن‌لاتین{CoAP} &
			گزارش شده &
			گزارش شده &
			گزارش شده &
			باز / بسته &
			ندارد &
			گزارش شده &
			واقعی \\
			\midrule
			\cite{sensors-20-06721} &
			ثابت &
			گزارش شده &
			\متن‌لاتین{LoRa} &
			ندارد &
			ندارد &
			گزارش شده &
			گزارش شده &
			گزارش شده &
			بسته &
			ندارد &
			گزارش شده &
			واقعی \\
			\midrule
			\cite{SanchezIborra2020} &
			متحرک &
			گزارش شده &
			\متن‌لاتین{LoRa \ NB-IoT} &
			\متن‌لاتین{UDP/TCP ov IPv6} &
			\متن‌لاتین{CoAP / ReST} &
			گزارش شده &
			گزارش شده &
			گزارش شده &
			باز &
			ندارد &
			گزارش شده &
			واقعی \\
			\midrule
			\cite{Lee2018} &
			ثابت &
			گزارش شده &
			\متن‌لاتین{LoRa} &
			ندارد &
			ندارد &
			گزارش نشده &
			گزارش شده &
			گزارش شده &
			باز / بسته &
			دارد &
			گزارش نشده &
			واقعی \\
			\midrule
			\cite{Marahatta2021} &
			ثابت &
			گزارش شده &
			\متن‌لاتین{LoRa} &
			ندارد &
			ندارد &
			گزارش شده &
			گزارش شده &
			گزارش شده &
			باز / بسته &
			دارد &
			گزارش نشده &
			\متن‌لاتین{ns-2} \\
			\bottomrule
		\end{tabularx}
	\end{frame}

	\begin{frame}{ارزیابی شبکه دسترسی}
	  \شروع{فقرات}
	  \فقره پژوهشگران \مرجع{sensors-20-06721} \مرجع‌پرانتزی{sensors-20-06721} تجربه بیش از دو سال نگهداری از شبکه‌ی سنسورهای فضای بسته دانشگاه \متن‌لاتین{oulu} کشور فلاند مبتنی بر \متن‌لاتین{LoRaWAN} در این پژوهش مرور می‌کنند.
	  \فقره یکی از موارد مهمی که در این پژوهش به آن اشاره می‌شود، از دست رفتن بسته‌ها به جز در شبکه‌ی دسترسی و
	  شبکه میان میان سرور شبکه و سرور پلتفرم است.
	  \فقره این پژوهش به بررسی بیشتر این موضوع نپرداخته است و دلیلی ارائه نمی‌دهد.
	  \پایان{فقرات}
	\end{frame}

	\begin{frame}{تاخیر انتها به انتها}
	  \شروع{فقرات}
	  \فقره \مرجع{FernandesCarvalho2019} روشی برای ارزیابی تاخیر \متن‌لاتین{uplink} در استقرارهای \متن‌لاتین{LoRaWAN} ارائه می‌کند.
	  \فقره \مرجع{Carvalho2019} تاخیر ناشی از زیرساخت \متن‌لاتین{LoRaWAN} را از دروازه تا سرور شبکه ارزیابی کرده و بیان می‌کند بیشترین تاخیر متعلق به سرور شبکه است.
	  \فقره \مرجع{Potsch2019} ارزیابی استقرار محلی و ابری زیرساخت شبکه‌ی \متن‌لاتین{LoRaWAN} از جهت تاخیر بسته از گره تا دریافت کامل بسته در برنامه کاربردی را انجام می‌دهد.
	  \پایان{فقرات}
	\end{frame}

	\begin{frame}{پروتکل‌های وب}
	  \شروع{فقرات}
	  \فقره \مرجع{sensors-20-00280-v2} الگوریتم \متن‌لاتین{Static Context Header Compression (SCHC)} را برای \متن‌لاتین{IPv6} پیاده‌سازی کرده و آن
	  برای انتقال پروتکل \متن‌لاتین{CoAP} بر بستر \متن‌لاتین{UDP} ارزیابی می‌کند.
	  \پایان{فقرات}
	\end{frame}

	\begin{frame}{جمع‌بندی}
	  \شروع{فقرات}
	  \فقره پژوهش‌ها به قسمت‌ها یا قسمت‌هایی از معماری \متن‌لاتین{LoRaWAN} پرداخته‌اند.
	  \فقره معماری \متن‌لاتین{LoRaWAN} به صورت انتها به انتها و با چگالی بالا مورد ارزیابی قرار نگرفته است.
	  \فقره پروتکل‌هایی مانند \متن‌لاتین{IPv6} در شبکه‌های \متن‌لاتین{LoRaWAN} به صورت انتها به انتها مورد ارزیابی قرار نگرفته‌اند.
	  \پایان{فقرات}
	\end{frame}

	\قسمت{شکاف تحقیقاتی}

	\begin{frame}
		\شروع{فقرات}
		\فقره شبکه \متن‌لاتین{LoRaWAN} از گره‌ها، دروازه‌ها، سرور شبکه و سرور اپلیکیشن تشکیل شده است.
		\فقره استاندارد \متن‌لاتین{LoRaWAN} بیشتر پردازش‌ها را در سمت شبکه هسته قرار داده است.
		\فقره همانطور که بیان شد بیشتر پژوهش‌ها بر روی ارزیابی شبکه‌ی دسترسی بی‌سیم تمرکز کرده‌اند.
		\فقره دیده نشده تاثیر پارامترهایی مانند نرخ ارسال، تعداد اشیا به صورت انتها به انتها
		\فقره دیده نشدن تاثیر پارامترهایی مانند ساختار داده‌ای و کدگذاری داده در ارتباط میان گره‌ها و برنامه‌های کاربردی اینترنت اشیا
		\پایان{فقرات}
	\end{frame}

	\قسمت{بیان مسائل}

	\begin{frame}{ارزیابی شبکه دسترسی در شبکه \متن‌لاتین{LoRaWAN} از دروازه تا سرور شبکه و سرور اپلیکیشن}
		\شروع{فقرات}
		\فقره ارزیابی شبکه هسته مبتنی بر \متن‌لاتین{IP} در معماری \متن‌لاتین{LoRaWAN}
		\فقره معیارهای کارایی
		\شروع{فقرات}
		\فقره نرخ دریافت صحیح بسته
		\فقره تاخیر
		\فقره بهره‌وری
		\پایان{فقرات}
		\پایان{فقرات}
	\end{frame}

	\begin{frame}{بخشی از پارامترهای موثر}
	  \شروع{فقرات}
	  \فقره روش‌های ارسال داده توسط اشیا
		\شروع{فقرات}
		\فقره \متن‌لاتین{Periodic}
		\فقره \متن‌لاتین{Self Trigger}
		\فقره \متن‌لاتین{Trigger by Event}
		\پایان{فقرات}
		\فقره نرخ ارسال داده
		\فقره تعداد اشیا
		\پایان{فقرات}
	\end{frame}

	\begin{frame}{ارزیابی لایه کاربرد شبکه \متن‌لاتین{LoRaWAN} از گره تا اپلیکیشن}
	  \شروع{فقرات}
	  \فقره ارزیابی تاثیر پروتکل‌هایی مانند \متن‌لاتین{IPv6} به صورت انتها به انتها
	  \پایان{فقرات}
	\end{frame}

	\begin{frame}{}
	  \تنظیم‌ازوسط
	  \درج‌تصویر[width=\textwidth]{./images/e2e-iot-lorawan.png}
	\end{frame}

	\قسمت{رویکرد حل مساله}

	\begin{frame}
	  \شروع{فقرات}
	  \فقره مدل‌های تحلیلی
	  \شروع{فقرات}
	  \فقره ارزیابی حالت میانگین با استفاده از تئوری صف
	  \فقره ارزیابی حالت‌های حدی با استفاده از \متن‌لاتین{Network Calculus}
	  \پایان{فقرات}

	  \فقره شبیه‌سازی نزدیک به واقعیت
	  \شروع{فقرات}
	  \فقره انجام شبیه‌سازی در شرایط مختلف مبتنی بر روش مونت کارلو
	  \فقره استفاده از پیاده‌سازی‌های واقعی برای اجزای معماری
	  \پایان{فقرات}

	  \فقره بهبودهای پیشنهادی
	  \شروع{فقرات}
	  \فقره نزدیک کردن مقادیر حاصل از شبیه‌سازی به مدل‌های تحلیلی
	  \فقره استفاده از پردازش لبه در راستای کاهش تاخیر و افزایش بهره‌وری
	  \فقره استفاده از الگوریتم‌های یادگیری تقویتی در مواردی مانند مشخص کردن تعداد بهینه دروازه‌ها
	  \پایان{فقرات}
	  \پایان{فقرات}
	\end{frame}

	\قسمت{شبیه‌سازی}

	\begin{frame}
	  \شروع{فقرات}
	  \فقره شبیه‌سازی مدل ساده شده از شبکه دسترسی \متن‌لاتین{LoRaWAN} از دروازه تا سرور شبکه و سرور اپلیکیشن
	  \فقره معیارهای تاخیر و نرخ از دست رفته بسته‌ها
	  \فقره ارزیابی تاثیر افزایش نرخ ارسال اشیا
	  \فقره ارزیابی تاثیر افزایش تعداد اشیا
	  \فقره ارزیابی تاثیر افزایش تعداد دروازه‌ها
	  \پایان{فقرات}
	\end{frame}

	\begin{frame}
		\begin{figure}
		\includegraphics[width=\textwidth]{./images/chirpstack-architecture.png}
		\centering
		\caption{معماری \متن‌لاتین{LoRaWAN} سرور متن باز \متن‌لاتین{Chirpstack}}
		\end{figure}
	\end{frame}

	\begin{frame}
	  \شروع{فقرات}
	  \فقره افزایش تعداد دروازه‌ها برخلاف شبکه دسترسی بی‌سیم در شبکه هسته نرخ دریافت صحیح بسته را افزایش نمی‌دهد.
	  \پایان{فقرات}
	\end{frame}

	\begin{frame}
	  \تنظیم‌ازوسط
	  \درج‌تصویر[height=\textheight]{../proposal/simulation/s1/e7.png}
	\end{frame}

	% -------------------------------------------------------------------------------
	\begin{frame}[allowframebreaks]{مراجع}
		\begin{latin}
			\printbibliography[title=مراجع]
		\end{latin}
	\end{frame}

\end{persian}
\end{document}
